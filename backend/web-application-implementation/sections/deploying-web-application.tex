\section{Deploying Web Application}\label{sect:deploying-web-application}

\subsection*{Deploying with Web Server Program}

지금까지 간단한 게시물 웹 애플리케이션을 구현하였으며, 이를 \shellref{shell:npm-run-script}\와 같이 node로 실행하여 사용해보았다. 이제 이 애플리케이션을 실제 서비스를 위해 deploy해보자. 사실 node로 실행하여도 웹 애플리케이션이 제 기능을 하지 못하는 것은 아니고, 외부에서도 잘 접속된다.

그러나 이러한 실행 방식은 여러 한계가 있는데, 특히 성능과 보안 문제가 가장 큰 문제이다. 먼저 외부에서 웹 서버에 HTTP 요청을 보내려면 서버에 연결되어 있는 포트가 개방(open)되어 있어야 하는데, 서비스가 실제로 사용중인 포트가 외부로 직접 노출되어 있는 것은 보안상 좋지 않다. 또한, Node.js는 기본적으로 싱글 스레드를 기반으로 동작하는 플랫폼이기 때문에 동시다발적인 요청이 들어왔을 때 그 처리 속도가 크게 저하될 수밖에 없다.

이러한 문제점을 해결하기 위해 production에서는 Apache HTTPd, nginx, Cloudflare 등의 웹 서버 프로그램을 클라이언트와 서버 사이에 위치시켜 서비스를 deploy한다. 이 중 nginx는 2002년 러시아의 이고르 시쇼브에 의해 개발된 웹 서버 프로그램으로, 이전까지 압도적인 점유율을 자랑하던 Apache HTTPd의 여러 한계점으로 인해 점유율이 빠르게 늘어나고 있다. 영국의 인터넷 서비스 회사인 Netcraft의 조사에 따르면 2021년 9월 기준 Apache HTTPd와 nginx의 점유율은 각각 24.21\%와 20.33\%로 nginx의 점유율이 Apache HTTPd의 점유율에 거의 근접하고 있다.\footnote{https://news.netcraft.com/archives/2021/09/29/september-2021-web-server-survey.html}

이번 장에서는 Linux 계열의 대표적인 OS인 Ubuntu 20.04에서 nginx를 이용하여 웹 애플리케이션을 deploy해본다.

\subsection*{Project Configuration and nginx Installation}

Shell을 열고 git(\shellref{shell:git-install-ubuntu}, \pageref{shell:git-install-ubuntu}쪽), Node.js(\shellref{shell:nodejs-linux-install}, \pageref{shell:nodejs-linux-install}쪽), MariaDB(\shellref{shell:mariadb-installation-ubuntu}, \pageref{shell:mariadb-installation-ubuntu}쪽)를 설치한다. 먼저 MariaDB를 실행하여 웹 애플리케이션에 사용할 DB 사용자 계정과 데이터베이스를 생성하고, DB 사용자에게 데이터베이스의 모든 권한을 부여한다. (\sectref{sect:maria-db} 참고)

\begin{shellenv}{shell:project-config}{Project Configuration}\begin{verbatim}
$ git clone <git-repo-url>
$ cd <repo-directory-name>
$ npm install --only=prod
$ mysql -u<db-user> -p<db-pass> -D<db-name> < schema/0000.sql
$ vim .env
\end{verbatim}
\end{shellenv}

\shellref{shell:project-config}\와 같이 git을 이용해 프로젝트를 clone하고, package.json에서 \cd{dependencies} 항목의 패키지를 설치한 뒤, schema/0000.sql 파일을 DB에 import한다. 이후 Vim을 이용하여 .env 파일을 생성하는데, \cd{MODE}의 값을 \cd{prod}로 설정한다.

\begin{shellenv}{shell:install-deploy-packages}{Install Deploy Packages}\begin{verbatim}
$ sudo apt update
$ sudo apt upgrade -y
$ sudo apt install -y ufw nginx
\end{verbatim}
\end{shellenv}

\shellref{shell:install-deploy-packages}\와 같이 ufw, nginx 패키지를 설치한다.

\begin{shellenv}{shell:get-server-ip}{Get Public IP of Server Machine}\begin{verbatim}
$ curl ifconfig.me
\end{verbatim}
\end{shellenv}

\shellref{shell:get-server-ip}\와 같이 서버 컴퓨터의 공용(public) IP 주소를 확인한다.

\begin{shellenv}{shell:allow-port-80}{Allow Port 80/TCP}\begin{verbatim}
$ sudo ufw allow 80/tcp
$ sudo ufw enable
$ sudo ufw verbose
\end{verbatim}
\end{shellenv}

\shellref{shell:allow-port-80}\과 같이 80번 포트의 TCP 연결을 허용하고, ufw를 실행한다. 이제 서버 컴퓨터가 아닌 외부 컴퓨터에서 웹 브라우저를 열고, 서버 컴퓨터의 공용 IP 주소에 접속하여 \figref{fig:nginx-root-page}\와 같은 응답을 받는지 확인한다.

\figures{fig:nginx-root-page}{nginx Root Page}{
    \fig{images/implementing-web-application/nginx-root-page.png}{.6}
}

\subsection*{nginx Configuration}

이제 nginx 서버와 Express 웹 애플리케이션을 연동해보자.

\begin{shellenv}{shell:open-nginx-config-file}{Open nginx Configuration File}\begin{verbatim}
$ sudo vim /etc/nginx/sites-available/node-simple-board
\end{verbatim}
\end{shellenv}

\shellref{shell:open-nginx-config-file}\과 같이 /etc/nginx/sites-available 디렉토리에 node-simple-board 파일을 Vim 편집기로 생성하며 연다.

\begin{codeenv}{code:nginx-config}{nginx Configuration}\begin{verbatim}
server {
    listen 80;
    server_name <server-public-ip>;

    charset utf-8;

    location / {
        proxy_pass http://localhost:<web-app-port>;
    }
}
\end{verbatim}
\end{codeenv}

/etc/nginx/sites-available/node-simple-board 파일을 \coderef{code:nginx-config}\와 같이 작성한다. \cd{<server-public-ip>}와 \cd{<web-app-port>}에는 각각 서버 컴퓨터의 공용 IP와 웹 애플리케이션의 실행 포트 번호를 작성한다. 이렇게 설정하면 외부에서 \cd{server\_name}과 \cd{listen}의 값, 즉 서버 공용 IP의 80번 포트로 들어오는 요청을 받아들이며, 이들 요청 중 \cd{location}, 즉 / 경로로 들어오는 모든 요청을 그 내부에 명시된 \cd{proxy\_pass}, 즉 Express.js 웹 애플리케이션의 주소로 보낸다. 이후 웹 애플리케이션으로부터 응답을 받으면 원래 사용자에게 다시 전송하여 응답한다.

\figures{fig:proxy-servers}{Proxy Servers}{
    \subfig{fig:forward-proxy}{Forward Proxy}
        {images/implementing-web-application/forward-proxy.png}{.45}
    \subfig{fig:reverse-proxy}{Reverse Proxy}
        {images/implementing-web-application/reverse-proxy.png}{.45}
}

클라이언트와 서버 사이에 위치하여 한 쪽의 실제 위치를 감추는 서버를 프록시 서버(\figref{fig:proxy-servers})라고 하며, nginx와 같이 서버의 실제 위치를 감추는 서버를 리버스 프록시 서버(\figref{fig:reverse-proxy})라고 한다. 이러한 리버스 프록시 서버를 이용하면 서버의 실제 위치가 직접적으로 노출되지 않아 외부의 공격으로부터 훨씬 안전하다.

또한, 이러한 server block을 여러 개 만들어서 HTTP 요청을 여러 서버에서 나누어 처리하게끔 할 수도 있다.

\begin{shellenv}{shell:symbolic-link-and-restart}{Create Symbolic Link and Restart nginx}\begin{verbatim}
$ sudo ln -s /etc/nginx/sites-available/node-simple-board \
    /etc/nginx/sites-enabled/node-simple-board
$ sudo service nginx restart
\end{verbatim}
\end{shellenv}

이제 \shellref{shell:symbolic-link-and-restart}\와 같이 /etc/nginx/sites-enabled 디렉토리 아래에 \coderef{code:nginx-config} 파일의 symbolic link\footnote{Windows의 바로가기와 비슷한 기능}를 생성하고, nginx를 재실행한다.

\begin{shellenv}{shell:run-web-app}{Run Express Web Application}\begin{verbatim}
$ cd <web-app-directory>
$ npm run start
\end{verbatim}
\end{shellenv}

마지막으로, Express.js 애플리케이션 프로젝트 디렉토리로 돌아와서 애플리케이션을 실행한다. 그리고 서버 컴퓨터의 공용 IP로 접속하여 이번 장에서 구현한 웹 애플리케이션이 잘 작동하는지 확인한다.
