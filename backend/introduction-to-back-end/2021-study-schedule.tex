\section{2021 Study Schedule}\label{sect:2021-study-schedule}

\subsection*{스터디 일정}

\begin{tblenv}
    {tab:backend-study-schedule}
    {2021 Back-end Study Schedule}
    {>{\colc}m{0.04\tw}>{\colc}m{0.15\tw}>{\coll}m{0.45\tw}>{\coll}m{0.25\tw}}
    \thickhline
    회차 & 스터디 날짜 & 스터디 내용 & 비고 \tabularnewline
    \hline
    1 & 09/13 - 09/17 & Introduction to Back-end & \tabularnewline
    2 & 09/27 - 10/01 & Advanced Javascript & \tabularnewline
    3 & 10/04 - 10/08 & Express.js and View Engine & \tabularnewline
    4 & 10/11 - 10/15 & Database Modeling & 중간고사 이전 회차\tabularnewline
    5 & 11/01 - 11/05 & Database Querying & 중간고사 이후 회차\tabularnewline
    6 & 11/08 - 11/12 & Authentication, Web Application Implementation & \tabularnewline
    7 & 11/15 - 11/19 & Web Application Implementation & \tabularnewline
    8 & 11/22 - 11/26 & Web Application Implementation & \tabularnewline
    9 & 11/29 - 12/03 & Web Application Implementation & 기말고사 이전 회차\tabularnewline
    \thickhline
\end{tblenv}

\subsection*{OUT COUNT 제도}

준회원 스터디 과정에서는 3 OUT COUNT 제도가 시행되며, OUT COUNT는 다음과 같은 경우에 적용된다.

\begin{itemize}
    \item 0.5 OUT : 스터디 사유 불참, 스터디 무단 지각, 과제 지각 제출
    \item 1.0 OUT : 스터디 무단 불참, 과제 미제출
    \item 2.0 OUT : 과제 Cheating
\end{itemize}

스터디 지각은 스터디 시작 이후 15분까지만 인정되며, 과제 지각 제출은 과제 제출 기한 이후 3일 이내에 제출되었을 경우에만 인정된다. OUT COUNT가 3.0을 초과하면 본 스터디의 Pass 자격이 박탈될 수 있다.

\subsection*{평가 시험}
본 스터디는 평가 시험에 통과하여야 Pass할 수 있으며, 본 스터디를 Pass하여야 정회원으로 승급할 수 있다. 평가 시험 일정은 KWEB 종강총회 당일이며, 종강총회는 12/23(목)로 예정되어 있다. 평가 시험은 약 60분간 진행되며, 일정 점수 이상 취득하여야 통과할 수 있고, 통과하지 못하였거나 응시하지 않은 경우 겨울방학에 보충 스터디\footnote{열리지 않을 수도 있음.}에 참여하여야 정회원으로 승급할 수 있다.
