\usepackage{amsthm}
\usepackage{mdframed}
\usepackage{fancyvrb}

\newcommand\cd\texttt

\definecolor{codebgcolor}{HTML}{E7E7E7}

\mdfdefinestyle{codeframe}{
    hidealllines=true,
    backgroundcolor=codebgcolor,
    innertopmargin=-3pt,
    innerbottommargin=8pt,
}

\makeatletter
\def\thmhead@plain#1#2#3{%
    \thmname{#1}\thmnumber{\@ifnotempty{#1}{ }\@upn{#2}}%
    \thmnote{ {\the\thm@notefont#3}}}
\let\thmhead\thmhead@plain
\makeatother

\newtheoremstyle{codestyle}% name of the style to be used
    {20pt}% measure of space to leave above the theorem. E.g.: 3pt
    {}% measure of space to leave below the theorem. E.g.: 3pt
    {}% name of font to use in the body of the theorem
    {}% measure of space to indent
    {\bfseries}% name of head font
    {}% punctuation between head and body
    {0pt}% space after theorem head
    {}% Manually specify head
\theoremstyle{codestyle}

\newtheorem{codethm}{Code}[chapter]
\newenvironment{codeenv}[2]
    {\vspace{10pt}\begin{mdframed}[style=codeframe]\begin{codethm}[#2]\label{#1}\linespread{1.1}{\ }}
    {\end{codethm}\end{mdframed}\vspace{6pt}}

\newtheorem{shellthm}{Shell}[chapter]
\newenvironment{shellenv}[2]
    {\vspace{10pt}\begin{mdframed}[style=codeframe]\begin{shellthm}[#2]\label{#1}\linespread{1.1}{\ }}
    {\end{shellthm}\end{mdframed}\vspace{6pt}}

\newtheorem{sqlthm}{SQL}[chapter]
\newenvironment{sqlenv}[2]
    {\vspace{10pt}\begin{mdframed}[style=codeframe]\begin{sqlthm}[#2]\label{#1}\linespread{1.1}{\ }}
    {\end{sqlthm}\end{mdframed}\vspace{6pt}}

\usepackage{listings}
\usepackage[newfloat,outputdir=../bin]{minted}

\definecolor{codebg}{HTML}{f2f2f2}
\definecolor{codelinenumber}{HTML}{707070}

\usemintedstyle{sas}

\setminted{
    linenos,
    bgcolor=codebg,
    baselinestretch=1,
    frame=single,
    framesep=10pt,
    framerule=0pt,
    numbersep=-5pt,
    xleftmargin=20pt,
    xrightmargin=2pt,
    escapeinside=@@
}

\renewcommand{\theFancyVerbLine}{\textcolor[HTML]{707070}{\small\arabic{FancyVerbLine}}}

\DeclareFloatingEnvironment[name=Code, within=chapter]{codefloat}
\newenvironment{code}[2]{
    \captionsetup{type=codefloat, font=normalsize}
    \caption{#2}\label{#1}
}{}

\DeclareFloatingEnvironment[name=Shell, within=chapter]{shellfloat}
\newenvironment{shell}[2]{
    \captionsetup{type=shellfloat, font=normalsize}
    \caption{#2}\label{#1}
}{}

\DeclareFloatingEnvironment[name=SQL, within=chapter]{sqlfloat}
\newenvironment{sql}[2]{
    \captionsetup{type=sqlfloat, font=normalsize}
    \caption{#2}\label{#1}
}{}
