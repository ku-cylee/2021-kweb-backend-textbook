\section{Database Designing Exercises}\label{sect:database-designing-exercises}

\subsection*{\excstref{exc:students-table-design}: \cd{students} Table Design}

학교에 소속된 학생들의 데이터를 저장하는 데이터베이스를 설계하고자 한다.

\begin{description}
    \item[Step 1] DB에 학생의 이름, 학번, 입학 연도, 전공, 전화번호, 주소, 누적 이수학점, 평균 평점, 재학 여부 정보를 저장하려고 한다. 이들 정보를 저장하는 테이블을 적절한 column 이름과 자료형을 제시하여 설계하여라.
    \item[Step 2] 학번은 입학 연도, 전공, 개별 번호로 구성되어 있다. 이를 이용하여 Step 1에서 설계한 테이블에서 학번 column을 세분화하여 redundancy를 최대한 없애보자.
    \item[Step 3] Step 2에서 설계한 테이블을 실제 DB에 생성하는 SQL문을 작성하여라.
\end{description}

이때 누적 이수학점, 평균 평점, 재학 여부의 기본값을 각각 0, 0.0, 참으로 설정하여라. Column의 기본값은 \cd{DEFAULT} 키워드를 이용하여 설정할 수 있다. 예를 들어, \coderef{code:sql-default-example}\은 \cd{title} column의 값을 빈 문자열(\cd{''}), \cd{level} column의 값을 1로 설정하는 예제이다.

\begin{codeenv}{code:sql-default-example}{\cd{DEFAULT} Keyword Example}\begin{verbatim}
`title` VARCHAR(20) DEFAULT '',
`level` INT DEFAULT 1,
\end{verbatim}
\end{codeenv}

\subsection*{\excstref{exc:messenger-db-design}: Messenger DB Design}

메신저 서비스의 DB를 설계하고자 한다. 먼저 각 테이블은 다음과 같은 조건을 만족해야 한다.

\begin{itemize}
    \item 모든 테이블은 기본 키 \cd{id}를 반드시 가져야 한다.
    \item 각 테이블의 column 이름과 자료형을 정하고, 필요에 따라 옵션을 정해야 한다.
\end{itemize}

각 테이블은 다음과 같은 정보를 포함해야 한다.

\begin{itemize}
    \item \cd{users} : 메신저 사용자에 관한 정보
    \begin{itemize}
        \item 사용자 아이디
        \item 사용자 비밀번호
        \item 사용자 닉네임
        \item 프로필 사진 링크
        \item 프로필 상태 메시지
        \item 탈퇴 여부; 기본값은 0
        \item 가입 날짜
    \end{itemize}
    \item \cd{channels}: 채팅 채널에 관한 정보
    \begin{itemize}
        \item 이름
        \item 채널을 생성한 사용자
        \item 채널의 링크
        \item 최대 수용 인원
        \item 탈퇴 여부; 기본값은 0
        \item 채널 생성 날짜
    \end{itemize}
    \item \cd{chats}: 각 채팅에 관한 정보
    \begin{itemize}
        \item 채팅 내용
        \item 채팅 작성자
        \item 채팅 채널
        \item 채팅 생성 날짜
    \end{itemize}
    \item \cd{follows}: 팔로우에 관한 정보
    \begin{itemize}
        \item 팔로우한 사람 (follower)
        \item 팔로우되는 사람 (followee)
        \item 팔로우 날짜
    \end{itemize}
    \item \cd{blocks}: 사용자 간 차단에 관한 정보
    \begin{itemize}
        \item 차단을 한 사람
        \item 차단을 당한 사람
        \item 차단 날짜
    \end{itemize}
\end{itemize}

5개 테이블이 서로 적절한 관계(relation)를 갖도록 외래 키를 설정하여 각 테이블을 생성하는 SQL문을 작성하여라. 외래 키가 있는 테이블을 생성할 때 참조되는 column이 있는 테이블이 없으면 테이블이 생성되지 않는다는 점을 주의해야 한다. 예를 들어 \sectref{sect:relational-designing}의 \cd{courses} 테이블과 \cd{departments} 테이블에서 \cd{courses.department} column이 \cd{departments.id} column을 참조하는데, \cd{departments} 테이블을 생성하지 않고 \cd{courses} 테이블을 생성하려고 하면 에러가 발생한다.

각 테이블의 SQL문을 모두 \coderef{code:messenger-db-design-example}\과 같이 messenger.sql 파일에 작성하여라. SQL문을 DB에서 실행시켜 잘 작동시켜 확인해보고, \cd{DESC} 키워드를 이용해 의도한 바와 같이 설계되었는지 확인해본다.

\begin{codeenv}{code:messenger-db-design-example}{\excref{exc:messenger-db-design} Example}\begin{verbatim}
CREATE TABLE `users` (
    -- Your code here
) ENGINE=InnoDB DEFAULT CHARSET=utf8;

CREATE TABLE `channels` (
    -- Your code here
) ENGINE=InnoDB DEFAULT CHARSET=utf8;

-- chats / blocks table code here

CREATE TABLE `follows` (
    -- Your code here
) ENGINE=InnoDB DEFAULT CHARSET=utf8;
\end{verbatim}
\end{codeenv}
