\section{Basics of Designing}\label{sect:basics-of-designing}

\subsection*{Data Types}

DB의 모든 column은 자료형(data type)을 가지며, 개발자는 각 column에 적절한 자료형을 부여해야 한다.\footnote{자료형과 무관하게 column의 값이 NULL이 될 수 있으나, NULL의 사용은 지양되어야 한다.}

\begin{tblenv}
    {tab:mariadb-data-types}
    {Data Types of MariaDB (MySQL)}
    {>{\coll}m{0.1\tw}>{\coll}m{0.65\tw}}
    \thickhline
    Category & Data Types\tabularnewline
    \hline
    문자형 & CHAR ($n<2^8$) / \textbf{VARCHAR} ($n<2^{16}$) / TINYTEXT ($n<2^8$)
    \\\textbf{TEXT} ($n<2^{16}$) / MEDIUMTEXT ($n<2^{24}$) / LONGTEXT ($n<2^{32}$)\tabularnewline
    숫자형 & \textbf{TINYINT} ($n<2^8$) / SMALLINT ($n<2^{16}$) / MEDIUMINT ($n<2^{24}$)
    \\\textbf{INT} ($n<2^{32}$) / \textbf{BIGINT} ($\infty$) / FLOAT / DECIMAL / \textbf{DOUBLE}\tabularnewline
    날짜형 & DATE / TIME / \textbf{DATETIME} / \textbf{TIMESTAMP} / YEAR\tabularnewline
    이진 데이터 & BINARY / BYTE / TINYBLOB / BLOB / MEDIUMBLOB / LONGBLOB\tabularnewline
    \thickhline
\end{tblenv}

\tblref{tab:mariadb-data-types}\는 MariaDB(MySQL)에서 제공되는 자료형의 일부를 나타낸 표이며, 굵은 글씨는 주로 사용되는 자료형이다. TINYINT형은 주로 boolean 값을 나타내기 위해 사용되며, VARCHAR형은 80자가 기본이지만 문자열의 길이를 항상 명시해주는 것이 권장된다. 이진 데이터는 이미지나 영상 등 텍스트로 표현되지 않는 데이터를 저장하기 위한 자료형이지만, DB에서 데이터 조회 효율을 떨어뜨리므로 권장되지 않는다. 대신, 이진 데이터를 별도의 파일로 저장하고 해당 파일의 경로를 DB에 저장하는 것이 일반적인 방법이다.

\subsection*{Designing Simple Table}

대학 강의들의 데이터를 저장하는 \cd{courses} 테이블을 설계하면서 간단한 테이블 설계 방법을 알아보자. 강의와 관련된 정보에는 강의명(\cd{name}), 강의 코드(\cd{code}), 이수구분(\cd{classification}), 학점(\cd{credit}), 시간(\cd{period}) 등이 있다. 이러한 정보를 정리하여 \tblref{tab:courses-table-basic}\과 같이 나타낼 수 있다.

\begin{tblenv}
    {tab:courses-table-basic}
    {\cd{courses} Table}
    {?>{\colc}m{0.16\tw}|>{\colc}m{0.12\tw}|>{\colc}m{0.15\tw}|>{\colc}m{0.1\tw}|>{\colc}m{0.1\tw}?}
    \thickhline
    \rowcolor{tblheadcolor}
    \cd{name} & \cd{code} & \cd{classification} & \cd{credit} & \cd{period}\tabularnewline
    \hline
    모두를위한파이썬 & COSE156 & 교양 & 3 & 4\tabularnewline
    \hline
    이산수학 & COSE211 & 전공선택 & 3 & 3\tabularnewline
    \hline
    운영체제 & COSE341 & 전공필수 & 3 & 3\tabularnewline
    \hline
    기계학습 & COSE362 & 전공선택 & 3 & 3\tabularnewline
    \hline
    캡스톤디자인 & COSE489 & 전공선택 & 3 & 6\tabularnewline
    \thickhline
\end{tblenv}
\clearpage

먼저 \tblref{tab:courses-table-basic}의 각 column에 자료형을 부여해보자. \cd{name} column은 문자열로 표현되고 매우 긴 텍스트가 아니므로 VARCHAR형을 사용할 수 있으며, 과목명의 최대 길이가 20자라고 한다면 VARCHAR(20)형으로 정할 수 있다. 유사하게 \cd{code}와 \cd{classification} column은 VARCHAR(8)로 정하면 적당한 길이가 될 것이다. \cd{credit}과 \cd{period}는 각각 정수이므로 INT형이 적당하다. 이렇게 부여한 자료형을 반영하여 \tblref{tab:courses-table-types}\와 같이 나타낼 수 있다.

\begin{tblenv}
    {tab:courses-table-types}
    {\cd{courses} Table with Data Types}
    {?>{\colc}m{0.16\tw}|>{\colc}m{0.12\tw}|>{\colc}m{0.15\tw}|>{\colc}m{0.1\tw}|>{\colc}m{0.1\tw}?}
    \thickhline
    \rowcolor{tblheadcolor}
    \cd{name} & \cd{code} & \cd{classification} & \cd{credit} & \cd{period}\tabularnewline
    \hline
    \rowcolor{tblheadcolor}
    \cd{VC(20)} & \cd{VC(8)} & \cd{VC(8)} & \cd{INT} & \cd{INT}\tabularnewline
    \hline
    모두를위한파이썬 & COSE156 & 교양 & 3 & 4\tabularnewline
    \hline
    이산수학 & COSE211 & 전공선택 & 3 & 3\tabularnewline
    \hline
    운영체제 & COSE341 & 전공필수 & 3 & 3\tabularnewline
    \hline
    기계학습 & COSE362 & 전공선택 & 3 & 3\tabularnewline
    \hline
    캡스톤디자인 & COSE489 & 전공선택 & 3 & 6\tabularnewline
    \thickhline
\end{tblenv}

\subsection*{Avoiding Redundancy}

\tblref{tab:courses-table-types}\는 컴퓨터학과에서 열리는 강의 데이터만 저장한 테이블이다. 이제 이 데이터베이스에 컴퓨터학과 이외의 화학과, 물리학과 등의 강의를 추가하려면 어떻게 하는 것이 좋겠는가?

가장 쉽게 떠올릴 수 있는 방법은 동일한 구조의 화학과 강의 테이블, 물리학과 강의 테이블 등을 만드는 것이다. 그러나 이 방법은 두 가지 문제점을 유발하는데, 첫번째는 구조가 동일한 테이블이 여러 개 생성되어 중복을 초래한다는 점이다. 프로그램을 설계할 때 코드의 중복을 최대한 피하기 위해 함수, 클래스의 상속 등을 활용하는 것처럼 DB를 설계할 때에도 중복을 가급적 피해야 한다.

두 번째 문제점은 새로운 학과의 강의를 추가하려면 그 DB를 사용하는 애플리케이션이 매번 테이블을 직접 생성해야 한다는 것이다. 애플리케이션은 DB의 데이터에 대한 CRUD 연산만 수행하여야 하고, 그 외에 테이블의 추가 및 삭제, column의 추가 및 삭제 등 설계 자체를 바꾸는 작업을 수행해서는 안 된다.

이렇듯 분류에 따라 테이블을 추가하는 방법은 지양되어야 하며, 모든 강의 데이터를 모두 한 테이블에 저장하고 분류를 위한 column을 생성하여 분류하는 것이 가장 이상적인 방법이다. \tblref{tab:courses-table-categories}\는 \tblref{tab:courses-table-types}에 \cd{department} column을 추가하여 각 강의의 학과를 명시하여 분류한 테이블이다. 이렇게 테이블을 설계하면 앞에서 제시된 문제점들이 발생하지 않고 분류에 따라 데이터를 조회하기도 간편하다.
\clearpage

\begin{tblenv}
    {tab:courses-table-categories}
    {\cd{courses} Table with Categories}
    {?>{\colc}m{0.16\tw}|>{\colc}m{0.12\tw}|>{\colc}m{0.12\tw}|>{\colc}m{0.15\tw}|>{\colc}m{0.1\tw}|>{\colc}m{0.1\tw}?}
    \thickhline
    \rowcolor{tblheadcolor}
    \cd{name} & \cd{department} & \cd{code} & \cd{classification} & \cd{credit} & \cd{period}\tabularnewline
    \hline
    \rowcolor{tblheadcolor}
    \cd{VC(20)} & \cd{VC(16)} & \cd{VC(8)} & \cd{VC(8)} & \cd{INT} & \cd{INT}\tabularnewline
    \hline
    모두를위한파이썬 & 컴퓨터학과 & COSE156 & 교양 & 3 & 4\tabularnewline
    \hline
    이산수학 & 컴퓨터학과 & COSE211 & 전공선택 & 3 & 3\tabularnewline
    \hline
    운영체제 & 컴퓨터학과 & COSE341 & 전공필수 & 3 & 3\tabularnewline
    \hline
    기계학습 & 컴퓨터학과 & COSE362 & 전공선택 & 3 & 3\tabularnewline
    \hline
    캡스톤디자인 & 컴퓨터학과 & COSE489 & 전공선택 & 3 & 6\tabularnewline
    \hline
    유기화학실험 & 화학과 & CHEM214 & 전공필수 & 2 & 4\tabularnewline
    \hline
    고체물리학 & 물리학과 & PHYS482 & 전공선택 & 3 & 3\tabularnewline
    \hline
    무기화학II & 화학과 & CHEM308 & 전공필수 & 3 & 3\tabularnewline
    \thickhline
\end{tblenv}

\tblref{tab:courses-table-categories}의 테이블에서 \cd{classification} column을 주목해보자. 이 column의 데이터에는 ``교양'', ``전공필수'', ``전공선택'' 등이 있는데, 이들 데이터는 전공 여부와 필수 여부로 나누어질 수 있다. 예를 들어, ``교양''은 전공도 아니고 필수도 아니며, ``전공선택''은 전공이지만 필수는 아니다. 이러한 데이터는 상황에 따라 유의미한 데이터가 될 수 있으며, 이렇게 column의 값을 최대한 유의미한 단위로 나누어 저장하면 중복을 최소화하고 데이터를 편리하게 조회할 수 있다.

\tblref{tab:courses-table-subdivisions}\은 \tblref{tab:courses-table-categories}의 \cd{classification} column을 \cd{is\_major}와 \cd{is\_required}로 나눈 테이블이다. 이때 두 column은 참(\cd{1})이나 거짓(\cd{0}) 값을 가져야 하므로 자료형을 TINYINT(1)\footnote{(1)은 붙이나 안 붙이나 차이점은 없으나, boolean형임을 나타내기 위해 관례적으로 쓴다.}로 설정하였다.

\begin{tblenv}
    {tab:courses-table-subdivisions}
    {\cd{courses} Table with \cd{classification} Subdivisions}
    {?>{\colc}m{0.16\tw}|>{\colc}m{0.12\tw}|>{\colc}m{0.11\tw}|>{\colc}m{0.1\tw}|>{\colc}m{0.11\tw}|>{\colc}m{0.1\tw}|>{\colc}m{0.1\tw}?}
    \thickhline
    \rowcolor{tblheadcolor}
    \cd{name} & \cd{department} & \cd{code} & \cd{is\_major} & \cd{is\_required} & \cd{credit} & \cd{period}\tabularnewline
    \hline
    \rowcolor{tblheadcolor}
    \cd{VC(20)} & \cd{VC(16)} & \cd{VC(8)} & \cd{TINYINT(1)} & \cd{TINYINT(1)} & \cd{INT} & \cd{INT}\tabularnewline
    \hline
    모두를위한파이썬 & 컴퓨터학과 & COSE156 & 0 & 0 & 3 & 4\tabularnewline
    \hline
    이산수학 & 컴퓨터학과 & COSE211 & 1 & 0 & 3 & 3\tabularnewline
    \hline
    운영체제 & 컴퓨터학과 & COSE341 & 1 & 1 & 3 & 3\tabularnewline
    \hline
    기계학습 & 컴퓨터학과 & COSE362 & 1 & 0 & 3 & 3\tabularnewline
    \hline
    캡스톤디자인 & 컴퓨터학과 & COSE489 & 1 & 0 & 3 & 6\tabularnewline
    \hline
    유기화학실험 & 화학과 & CHEM214 & 1 & 1 & 2 & 4\tabularnewline
    \hline
    고체물리학 & 물리학과 & PHYS482 & 1 & 0 & 3 & 3\tabularnewline
    \hline
    무기화학II & 화학과 & CHEM308 & 1 & 1 & 3 & 3\tabularnewline
    \thickhline
\end{tblenv}

\subsection*{Create Table With SQL}

앞에서 설계한 \cd{courses} 테이블(\tblref{tab:courses-table-subdivisions})을 SQL을 이용하여 DB에 생성해보자.

\begin{codeenv}{code:create-courses-table}{Create \cd{courses} Table}\begin{verbatim}
CREATE TABLE `courses` (
    `name` VARCHAR(20) NOT NULL,
    `department` VARCHAR(16) NOT NULL,
    `code` VARCHAR(8) NOT NULL,
    `is_major` TINYINT(1) NOT NULL,
    `is_required` TINYINT(1) NOT NULL,
    `credit` INT NOT NULL,
    `period` INT NOT NULL
) ENGINE=InnoDB DEFAULT CHARSET=utf8;
\end{verbatim}
\end{codeenv}

\coderef{code:create-courses-table}\는 데이터베이스에 \cd{courses} 테이블을 생성하는 SQL문이다. 각 column을 정의할 때는 column의 이름, 자료형, 옵션\footnote{주어진 SQL문의 \cd{NOT NULL} 옵션은 NULL 값을 허용하지 않는다는 뜻이다.}을 공백문자(whitespace)로 구분하여 나열하고, 각 column을 쉼표(\cd{,})로 구분하여 나열한다. SQL 문법에서는 여러 공백 문자도 하나의 공백 문자로 취급(값은 제외)되므로 개행없이 한 줄에 모두 나열하여도 상관없고, \cd{CREATE}, \cd{INT}와 같은 예약어는 모두 대문자로 작성하는 것이 원칙이지만 소문자로 작성해도 차이는 없다. Table이나 column 이름 등은 백틱(\cd{`})으로 감싸는 것이 원칙이지만 예약어가 아니라면 감싸지 않아도 된다.\footnote{예를 들어 column 이름이 \cd{student}라면 감싸지 않아도 되나 \cd{order}라면 반드시 감싸야 한다.}

\begin{sqlenv}{sql:desc}{Description of Table}\begin{verbatim}
DESC <table-name>
\end{verbatim}
\end{sqlenv}

\sqlref{sql:desc}\과 같이 \cd{DESC} 키워드를 사용하면 테이블의 설계 정보를 확인할 수 있다.

\begin{sqlenv}{sql:truncate-and-drop}{Truncate and Drop Table}\begin{verbatim}
TRUNCATE <table-name>
DROP TABLE <table-name>
\end{verbatim}
\end{sqlenv}

\sqlref{sql:truncate-and-drop}\과 같이 테이블을 삭제할 수 있다. \cd{TRUNCATE} 키워드는 테이블의 구조는 유지한 채 데이터만 삭제하고, \cd{DROP TABLE} 키워드는 테이블 자체를 삭제한다는 차이가 있다.
