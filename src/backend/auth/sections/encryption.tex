\section{Hash Functions and Encryption}\label{sect:encryption}

\subsection*{Encryption}

많은 웹 사이트에서는 사용자를 식별하기 위해 사용자 계정을 이용한 인증 시스템을 사용하며, 사용자는 자신이 만든 계정에 로그인이라는 방식을 이용하여 인증한 뒤 웹 사이트에 접근한다. 이때 사용자는 아이디와 비밀번호 등 인증에 필요한 정보를 서버에 전달하여 본인임을 인증하고, 서버에서는 각 사용자의 아이디와 비밀번호 등을 DB에 저장해두고, 사용자로부터 받은 인증 데이터를 DB에 저장된 데이터와 비교하여 요청을 보낸 사용자가 계정에 접근 권한을 갖고 있는지 검증한다.

비밀번호와 같은 민감한 정보를 DB에 저장할 때 plain text로 저장하면 DB가 해킹되었을 때 사용자의 암호가 공격자의 손에 고스란히 넘어갈 수 있다. 따라서 이러한 정보는 회원가입 과정 등에서 반드시 암호화(encryption)하여 저장하여야 한다.

암호화 알고리즘은 크게 단방향 암호화와 양방향 암호화로 나뉘며, 양방향 암호화 알고리즘은 세부적으로 비대칭형 암호화와 대칭형 암호화로 나뉘기도 한다. 암호화된 데이터를 원래 데이터로 변환하는 과정을 복호화(decryption)라고 하는데, 양방향 암호화 알고리즘은 복호화가 가능한 알고리즘인 반면, 단방향 암호화 알고리즘은 복호화가 불가능한 알고리즘이다. 웹 서버에서는 이러한 암호화 알고리즘을 사용하여 사용자의 비밀번호를 암호화한 후, 암호화된 비밀번호를 DB에 저장해야 한다.

웹 서버에서 사용자 인증 과정을 어떻게 수행할지 생각해보자. 먼저 사용자가 보낸 HTTP 요청으로부터 아이디와 비밀번호를 전달받으면, DB에서 아이디 값이 입력받은 아이디와 일치하는 row를 조회하고 조회 결과가 없으면 가입되지 않은 아이디로 판별한다. 아이디가 존재하면 입력받은 비밀번호와 해당 row의 비밀번호 값이 일치하는지 검증하여 일치 여부에 따라 인증의 성공 여부를 판별한다.

비밀번호 등의 암호화 과정에서 사용되는 알고리즘이 동일한 입력에 대해 항상 동일한 결과를 반환하는, 즉 결정론적(deterministic)으로 작동하는 알고리즘이라면, 웹 서버는 인증 과정에서 암호화된 비밀번호로부터 원래 비밀번호를 도출할 필요가 없다. 대신 처음 비밀번호를 암호화한 방법과 같은 방법으로 사용자가 전달한 비밀번호를 암호화하여 DB에 저장된 비밀번호와 같은지 비교하여 판별할 수 있다. 따라서, 비밀번호를 통한 인증 과정에서는 결정론적으로 작동하는 단방향 암호화 알고리즘을 이용한다.

\subsection*{Hash Function}

해시 함수(hash function)는 입력받은 이진 데이터(입력값)를 또 다른 이진 데이터로 암호화(해시값)하여 반환하는 함수로, 비밀번호 암호화뿐만 아니라 데이터의 무결성 검증을 위한 체크섬 생성 등에도 널리 사용된다. 이러한 해시 함수는 다음과 같은 몇 가지 특징을 갖는다.

먼저, 해시 함수는 결정론적 알고리즘으로, 동일한 두 입력값의 해시값은 반드시 같다. 이는 곧 해시값이 다른 두 입력값은 반드시 다르다는 것을 의미한다. 그러나 두 입력값이 다르더라도 해시값이 반드시 다르다는 보장은 없으며, 이를 해시 충돌(hash collision)이라고 한다.

해시 함수의 두 번째 특징은 입력값과 무관하게 해시값의 길이는 일정하다는 것이며, 그 길이는 알고리즘마다 다르다. 입력값의 길이는 제한이 없는데 해시값의 길이는 고정되어 있으므로 이론적으로 해시 충돌은 필연적으로 발생한다. 그러나, 실제로는 입력값에 해당하는 비밀번호 등은 길이 제한이 존재하고, 해시값이 길이는 충분히 기므로 해시 충돌이 발생할 확률은 극히 낮다.\footnote{키보드로 입력할 수 있는 문자는 95개이고, 비밀번호의 길이를 8-30자로 제한한다면 $\sum_{i=8}^{30}95^i\approx 2.17\times 10^{59}$가지; 해시값이 512비트인 SHA-512의 경우의 수는 $2^{512}=1.34\times 10^{154}$가지; 해시 충돌 발생 확률은 $1.62\times 10^{-95}$. 참고로 우주 전체의 원자 수가 약 $10^{82}$개이다.}

해시 함수의 세 번째 특징은 입력값에서 발생한 아주 작은 변화가 해시값에서 아주 큰 변화를 발생시킨다는 점이다. 크기가 매우 큰 두 입력값이 단 1비트만 달라도 생성되는 해시값은 전혀 다르며, 이러한 특징을 눈사태 효과(avalanche effect)라고 한다. 눈사태 효과는 해시값을 이용하여 입력값을 유추할 수 없게 만들고, 해시 함수의 연산 속도가 매우 빠르다는 특징과 맞아떨어져 매우 큰 파일이 전송 과정에서 손상되었는지 확인하는 무결성 검증(checksum)에서도 널리 사용된다.

통상적으로 사용되는 해시 함수에는 MD5, SHA-1, SHA-256, SHA-512, SHA-3 등이 있다. 이 중 MD5나 SHA-1 알고리즘은 출시된 지 오래되었고 해시 충돌 알고리즘이 발견되어 매우 취약하므로 보안 목적으로 사용하는 것이 권장되지 않는다. 최근에는 주로 SHA-256과 SHA-512 등 SHA-2 계열의 해시 함수가 널리 사용되고 있고, 앞으로는 SHA-3 해시 함수를 사용할 것이 권장되고 있다. 본 교재에서는 SHA-512 함수를 이용하여 비밀번호를 암호화하여 저장하고 검증하는 실습을 진행한다.

\subsection*{Simple Encryption using PBKDF2}

crypto는 입력값을 암호화하여 해시값을 반환하는 해시 함수를 제공하는 Node.js 내장 모듈이다.

\begin{code}{code:simple-sha512-hashing}{Simple example of SHA-512 hashing}
\begin{minted}{js}
const util = require('util');
const crypto = require('crypto');

const pbkdf2 = util.promisify(crypto.pbkdf2);

const encrypt = async text => {
    const ALGO = 'sha512';
    const KEY_LEN = 64;
    const digest = await pbkdf2(text, '', 1, KEY_LEN, ALGO);
    console.log(`${text} | ${digest.toString('base64')}`);
};

(async () => await encrypt('samplepassword'))();
\end{minted}
\end{code}

\coderef{code:simple-sha512-hashing}의 \cd{encrypt} 함수는 인자로 받은 문자열을 crypto 모듈의 \cd{pbkdf2} 메서드를 이용하여 SHA-512 알고리즘으로 암호화한 후, 암호화된 데이터(\cd{digest})를 Base64 방식\footnote{6bits의 이진 데이터를 하나의 ASCII 문자로 표현하기 위한 인코딩 방식. 0-63의 값이 A-Z, a-z, 0-9, +, /의 64개 문자에 대응되며, 맨 끝에 빈 부분을 = 문자로 채우기도 한다.}으로 인코딩하여 출력하는 함수이다. \cd{KEY\_LEN}은 해시값 중 사용할 바이트 수를 나타내는 인자로, 예제에서는 64B(=512bits)를 사용한다.

코드 실행 결과 ``samplepassword'' 텍스트는 88자(= 문자를 제외하면 $86=\left\lceil\frac{512}{6}\right\rceil$자)의 문자열로 변환되었음을 확인할 수 있고, 코드를 여러 번 실행하여도 항상 같은 결과가 도출되어 결정론적으로 작동한다는 것을 알 수 있다.

\begin{code}{code:avalanche-effect}{Avalanche effect (Derived from \coderef{code:simple-sha512-hashing})}
\begin{minted}[escapeinside=||]{js}
(async () => {
    await encrypt(|\PYG{l+s}{\PYGZsq{}samplepasswordsamplepas\textbf{\textcolor{red}{s}}wordsamplepasswordsamplepasswordsample\PYGZsq{}}|);
    await encrypt(|\PYG{l+s}{\PYGZsq{}samplepasswordsamplepas\textbf{\textcolor{red}{t}}wordsamplepastwordsamplepasswordsample\PYGZsq{}}|);
})();
\end{minted}
\end{code}

\coderef{code:avalanche-effect}\는 \coderef{code:simple-sha512-hashing}에서 \cd{encrypt} 함수의 인자를 바꾼 것이다. 62자의 문자로 이루어진 두 문자열은 단 한 자만 다르지만 코드를 실행해보면 두 문자열의 해시값이 전혀 다르다는 것, 즉 눈사태 효과를 확인할 수 있다. 이렇게 해시 함수는 매우 큰 데이터에서 매우 작은 차이를 탐지하는데 유용하며, 실제로 크기가 큰 파일이 전송 과정에서 손상되었는지 확인하기 위해 사용되는 체크섬(checksum)을 계산하기 위해 널리 사용된다.

이처럼 해시 함수를 이용하면 비밀번호 등의 복호화가 필요없는 민감한 정보를 암호화한 후 저장하여 그 정보를 안전하게 보호할 수 있다. 그러나 입력값을 해시값으로부터 수학적으로 유추할 수는 없지만, 공격자가 brute-force 방식으로 모든 입력값에 대한 해시값을 계산하여 저장해둔다면, 결정론적인 특징 때문에 해시값에 대한 입력값을 알아낼 수 있다. 이처럼 무수히 많은 입력값에 대한 해시값을 저장한 테이블을 레인보우 테이블(rainbow table) 이라고 하며, 해시 함수는 연산 속도가 빠르기 때문에 입력값에 제한 조건이 있다면 레인보우 테이블을 계산하는 것은 상대적으로 쉽다. 따라서 이러한 레인보우 테이블을 사용하기 어렵게 하는 조치를 취해야 한다.

\subsection*{Salting}

레인보우 테이블 사용을 어렵게 하는 첫 번째 방법은 salt를 이용하는 것이다. Salt란 랜덤하게 생성된 충분히 긴 길이의 문자열이며, salt를 이용하여 암호화하고자 하는 텍스트를 변조(salting)한 뒤 해시 함수의 입력값으로 사용하는 것이다. 이렇게 얻은 해시값을 salt와 함께 저장하면 검증하고자 하는 비밀번호를 salting하여 도출된 해시값과 저장되어 있는 해시값을 비교하여 검증할 수 있다. 공격자 입장에서는 입력될 수 있는 경우의 수가 매우 많아져 레인보우 테이블을 통해 입력값을 유추하는 것이 거의 불가능하다.

\begin{code}{code:salting}{Salting input text (Derived from \coderef{code:simple-sha512-hashing})}
\begin{minted}[escapeinside=||]{js}
|\textbf{\PYG{k+kr}{const} \PYG{n+nx}{randomBytes} \PYG{o}{=} \PYG{n+nx}{util}\PYG{p}{.}\PYG{n+nx}{promisify}\PYG{p}{(}\PYG{n+nx}{crypto}\PYG{p}{.}\PYG{n+nx}{randomBytes}\PYG{p}{)}\PYG{p}{;}}|

const encrypt = async text => {
    const ALGO = 'sha512';
    const KEY_LEN = 64;
    |\textbf{\PYG{k+kr}{const} \PYG{n+nx}{salt} \PYG{o}{=} \PYG{n+nx}{await} \PYG{n+nx}{randomBytes}\PYG{p}{(}\PYG{l+m+mi}{32}\PYG{p}{)}\PYG{p}{;}}|
    const digest = await pbkdf2(text, |\textbf{salt}|, 1, KEY_LEN, ALGO);
    console.log(`${text} | ${salt.toString('base64')} | ${digest.toString('base64')}`);
};
\end{minted}
\end{code}

% const randomBytes = util.promisify(crypto.randomBytes);

% const encrypt = async text => {
%     const ALGO = 'sha512';
%     const KEY_LEN = 64;
%     const salt = await randomBytes(32);
%     const digest = await pbkdf2(text, salt, 1, KEY_LEN, ALGO);
%     console.log(`${text} | ${salt.toString('base64')} | ${digest.toString('base64')}`);
% };

\coderef{code:salting}\은 \coderef{code:simple-sha512-hashing}\을 수정한 코드로, crypto 모듈의 \cd{randomBytes} 메서드를 이용하여 32B의 salt를 생성한 후, 이를 이용하여 입력값을 salting한 뒤 암호화한 코드이다.

\subsection*{Key Stretching}

레인보우 테이블 사용을 어렵게 하는 두 번째 방법은 암호화를 여러 번 수행하는 것이다. 해시 함수를 통해 salting한 입력값의 해시값을 얻은 후, 이 해시값을 다시 salting한 값의 해시값을 얻는다. 이를 정해진 횟수만큼 반복하여 최종적으로 도출된 해시값을 저장하며, 반복 횟수는 10만 회 이상, 랜덤하게 결정하는 것이 권장된다. 검증 단계에서는 동일한 횟수만큼 반복하여 얻은 해시값과 저장된 해시값을 비교한다.

\begin{code}{code:hashing-key-stretching}{Key stretching of hashing (Derived from \coderef{code:salting})}
\begin{minted}[escapeinside=||]{js}
const encrypt = async text => {
    const ALGO = 'sha512';
    const KEY_LEN = 64;
    const salt = await randomBytes(32);
    |\textbf{\PYG{k+kr}{const} \PYG{n+nx}{iter} \PYG{o}{=} \PYG{n+nb}{Math}\PYG{p}{.}\PYG{n+nx}{floor}\PYG{p}{(}\PYG{n+nb}{Math}\PYG{p}{.}\PYG{n+nx}{random}\PYG{p}{(}\PYG{p}{)} \PYG{o}{*} \PYG{l+m+mi}{20000}\PYG{p}{)} \PYG{o}{+} \PYG{l+m+mi}{200000}\PYG{p}{;}}|
    const digest = await pbkdf2(text, salt, |\textbf{\PYG{n+nx}{iter}}|, KEY_LEN, ALGO);
    console.log(`${text} | ${iter} | ${digest.toString('base64')}`);
};
\end{minted}
\end{code}

% const encrypt = async text => {
%     const ALGO = 'sha512';
%     const KEY_LEN = 64;
%     const salt = await randomBytes(32);
%     const iter = Math.floor(Math.random() * 20000) + 200000;
%     const digest = await pbkdf2(text, salt, iter, KEY_LEN, ALGO);
%     console.log(`${text} | ${iter} | ${digest.toString('base64')}`);
% };

\coderef{code:hashing-key-stretching}\은 \coderef{code:salting}\을 수정한 코드로, 암호화를 20만 이상, 22만 미만의 랜덤 횟수만큼 반복적으로 수행하는 코드이다.

\subsection*{Password Generation and Verification}

이제까지 비밀번호 등의 민감한 정보를 적절히 암호화하는 방법에 대해 학습하였다. 이러한 해시값은 암호화 과정에서 사용된 암호화 알고리즘, 해시값의 길이, salt, 반복 횟수 등의 인자들과 함께 저장되어야 한다.

\begin{code}{code:generating-password}{Password generation (Derived from \coderef{code:hashing-key-stretching})}
\begin{minted}[escapeinside=||]{js}
const generatePassword = async password => {
    const ALGO = 'sha512';
    const KEY_LEN = 64;
    const salt = await randomBytes(32);
    const iter = Math.floor(Math.random() * 20000) + 200000;
    const digest = await pbkdf2(password, salt, iter, KEY_LEN, ALGO);
    |\textbf{\PYG{k}{return} \PYG{l+s+sb}{\textasciigrave}\PYG{l+s+si}{\PYGZdl{}\PYGZob{}}\PYG{n+nx}{ALGO}\PYG{l+s+si}{\PYGZcb{}}\PYG{l+s+sb}{:}\PYG{l+s+si}{\PYGZdl{}\PYGZob{}}\PYG{n+nx}{salt}\PYG{p}{.}\PYG{n+nx}{toString}\PYG{p}{(}}|
        |\textbf{\PYG{l+s+s1}{\PYGZsq{}base64\PYGZsq{}}\PYG{p}{,}}|
    |\textbf{\PYG{p}{)}\PYG{l+s+si}{\PYGZcb{}}\PYG{l+s+sb}{:}\PYG{l+s+si}{\PYGZdl{}\PYGZob{}}\PYG{n+nx}{iter}\PYG{l+s+si}{\PYGZcb{}}\PYG{l+s+sb}{:}\PYG{l+s+si}{\PYGZdl{}\PYGZob{}}\PYG{n+nx}{KEY\PYGZus{}LEN}\PYG{l+s+si}{\PYGZcb{}}\PYG{l+s+sb}{:}\PYG{l+s+si}{\PYGZdl{}\PYGZob{}}\PYG{n+nx}{digest}\PYG{p}{.}\PYG{n+nx}{toString}\PYG{p}{(}\PYG{l+s+s1}{\PYGZsq{}base64\PYGZsq{}}\PYG{p}{)}\PYG{l+s+si}{\PYGZcb{}}\PYG{l+s+sb}{\textasciigrave}\PYG{p}{;}}|
};
\end{minted}
\end{code}

% const generatePassword = async password => {
%     const ALGO = 'sha512';
%     const KEY_LEN = 64;
%     const salt = await randomBytes(32);
%     const iter = Math.floor(Math.random() * 20000) + 200000;
%     const digest = await pbkdf2(password, salt, iter, KEY_LEN, ALGO);
%     return `${ALGO}:${salt.toString(
%         'base64',
%     )}:${iter}:${KEY_LEN}:${digest.toString('base64')}`;
% };

\coderef{code:generating-password}의 \cd{generatePassword} 함수는 암호화 인자들과 해시값을 콜론(\cd{:})으로 구분한 문자열을 생성하여 반환하는 함수이다. 비밀번호는 이러한 형태로 DB에 저장된다.
\clearpage

\begin{code}{code:verifying-password}{Password verification (Derived from \coderef{code:generating-password})}
\begin{minted}{js}
const verifyPassword = async (password, hashedPassword) => {
    const [algo, encodedSalt, iterStr, keyLenStr, encodedDigest] =
        hashedPassword.split(':');
    const salt = Buffer.from(encodedSalt, 'base64');
    const iter = parseInt(iterStr, 10);
    const keyLen = parseInt(keyLenStr, 10);
    const storedDigest = Buffer.from(encodedDigest, 'base64');
    const digest = await pbkdf2(password, salt, iter, keyLen, algo);
    return Buffer.compare(digest, storedDigest) === 0;
};
\end{minted}
\end{code}

\coderef{code:verifying-password}의 \cd{verifyPassword} 함수는 저장된 비밀번호를 이용하여 입력값을 검증하는 함수이다. 저장된 비밀번호로부터 알고리즘, salt, 반복 횟수, 해시값 길이 등 암호화 과정에 사용된 인자들과 해시값을 도출하고, 암호화 과정의 인자들을 이용하여 입력값을 똑같이 암호화한다. 그 결과 도출된 해시값과 저장된 비밀번호의 해시값이 동일하면 옳은 비밀번호로, 다르면 틀린 비밀번호로 판별한다.

\begin{code}{code:password-example}{Example of password generation and verification (Derived from \coderef{code:verifying-password})}
\begin{minted}{js}
(async () => {
    const hashedPassword = await generatePassword('password');
    const result1 = await verifyPassword('password', hashedPassword);
    const result2 = await verifyPassword('passsword', hashedPassword);
    console.log(`hashed: ${hashedPassword}`);
    console.log(`password: ${result1} / passsword: ${result2}`);
})();
\end{minted}
\end{code}

\coderef{code:password-example}\은 \cd{generatePassword} 함수를 이용하여 ``passsword'' 문자열을 암호화한 뒤, \cd{verifyPassword} 함수와 암호화된 문자열을 이용하여 두 입력값 ``password''와 ``passsword''가 옳은 비밀번호인지 판별하는 예제이다. ``password''는 옳은 문자열로, ``passsword''는 틀린 문자열로 판별되는 것을 확인할 수 있다.
