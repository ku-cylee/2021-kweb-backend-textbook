\section{Introduction to Database}\label{sect:db-intro}

\subsection*{Necessity of Database}

프로그램은 그 내에서 필요한 데이터의 값을 기본적으로 메모리에 저장하고, 접근하여 사용한다. 메모리에 저장된 데이터는 access time이 매우 빨라 성능 면에서는 유리하지만 휘발성(volatile) 기억장치이므로 기기가 종료되면 데이터가 사라지게 되며, 용량 대비 가격이 매우 비싸기 때문에 대용량의 데이터를 저장하기에는 적합하지 않다.

대부분의 웹 서버에서 다루는 데이터는 프로그램이 종료되더라도 사라지면 안되고, 때로는 그 크기가 매우 크기 때문에 데이터를 메모리에 저장하는 것은 적절하지 않다. 예를 들어 웹 사이트를 이용하는 사용자의 아이디나 비밀번호, 사용자가 작성한 게시글 등이 서버가 종료되었다고 해서 사라진다면 제대로 된 서비스라고 할 수 없다. 특히 금융기관의 서버에서 고객들의 송금 내역, 통장 잔고 등의 데이터나 온라인 쇼핑몰 서버에서 고객들의 주문 내역 등이 사라진다면 금전적인 피해로 이어질 수 있다. 따라서 웹 서버는 프로그램이 종료되더라도 데이터가 유실되지 않는 비휘발성(non-volatile) 기억장치에 데이터를 저장해야 한다.

SSD, HDD와 같은 디스크는 대표적인 비휘발성 기억장치이다. 많은 프로그램은 디스크에 파일을 써서 프로그램이 종료되더라도 데이터가 유실되지 않도록 저장하며, 이러한 파일을 읽어 저장된 데이터를 사용한다. 그러나 plain text 형태로 데이터를 저장하는 방식은 매우 많은 데이터를 다루기 어렵다는 점, 여러 사용자가 동시에 수정할 수 없다는 점 등 많은 문제점을 갖는다.

웹 서버를 비롯한 수많은 애플리케이션은 대용량의 정형화된 데이터를 저장하기 위해 데이터베이스(database; DB)를 사용한다. DB는 저장된 데이터를 프로그램이나 컴퓨터가 종료되더라도 안전하게 보관될 뿐만 아니라, 수많은 데이터를 빠르게 탐색할 수 있는 기능, 여러 사용자가 DB의 데이터를 읽거나 쓸 수 있는 기능, 변경된 데이터를 DB에 거의 실시간으로 반영하는 기능 등 수많은 기능을 제공한다.

\subsection*{CRUD Functions}

CRUD 기능은 데이터를 다루는 프로그램이 기본적으로 갖추고 있어야 하는 데이터 처리 기능인 Create(생성), Read(조회), Update(수정), Delete(삭제)를 묶어 부르는 용어이다. Create는 데이터를 새로 생성하는 기능, Read는 저장된 데이터를 조회하여 읽는 기능, Update는 이미 저장된 데이터를 수정하는 기능, Delete는 저장된 데이터를 삭제하는 기능을 의미한다. DB 역시 이러한 CRUD 연산에 대응되는 기능을 제공함으로써 데이터를 편리하게 다룰 수 있도록 한다.

여담으로, HTTP에도 CRUD 기능에 대응되는 메서드가 있다. GET은 Read(조회), POST는 Create(생성), PUT은 Update(수정), DELETE는 Delete(삭제) 기능에 해당된다.

\subsection*{Relational DB}

Relational DB, 즉 관계형 데이터베이스(이하 RDB) 모델은 체계화된 DB가 개발된 이래 오늘날까지도 가장 널리 쓰이고 있는 DB 모델로, key와 value들로 이루어진 데이터의 형태를 테이블로 나타내고, row와 column의 형태로 관리하는 모델이다. Row는 각각의 데이터를 나타내며, record라고도 한다. Column은 각 데이터를 구성하는 속성값으로, attribute나 field라고도 하며, 각 데이터의 이름은 column 이름이라고 한다.

\figures{fig:rdb-model-table-example}{Example of RDB model table}{
    \fig{rdb-model-table-example.PNG}{.6}
}

\figref{fig:rdb-model-table-example}\은 컴퓨터학과의 전공과목 목록의 일부를 RDB 모델로 나타낸 것이다. 각 전공과목의 정보는 row 단위로 기록되어 있고, 각 row의 이수구분, 학수번호, 교과목명, 학점 등의 정보가 각 column에 저장되어 있어 모든 row의 데이터 형태가 일정하다. 이러한 row와 column의 집합을 table이라고 하고, 하나의 DB는 다수의 table을 가질 수 있다.

RDB 모델은 이처럼 데이터가 매우 일관적으로 저장되어 있고 직관적이라는 장점이 있으며, 그로 인해 데이터의 분류, 정렬, 탐색 속도가 매우 빠르다는 장점이 있다. 이렇듯 RDB 모델은 가장 기본적이고 대중적인 DB 모델이며, 대부분의 상용 DB는 RDB 모델로 설계되어 있고, 많은 웹 서버에서도 RDB 모델을 기반으로 데이터를 저장한다.

\subsection*{SQL}

SQL(Structured Query Language)은 대부분의 RDB형 DB에서 데이터를 다루기 위해 사용되는 스크립트 언어로, SQL을 이용하여 DB나 table을 생성하거나 CRUD 기능 등의 작업을 수행할 수 있다. 물론 DBeaver, Beekeeper Studio, HeidiSQL와 같이 SQL을 굳이 사용하지 않고도 DB를 관리할 수 있는 프로그램들이 존재하지만, 고수준의 연산을 수행하기 위해서는 여전히 SQL이 필요하고 애플리케이션에서 DB 내의 데이터에 접근할 때에도 여전히 SQL이 필요하다.
