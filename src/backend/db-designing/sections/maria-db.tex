\section{MariaDB}\label{sect:maria-db}

\subsection*{DBMS}

DBMS는 데이터베이스 관리 시스템(Database Management System)의 줄임말로, 데이터베이스를 사용하기 편리하도록 만들어진 프로그램이다. DBMS는 데이터를 다루는 작업, 데이터베이스 생성, 사용자 생성, 사용자 권한 설정 등 데이터베이스 전반을 관리하고 호스팅하는 역할을 한다.

RDB형 DBMS에는 대표적으로 MySQL, Oracle, MSSQL, PostgreSQL 등이 있고, 이 중 MySQL은 오픈 소스 소프트웨어라는 점, 표준에 가까운 SQL을 사용한다는 점 등으로 인해 RDB형 DBMS 중 가장 높은 점유율을 자랑해왔다. 그러나 MySQL이 오픈 소스에 적대적인 Oracle에 인수되어 라이선스 상태가 불안정해지자 MySQL의 개발팀이 MySQL을 기반으로 MariaDB라는 DBMS를 개발하였다. MariaDB는 MySQL과 완벽히 호환되며 최근에는 MySQL의 상위호환으로 거듭나 MariaDB의 사용률이 급증하는 추세이다. 따라서 본 교재에서는 가장 최신 LTS 버전\footnote{2024년 9월 기준 11.4.3}을 이용하여 DB와 관련된 실습을 진행한다.

\subsection*{Installation (Windows)}

MariaDB 다운로드 홈페이지(https://mariadb.org/download/)에서 자신의 OS 버전과 맞는 msi 패키지를 다운로드하여 실행한다.

\figures{fig:mariadb-install-windows}{Installation of MariaDB server on Windows}{
    \subfig{fig:mariadb-install-windows-location}{Copy Install Directory}
        {mariadb-install-windows-location.png}{.31}
    \subfig{fig:mariadb-install-windows-password}{Set Password and Charset}
        {mariadb-install-windows-password.png}{.31}
    \subfig{fig:mariadb-install-windows-port}{Set TCP Port}
        {mariadb-install-windows-port.png}{.31}
}

먼저 \figref{fig:mariadb-install-windows-location}\과 같이 DBMS의 설치 경로\footnote{11.4 버전의 경우 C:\textbackslash{}Program Files\textbackslash{}MariaDB 11.4}를 미리 파악해둔다. 이후 \figref{fig:mariadb-install-windows-password}\와 같이 root 계정(관리자 계정)의 비밀번호를 설정하고 문자 집합을 UTF8로 체크하여 한글 및 기타 특수문자를 사용할 수 있도록 설정한다. root 계정은 DBMS 내의 모든 DB와 데이터를 다룰 수 있는 권한을 가지고 있고, root 비밀번호를 분실하면 DBMS를 재설치해야 하므로 신중하게 정하고 기억해야 한다.

마지막으로 \figref{fig:mariadb-install-windows-port}\와 같이 MariaDB에 접속하기 위한 port 번호를 TCP port란에 설정한다. MariaDB는 기본적으로 3306번 port를 사용하며, 여타 이유로 선점되어 있다면 3307번 등 다른 port로 변경한다.\footnote{보안을 위해 일부러 다른 port로 바꾸기도 한다.}

\figures{fig:mariadb-env-variable}{Setting environment variable}{
    \subfig{fig:env-var-1}{Step 1}
        {env-var-1.png}{.2940}
    \subfig{fig:env-var-2}{Step 2}
        {env-var-2.png}{.2926}
    \subfig{fig:env-var-3}{Step 3}
        {env-var-3.png}{.3434}
}

이제 윈도우의 shell인 cmd에서 MariaDB에 접속할 수 있도록 환경 변수를 설정한다. 먼저 윈도우 탐색기를 열고, \figref{fig:mariadb-install-windows-location}에서 파악한 설치 경로를 연다. 설치 폴더 아래의 bin 폴더에 mariadb.exe 파일이 있는 것을 확인한 뒤 bin 폴더의 전체 경로\footnote{11.4 버전의 경우 C:\textbackslash{}Program Files\textbackslash{}MariaDB 11.4\textbackslash{}bin}를 복사한다. 이후 윈도우 검색을 이용해 [시스템 환경 변수 편집]을 실행하고, \figref{fig:mariadb-env-variable}\과 같이 환경 변수 편집 창에 진입한 뒤, [새로 만들기] 버튼을 클릭하여 bin 폴더의 전체 경로를 붙여넣어 추가한다.

\begin{shell}{shell:mariadb-check-install}{Checking MariaDB installation}
\begin{minted}[linenos=false, xleftmargin=2pt]{text}
$ mariadb --version
\end{minted}
\end{shell}

이제 cmd 창을 열고, \shellref{shell:mariadb-check-install}\와 같이 MariaDB가 제대로 설치되었는지 확인한다.

\subsection*{Installation (macOS)}

\begin{shell}{shell:install-homebrew}{Installing Homebrew}
\begin{minted}[linenos=false, xleftmargin=2pt]{text}
$ /bin/bash -c \
    "$(curl -fsSL https://raw.githubusercontent.com/Homebrew/install/HEAD/install.sh)"
$ brew update
\end{minted}
\end{shell}

macOS에서는 먼저 \shellref{shell:install-homebrew}\를 따라 패키지 관리자인 Homebrew를 설치하고, 설치되었는지 확인한다.

\begin{shell}{shell:mariadb-installation-macos}{Installing MariaDB (macOS)}
\begin{minted}[linenos=false, xleftmargin=2pt]{text}
$ brew info mariadb
$ brew install mariadb
$ brew services start mariadb
\end{minted}
\end{shell}

Homebrew를 이용해 \shellref{shell:mariadb-installation-macos}\를 참고하여 MariaDB를 설치한 뒤 실행한다. 설치가 완료되면 \shellref{shell:mariadb-check-install}\와 같이 확인해본다.

\subsection*{Installation (Linux)}\label{ssect:mariadb-install-linux}

Linux에서는 다음 링크에 접속하여 Linux의 배포판, 버전, MariaDB의 버전을 선택한 뒤 제공되는 안내에 따라 MariaDB를 설치한다. 설치가 완료되면 \shellref{shell:mariadb-check-install}\와 같이 확인해본다.

\begin{itemize}
    \item https://mariadb.org/download/?t=repo-config
\end{itemize}

\subsection*{Accessing MariaDB Server}

\begin{shell}{shell:mariadb-login}{Basic command logging In MariaDB}
\begin{minted}[linenos=false, xleftmargin=2pt]{text}
$ mariadb –u<username> -p<password>
\end{minted}
\end{shell}

이제 shell을 통해 MariaDB에 접속할 수 있다. \shellref{shell:mariadb-login}\은 MariaDB에 로그인하는 기본적인 명령어로, 현재는 root 계정만 존재하므로 root 계정으로 로그인한다. \cd{<username>} 부분을 root으로, \cd{<password>} 부분을 root 계정의 비밀번호로 바꾸어 명령어를 작성한다.

\begin{shell}{shell:mariadb-sudo-login}{Accessing root user using sudo}
\begin{minted}[linenos=false, xleftmargin=2pt]{text}
$ sudo mariadb
\end{minted}
\end{shell}

Unix 계열의 OS에서는 \shellref{shell:mariadb-login}의 방법으로 root 계정에 접속하지 못하는 경우가 있는데, 이때 \shellref{shell:mariadb-sudo-login}\과 같이 sudo 권한으로 실행하면 자동으로 root 계정으로 접속된다.

\begin{shell}{shell:query-db-list}{Querying list of databases}
\begin{minted}[linenos=false, xleftmargin=2pt]{text}
MariaDB [(none)]> SHOW DATABASES;
\end{minted}
\end{shell}

\shellref{shell:query-db-list}\와 같이 SQL을 사용하여 현재 DBMS에 있는 데이터베이스의 목록을 조회할 수 있다. 커서 부분의 대괄호(\cd{[]}) 내부에는 현재 사용자가 작업 중인 데이터베이스의 이름이 표시되며, 현재는 작업 중인 데이터베이스가 없으므로 \cd{(none)}으로 표시된다.

\begin{shell}{shell:create-and-use-db}{Creating and Using Database}
\begin{minted}[linenos=false, xleftmargin=2pt]{text}
MariaDB [(none)]> CREATE DATABASE kweb_db;
MariaDB [(none)]> USE kweb_db;
MariaDB [kweb_db]>
\end{minted}
\end{shell}

\shellref{shell:create-and-use-db}\와 같이 \cd{CREATE DATABASE} 키워드를 이용하여 \cd{kweb\_db}라는 데이터베이스를 생성할 수 있고, \cd{USE} 키워드를 이용하여 작업 중인 데이터베이스를 \cd{kweb\_db}로 바꿀 수 있다.

본 교재에서는 이후 DB shell을 표시할 때 커서 부분에서 DBMS와 DB의 이름은 생략한다.

\subsection*{Database User}

DBMS에는 여러 DB를 생성할 수 있고, 하나의 애플리케이션은 하나의 DB를 사용하는 것이 원칙이므로 여러 애플리케이션이 하나의 DBMS에 접속할 수 있다. root 계정은 소위 관리자 계정에 해당하는 계정으로 DBMS 전체를 조작할 수 있고 모든 DB에 접근하여 데이터를 조작할 수 있다. 여러 애플리케이션에서 각자의 DB를 사용하기 위해 모두 root 계정으로 접속하면 다른 애플리케이션이 사용하는 DB의 데이터를 수정하거나 DB 자체를 없애버리는 등의 조작이 일어나는 등 보안상의 문제가 발생할 수 있다. 따라서 관리자는 애플리케이션마다 적어도 하나의 사용자 계정을 생성하고, 각 사용자의 권한을 제한하여 DB 전체가 공격받는 일이 없도록 해야한다.

\begin{sql}{sql:create-user}{Create user}
\begin{minted}[linenos=false, xleftmargin=2pt]{sql}
CREATE USER '<username>'@'<host>' IDENTIFIED BY '<password>'
\end{minted}
\end{sql}

먼저 root 계정으로 DB에 로그인한 후, \sqlref{sql:create-user}\과 같이 사용자 계정을 생성한다. 이때 \cd{host}는 사용자가 접속하고자 하는 host로, 생성된 계정은 host에 해당하는 컴퓨터에서만 접속할 수 있다. \cd{host} 값을 \cd{\%}로 설정하면 모든 host에서 접속할 수 있다.

\begin{code}{code:create-kwebuser-user}{Create \cd{kwebuser} user}
\begin{minted}{sql}
CREATE USER 'kwebuser'@'%' IDENTIFIED BY 'kwebpw';
\end{minted}
\end{code}

\coderef{code:create-kwebuser-user}\는 모든 host로부터 접근할 수 있고, 비밀번호는 ``kwebpw''인 ``kwebuser'' 계정을 생성하는 예시이다.

\begin{code}{code:create-two-dbs}{Create two DBs}
\begin{minted}{sql}
CREATE DATABASE kwebdb1;
CREATE DATABASE kwebdb2;
\end{minted}
\end{code}

\coderef{code:create-two-dbs}\와 같이 두 DB를 생성하고, ``kwebuser'' 사용자에게 \cd{kwebdb1}의 권한만 부여해보자.

\begin{sql}{sql:grant-all-privileges}{Grant all privileges to user}
\begin{minted}[linenos=false, xleftmargin=2pt]{sql}
GRANT ALL PRIVILEGES ON <db-name>.* TO '<username>'@'<host>'
\end{minted}
\end{sql}

\sqlref{sql:grant-all-privileges}\는 사용자에게 \cd{db-name} DB의 모든 권한을 부여하는 SQL문이다. 모든 권한 외에도 옵션에 따라 특정 권한만 부여할 수 있다.

\begin{code}{code:grant-kwebdb1-to-kwebuser}{Grant all privileges on \cd{kwebdb1} table to ``kwebuser'' user}
\begin{minted}{sql}
GRANT ALL PRIVILEGES ON kwebdb1.* TO 'kwebuser'@'%';
\end{minted}
\end{code}

\coderef{code:grant-kwebdb1-to-kwebuser}\와 같이 \cd{kwebuser} 사용자에게 \cd{kwebdb1} DB의 권한만 부여한다. 이제 root 계정에서 로그아웃하고 \shellref{shell:mariadb-login}\을 참고하여 \cd{kwebuser} 사용자로 로그인해본다.

\begin{shell}{shell:kwebuser-access-db}{Accessing DB with ``kwebuser'' User}
\begin{minted}[linenos=false, xleftmargin=2pt]{text}
$ mariadb -ukwebuser -pkwebpw
> USE kwebdb1;
> USE kwebdb2;
\end{minted}
\end{shell}

\shellref{shell:kwebuser-access-db}\와 같이 ``kwebuser'' 사용자로 로그인한 후, \cd{kwebdb1} DB와 \cd{kwebdb2} DB에 각각 접속을 시도해보면 \cd{kwebdb1}에는 접속할 수 있으나 \cd{kwebdb2}에는 접속이 불가능하다는 것을 확인할 수 있다.
