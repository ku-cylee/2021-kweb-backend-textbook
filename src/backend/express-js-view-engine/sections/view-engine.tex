\section{View Engine}\label{sect:view-engine}

\subsection*{Necessity of View Engine}

\sectref{sect:express-js}에서는 Express.js를 이용하여 HTTP 요청을 받고 그 요청에 따라 plain text 형태로 적절한 HTTP 응답을 보내는 웹 서버를 구현하였다. 그러나 일반 사용자가 웹 사이트를 이용하기 위해서는 브라우저에서 응답 내용을 렌더링할 수 있도록 HTML 형태의 응답을 보내야 한다. 이러한 서버는 \coderef{code:express-server-responding-html}\과 같이 구현될 수 있다.

\begin{codeenv}{code:express-server-responding-html}{Express Server Responding HTML}\begin{verbatim}
const express = require('express');
const app = express();
const port = 3000;

app.get('/', (req, res) => res.send(`
    <h1>GET /</h1>
    <h3>Hello World!</h3>
`));

app.listen(port, () => console.log(`Server listening on port ${port}!`));
\end{verbatim}
\end{codeenv}

\coderef{code:express-server-responding-html}\은 의도한 기능을 문제없이 수행하는 코드이다. 그러나 이렇게 HTML 문서를 직접 코드에 넣어 응답하는 방식은 여러 문제점을 갖는다. 먼저 가장 큰 문제점은 코드의 가독성을 매우 저하시킨다는 점이다. 대부분 웹 페이지의 HTML 문서는 매우 긴데, 이러한 HTML 문서를 그대로 코드 내에 하드 코딩 해버리면 코드에서 잘 나타나야 하는 로직과 흐름이 잘 보이지 않게 된다. 또한, HTTP 요청의 처리 결과에 따라 문서 내의 값이 바뀌거나, 특정 부분이 조건에 따라 바뀌거나, 반복되어야 할 때는 JS 문법을 이용하여 HTML 문서를 생성해야 하는데, HTML 문서와 JS 코드가 난잡하게 섞여 가독성이 저하될 수밖에 없다.

상기한 문제점은 HTML 문서를 응답으로 보낼 때 view engine으로 템플릿 파일을 작성 사용함으로써 해소될 수 있다. 대표적인 view engine으로는 pug, jinja2, ejs 등이 있으며, 이 중 본 교재에서는 간결하여 가독성이 좋고 간편한 pug를 사용한다. 새 Node.js 애플리케이션을 생성하고, Express.js에서 pug로 작성된 파일을 HTML 문서로 렌더링할 수 있도록 \shellref{shell:installing-pug}\와 같이 pug 모듈을 express 모듈과 함께 설치한다.

\begin{shellenv}{shell:installing-pug}{Installing pug}\begin{verbatim}
$ npm install express pug
\end{verbatim}
\end{shellenv}

index.js를 생성하고 \coderef{code:pug-example-index-js}\와 같이 작성한다. \cd{app.set} 메서드를 이용하여 프로젝트 디렉토리(\cd{\_\_dirname})의 views 디렉토리를 템플릿 파일들을 모아둔 디렉토리로 설정하고, pug를 view engine으로 설정한다. 이제 GET /, GET /page/:page, GET /posts/:until의 controller 함수를 각각 구현하며 pug 문법을 알아보자.

\begin{codeenv}{code:pug-example-index-js}{pug Example Application - index.js}\begin{verbatim}
const express = require('express');

const port = 3000;
const app = express();

app.set('views', `${__dirname}/views`);
app.set('view engine', 'pug');

app.get('/', (req, res) => {});

app.get('/page', (req, res) => {});

app.get('/posts', (req, res) => {});

app.listen(port, () => console.log(`Server listening on port ${port}!`));
\end{verbatim}
\end{codeenv}

\subsection*{Basic Syntaxes}

먼저 views/index.pug를 생성하여 \coderef{code:pug-example-index-pug}\와 같이 작성한다.

\begin{codeenv}{code:pug-example-index-pug}{pug Example Application - views/index.pug}\begin{verbatim}
doctype html
html
    head
        title Index Page
        meta(charset='utf-8')
    body
        .page-header
            h1#page-title Index Title
            ul
                li
                    a(href='/board') Go to Board
                li
                    a(href='/article/compose') Compose New Article
        .page-body
            | Page Body
            | Hello World!
\end{verbatim}
\end{codeenv}

\coderef{code:pug-example-index-pug}\를 통해 pug는 다음과 같은 특징을 갖는다는 것을 알 수 있다.

\begin{itemize}
    \item HTML 태그는 태그의 이름만 작성하고, 태그의 열고 닫음은 들여쓰기(indent)로 표현한다.
    \item class와 id는 CSS 선택자를 작성하는 방식과 동일하게 작성하며, class나 id가 있는 \cd{div} 태그는 태그 이름을 생략할 수 있다. 여타 태그는 반드시 태그명을 명시해야 한다.
    \item class, id를 제외한 요소의 속성(attribute)과 속성값(value)은 괄호 내에 \cd{attribute=value}의 형태로 작성하며, 각 key-value pair는 쉼표(\cd{,})로 구분한다.
    \item 태그 내부의 내용이 한 줄이면 태그 부분 뒤에 한 칸을 띄고 내용을 작성한다.
    \item 태그 내부의 내용이 여러 줄이면 다음 줄부터 줄마다 pipe(\cd{|})를 삽입한 뒤 한 칸을 띄고 내용을 작성한다.
\end{itemize}

\begin{codeenv}{code:pug-example-get-index}{pug Example Application - index.js (GET /)}\begin{verbatim}
app.get('/', (req, res) => res.render('index.pug'));
\end{verbatim}
\end{codeenv}

\coderef{code:pug-example-index-js}에서 GET /의 controller 함수를 \coderef{code:pug-example-get-index}\와 같이 수정한다. \cd{render} 메서드에 템플릿 파일의 상대 경로를 인자로 전달하면 템플릿 파일이 pug 모듈에 의해 HTML 문서로 변환(render)되어 응답으로 보내지게 된다. 이때 템플릿 파일의 상대 경로는 \coderef{code:pug-example-index-js}에서 \cd{set} 메서드를 통해 설정한 템플릿 디렉토리를 기준으로 한다.

이제 서버를 실행하고, GET /에 요청을 보내 템플릿 문서가 잘 render되어 응답되는지 확인한다.

\subsection*{Interpolation Syntaxes}

\begin{codeenv}{code:pug-example-board-pug}{pug Example Application - views/board.pug}\begin{verbatim}
doctype html
html
    head
        title Board Page
    body
        h1 Board Page
        h3 This is page #{page}
\end{verbatim}
\end{codeenv}

views/board.pug를 생성하여 \coderef{code:pug-example-board-pug}\와 같이 작성한다. \sectref{sect:advanced-javascript-syntaxes}에서 다룬 template literal과 유사하게 pug에서는 \cd{\#\{\}} 표현을 이용하여 템플릿에 JS 표현식의 반환값을 삽입한다. \cd{page}와 같이 템플릿에서 사용되는 값은 객체의 형태로 \cd{render} 메서드의 두 번째 인자에 전달하면 해당 부분이 JS 표현식의 반환값으로 대체되어 render된다.

\begin{codeenv}{code:pug-example-get-board}{pug Example Application - index.js (GET /page)}\begin{verbatim}
app.get('/page', (req, res) => {
    const { page } = req.query;
    res.render('board.pug', { page });
});
\end{verbatim}
\end{codeenv}

\coderef{code:pug-example-index-js}에서 GET /page의 controller 함수를 \coderef{code:pug-example-get-board}\와 같이 수정하여 \cd{render} 메서드의 인자로 \cd{page}의 값을 전달한다. 그리고 GET /page?page=1, GET /page?page=2 등으로 요청을 보내어 템플릿 문서가 잘 render되는지 확인한다.
\clearpage

\subsection*{Conditional and Iteration Syntaxes}

\begin{codeenv}{code:pug-example-posts-pug}{pug Example Application - views/posts.pug}\begin{verbatim}
doctype html
html
    head
        title Posts Page
    body
        if posts.length
            h1 Posts Count: #{posts.length}
            each post, index in posts
                h3 ##{index} - #{post}
        else
            h1 Invalid Page!
\end{verbatim}
\end{codeenv}

views/posts.pug를 \coderef{code:pug-example-posts-pug}\와 같이 작성한다. 이처럼 pug 템플릿에서는 if-else문 등을 이용하여 조건문을, each-in문 등을 이용하여 반복문을 사용할 수 있다.

\begin{codeenv}{code:pug-example-get-posts}{pug Example Application - index.js (GET /posts)}\begin{verbatim}
app.get('/posts', (req, res) => {
    const { until } = req.query;
    const untilParsed = parseInt(until, 10);

    const posts = [];
    if (!isNaN(untilParsed)) {
        for (let i = 0; i < untilParsed; i++) {
            posts.push(`Post ${i + 1}`);
        }
    }
    res.render('posts.pug', { posts });
});
\end{verbatim}
\end{codeenv}

\coderef{code:pug-example-index-js}에서 GET /posts의 controller 함수를 \coderef{code:pug-example-get-posts}\와 같이 수정하여 \cd{posts} 값을 \cd{render} 메서드의 인자로 전달한다. 이후 GET /posts?until=10 요청에 대한 응답과 GET /posts?until=tenth 요청에 대한 응답을 비교해본다.

pug의 자세한 문법은 공식 홈페이지\footnote{https://pugjs.org/api/getting-started.html}에서 확인할 수 있다.
