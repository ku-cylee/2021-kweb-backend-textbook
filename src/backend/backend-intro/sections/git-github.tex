\section{git and GitHub}\label{sect:git-github}

\subsection*{Version Control System(VCS) and git}

소프트웨어의 개발 과정에서 소스 코드의 백업, 개발 버전의 관리, 소스 코드의 변경사항 기록 및 추적 등의 작업은 필수적이다. 버전 관리 시스템(Version Control System, VCS)은 소프트웨어 개발자를 위해 이러한 작업을 수행해주는 시스템이다.

git은 소프트웨어 개발 분야에서 압도적인 점유율을 차지하는 버전 관리 시스템이다. git은 버전 관리뿐만 아니라 branch를 이용하여 여러 사용자가 하나의 프로젝트를 개발하고 merge를 통해 각자 개발한 부분을 조율하여 반영할 수 있게 하고, 별도의 서버에 소스 코드를 저장할 수 있게 하여 소스 코드의 백업과 공유를 용이하게 한다.

\subsection*{How git works}

git은 프로젝트를 저장소(repository)에 저장하여 관리한다. 특정 디렉토리에서 git 저장소를 생성하면, 해당 디렉토리에 속한 모든 하위 파일과 하위 디렉토리는 프로젝트를 구성하는 요소가 되고, 그 디렉토리에 있는 파일에서 발생한 변경사항은 git에 의해 추적(track)된다.

로컬 디렉토리에서 발생한 변경사항은 \cd{add} 명령을 통해 staging area에 추가되고, \cd{commit} 명령을 통해 로컬 저장소(local repository)에 변경사항이 반영된다. Staging area는 \cd{commit}을 실행하기 전 commit할 내용을 정리할 수 있게 도와주는 중간 단계 역할을 수행한다. 이렇게 \cd{commit}을 이용하여 변경사항을 반영할 때마다 하나의 commit이 생성되는데, git은 버전을 각 commit 단위로 관리하고 다른 commit과의 변경사항을 확인할 수 있게 한다.

이렇게 생성된 commit들은 \cd{push} 명령을 통해 원격 저장소(remote repository)에 업로드되어 반영된다. 반대로 \cd{pull} 명령은 원격 저장소에 반영된 변경사항을 다운로드하여 로컬 저장소에 반영하며, \cd{clone} 명령은 원격 저장소를 복사하여 로컬 디렉토리에 저장하는 명령이다.

\figures{fig:how-git-works}{How git works}{
    \fig{how-git-works.png}{.6}
}

\subsection*{gitignore}

소프트웨어를 개발하는 과정에서 로그 파일, 빌드 파일(.exe, .out 등), 설정(config) 파일 등 소스 코드가 아닌 파일이 프로젝트 디렉토리 내에 생성되곤 한다. 설정 파일은 로컬 개발 환경에 따라 달리 설정해야 하고, 비밀번호나 비밀 키 등과 같이 제 3자가 알 경우 보안 취약점이 발생할 수 있는 내용이 포함되어 있을 수 있어 추적되어서는 안 된다. 빌드 파일은 기계어로 되어있기 때문에 변경사항을 추적하는 것이 용량만 차지하고 무의미하며, 빌드 방법만 정확히 서술되어 있다면 다른 사용자가 충분히 빌드할 수 있기 때문에 추적할 필요가 없다. 이렇게 소프트웨어 그 자체와는 무관한 파일들은 git을 통해 추적하지 않는 것이 권장된다.

이러한 파일이나 디렉토리를 gitignore 파일에 정규 표현식 형태로 명시하여 추적 대상에서 제외할 수 있다. gitignore 파일은 프로젝트 디렉토리의 최상위 디렉토리에 ``.gitignore''이라는 이름으로 생성한다.

추적에서 제외되는 파일이나 디렉토리의 목록은 애플리케이션에서 사용되는 언어, 프레임워크, IDE 등 다양한 요소에 영향을 받는다. 이러한 요소에 따라 통상적인 gitignore 파일을 자동으로 생성하는 유용한 기능을 제공해주는 웹 사이트를 소개한다.

\begin{itemize}
    \item https://www.toptal.com/developers/gitignore
\end{itemize}

\subsection*{Installing git}

git은 소프트웨어 개발에서 거의 빠지지 않고 사용되기 때문에 미리 설치되어 있는 경우가 종종 있다. macOS에서는 Xcode를 설치할 때 같이 설치되며, Linux 계열의 운영체제에는 사전에 설치되어 있는 경우가 많다. \shellref{shell:git-version}\과 같이 git이 설치되어 있는지 확인할 수 있다.

\begin{shellenv}{shell:git-version}{Checking git version}\begin{verbatim}
$ git --version
\end{verbatim}
\end{shellenv}

Windows와 macOS에서는 git 홈페이지(https://git-scm.com/downloads)에서 자신의 운영체제에 맞는 설치파일을 다운받고, 실행하여 추가 옵션 없이 설치한다. Ubuntu와 같은 Linux 계열의 OS에서는 \shellref{shell:git-install-ubuntu}과 같이 설치한다.

\begin{shellenv}{shell:git-install-ubuntu}{git Installation (Ubuntu)}\begin{verbatim}
$ sudo apt install -y git
\end{verbatim}
\end{shellenv}

이후 설치가 완료되면 \shellref{shell:git-version}\과 같은 방법으로 설치가 잘 되었는지 확인한다.
\clearpage

\subsection*{Starting GitHub and Creating Repository}

GitHub은 대표적인 git 원격 저장소 무료 호스팅 사이트로, git을 이용하여 로컬 컴퓨터에 있는 프로젝트를 저장할 수 있는 원격 저장소를 제공한다. 다른 사람과 소스 코드를 공유하기 쉬울뿐만 아니라 git의 버전 관리 시스템과 branch 등을 이용한 협업 등의 기능, 그리고 그 기능을 보조해주는 여러 기능들을 간단하고 깔끔한 UI로 제공하여 오늘날 가장 대중적인 프로젝트 관리 및 오픈소스 공유 플랫폼이 되었다. 이러한 GitHub에 가입하고, 저장소를 만들어 GitHub 상에 간단한 프로젝트를 업로드하고 변경사항을 반영하는 실습을 진행한다.

먼저 GitHub 홈페이지에 접속하여 우측 상단의 [Sign Up]을 클릭하여 회원가입 페이지에 접속한 뒤, 사용자명, 비밀번호, 이메일 주소를 입력하고 계정 인증 절차를 완료한다. 이후 하단의 [Next: Select a plan]에서 [Free Plan]을 선택한 후, 본인의 정보를 간단하게 입력하고 [Complete setup]을 클릭하여 완료한다. 마지막으로 입력한 이메일을 통해 인증하여 가입을 마무리한다.

먼저 GitHub에 원격 저장소를 생성하는 단계이다.\footnote{로컬 저장소를 먼저 생성하는 방법도 있으나, 여러 설정을 해주어야 하기 때문에 GitHub에서 먼저 원격 저장소를 생성하는 것이 훨씬 간단하다.} GitHub 홈페이지 좌측의 [Repositories] 메뉴에서 [New] 버튼을 클릭하여 새 저장소를 생성할 수 있는 페이지로 이동하고, \figref{fig:create-new-repo}\와 같이 [Repository Name]란에 저장소 이름을 ``my-first-repo''로 지정하고, [Create Repository]를 클릭하여 새 저장소를 생성한다. 이후 해당 저장소를 열람할 수 있는 페이지에서 \figref{fig:copying-remote-repo-url}\과 같이 원격 저장소의 주소를 복사해둔다.

\figures{fig:github-start}{Starting GitHub}{
    \subfig{fig:create-new-repo}{Create New Repository}
        {create-new-repo.png}{.3746}
    \subfig{fig:copying-remote-repo-url}{Copy Remote Repository URL}
        {copying-remote-repo-url.png}{.4254}
}

이제 이 프로젝트를 \cd{clone} 명령어를 통해 로컬 환경에 다운로드 받을 수 있다. 프로젝트를 다운받고자 하는 디렉토리에서 shell을 실행하고, \shellref{shell:git-clone}에서 \cd{<repo-url>} 부분에 원격 저장소 주소를 입력하여 실행한다. 이후 프로젝트가 원하는 디렉토리에 정상적으로 clone 되었는지 확인한다.

\begin{shellenv}{shell:git-clone}{Clone remote repository to local}\begin{verbatim}
$ git clone <repo-url>
\end{verbatim}
\end{shellenv}
\clearpage

\subsection*{Commit Project Updates}

이제 \sectref{sect:node-js}에서 작성한 코드를 활용하여 로컬 환경에 clone된 프로젝트를 수정하고, 수정한 내역을 commit 해본다. 먼저 프로젝트 내에 .gitignore 파일을 생성\footnote{파일 시스템은 . 으로 시작하는 파일을 만드는 것을 그다지 좋아하지 않으므로, VS Code 등의 IDE를 이용하여 만드는 것을 추천한다.}하고, \coderef{code:gitignore-example}\과 같이 작성하여 확장자가 txt인 모든 파일을 추적하지 않도록 한다.

\begin{codeenv}{code:gitignore-example}{.gitignore}\begin{verbatim}
*.txt
\end{verbatim}
\end{codeenv}

프로젝트 디렉토리에 index.js를 생성하고, \coderef{code:nodejs-simple-web-server}(\pageref{code:nodejs-simple-web-server}쪽)의 코드를 작성한다. 이후 \shellref{shell:commit-and-push}\와 같은 과정을 통해 변경 사항을 commit하고, 이를 원격 저장소에 push한다. 이때 \cd{git add .} 명령어는 프로젝트 내 모든 파일의 변경 사항을 추적하여 staging area로 올린다는 뜻이다. 이후 원격 저장소에서 commit 내역을 확인하여, 변경 사항이 정상적으로 반영되었는지 확인한다.

\begin{shellenv}{shell:commit-and-push}{Commit Changes and Push Commits}\begin{verbatim}
$ git add .
$ git commit -m <commit-message>
$ git push
\end{verbatim}
\end{shellenv}

이제 index.js에서 \cd{port}의 값을 4000으로 바꾸고, config.txt 파일을 생성하여 아무 내용이나 입력해보자. 그러면 index.js에서는 \cd{port}가 선언된 네 번째 줄에서 변경이 발생하였고, config.txt는 .gitignore에 의해 추적되지 않으므로 변경 사항으로 간주되지 않을 것임을 예상할 수 있다. \shellref{shell:commit-and-push} 과정을 통해 변경사항을 다시 commit 및 push하고, 원격 저장소에서 commit 내역을 확인하여 이와 같은 예상이 맞는지 확인해본다.
