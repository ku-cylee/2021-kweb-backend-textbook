\section{HyperText Transfer Protocol (HTTP)}\label{sect:http}

\subsection*{What is HTTP?}

\sectref{sect:web-intro}에서 다룬 바와 같이 웹 애플리케이션은 클라이언트와 서버가 요청과 응답을 통해 데이터를 교환하는 애플리케이션이다. 이때 요청과 응답에 의해 교환되는 데이터에는 발신 및 수신 IP와 같은 요청과 응답에 관한 여러 정보가 포함되어 있어야 하는데, 이러한 정보를 작성하는 방식이 개발 주체에 따라 다를 경우 클라이언트가 보낸 요청을 서버가 해석하지 못하는 등의 혼란이 발생한다. 이를 방지하기 위해 웹 애플리케이션에서 클라이언트와 서버 간의 통신에서 준수해야 하는 표준 규약(protocol)이 존재하며, HTTP(HyperText Transfer Protocol)는 웹 페이지를 전송하기 위해 사용되는 표준 규약이다.

HTTP는 헤드와 본문으로 이루어져 있고, 헤드(HTTP head)는 요청/응답 라인(request/response line)과 헤더(HTTP header)로 이루어져 있다. 요청/응답 라인에는 요청/응답에 대한 간단한 정보가 한 줄로 기록되어 있고, 헤더에는 요청이나 응답에 대한 상세한 정보나 옵션이 포함된다. 본문(HTTP body)에는 전송하고자 하는 데이터가 포함되어 있으며, 요청에서는 요청에 대한 여러 조건과 값, 응답에서는 웹 페이지의 HTML 문서 등이 포함되어 있다.

\subsection*{HTTP Request}

HTTP 요청(request)은 클라이언트가 서버에 전송하는 메시지이다.

\begin{codeenv}{code:http-request-example}{Head of HTTP Request Example}\begin{verbatim}
GET /api/articles?bid=12&page=3&count=10 HTTP/1.1
Host: kweb.korea.ac.kr
Connection: keep-alive
Upgrade-Insecure-Requests: 1
User-Agent: Mozilla/5.0 (Windows NT 10.0; Win64; x64; rv:79.0) Firefox/79.0
Accept: text/html,application/xhtml+xml,application/xml;q=0.9,*/*;q=0.8
Accept-Encoding: gzip, deflate
Accept-Language: ko-KR,ko;q=0.9,en-US;q=0.8,en;q=0.7
Cookie: sid=s%3A_2uUdS9bmos5wPmQmMuHvVXHrcdIYHmk.%2F3FX1NwUkH5fFK3cd7gDr8qfMHsrvfMWGSO
\end{verbatim}
\end{codeenv}

HTTP 요청의 헤더는 \coderef{code:http-request-example}\과 같은 형태로 구성되어 있다. 요청 라인에는 차례대로 요청 메서드(\cd{GET}), URL(\cd{/api/articles?bid=12\&page=3\&count=10}), HTTP 버전(\cd{HTTP/1.1})이 차례대로 나열된다. 헤더에는 Host, User-Agent, Cookie등의 여러 정보와 그에 대한 값으로 구성되어 있다. User-Agent는 사용자가 요청을 보내기 위해 사용한 애플리케이션에 대한 정보를 제시하며, 예시에서의 요청은 Mozilla Firefox 브라우저에서 보낸 요청이다. HTTP 요청에 열거되는 중요한 요소 중 하나인 쿠키(cookie)는 \sectref{sect:cookie-session}에서 다룬다.

\subsection*{HTTP Method}

HTTP 메서드(method)는 서버가 수행해야 할 동작의 종류를 나타내는 요소이다. 가장 흔히 쓰이는 메서드는 GET과 POST이고, RESTful하게 구현된 애플리케이션에서는 PUT과 DELETE가 추가로 사용되며, 이 외에도 HEAD, OPTIONS, PATCH 등의 메서드가 사용되기도 한다. 가장 많이 사용되는 네 메서드는 각각 다음과 같은 동작을 명시한다.

\begin{itemize}
    \item GET: 서버에 존재하는 리소스에 대한 열람(read)을 요청
    \item POST: 서버에 특정 리소스를 새로 생성(create)할 것을 요청
    \item PUT: 서버에 존재하는 특정 리소스를 수정(update)할 것을 요청
    \item DELETE: 서버에 존재하는 특정 리소스를 삭제(delete)할 것을 요청
\end{itemize}

예를 들어 서버에 저장되어 있는 회원들에 대한 데이터에 관한 요청을 보낼 때, 특정 회원 또는 여러 회원들에 대한 데이터를 열람하고자 할 때는 GET, 새로 가입한 회원의 정보를 추가하고자 할 때는 POST, 이미 존재하는 회원의 정보를 수정하고자 할 때는 PUT, 특정 회원이 사이트를 탈퇴하여 그 정보를 지우고자 할 때는 DELETE를 사용하여 요청한다.

다만, 초기의 HTTP는 GET과 POST 메서드만 있어 많은 웹 애플리케이션들이 두 메서드만으로 구현되어왔고, 이러한 이유로 대부분의 웹 브라우저들이 GET과 POST 메서드만 지원한다. 이러한 경우 GET 메서드는 요청의 주된 목적이 리소스의 열람인 경우, POST 메서드는 요청의 주된 목적이 리소스에 대한 변화인 경우 사용된다.

\subsection*{URL}

URL(Uniform Resource Locator)은 인터넷 상에서 특정 리소스의 위치를 나타내는 문자열로, 흔히 링크 혹은 주소라고 부르기도 한다. URL은 정해진 구조에 따라 리소스에 대한 위치 정보를 세부적으로 담고 있다.

\figures{fig:url-structure}{Brief Structure of URL}
    {\fig{url-structure.png}{.93}}

\figref{fig:url-structure}\는 간단한 URL의 구조이다. 각 요소는 다음과 같은 정보를 나타낸다.

\begin{itemize}
    \item protocol: 사용자가 요청을 보내 사용하고자 하는 애플리케이션의 프로토콜; http, https 뿐만 아니라 ftp, mailto 등의 프로토콜도 많이 사용된다.
    \item hostname: 요청을 보내고자 하는 서버의 주소; 도메인 이름이나 아이피 주소가 사용된다 (예: google.com / korea.ac.kr / 8.8.8.8)
    \item port: 서버에 접근하기 위한 관문의 번호; 이론적으로 0 이상, $2^{16}=65536$ 미만의 정수가 가능하고, 프로토콜의 기본 포트에 요청을 보낼 때에는 생략할 수 있다. http와 https의 기본 포트는 각각 80번, 443번.
    \item path: 서버상에 존재하는 리소스의 위치 경로; 최근에는 서버에 요청하고자 하는 작업을 추상적으로 나타낸다.
    \item query: 요청에 대한 상세한 조건과 그에 대응하는 값을 key-value pair 형태로 제시하는 요소; \cd{?} 문자로 query임을 나타내고, \cd{key=value}의 형태로 나타내며, 각 key-value pair는 \cd\& 문자를 사용하여 구분한다. (예: \cd{?bid=12\&page=3\&count=10})
\end{itemize}

웹 서버는 받은 요청의 메서드와 URL을 기반으로 요청을 수행하고 처리하여 클라이언트에게 응답을 보내는 일을 수행한다. 특히 메서드와 경로(path)를 묶어 라우트(route)라고 하며, ``GET /session/view''와 같이 표시한다.

\subsection*{HTTP Response and Response Status}
HTTP 응답(response)은 클라이언트로부터 받은 요청에 대해 서버가 클라이언트에 전송하는 메시지이다.

\begin{codeenv}{code:http-response-example}{Head of HTTP Response Example}\begin{verbatim}
HTTP/1.1 200 OK
Server: nginx
Date: Sat, 12 Sep 2020 13:43:29 GMT
Content-Type: text/html
Connection: keep-alive
Content-Length: 224
\end{verbatim}
\end{codeenv}

HTTP 응답의 헤더는 \coderef{code:http-response-example}\과 같은 형태로 구성되어 있다. 응답 라인에는 차례대로 HTTP 버전(\cd{HTTP/1.1}), 상태 코드(\cd{200}), 상태 메시지(\cd{OK})가 나열된다. 응답 상태(status)는 서버가 보낸 응답에 대한 상태를 대략적으로 나타내며, 상태 코드(status code)와 상태 메시지(status message)는 해당 상태를 각각 세 자리의 정수 형태와 문자열로 나타낸 것이다.

표준 HTTP 상태 코드는 63개가 있으며, 다음과 같이 크게 5가지로 분류된다.

\begin{itemize}
    \item 1XX (Informational Response): 요청을 받았으며, 작업을 수행 중
    \item 2XX (Successful): 요청한 작업을 성공적으로 수행하였음
    \item 3XX (Redirection): 요청을 완료하기 위해 클라이언트는 추가 작업을 수행해야 함
    \item 4XX (Client Error): 클라이언트에서 잘못된 요청을 보내, 요청을 완료할 수 없음
    \item 5XX (Server Error): 서버상의 문제로 요청을 완료할 수 없음
\end{itemize}

다음 상태 코드들은 63개의 상태 코드 중 널리 사용되는 코드들이다.

\begin{itemize}
    \item 200 OK: 요청이 성공적으로 완료됨
    \item 302 Found: 요청한 리소스가 다른 URL에 존재함
    \item 304 Not Modified: 응답할 내용이 클라이언트가 이전에 받은 응답과 동일함
    \item 400 Bad Request: 요청한 데이터의 형태가 잘못되어, 요청을 수행할 수 없음
    \item 401 Unauthorized: 인증이 실패하였거나 이루어지지 않아, 요청을 수행할 수 없음
    \item 403 Forbidden: 요청한 클라이언트가 적절한 접근 권한을 갖지 않음
    \item 404 Not Found: 요청한 리소스가 서버에 존재하지 않음
    \item 500 Internal Server Error: 서버 내부의 문제로 요청을 수행할 수 없음
    \item 503 Service Unavailable: 서버의 과부하나 점검으로 인해 요청을 수행할 수 없음
\end{itemize}

이처럼 상태 코드는 클라이언트(사용자)에게 요청에 대한 응답이 어떤 상태인지 알려준다. 상용 웹 브라우저는 상태 코드에 따라 별도의 작업을 수행하기도 하는데, 예를 들어 클라이언트(브라우저)가 /articles로 요청을 보냈을때 서버가 302 Found라는 상태 코드와 함께 /articles/page/1이라는 URL을 응답하면, 브라우저는 자동으로 이 URL에 요청을 보내 응답을 받아온다. 이러한 과정을 리다이렉트(redirect)라고 한다.

\subsection*{Insomnia}

HTTP 통신은 그 목적에 맞게 인터넷 브라우저를 통해서 이루어질 수 있지만, 브라우저는 일반 사용자를 위한 프로그램인 만큼 HTTP 요청의 속성을 수정하고 응답의 속성을 확인하는 기능을 잘 지원하지 않는다. 따라서 개발자들은 이러한 웹 서버를 개발하면서 HTTP 통신을 용이하게 하기 위해 개발용 클라이언트 프로그램을 사용하곤 한다. 개발용 클라이언트 프로그램에는 Postman이나 Insomnia 등이 있으며, 본 교재에서는 Insomnia를 사용한다.

Insomnia 홈페이지\footnote{https://insomnia.rest/download}에서 자신의 OS에 맞는 설치파일을 다운로드 받아 설치한다. 이후 Insomnia를 실행하면 나타나는 Dashboard 창에서 우상단의 [Create] \textgreater{} [Request Collection] 버튼을 눌러 Collection을 생성하고, 좌상단의 [+] \textgreater{} [New Request] 버튼을 눌러 새로운 HTTP 요청을 생성한다.

\figures{fig:insomnia}{Insomnia}
    {\fig{insomnia-example.png}{.7}}

\figref{fig:insomnia}\와 같이 URL란에 ``http://www.example.com/''을 입력하고, [Send] 버튼을 눌러 응답을 받아 응답을 확인한다. Insomnia를 사용하면 요청을 보낼 때 메서드나 내용, query, 헤더 등을 손쉽게 수정할 수 있고, 받은 응답의 상태 코드, 소요 시간, 크기, 헤더 등도 쉽게 확인할 수 있다. 또한, 응답 내용이 HTML 문서라면 렌더링하여 보여주고, JSON이나 XML 형태라면 formatting하여 보여주며, Raw HTML을 확인할 수 있는 등 여러 편의 기능을 제공한다.
