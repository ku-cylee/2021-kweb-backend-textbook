\section{Routing and Controller Implementation}\label{sect:web-app-routing-controller}

이번 장에서는 웹 서버의 각 기능을 설계하고 라우팅한 뒤, 구현해본다.

\subsection*{Routing}

먼저 웹 사이트의 홈 페이지 라우트는 다음과 같이 정할 수 있다.

\begin{itemize}
    \item GET /: 홈 페이지
\end{itemize}

사용자 인증 기능을 담당하는 라우팅을 설계하자. 인증과 관련된 기능은 크게 회원 가입(sign up), 로그인(sign in), 로그아웃(sign out) 등이 있다. \sectref{sect:express-js-view-engine-exercises}의 \excref{exc:form-login}에서 이미 GET 메서드를 통해 로그인 페이지를 응답하고, POST 메서드를 통해 로그인 데이터를 받은 바 있다. 따라서 다음과 같이 라우트를 설계할 수 있다.

\begin{itemize}
    \item GET /auth/sign\_in : 로그인 페이지
    \item POST /auth/sign\_in : 로그인 데이터를 받아서 인증
    \item GET /auth/sign\_up : 회원가입 페이지
    \item POST /auth/sign\_up : 회원가입 데이터를 받아서 인증
    \item GET /auth/sign\_out : 로그아웃 처리\footnote{로그아웃 기능은 POST 메서드로 처리하는 것이 좋으나, 본 교재에서는 일단 GET 메서드로 처리한다.}
\end{itemize}

게시물과 관련된 기능을 담당하는 라우팅을 설계하자. 먼저 각 게시물에 대한 CRUD 기능, 즉 생성, 조회, 수정 삭제 기능을 수행할 라우트가 필요하다.\footnote{정책에 따라 특정 기능들은 제외될 수 있다.} 게시물 생성 기능과 수정 기능은 사용자가 데이터를 입력할 수 있는 페이지가 제공되어야 하므로 다음과 같이 라우트를 설계할 수 있다.

\begin{itemize}
    \item GET /article/:articleId(\textbackslash{}d+) : \cd{id}가 \cd{articleId}인 게시물을 열람
    \item GET /article/compose : 새 게시물 작성 페이지
    \item POST /article/compose : 새 게시물 데이터 반영
    \item GET /article/edit/:articleId(\textbackslash{}d+) : \cd{id}가 \cd{articleId}인 게시물 수정 페이지
    \item POST /article/edit/:articleId(\textbackslash{}d+) : \cd{id}가 \cd{articleId}인 게시물 수정사항 반영
    \item GET /article/delete/:articleId(\textbackslash{}d+) : \cd{id}가 \cd{articleId}인 게시물 삭제\footnote{게시물을 삭제하는 것은 POST 메서드를 사용하는 것이 옳으나, 본 교재에서는 일단 GET 메서드로 처리한다.}
\end{itemize}

여기에서 \cd{\textbackslash{}d+}는 정수와 매칭되는 정규표현식으로, 이렇게 작성하면 \cd{articleId}는 정수 형태의 문자열임이 보장된다.

게시판에서는 사용자의 편의를 위해 게시물의 목록도 조회할 수 있어야 한다. 이 웹 애플리케이션에서는 게시물 목록을 최신 게시물부터 페이지 단위로 조회할 수 있도록 한다. 이를 반영하면 다음과 같이 라우트를 설계할 수 있다.

\begin{itemize}
    \item GET /articles/page/:page(\textbackslash{}d+) : \cd{page}번째 페이지의 게시물 목록 조회
    \item GET /articles : 1번째 페이지의 게시물 목록 조회
\end{itemize}

\subsection*{Home and Auth Routes' Controller}

각 라우트에 대한 컨트롤러를 구현해보자. 모든 컨트롤러는 \coderef{code:controller-format}\와 같은 형태를 가지며, 컨트롤러에서 에러가 발생하면 \cd{next} 함수에 에러 객체 \cd{err}를 전달하여 호출한다.

\begin{code}{code:controller-format}{Controller Format}
\begin{minted}{js}
const controllerName = async (req, res, next) => {
    try {
        // Controller function
    } catch (err) {
        return next(err);
    }
};
\end{minted}
\end{code}

src/controller/ctrl.js를 생성하고, 다음 controller를 구현한 뒤 export하여라. 결과는 \coderef{code:index-ctrl}\를 참고하여라.

\subsubsection*{\cd{indexPage}}
\begin{itemize}
    \item GET / → 웹 사이트의 홈페이지를 응답
    \item 템플릿: views/index.pug
    \begin{itemize}
        \item \cd{user}: 현재 로그인된 사용자 객체
    \end{itemize}
    \item Hint
    \begin{itemize}
        \item 템플릿 파일의 렌더링: \sectref{sect:view-engine} 참고
        \item 사용자 정보는 세션 객체의 \cd{user} 객체에 저장된다. (\sectref{sect:cookie-session})
    \end{itemize}
\end{itemize}

src/controller/auth/ctrl.js를 생성하고, 다음 controller를 구현한 뒤 export하여라. 필요에 따라 적절한 모듈을 import하여 사용하여라. 결과는 \coderef{code:auth-ctrl}\를 참고하여라.

\subsubsection*{\cd{signInForm}}
\begin{itemize}
    \item GET /auth/sign\_in
    \begin{itemize}
        \item 로그인된 사용자 → 홈으로 redirect
        \item 로그인되지 않은 사용자 → 로그인 페이지 응답
    \end{itemize}
    \item 템플릿: views/auth/sign-in.pug
    \begin{itemize}
        \item \cd{user}: 현재 로그인된 사용자 객체
    \end{itemize}
    \item Hint: 로그인되지 않은 사용자의 \cd{user} 값은 \cd{undefined}이다.
\end{itemize}

\subsubsection*{\cd{signIn}}
\begin{itemize}
    \item POST /auth/sign\_in → HTTP 요청의 body의 아이디와 비밀번호를 받아 로그인 처리
    \begin{itemize}
        \item 아이디(\cd{username})나 비밀번호(\cd{password})가 없으면 \cd{BAD\_REQUEST} 에러 발생
        \item DB에서 \cd{username}이 일치하는 사용자가 없으면 \cd{UNAUTHORIZED} 에러 발생
        \item 요청받은 비밀번호와 사용자의 비밀번호가 불일치하면 \cd{UNAUTHORIZED} 에러 발생
        \item 사용자의 \cd{id}, 아이디(\cd{username}), 닉네임(\cd{displayName}), 탈퇴 여부(\cd{isActive}), 관리자 여부(\cd{isStaff}) 정보를 담은 객체를 생성하여 세션의 \cd{user} 객체로 등록
        \item 홈으로 redirect
    \end{itemize}
    \item Hint
    \begin{itemize}
        \item \cd{ERROR\_NAME} 에러를 발생시키는 방법: \cd{throw new Error('ERROR\_NAME')}
        \item 세션을 생성하는 방법: \sectref{sect:cookie-session} 참고
    \end{itemize}
\end{itemize}

\subsubsection*{\cd{signUpForm}}
\begin{itemize}
    \item GET /auth/sign\_up → 회원가입 페이지 응답
    \item 템플릿: views/auth/sign-up.pug
    \begin{itemize}
        \item \cd{user}: 현재 로그인된 사용자 객체
    \end{itemize}
\end{itemize}

\subsubsection*{\cd{signUp}}
\begin{itemize}
    \item POST /auth/sign\_up → HTTP 요청의 body의 아이디와 비밀번호, 닉네임을 받아 회원가입 처리
    \begin{itemize}
        \item 아이디(\cd{username}), 비밀번호(\cd{password}), 닉네임(\cd{displayName})이 없거나 저장 가능한 최대 길이를 초과하면 \cd{BAD\_REQUEST} 에러 발생
        \item 사용자가 입력한 비밀번호를 이용하여 암호화된 비밀번호 생성
        \item DB에 새로운 사용자 데이터 생성
        \item 로그인 페이지로 redirect
    \end{itemize}
    \item Hint: 저장 가능한 최대 길이는 해당 column의 자료형을 확인
\end{itemize}

\subsubsection*{\cd{signOut}}
\begin{itemize}
    \item GET /auth/sign\_out → 로그아웃 처리, 완료 시 홈으로 redirect
    \item Hint: 세션을 삭제하는 방법은 \sectref{sect:cookie-session} 참고
\end{itemize}

\begin{code}{code:auth-routing}{Auth Routing}
\begin{minted}{js}
const { Router } = require('express');

const ctrl = require('./ctrl');

const router = Router();

router.get('/sign_in', ctrl.signInForm);
router.post('/sign_in', ctrl.signIn);

router.get('/sign_up', ctrl.signUpForm);
router.post('/sign_up', ctrl.signUp);

router.get('/sign_out', ctrl.signOut);

module.exports = router;
\end{minted}
\end{code}

이제 각 controller 함수를 각 라우트에 연결하는 라우팅을 해보자. src/controller/auth/index.js를 생성하고, \coderef{code:auth-routing}\과 같이 작성한다. \coderef{code:auth-routing}\은 express의 \cd{Router} 객체를 이용하여 경로가 /auth로 시작하는 라우트만 라우팅하며, 각 라우트의 경로에서 /auth는 생략한다. 그리고 이 \cd{Router} 객체를 export한다.

\begin{code}{code:index-routing}{Index Routing}
\begin{minted}{js}
const { Router } = require('express');

const ctrl = require('./ctrl');
const auth = require('./auth');

const router = Router();

router.get('/', ctrl.indexPage);

router.use('/auth', auth);

module.exports = router;
\end{minted}
\end{code}

src/controller/index.js를 생성하고, \coderef{code:index-routing}\과 같이 작성한다. \coderef{code:index-routing}에서는 express의 \cd{Router} 객체에 GET / 라우트의 controller 함수를 등록하고, \coderef{code:auth-routing}에서 작성한 \cd{Router} 객체를 /auth로 시작하는 경로에 대해서만 실행되는 미들웨어로 bind한다. 이러한 구조는 인증(auth)과 관련된 라우트를 따로 분리하여 작성할 수 있도록 해준다.

\begin{code}{code:bind-controller}{Bind Controller Module on \cd{app}}
\begin{minted}[escapeinside=||]{js}
// omit

|\textbf{\PYG{k+kr}{const} \PYG{n+nx}{controller} \PYG{o}{=} \PYG{n+nx}{require}\PYG{p}{(}\PYG{l+s+s1}{\PYGZsq{}./controller\PYGZsq{}}\PYG{p}{);}}|
const { errorHandler } = require('./lib/error-handler');

// omit

|\textbf{\PYG{n+nx}{app}\PYG{p}{.}\PYG{n+nx}{use}\PYG{p}{(}\PYG{l+s+s1}{\PYGZsq{}/\PYGZsq{}}\PYG{p}{,} \PYG{n+nx}{controller}\PYG{p}{);}}|

app.use(errorHandler);

module.exports = app;
\end{minted}
\end{code}

% // omit

% const controller = require('./controller');
% const { errorHandler } = require('./lib/error-handler');

% // omit

% app.use('/', controller);

% app.use(errorHandler);

% module.exports = app;
% \end{minted}

src/app.js를 \coderef{code:bind-controller}\와 같이 수정하여 /로 시작하는 경로에 대해 src/controller 모듈이 실행될 수 있도록 한다. 이제 \shellref{shell:npm-run-script}\와 같이 서버를 실행하고, 지금까지 작성한 기능들을 직접 테스트 해보자.
\clearpage

\subsection*{Article Routes' Controller}

src/controller/article/ctrl.js를 생성하고, 다음 controller를 구현한 뒤 export하여라. 결과는 \coderef{code:article-ctrl}\를 참고하여라.

\subsubsection*{\cd{readArticle}}
\begin{itemize}
    \item GET /article/:articleId(\textbackslash{}d+) → \cd{articleId}에 해당하는 게시물 상세 페이지 응답
    \begin{itemize}
        \item DB에서 \cd{id}가 \cd{articleId}인 게시물 조회 → 없으면 \cd{NOT\_EXIST} 에러 발생
        \item 상세 페이지 응답
    \end{itemize}
    \item 템플릿: views/articles/details.pug
    \begin{itemize}
        \item \cd{user}: 현재 로그인된 사용자 객체
        \item \cd{article}: 페이지에 띄우고자 하는 게시물 객체
    \end{itemize}
\end{itemize}

\subsubsection*{\cd{writeArticleForm}}
\begin{itemize}
    \item GET /article/compose → 게시물 에디터 페이지 응답
    \item 템플릿: views/articles/editor.pug
    \begin{itemize}
        \item \cd{user}: 현재 로그인된 사용자 객체
    \end{itemize}
\end{itemize}

\subsubsection*{\cd{writeArticle}}
\begin{itemize}
    \item POST /article/compose → HTTP 요청의 body의 제목과 내용을 받아 게시물 생성
    \begin{itemize}
        \item \cd{trim} 메서드를 이용하여 제목(\cd{title})과 내용(\cd{content})에서 whitespace 제거
        \item Trim된 제목이나 내용이 빈 문자열이거나 저장 가능한 최대 길이를 초과하면 \cd{BAD\_REQUEST} 에러 발생
        \item DB에 새 게시물을 생성하고, 새 게시물의 상세 페이지로 redirect
    \end{itemize}
    \item Hint: 게시물의 작성자(\cd{author}) 정보가 어디에 저장되어 있는지 생각해본다.
\end{itemize}

\subsubsection*{\cd{editArticleForm}}
\begin{itemize}
    \item GET /article/edit/:articleId(\textbackslash{}d+) → \cd{articleId}에 해당하는 게시물 에디터 페이지 응답
    \begin{itemize}
        \item DB에서 \cd{id}가 \cd{articleId}인 게시물 조회 → 없으면 \cd{NOT\_EXIST} 에러 발생
        \item 수정할 게시물의 제목과 내용이 포함된 게시물 에디터 페이지 응답
    \end{itemize}
    \item 템플릿: views/articles/editor.pug
    \begin{itemize}
        \item \cd{user}: 현재 로그인된 사용자 객체
        \item \cd{article}: 수정할 게시물 객체
    \end{itemize}
\end{itemize}

\subsubsection*{\cd{editArticle}}
\begin{itemize}
    \item POST /article/edit/:articleId(\textbackslash{}d+) → \cd{articleId}에 해당하는 게시물의 제목과 내용을 HTTP 요청의 body의 제목과 내용으로 수정
    \begin{itemize}
        \item \cd{trim} 메서드를 이용하여 제목(\cd{title})과 내용(\cd{content})에서 whitespace 제거
        \item Trim된 제목이나 내용이 빈 문자열이거나 저장 가능한 최대 길이를 초과하면 \cd{BAD\_REQUEST} 에러 발생
        \item DB에서 \cd{id}가 \cd{articleId}이고 요청을 보낸 사용자가 작성한 게시물 조회 → 없으면 \cd{NOT\_EXIST} 에러 발생
        \item DB에 해당 게시물의 변경 사항을 반영한 뒤, 해당 게시물의 상세 페이지로 redirect
    \end{itemize}
\end{itemize}
\clearpage

\subsubsection*{\cd{deleteArticle}}
\begin{itemize}
    \item GET /article/delete/:articleId(\textbackslash{}d+) → \cd{articleId}에 해당하는 게시물을 삭제
    \begin{itemize}
        \item DB에서 \cd{id}가 \cd{articleId}이고 요청을 보낸 사용자가 작성한 게시물 조회 → 없으면 \cd{NOT\_EXIST} 에러 발생
        \item DB에 해당 게시물의 \cd{is\_deleted} 값을 수정하여 게시물 삭제
        \item 게시물 목록의 1페이지로 redirect
    \end{itemize}
\end{itemize}

게시물 작성, 수정, 삭제 기능을 수행할 때 모두 요청한 사용자의 정보가 필요하므로 이러한 기능에 관여하는 5개 라우트는 모두 로그인하지 않은 사용자에게 404 Not Found 상태를 응답하여야 한다. 이 로직은 5개 라우트에 각각 구현하여도 되지만, 미들웨어를 이용하여 이 로직을 구현해보자.

\begin{code}{code:auth-middleware}{Auth Middleware}
\begin{minted}{js}
const authRequired = async (req, res, next) => {
    try {
        if (req.session.user) return next();
        else return res.redirect('/auth/sign_in');
    } catch (err) {
        return next(err);
    }
};

module.exports = { authRequired };
\end{minted}
\end{code}

src/controller/auth/middleware.js를 생성하고, \coderef{code:auth-middleware}\와 같이 작성한다. \cd{authRequired}는 세션 객체에 \cd{user} 객체가 있으면 다음 미들웨어를 실행하고, 없으면 로그인 페이지로 redirect하는 미들웨어이다.

\begin{code}{code:article-routing}{Article Routing}
\begin{minted}{js}
const { Router } = require('express');

const ctrl = require('./ctrl');
const { authRequired } = require('../auth/middleware');

const router = Router();

router.get('/:articleId(\\d+)', ctrl.readArticle);

router.get('/compose', authRequired, ctrl.writeArticleForm);
router.post('/compose', authRequired, ctrl.writeArticle);

router.get('/edit/:articleId(\\d+)', authRequired, ctrl.editArticleForm);
router.post('/edit/:articleId(\\d+)', authRequired, ctrl.editArticle);

router.get('/delete/:articleId(\\d+)', authRequired, ctrl.deleteArticle);

module.exports = router;
\end{minted}
\end{code}
\clearpage

이제 게시물 controller 함수를 라우팅하자. src/controller/article/index.js를 생성하고, \coderef{code:article-routing}\과 같이 작성한다. 여기에서 로그인이 필요한 5개 라우트에 대해 컨트롤러 미들웨어 직전에 \cd{authRequired} 미들웨어가 실행되게 함으로써 로그인하지 않은 사용자로부터 온 요청은 미리 걸러내도록 한다.

src/controller/ctrl.js에 다음 controller를 추가로 구현한 뒤 export하여라. 결과는 \coderef{code:articles-ctrl}\를 참고하여라.

\subsubsection*{\cd{listArticles}}
\begin{itemize}
    \item GET /articles/page/:page(\textbackslash{}d+) → \cd{page}번째 페이지에 나열되어야 하는 게시물 목록 응답
    \begin{itemize}
        \item 요청에서 \cd{page} 값 도출 → 0 이하이면 \cd{BAD\_REQUEST} 에러 발생
        \item 한 페이지에는 최대 10개의 게시물 정보를 나열 → 나열될 게시물들이 전체 게시물의 몇 번째 게시물부터 10개인지 계산
        \item DB에서 나열할 게시물 데이터의 배열과 전체 게시물 데이터의 개수를 조회
        \item 전체 게시물 데이터와 현재 페이지 번호를 이용하여 이전 페이지와 다음 페이지 존재 여부를 계산
        \item 게시물 목록 페이지 응답
    \end{itemize}
    \item 템플릿: views/articles/index.pug
    \begin{itemize}
        \item \cd{user}: 현재 로그인된 사용자 객체
        \item \cd{articles}: 나열할 게시물 객체들의 배열
        \item \cd{page}: 현재 페이지 번호 (정수 형태)
        \item \cd{hasPrev}: 이전 페이지의 존재 여부
        \item \cd{hasNext}: 다음 페이지의 존재 여부
    \end{itemize}
\end{itemize}

\subsubsection*{\cd{latestArticles}}
\begin{itemize}
    \item GET /articles → 게시물 목록 페이지의 1페이지로 redirect
\end{itemize}

마지막으로, 이제 src/controller/index.js를 \coderef{code:index-routing-added}\와 같이 수정하여, 게시물과 관련된 기능을 담당하는 컨트롤러 함수들의 라우팅을 완료한다.

\begin{code}{code:index-routing-added}{Index Routing (Added)}
\begin{minted}[escapeinside=||]{js}
const { Router } = require('express');

const ctrl = require('./ctrl');
|\textbf{\PYG{k+kr}{const} \PYG{n+nx}{article} \PYG{o}{=} \PYG{n+nx}{require}\PYG{p}{(}\PYG{l+s+s1}{\PYGZsq{}./article\PYGZsq{}}\PYG{p}{)}\PYG{p}{;}}|
const auth = require('./auth');

const router = Router();

router.get('/', ctrl.indexPage);
|\textbf{\PYG{n+nx}{router}\PYG{p}{.}\PYG{n+nx}{get}\PYG{p}{(}\PYG{l+s+s1}{\PYGZsq{}/articles/page/:page(\PYGZbs{}\PYGZbs{}d+)\PYGZsq{}}\PYG{p}{,} \PYG{n+nx}{ctrl}\PYG{p}{.}\PYG{n+nx}{listArticles}\PYG{p}{)}\PYG{p}{;}}|
|\textbf{\PYG{n+nx}{router}\PYG{p}{.}\PYG{n+nx}{get}\PYG{p}{(}\PYG{l+s+s1}{\PYGZsq{}/articles\PYGZsq{}}\PYG{p}{,} \PYG{n+nx}{ctrl}\PYG{p}{.}\PYG{n+nx}{latestArticles}\PYG{p}{)}\PYG{p}{;}}|

|\textbf{\PYG{n+nx}{router}\PYG{p}{.}\PYG{n+nx}{use}\PYG{p}{(}\PYG{l+s+s1}{\PYGZsq{}/article\PYGZsq{}}\PYG{p}{,} \PYG{n+nx}{article}\PYG{p}{)}\PYG{p}{;}}|
router.use('/auth', auth);

module.exports = router;
\end{minted}
\end{code}

% const { Router } = require('express');

% const ctrl = require('./ctrl');
% const article = require('./article');
% const auth = require('./auth');

% const router = Router();

% router.get('/', ctrl.indexPage);
% router.get('/articles/page/:page(\\d+)', ctrl.listArticles);
% router.get('/articles', ctrl.latestArticles);

% router.use('/article', article);
% router.use('/auth', auth);

% module.exports = router;

이제 \shellref{shell:npm-run-script}\와 같이 서버를 실행하고, 지금까지 작성한 기능들을 직접 테스트 해보자.
