\section{Sending POST Request From Browser}\label{sect:browser-post-request}

\subsection*{\texttt{form} Tag}

지금까지는 Express.js의 작동 방식을 이해하기 위해 Insomnia를 이용하여 웹 서버에 POST 요청을 보냈다. 그러나 대부분의 사용자는 이러한 프로그램을 사용하지 않고, 인터넷 브라우저에 띄워진 웹 페이지에서 요청을 보내게 되며, 따라서 HTML 문서에서 \cd{body} 데이터와 함께 POST 요청을 보낼 수 있어야 한다.

\cd{form} 태그는 HTML 문서에서 GET과 POST 요청을 보내는 기능을 수행할 수 있는 태그이다. \cd{form} 요소 내에 태그가 \cd{input}이나 \cd{button}이고, \cd{type} 속성의 값이 \cd{submit}인 버튼을 클릭했을 때 \cd{form} 요소의 \cd{method} 속성의 값을 메서드, \cd{action} 속성의 값을 경로로 정하여 HTTP 요청을 보낸다.

이때 \cd{form} 요소의 body는 요소 내부의 모든 입력 요소(\cd{input} 또는 \cd{textarea} 태그)의 데이터를 보내게 되는데, 입력 요소의 \cd{name} 속성의 값이 key, \cd{value} 속성의 값이 value가 된다.

\begin{codeenv}{code:form-tag}{form Tag Example}\begin{verbatim}
<form method="post" action="/login">
    <div>
        <label>Username:</label>
        <input id="username-input" name="username" type="text">
    </div>
    <div>
        <label>Password:</label>
        <input id="password-input" name="password" type="password">
    </div>
    <div>
        <div>Introduce yourself</div>
        <textarea id="introduction-input" name="introduction"></textarea>
    </div>
    <button type="submit">Submit</button>
</form>
\end{verbatim}
\end{codeenv}

\coderef{code:form-tag}\는 \cd{form} 요소를 작성한 예제로, 사용자가 ``Submit'' 버튼을 클릭하면 \cd{\#username-input}, \cd{\#password-input}, \cd{\#introduce-input} 요소의 \cd{value} 값이 각각 HTTP 요청의 body(GET 요청일 경우 query)에 \cd{username}, \cd{password}, \cd{introduction} 속성의 값이 되고, 이 요청이 POST /login으로 보내지게 된다.

여담으로, \cd{form} 태그에서 \cd{method} 속성의 기본값은 \cd{get}, \cd{action} 속성의 기본값은 현재 페이지의 경로이다.
