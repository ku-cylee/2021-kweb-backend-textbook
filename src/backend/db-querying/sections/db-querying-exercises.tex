\section{Database Querying Exercises}\label{sect:db-querying-exercises}

\subsection*{\excstref{exc:simple-crud}: Simple CRUD Functions Exercises}

\shellref{shell:import-sql-file}\을 참고하여 DB에 \coderef{code:crud-table-sql}\을 import하여 테이블을 생성하고 데이터를 삽입하여라. \cd{crud} 테이블에는 100개의 데이터가 있다. 다음 작업들을 수행하는 SQL문을 각각 작성하여라.

\begin{enumerate}
    \item \cd{c1}이 11이고 \cd{c2}가 2인 row의 \cd{id}, \cd{c3}, \cd{c5} column을 조회
    \item \cd{c1}이 18보다 크거나 \cd{c2}가 2보다 작은 row의 모든 column을 조회
    \item \cd{id}와 \cd{c4}는 기본값, \cd{c1}은 7, \cd{c2}는 4, \cd{c3}는 ``col101'', \cd{c5}는 0인 row를 생성
    \item \cd{id}는 103, \cd{c1}은 3, \cd{c2}는 3, \cd{c3}는 ``col103'', \cd{c4}는 기본값, \cd{c5}는 1인 row를 생성
    \item \cd{id}가 100보다 큰 row의 모든 column을 조회
    \item \cd{c1}이 4보다 크고 9보다 작고, \cd{c2}가 1인 row의 \cd{c3}를 ``col0'', \cd{c5}를 0으로 수정
    \item \cd{c1}이 4보다 크고 9보다 작고, \cd{c2}가 1인 row의 모든 column을 조회
    \item \cd{c5}가 0인 row를 삭제
    \item \cd{c5}가 0인 row의 모든 column을 조회
\end{enumerate}

\subsection*{Intercity Trains and Tickets System}

\excref{exc:train-system-sql}\과 \excref{exc:train-system-server}\는 우리나라 간선열차 노선과 예매 현황을 나타내는 데이터베이스를 다루는 문제이다. \shellref{shell:import-sql-file}\을 참고하여 DB에 \coderef{code:train-system-tables}\를 import하여 5개 테이블 \cd{stations}, \cd{types}, \cd{trains}, \cd{users}, \cd{tickets}를 생성하여라. 이후 \coderef{code:train-system-data}\를 import하여 각 테이블의 데이터를 모두 import하여라.

\cd{stations}는 6개의 역, \cd{types}는 5개의 열차 종류, \cd{trains}는 20개의 열차 노선, \cd{users}는 50명의 사용자, \cd{tickets}는 80개의 예매 현황을 나타내는 테이블이다.

각 열차에는 그 열차 종류(\cd{trains.type})의 최대 좌석 수(\cd{types.max\_seats})만큼의 사용자만 예매할 수 있으며, 좌석 번호(\cd{tickets.seat\_number})는 1부터 최대 좌석 수에 해당하는 번호까지 있다. 즉, 최대 10개의 좌석이 있는 열차에는 1번, 2번, \ldots, 10번 좌석이 존재한다.

운임률(\cd{types.fare\_rate})의 단위는 100km당 원(KRW)이며, 노선 길이(\cd{trains.distance})의 단위는 100m이다. 즉, \cd{types.fare\_rate}의 값이 30,000인 열차의 운임률은 1km당 300원이며, \cd{trains.distance}의 값이 200인 노선의 길이는 20km이다. 노선의 운임은 운임률과 노선의 거리를 이용하여 도출한 값을 백의 자리로 반올림하여 결정한다.

\subsection*{\excstref{exc:train-system-sql}: Intercity Trains and Tickets SQL Querying}

다음 작업들을 수행하는 SQL문을 각각 작성하여라.

\begin{enumerate}
    \item ID가 11인 노선을 예매한 모든 승객의 ID(\cd{id}), 이름(\cd{name}), 좌석 번호(\cd{seat\_number})를 좌석 번호의 오름차순으로 조회
    \item 각 사용자의 ID(\cd{id}), 이름(\cd{name}), 탑승 열차 수(\cd{trains\_count}), 총 거리(\cd{total\_distance})를 총 거리의 내림차순으로 상위 6명만 조회
    \item 각 노선의 ID(\cd{id}), 열차 종류(\cd{type}), 출발역(\cd{src\_stn}), 도착역(\cd{dst\_stn}), 여행 시간(\cd{travel\_time})을 여행 시간의 내림차순으로 상위 6개만 조회
    \item 각 노선의 열차 종류(\cd{type}), 출발역(\cd{src\_stn}), 도착역(\cd{dst\_stn}), 출발 시각(\cd{departure}), 도착 시각(\cd{arrival}), 운임(\cd{fare}; 원 단위)을 출발 시각의 오름차순으로 모두 조회
    \item 각 노선의 ID(\cd{id}), 열차 종류(\cd{type}), 출발역(\cd{src\_stn}), 도착역(\cd{dst\_stn}), 예매된 좌석 수(\cd{occupied}), 최대 좌석 수(\cd{maximum})를 노선의 ID의 오름차순으로 모두 조회 (예매한 사용자가 없는 노선은 제외)
    \item 각 노선의 ID(\cd{id}), 열차 종류(\cd{type}), 출발역(\cd{src\_stn}), 도착역(\cd{dst\_stn}), 예매된 좌석 수(\cd{occupied}), 최대 좌석 수(\cd{maximum})를 노선의 ID의 오름차순으로 모두 조회 (예매한 사용자가 없는 노선도 포함)
\end{enumerate}

여행 시간은 출발 시각과 도착 시각의 시간 차이로 계산하며, 두 시각의 시간 차는 \cd{Timediff} 함수를 이용하여 구할 수 있다. \coderef{code:sql-timediff}\는 \cd{former} 값과 \cd{latter} 값의 시간 차 \cd{diff}를 구하는 SQL문이다.

\begin{code}{code:sql-timediff}{\cd{Timediff} Function Example}
\begin{minted}{sql}
SELECT Timediff(`latter`, `former`) AS `diff` FROM `table`;
\end{minted}
\end{code}

각 문제의 SQL문 실행 결과는 \figref{fig:train-system-sql-result}\와 같아야 한다.

\figures{fig:train-system-sql-result}{\excref{exc:train-system-sql} Results}{
    \subfig{fig:train-system-passenger-list}{\excref{exc:train-system-sql}.1}
        {train-system-passenger-list.png}{.2271}
    \subfig{fig:train-system-total-distance}{\excref{exc:train-system-sql}.2}
        {train-system-total-distance.png}{.3041}
    \subfig{fig:train-system-travel-time}{\excref{exc:train-system-sql}.3}
        {train-system-travel-time.png}{.3988}
    \subfig{fig:train-system-time-table}{\excref{exc:train-system-sql}.4}
        {train-system-time-table.png}{.3049}
    \subfig{fig:train-system-seats-table-nonempty}{\excref{exc:train-system-sql}.5}
        {train-system-seats-table-nonempty.png}{.3202}
    \subfig{fig:train-system-seats-table}{\excref{exc:train-system-sql}.6}
        {train-system-seats-table.png}{.3049}
}
\clearpage

\subsection*{\excstref{exc:train-system-server}: Intercity Trains and Tickets Server}

다음 두 라우트를 갖는 웹 서버를 구현하여라. 모든 라우트는 plain text 형태로 응답하고, 인자값에 대한 검증은 하지 않아도 되며, 에러가 발생하면 \cd{console.error} 메서드를 이용하여 에러를 출력하고, 500 Internal Server Error를 응답하여야 한다.

\begin{itemize}
    \item GET /fare: Query로 사용자의 ID인 \cd{uid}를 받아 해당 사용자가 지불해야 하는 총 요금을 응답하는 라우트
    \item GET /train/status: Query로 열차의 ID인 \cd{tid}를 받아 해당 열차가 매진되었는지 판단하여 응답하는 라우트 (Hint: 예매된 좌석 수와 최대 좌석 수를 조회하여 비교한다)
\end{itemize}

\figures{fig:train-system-server-example-result}{\excref{exc:train-system-server} Example Results}{
    \subfig{fig:train-system-fare-uid22}{GET /fare?uid=22}
        {train-system-fare-uid22.png}{.45}
    \subfig{fig:train-system-fare-uid36}{GET /fare?uid=36}
        {train-system-fare-uid36.png}{.45}
    \subfig{fig:train-system-trainstatus-tid5}{GET /train/status?tid=5}
        {train-system-trainstatus-tid5.png}{.45}
    \subfig{fig:train-system-trainstatus-tid16}{GET /train/status?tid=16}
        {train-system-trainstatus-tid16.png}{.45}
}

\figref{fig:train-system-server-example-result}\는 라우트에 따른 서버의 응답 예시이다.
