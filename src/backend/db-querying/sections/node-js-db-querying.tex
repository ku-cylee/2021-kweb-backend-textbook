\section{Querying in Node.js}\label{sect:node-js-db-querying}

이번 장에서는 Node.js에서 DB에 CRUD SQL문을 보내어 데이터를 조작하고 읽는 방법에 대해 학습한다. Production 단계에서는 ORM(Object Relational Mapping)을 사용하여 raw SQL문을 사용하지 않고도 DB에서 데이터를 조작하고 읽는 작업을 수행하는 것이 일반적이며, Node.js에서는 Sequelize, TypeORM 등의 ORM이 있다. 그러나 SQL을 다룰 줄 알아야 ORM도 효과적으로 다룰 수 있으므로, 본 교재에서는 raw SQL문을 이용하여 DB에서 데이터를 조작하고 읽는 방법을 학습한다.

\subsection*{DB Querying in Node.js App}

mysql2 모듈은 Node.js에서 MariaDB(MySQL) DBMS에 접속하고 SQL문을 실행하는 메서드를 제공한다. Node.js 프로젝트를 생성하고 npm을 이용하여 mysql2를 설치한 뒤, database.js를 생성하여 \coderef{code:runquery-function}\과 같이 작성한다.

\begin{code}{code:runquery-function}{Implementation of \cd{runQuery} function}
\begin{minted}{js}
const mysql = require('mysql2/promise');

const pool = mysql.createPool({
    host: <db-host>,
    port: <db-port>,
    user: <db-username>,
    password: <db-password>,
    database: <db-name>,
});

const runQuery = async sql => {
    const conn = await pool.getConnection();
    try {
        const [result] = await conn.query(sql);
        return result;
    } finally {
        conn.release();
    }
};

module.exports = { runQuery };
\end{minted}
\end{code}

\coderef{code:runquery-function}\은 mysql2 모듈의 \cd{createPool} 메서드에 연결하고자 하는 DB의 host, port, 사용자 이름과 비밀번호, DB 이름 등을 인자로 넘겨주어 connection pool을 생성하고, SQL문을 인자로 받아 실행시켜 그 결과를 반환하는 \cd{runQuery} 함수를 구현한 코드이다.

Connection pool은 미리 DB와의 연결들을 생성하여 pool에 저장해둔 후, 필요할 때마다 연결을 가져가서 사용하고  pool에 반납(release)하도록 설계된 연결 방식이다. 이 방식은 DB에서 SQL문을 실행할 때마다 DB 연결에 필요한 여러 정보를 이용해 연결하고 SQL문을 실행한 뒤 연결을 닫는 과정에 비해 높은 성능을 보인다.

\begin{code}{code:runquery-function-usage}{Usage of \cd{runQuery} function}
\begin{minted}{js}
const { runQuery } = require('./database');

const getScoreStats = async () => {
    const sql = 'SELECT course, Count(*) AS cnt, Avg(final) AS avg, ' +
                'Stddev(final) AS stddev FROM scores GROUP BY course';
    const results = await runQuery(sql);
    return results;
};

const getScoreByIdName = async (id, name) => {
    const sql = `SELECT * FROM scores WHERE id = ${id} AND student = '${name}'`;
    const results = await runQuery(sql);
    return results[0];
};

const createScore = async (name, course, midterm, final) => {
    const sql = 'INSERT INTO scores ' +
                `VALUES (DEFAULT, '${name}', '${course}', ${midterm}, ${final})`;
    const result = await runQuery(sql);
    return result;
};

(async () => {
    const stats = await getScoreStats();
    stats.forEach(stat => {
        const { course, cnt, avg, stddev } = stat;
        console.log(`${course} (${cnt} people): Average ${avg}, Std.Dev. ${stddev}`);
    });

    const scoreData = await getScoreByIdName(2, 'Joe');
    const { course, final } = scoreData;
    console.log(`Course: ${course} / Final score: ${final}`);

    console.dir(await getScoreByIdName(9, 'Barack'));
    const newScore = await createScore('Barack', 'Operating Systems', 83, 62);
    console.dir(await getScoreByIdName(9, 'Barack'));
    console.dir(await getScoreByIdName(newScore.insertId, 'Barack'));
})();
\end{minted}
\end{code}

\coderef{code:runquery-function-usage}\는 \cd{runQuery} 함수를 이용하여 \tblref{tab:scores-table}의 \cd{scores} 테이블 데이터를 조회한 결과이다. 조회 결과 데이터는 항상 row들의 배열 형태로 반환되며, 각 row는 column 이름이 key, column 값이 value인 객체의 형태로 반환된다.\footnote{엄밀히는 \cd{TextRow}, \cd{ResultSetHeader} 등 클래스의 인스턴스가 반환된다.}

\cd{getScoreStats} 메서드는 \pageref{code:group-by-example}쪽의 \coderef{code:group-by-example}\을 실행하는 메서드로, 조회 결과는 \cd{course}, \cd{cnt}, \cd{avg}, \cd{stddev}를 속성으로 갖는 객체들의 배열 형태이다.

\cd{getScoreByIdName} 메서드는 \cd{id}와 \cd{name}을 인자로 받아 \cd{id} column의 값이 \cd{id}, \cd{student} column의 값이 \cd{name}인 row를 반환한다. 조회 결과는 언제나 row들의 배열 형태이므로 \cd{results[0]}의 값은 row 객체거나 \cd{undefined}이다.

\cd{createScore} 메서드는 \cd{name}, \cd{course}, \cd{midterm}, \cd{final}을 인자로 받아 \cd{student}, \cd{course}, \cd{midterm}, \cd{final} column의 값이 각각 \cd{name}, \cd{course}, \cd{midterm}, \cd{final}인 row를 새로 생성하는 메서드이다. 이때 \cd{runQuery} 함수는 새로 생성된 row의 \cd{id} 값(\cd{insertId})이나 영향받은 row의 수(\cd{affectedRows}) 등 SQL문 수행 결과를 반환한다.

이렇게 mysql2 모듈을 이용하여 DB의 데이터를 조작하거나 읽을 수 있다. \cd{runQuery} 함수에서 \cd{result}의 값을 \cd{console.dir} 메서드를 이용하여 출력하면서 결과 객체의 형태를 익혀보자.

\subsection*{SQL Injection}

\coderef{code:runquery-function}의 \cd{runQuery} 함수는 DB의 데이터를 조작하는 기능을 잘 수행하지만, 보안상의 문제점이 존재한다.

\begin{code}{code:sql-injection-example-true-condition}{SQL injection example - Force true condition}
\begin{minted}[escapeinside=||]{js}
const scoreData = await getScoreByIdName(3, "abc' OR '1'='1");

|\PYG{c+c1}{\slash\slash (1) SELECT * FROM scores WHERE id = \textbf{\textcolor{blue}{3}} AND student = \PYGZsq{}\textbf{\textcolor{blue}{abc\PYGZsq{} OR \PYGZsq{}1\PYGZsq{}=\PYGZsq{}1}}\PYGZsq{}}|
|\PYG{c+c1}{\slash\slash (2) SELECT * FROM scores WHERE id = 3 AND student = \PYGZsq{}abc\PYGZsq{} \textbf{\textcolor{red}{OR \PYGZsq{}1\PYGZsq{}=\PYGZsq{}1\PYGZsq{}}}}|
// (3) SELECT * FROM scores
\end{minted}
\end{code}

\cd{scores} 테이블에서 성적을 조회하기 위해서는 \cd{id}와 \cd{student} 값을 알아야 하고, \cd{id} 값은 본인 이외의 다른 사람은 모르는 값이라고 가정하자. 즉, \cd{id} 값이 비밀번호인 셈이다. 그렇다면 자신의 성적을 조회하고자 하는 학생은 \coderef{code:runquery-function-usage}에서 \cd{getScoreByIdName} 메서드에 적절한 \cd{id}와 \cd{name} 값을 제시해야만 본인의 성적을 확인할 수 있다.

그런데 만약 어떤 사용자가 \coderef{code:sql-injection-example-true-condition}\과 같이 \cd{name} 값에 ``abc' OR '1'='1'' 이라는 값을 넘긴다고 가정해보자. 이때 SQL문에 각 값을 대입해보면 (1)과 같은 SQL문이 생성되고, 이는 (2)와 같이 \cd{'1'='1'}, 즉 항상 참인 조건식을 포함하게 된다. 참값과의 OR 연산은 항상 참이므로 이는 (3)과 같은 SQL문이 된다. 즉 사용자가 \cd{id}와 \cd{name}을 전혀 알지 못하는 row의 성적을 모두 조회할 수 있는 것이다.

이렇게 변수를 이용하여 SQL문을 생성하는 과정에 의도적으로 조작된 변수를 넘겨 DB를 공격하는 방식을 SQL Injection이라고 한다. SQL Injection은 예제와 같이 공격 난이도가 높지 않으면서도 공격당했을 때 치명적인 문제를 초래할 수 있는 보안 취약점이다.

\begin{code}{code:sql-injection-example-drop-table}{SQL injection example - Drop table}
\begin{minted}[escapeinside=||]{js}
const newScore = await createScore('n', "c'); DROP TABLE scores;", 0, 0);

|\PYG{c+c1}{\slash\slash (1) INSERT INTO scores VALUES (DEFAULT, \PYGZsq{}\textbf{\textcolor{blue}{n}}\PYGZsq{}, \PYGZsq{}\textbf{\textcolor{blue}{c\PYGZsq{}); DROP TABLE scores;}}\PYGZsq{}, \textbf{\textcolor{blue}{0}}, \textbf{\textcolor{blue}{0}});}|
|\PYG{c+c1}{\slash\slash (2) INSERT INTO scores VALUES (DEFAULT, \PYGZsq{}n\PYGZsq{}, \PYGZsq{}c\PYGZsq{}); \textbf{\textcolor{red}{DROP TABLE scores;}}\PYGZsq{}, 0, 0);}|
\end{minted}
\end{code}

database.js의 \cd{createPool} 메서드의 인자 객체에 \cd{multipleStatements: true}를 추가\footnote{실제 서비스에서는 반드시 \cd{false}로 설정해야 한다. (명시하지 않으면 기본값이 \cd{false})}하고 \coderef{code:sql-injection-example-drop-table}\을 실행하면 \cd{scores} 테이블이 \cd{DROP TABLE} 키워드로 인해 삭제된다. 이렇게 SQL Injection이 발생하면 모든 데이터가 삭제되는 불상사가 발생할 수 있다.

SQL Injection은 매우 기본적이고 대중적이면서도 치명적인 문제를 초래할 수 있기 때문에 DB에 접속할 수 있는 기능을 제공하는 대부분의 모듈에는 이러한 공격을 방지하는 기능이 탑재되어 있다. 가장 대표적인 방법은 prepared statement를 이용하는 방법으로, SQL문에서 사용자로부터 값을 받는 변수 부분을 미리 약속된 문자로 표시하여 prepared statement를 생성한다. 모듈이 제공하는 메서드에 이 statement와 사용자로부터 받은 인자를 전달하면 모듈은 입력값들을 검증하며 의도치 않은 작업이 수행되지 않도록 SQL문을 생성한다.

Prepared statement의 변수 부분을 작성하는 방식은 애플리케이션의 언어, 모듈, DBMS마다 조금씩 다르지만 대부분은 물음표(\cd{?}) 문자를 사용하여 변수 부분을 표시한다. mysql2 모듈 역시 prepared statement에 \cd{?} 문자를 사용하여 SQL Injection을 방지할 수 있도록 한다.

\begin{code}{code:runquery-prepared-statement}{\cd{runQuery} function with prepared statement}
\begin{minted}{js}
const runQuery = async (pstmt, data) => {
    const conn = await pool.getConnection();
    try {
        const sql = conn.format(pstmt, data);
        const [result] = await conn.query(sql);
        return result;
    } finally {
        conn.release();
    }
};
\end{minted}
\end{code}

\coderef{code:runquery-prepared-statement}\는 \cd{runQuery} 함수를 수정한 코드로, 인자로 prepared statement(\cd{pstmt})와 \cd{?} 문자 자리에 들어갈 값들이 순서대로 할당된 배열(\cd{data})을 전달받아 SQL문을 생성(\cd{sql})한 뒤 DB에 전달하여 데이터를 받는다.

\begin{code}{code:using-prepared-statement}{Using prepared statement}
\begin{minted}{js}
const getScoreByIdName = async (id, name) => {
    const sql = 'SELECT * FROM scores WHERE id = ? AND student = ?';
    const results = await runQuery(sql, [id, name]);
    return results[0];
};

const createScore = async (name, course, midterm, final) => {
    const sql = 'INSERT INTO scores VALUES (DEFAULT, ?, ?, ?, ?)';
    const result = await runQuery(sql, [name, course, midterm, final]);
    return result;
};
\end{minted}
\end{code}

\cd{runQuery} 함수의 형태가 수정되었으므로 \coderef{code:runquery-function-usage}에서 SQL문에 변수가 들어가는 함수들은 \coderef{code:using-prepared-statement}\와 같이 수정되어야 한다. \coderef{code:sql-injection-example-true-condition}\과 \coderef{code:sql-injection-example-drop-table}\을 실행해보면 SQL Injection이 방어됨을 확인할 수 있으며, \cd{runQuery} 함수에서 \cd{sql} 값을 확인해보면 조작된 변수가 잘 escape된 것을 확인할 수 있다.
