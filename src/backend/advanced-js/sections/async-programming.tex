\section{Asynchronous Programming}\label{sect:async-programming}

\subsection*{Synchronous Programming}

동기식(synchronous) 프로그래밍 모델이란 프로그램의 코드가 순차적(sequentially)으로 실행되도록 설계하는 프로그래밍 모델로, 이렇게 설계 및 구현된 프로그램은 코드가 나열된 순서대로 실행된다. 여러분이 지금까지 구현한 대부분의 프로그램은 동기식으로 구현되었을 가능성이 높다.

이러한 동기식 프로그램은 코드가 순차적으로 실행되기 때문에 실행 흐름을 직관적으로 파악할 수 있다는 장점이 있고, 특히 코드가 수행하는 작업이 메모리에 접근하는 횟수가 많은 작업이거나 병렬 처리가 불가능 또는 불필요한 계산 위주의 작업인 경우 동기식 프로그래밍이 적합할 수 있다.

\begin{code}{code:file-read-sync}{Reading file synchronously}
\begin{minted}{js}
const fs = require('fs');

console.log('String 1');

const data = fs.readFileSync('./file.bin');
console.log(`Data length: ${data.length} bytes`);

console.log('String 2');
\end{minted}
\end{code}

\coderef{code:file-read-sync}\는 file.bin이라는 파일을 처음부터 끝까지 읽어 내용의 길이, 즉 파일의 크기를 표시하는 코드로, \cd{fs.readFileSync} 메서드는 인자로 받은 파일을 읽어 그 내용을 반환하는 메서드이다. 이 프로그램을 실행하면 ``String 1'' 문자열이 출력되고, file.bin 파일의 내용을 읽어들인 후, 파일의 크기가 출력되고 ``String 2'' 문자열이 출력된다.

그런데 file.bin이 크기가 매우 큰 파일이어서 \cd{fs.readFileSync} 메서드로 그 내용을 읽는 시간이 오래 소요된다고 가정해보자. 이때 파일을 읽는 과정과는 무관한 ``String 2'' 문자열은 file.bin 파일을 읽는 동안 출력되지 못한 채 기다려야 한다. 이렇게 동기식으로 설계된 프로그램에서는 프로그램이 특정 작업을 수행하고 있다면 다른 작업들은 그 작업이 끝날 때까지 기다리고 있어야 한다.

동기식으로 설계된 소프트웨어에서는 특정 작업으로 인해 다른 작업의 실행이 막히는 경우는 많이 발생한다. 가장 대표적인 예시는 예제 코드와 같이 입/출력 작업이 수반되는 경우로, 디스크, 네트워크 등 주변 장치와의 통신 과정에서 발생되는 지연으로 인해 코드의 흐름이 막히고, 다른 요청이나 작업을 처리할 수 없는 상황이 발생하곤 한다.

\subsection*{Asynchronous Programming with Callbacks}

비동기(asynchronous) 프로그래밍 모델은 이러한 문제점을 해결할 수 있는 프로그래밍 모델로, 프로그램이 코드의 순서대로 실행되지 않을 수 있는 프로그래밍 모델이다.

\figures{fig:sync-async-code-flow}{Code Flow of Synchronous Code and Asynchronous Code}{
    \subfig{fig:sync-code-flow}{Synchronous Code Flow}
        {sync-code-flow.png}{.772}
    \par\bigskip
    \subfig{fig:async-code-flow}{Asynchronous Code Flow}
        {async-code-flow.png}{.6}
}

\figref{fig:sync-async-code-flow}\는 코드를 동기식으로 구현했을 때와 비동기식으로 구현했을 때 코드의 흐름을 개략적으로 나타낸 도식으로, Task 2는 소요 시간이 Task 1과 Task 3에 비해 매우 긴 작업이다. 동기식 코드(\figref{fig:sync-code-flow})에서 Task 3은 Task 2가 종료되고 나서야 실행될 수 있다. 그러나 비동기식 코드(\figref{fig:async-code-flow})에서는 Task 2는 Task 1이 종료된 직후 실행되고, Task 3은 Task 2의 종료 여부와 상관없이 Task 2가 실행된 직후에 실행되며, Task 3은 Task 2보다 먼저 끝나고 Task 2의 수행 결과는 종료 이후 따로 처리된다.

\begin{code}{code:file-read-async}{Reading file asynchronously}
\begin{minted}[escapeinside=||]{js}
const fs = require('fs');

console.log('String 1');

fs.readFile('./file.bin', |\textbf{\PYG{p}{(}\PYG{n+nx}{err}\PYG{p}{,} \PYG{n+nx}{data}\PYG{p}{)} \PYG{p}{=\PYGZgt{}} \PYG{p}{\PYGZob{}}}|
    |\textbf{\PYG{k}{if} \PYG{p}{(}\PYG{o}{!}\PYG{n+nx}{err}\PYG{p}{)} \PYG{n+nx}{console}\PYG{p}{.}\PYG{n+nx}{log}\PYG{p}{(}\PYG{l+s+sb}{\textasciigrave}\PYG{l+s+sb}{Data length: }\PYG{l+s+si}{\PYGZdl{}\PYGZob{}}\PYG{n+nx}{data}\PYG{p}{.}\PYG{n+nx}{length}\PYG{l+s+si}{\PYGZcb{}}\PYG{l+s+sb}{\textasciigrave}\PYG{p}{)}\PYG{p}{;}}|
    |\textbf{\PYG{k}{else} \PYG{n+nx}{console}\PYG{p}{.}\PYG{n+nx}{error}\PYG{p}{(}\PYG{n+nx}{err}\PYG{p}{)}\PYG{p}{;}}|
|\textbf{\PYG{p}{\PYGZcb{}}}|);

console.log('String 2');
\end{minted}
\end{code}

% const fs = require('fs');

% console.log('String 1');

% fs.readFile('./file.bin', (err, data) => {
%     if (!err) console.log(`Data length: ${data.length}`);
%     else console.error(err);
% });

% console.log('String 2');

\coderef{code:file-read-async}\는 \coderef{code:file-read-sync}\를 비동기식으로 구현한 코드로, file.bin 파일을 읽는 작업을 먼저 실행한 뒤, 다 읽을 때까지 기다리지 않고 다음 코드를 실행하여 코드의 흐름이 갈라지게 된다. \figref{fig:sync-async-code-flow}에서 ``String 1''을 출력하는 작업은 Task 1, 파일을 읽어 크기를 출력하는 작업은 Task 2, ``String 2''를 출력하는 작업은 Task 3에 해당하고, file.bin의 크기가 충분히 클 경우 ``String 1'', ``String 2''가 출력된 후 파일의 크기가 출력되게 된다.

\coderef{code:file-read-async}에서 볼드체로 표시된 부분을 콜백 함수(callback function)라고 한다. 콜백 함수는 메서드의 실행 결과를 인자로 받는 함수로, \cd{fs.readFile} 메서드는 파일 읽기를 실패한 경우 \cd{err} 인자에 에러 객체를 넘겨주고, 성공한 경우 읽은 데이터를 \cd{data} 인자에 넘겨준다. 이렇듯 콜백 함수를 사용하여 비동기적으로 수행된 코드의 결과를 사용할 수 있다.

그러나 콜백을 이용한 비동기 프로그래밍은 여러 가지 문제점이 있는데, 가장 대표적인 문제점은 콜백 지옥(callback hell)이다. 콜백 함수 내에서 또 비동기 함수를 사용하여 콜백 함수를 호출하는 코드가 반복되면, 들여쓰기(indent)가 계속되어 \coderef{code:callback-hell}\과 같이 코드의 깊이가 점점 깊어진다. 적당한 들여쓰기는 코드의 흐름을 파악하기 쉽게 하지만, 지나친 들여쓰기는 코드의 가독성을 현저히 저해시킨다.

\begin{code}{code:callback-hell}{Example of callback hell (Abstractly)}
\begin{minted}{js}
const fs = require('fs');

fs.readFile('db.info', (err, data) => {
    const tableLocation = identifyTable(data, 'A');
    fs.readFile(tableLocation, (err, data) => {
        const recordLocation = identifyRecord(data, 'B');
        fs.readFile(recordLocation, (err, data) => {
            const record = parseRecord(data);
            // do something based on record's data
            console.log(processedData);
        });
    });
});
\end{minted}
\end{code}

\subsection*{async/await}

ES6부터는 비동기 프로그래밍을 보다 간편하게 할 수 있는 async/await 방식이 도입되었다. 이 방식을 이용하면 비동기식 코드를 기존의 동기식 코드에 가까운 형태로 작성할 수 있다.

\begin{code}{code:file-read-async-await}{Reading file using async/await}
\begin{minted}{js}
const fs = require('fs');
const util = require('util');

const readFile = util.promisify(fs.readFile);

const printFileSize = async filename => {
    try {
        const data = await readFile(filename);
        console.log(`Data length: ${data.length} bytes`);
    } catch (err) {
        console.error(err);
    }
};

console.log('String 1');
printFileSize('./file.bin');
console.log('String 2');
\end{minted}
\end{code}

\coderef{code:file-read-async-await}\은 \coderef{code:file-read-async}\를 async/await 방식으로 작성한 코드로, 원래 코드에 비해 동기식 코드와 매우 유사해진 것을 확인할 수 있다. async/await 방식은 몇 가지 특징이 있다.

\begin{itemize}
    \item 비동기 함수는 함수의 앞에 \cd{async} 키워드를 붙여 비동기 함수임을 표시한다.
    \item 비동기 함수를 호출할 때 \cd{await} 키워드를 사용하면 함수가 끝날 때까지 기다리며, 사용하지 않으면 기다리지 않고 다음 코드를 실행한다.
    \item \cd{await} 키워드는 비동기 함수 내에서만 사용 가능하다.
\end{itemize}

여기에서 \cd{util.promisify} 메서드는 일반적인 메서드를 async/await 키워드를 사용할 수 있도록 변형(Promise화\footnote{async/await 방식은 Promise 방식을 통한 비동기 프로그래밍을 간결화한 방식이다. Promise에 관한 자세한 내용은 MDN 문서(https://developer.mozilla.org/en-US/docs/Web/JavaScript/Guide/Using\_promises)에서 확인할 수 있다.})시켜주는 메서드이다.

\coderef{code:file-read-async-await-contd}\는 \coderef{code:file-read-async-await}의 \cd{printFileSize} 함수를 이용한 코드이다. file1.bin과 file2.bin 파일의 크기가 충분히 클 때 출력 순서를 예상할 수 있겠는가?

\begin{code}{code:file-read-async-await-contd}{Reading big file using async/await (Derived from \coderef{code:file-read-async-await})}
\begin{minted}{js}
(async () => {
    console.log('String 1');
    await printFileSize('./file1.bin');
    console.log('String 2');
    printFileSize('./file2.bin');
    console.log('String 3');
})();
\end{minted}
\end{code}

file1.bin 파일의 크기를 출력하는 코드는 \cd{await} 키워드가 있으므로 \cd{printFileSize} 함수의 종료를 기다리고, file2.bin 파일의 크기를 출력하는 코드는 \cd{await} 키워드가 없으므로 함수의 종료를 기다리지 않는다. 따라서 ``String 1'', file1.bin 파일의 크기, ``String 2'', ``String 3'', file2.bin 파일의 크기 순으로 출력된다.

\begin{code}{code:callback-hell-resolved}{Callback hell resolved with async/await}
\begin{minted}{js}
const fs = require('fs');
const readFile = util.promisify(fs.readFile);

(async () => {
    try {
        const dbInfo = await readFile('db.info');
        const tableInfo = await readFile(identityTable(dbInfo, 'A'));
        const rawRecord = await readFile(identifyRecord(tableInfo));
        const record = parseRecord(data);
        // do something based on record's data
        console.log(processedData);
    } catch (err) {
        console.error(err);
    }
})();
\end{minted}
\end{code}

async/await 방식을 사용하면 callback hell 역시 해결된다. \coderef{code:callback-hell-resolved}\는 \coderef{code:callback-hell}\을 async/await 방식으로 작성한 코드로, 콜백 함수의 반복적인 호출로 발생하는 callback hell이 해결되어 가독성이 높아졌다.

이렇듯 코드를 간결하게 바꾸어주는 async/await 방식도 단점이 존재한다. async/await 방식은 자체적인 에러 처리 방식이 없으므로 \coderef{code:file-read-async-await}\과 같이 try-catch문을 사용하여야 한다. try-catch문을 사용한 에러 처리는 가독성 측면에서는 깔끔하지만 한 칸의 indent가 강제되고, JS의 try-catch문은 속도가 느려 성능 저하가 발생할 수 있다는 단점이 있다.
