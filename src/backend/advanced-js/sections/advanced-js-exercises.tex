\section{Advanced JavaScript Exercises}\label{sect:advanced-js-exercises}

\subsection*{\excstref{exc:cases-module}: Number of Cases Calculating Module}

주어진 식을 참고하여 \cd{n}과 \cd{r}를 인자로 받아 순열(permutation), 조합(combination), 중복 순열(multiset permutation), 중복 조합(multiset combination) 값을 반환하는 메서드 \cd{permutation}, \cd{combination}, \cd{multiPermutation}, \cd{multiCombination}을 갖는 모듈 cases를 구현하여라.

$$ \mathrm{Permutation}:~{}_n\mathrm{P}_r = \frac{n!}{(n-r)!}~~/~~\mathrm{Multiset~Permutation}:~n^r $$
$$ \mathrm{Combination}:~\binom{n}{r} = \frac{n!}{(n-r)!~r!}~~/~~\mathrm{Multiset~Combination}:~\left(\kern-.3em\binom{n}{r}\kern-.3em\right) = \binom{n+r-1}{r} $$

\begin{code}{code:cases-module-example}{\excref{exc:cases-module} Example}
\begin{minted}{js}
const cases = require('./cases');

const n = 8;
const r = 3;

console.log(`n = ${n}, r = ${r}`)
console.log(`Permutation: ${cases.permutation(n, r)}`);
console.log(`Combination: ${cases.combination(n, r)}`);
console.log(`Multi Permutation: ${cases.multiPermutation(n, r)}`);
console.log(`Multi Combination: ${cases.multiCombination(n, r)}`);
\end{minted}
\end{code}

\coderef{code:cases-module-example}\은 cases 모듈을 이용하여 작성한 예시 코드이다. 이 코드를 실행하였을 때 결과는 \shellref{shell:cases-module-example-result}\와 같아야 한다.

\begin{shell}{shell:cases-module-example-result}{\excref{exc:cases-module} Example result}
\begin{minted}[linenos=false, xleftmargin=2pt]{text}
$ node index.2.1.js
n = 8, r = 3
Permutation: 336
Combination: 56
Multi Permutation: 512
Multi Combination: 120
\end{minted}
\end{shell}
\clearpage

\subsection*{\excstref{exc:directory-search}: Directory Search}

프로젝트 내의 test 디렉토리 내부를 모두 탐색하여 확장자가 js인 파일의 목록을 출력하고, 탐색 과정에서 에러가 발생하면 \cd{console.error} 메서드를 이용하여 에러를 출력하는 프로그램을 내장 모듈인 fs와 path 모듈의 다음 메서드들을 활용하여 구현하여라. 두 모듈의 API 문서\footnote{16.17.0 버전의 경우 https://nodejs.org/dist/latest-v16.x/docs/api}는 Node.js 홈페이지에서 확인할 수 있으며, 필요에 따라 \coderef{code:file-read-async-await}\과 같이 \cd{util.promisify} 메서드를 사용해야 한다.

\begin{itemize}
    \item fs (File System): \cd{readdir}, \cd{stat}, \cd{stats.isDirectory}
    \item path: \cd{extname}, \cd{join}
\end{itemize}

\begin{shell}{shell:directory-search-example}{\excref{exc:directory-search} Example (On Linux)}
\begin{minted}[linenos=false, xleftmargin=2pt]{text}
$ tree test
test
├── 01.js
├── 01.py
├── 02.js
├── test1
│   ├── 11.c
│   ├── 11.js
│   ├── 11.py
│   └── 12.js
├── test2
│   ├── 21.js
│   └── 21.ts
├── test3
│   ├── 31.cs
│   └── 31.java
├── test4
│   ├── 41.js
│   └── 41.py
└── test5
    └── test51
        ├── 511.js
        └── 511.py
\end{minted}
\end{shell}

예를 들어 test 디렉토리의 내부 구조가 \shellref{shell:directory-search-example}\과 같을 때, 프로그램의 실행 결과는 \shellref{shell:directory-search-example-result}\와 같아야 한다.

\begin{shell}{shell:directory-search-example-result}{\excref{exc:directory-search} Example result}
\begin{minted}[linenos=false, xleftmargin=2pt]{text}
$ node index.2.2.js
test\01.js
test\02.js
test\test1\11.js
test\test1\12.js
test\test2\21.js
test\test4\41.js
test\test5\test51\511.js
\end{minted}
\end{shell}
