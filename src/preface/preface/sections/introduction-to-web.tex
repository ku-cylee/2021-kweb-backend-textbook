\section{Introduction to WEB}\label{sect:introduction-to-web}

\subsection*{WEB}

웹(WEB)이란, 인터넷에 연결된 컴퓨터들을 통해 사람들이 정보를 공유할 수 있는 전 세계적인 정보 공간이다. 웹을 통해 공유되는 정보들은 대개 웹 페이지(web page)의 형태로 공유되며, 웹 페이지는 특수한 양식을 갖춘 텍스트로 구성된다. 각 웹 페이지는 일반적인 텍스트(plain text)뿐만 아니라 이미지, 동영상, 다른 웹 페이지로 연결되는 하이퍼링크(hyperlink) 등의 웹 자원(web resource)으로 다양하게 구성되고, 하나의 주제, 하나의 영역을 공유하는 여러 웹 자원과 웹 페이지는 웹 사이트(website)를 구성한다. 웹 사이트는 웹 서버(web server)라는 디바이스에서 동작하는 웹 애플리케이션(web application)이라고 하는 프로그램의 형태로 구현되고 작동하며, 웹 애플리케이션을 통해 특수한 제한이 없다면 전 세계 어디서든 웹 사이트에 접속할 수 있다.

본 교재에서 다루는 웹의 영역은 흔히 웹 개발자가 다루는 웹의 영역과 밀접하게 관련이 있는 웹 애플리케이션과 관련된 영역이다. 앞으로 여러분은 이 교재를 통해 웹 애플리케이션의 작동 원리와 구조를 이해하고, 웹을 디자인하고 설계하는 학습을 할 것이다.

\subsection*{Frontend and Backend}

우리가 일상에서 사용하는 웹 사이트, 웹 애플리케이션이 어떻게 작동할 지 상상해보자. 예를 들어, 사용자가 블로그 형태의 웹 사이트에 접속했을 때 사용자에게 보여지는 영역에는 어떤 것이 있는가? 블로그의 블로거, 블로거에 대한 상세 정보, 게시물, 게시물에 대한 상세 정보, 댓글, 댓글에 대한 상세 정보 등이 있을 것이다. 뿐만 아니라 블로그를 예쁘게 꾸민 디자인 등도 사용자에게 보여지는 영역이다. 이렇듯 사용자에게 보여지는 영역을 \textbf{frontend}라고 부르며, 사용자(client) 쪽에서 동작하는 영역이라고 하여 \textbf{client-side}라고도 한다.

반대로, 사용자에게 직접적으로 보여지지 않는 영역도 있다. 사용자의 요청에 맞게 적절한 웹 페이지를 구성하여 전달해주는 로직, 게시물이나 댓글, 블로거의 정보를 저장하고 읽어오는 기능, 회원에 따라 글을 작성할 권한을 부여할지, 댓글을 수정할 권한을 부여할지 결정하는 로직 등이 있을 것이다. 이렇게 사용자에게 직접적으로 보여지지 않는 영역을 \textbf{backend}라고 부르고, 이러한 로직은 서버상에서 동작하기 때문에 \textbf{server-side}라고도 한다.

\figures{fig:webapp-schema}{How typical web application works}{
    \fig{webapp-schema.png}{.6}
}

\begin{fullstack}
    \partref{part:frontend}
\end{fullstack}
\begin{frontend}
    본 교재
\end{frontend}
\begin{notbackend}
    에서는 frontend에서 기본적으로 사용되는 HTML, CSS, Javascript를 학습
\end{notbackend}
\begin{fullstack}
    하며, 
\end{fullstack}
\begin{frontend}
    한다.
\end{frontend}
\begin{fullstack}
    \partref{part:backend}
\end{fullstack}
\begin{backend}
    본 교재
\end{backend}
\begin{notfrontend}
    에서는 backend 분야에서 필요한 기초 개념들과 사용되는 각종 기술 스택들을 다루며, Node.js에서 작동하는 Express.js 프레임워크와 데이터베이스를 이용하여 간단한 웹 애플리케이션을 제작하는 실습을 진행한다.
\end{notfrontend}
