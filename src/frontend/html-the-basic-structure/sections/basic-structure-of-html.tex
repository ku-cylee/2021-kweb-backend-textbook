\section{Basic Structure of HTML}\label{sect:basic-structure-of-html}

\subsection*{What is HTML?}

HTML(HyperText Markup Language)은 웹 페이지의 구조를 체계적으로 표현하는 컴퓨터 언어이다. HTML은 웹 페이지의 가장 기본적인 뼈대를 표현하고, 웹 브라우저가 웹 페이지를 화면에 표시하는 과정인 렌더링(rendering)은 이러한 HTML 문서를 기반으로 이루어진다.

``HTML is not a programming language''와 같은 밈(meme)에서 보듯, HTML은 구조를 서술하기 위한 언어이며 C, Java, Python 등에 존재하는 조건문, 반복문, 변수 선언 등과 관련된 기능이 전혀 존재하지 않는다. 미리 정해진 규칙에 따라 HTML을 작성하여 나타내고자 하는 \textbf{구조를 표현}한다는 개념으로 접근하면 HTML이라는 언어를 다소 쉽게 이해할 수 있을 것이다.

\subsection*{Tags and Elements}
HTML은 그 원말에서 보듯, 구조를 서술할 때 정해진 마크로 시작하여 마크로 끝나는 언어인 마크업 언어이다. \coderef{code:html-example}\은 웹 페이지 https://www.example.com/의 HTML 문서의 일부로, 이를 관찰하며 HTML의 구조를 이해해보자. 

\begin{codeenv}{code:html-example}{Example of HTML}\begin{verbatim}
<!doctype html>
<html>
<head>
    <title>Example Domain</title>
    <meta charset="utf-8">
    <meta http-equiv="Content-type" content="text/html; charset=utf-8">
    <meta name="viewport" content="width=device-width, initial-scale=1">
</head>
<body>
    <div>
        <h1>Example Domain</h1>
        <p>This domain is for use in illustrative examples in documents. You may use this
        domain in literature without prior coordination or asking for permission.</p>
        <p><a href="https://www.iana.org/domains/example">More information...</a></p>
    </div>
</body>
</html>
\end{verbatim}
\end{codeenv}

Line 4를 관찰해보면, ``Example Domain''이라는 텍스트를 \cd{<title>}과 \cd{</title>}이 감싸고 있다. 이때 \cd{<title>}과 \cd{</title>}을 \textbf{태그(tag)}라고 하며, \cd{title}을 \textbf{태그 이름(tag name)}이라고 한다. 두 태그 중 앞쪽에 위치하는 태그를 시작 태그 또는 여는 태그라고 하고 뒤쪽에 위치하는 태그를 끝 태그 또는 닫는 태그라고 하며, 두 태그의 이름은 반드시 같아야 한다. 한 쌍의 태그에 둘러싸인 부분을 \textbf{내용(content)}이라고 하며, 두 태그와 그 사이에 위치한 내용이 하나의 HTML \textbf{요소(element)}를 구성한다. 요소는 HTML에서 구조를 표현하는 가장 기본 단위이며, 요소 간의 관계를 통해 구조가 표현된다.

Line 4에서 각 용어에 대응하는 부분은 다음과 같다.

\begin{itemize}
    \item 태그 이름: \cd{title}
    \item 시작 태그: \cd{<title>}
    \item 끝 태그: \cd{</title>}
    \item 태그: \cd{<title>}, \cd{</title>}
    \item 내용: \cd{Example Domain}
    \item 요소: \cd{<title>Example Domain</title>}
\end{itemize}

다만, Line 5-7에서 보듯 내용이 없고 시작 태그 그 자체로만 이루어진 요소도 존재한다. 이렇게 내용이 필요하지 않은 태그를 단일 태그라고 하며, 단일 태그에는 \cd{br}, \cd{hr}, \cd{img}, \cd{input}, \cd{meta} 등이 존재한다. 과거에는 단일 태그에도 태그의 끝을 알리기 위해 \cd{<meta charset="utf-8" />}와 같이 별도의 슬래시(\cd{/})를 삽입하기도 하였으나, 최근에는 생략하는 것이 권장된다.

Line 3-8에서는 하나의 \cd{title} 요소와 세 개의 \cd{meta} 요소가 \cd{head} 태그의 내용을 구성하는 것을 볼 수 있다. 이렇게 HTML 요소는 내부에 다른 HTML 요소를 포함할 수 있고, 다른 HTML 요소의 내용 부분에 위치함으로써 다른 HTML 요소에 포함될 수 있다. HTML에서는 요소 A와 요소 B가 요소 C에 포함되어 있을 때, 요소 A와 B를 요소 C의 하위 요소(sub element) 혹은 \textbf{자식 요소(child element)}라고 하고, 요소 C를 요소 A와 요소 B의 상위 요소(super element) 혹은 \textbf{부모 요소(parent element)}라고 한다. 또한, 요소 A를 요소 B의 \textbf{형제 요소(sibling element)}라고 한다. 이러한 포함 관계는 중첩(nested)될 수 있어, 모든 요소는 하나의 tree 구조를 이룬다.

즉, Line 3-8에서 \cd{head} 요소는 \cd{title} 요소와 세 \cd{meta} 요소의 부모 요소이며, \cd{title} 요소와 세 \cd{meta} 요소는 각각 \cd{head} 요소의 자식 요소이다. 또한, \cd{title} 요소와 세 \cd{meta} 요소는 서로 형제 요소이다.

Line 5-7의 \cd{meta} 요소에서는 태그 내에 태그 이름 외의 다른 텍스트가 존재하는데, 이는 각 HTML 요소의 세부적인 성질을 표현하는 문자열이다. 예를 들어, Line 5의 \cd{meta} 요소는 \cd{charset}의 값이 \cd{utf-8}, Line 6의 \cd{meta} 요소는 \cd{http-equiv}의 값이 \cd{Content-type}, \cd{content}의 값이 \cd{text/html; charset=utf-8}이라는 뜻이다. \footnote{이렇게 데이터의 정의나 이름과 데이터의 값으로 이루어진 데이터 집합을 \textbf{키/값 쌍(key-value pair)}이라고 하며, 데이터의 정의를 키(key), 데이터의 값을 값(value)라고 한다.}\footnote{예를 들어, 어떠한 학생의 이름이 홍길동, 학번이 2021990999, 성별이 남성, 거주지는 서울특별시라고 가정해보자. 이 학생의 인적 정보를 저장한 데이터 집합에서 ``이름'', ``학번'', ``성별'', ``거주지'' 등은 key가 되고, ``홍길동'', ``2021990999'', ``남성'', ``서울특별시 성북구'' 등은 각 key에 대한 value가 된다. 대부분의 데이터는 이러한 key-value pair 형태로 표현될 수 있다.} 이렇듯 HTML의 속성은 key-value pair 형태의 데이터 집합이며, 이때 key에 해당하는 부분을 \textbf{속성(attribute)}, value에 해당하는 부분을 속성값이라고 한다. 속성값은 기본적으로 큰따옴표(\cd{"}) 내에 작성되며, 속성과 속성값은 등호(\cd{=})로 연결되고, 각 key-value pair는 공백을 사이에 두고 나열된다.

이제까지 살펴본 HTML 요소의 구조를 종합해보면 \figref{fig:html-elmt-struct}\와 같이 표현될 수 있다.

\figures{fig:html-elmt-struct}{Structure of HTML Element}
    {\fig{html-elmt-struct.png}{.7}}

HTML에서는 예약된(reserved) 태그 이름과 속성, 그리고 속성값이 존재하며, 이들 중 자주 사용되는 것들은 \sectref{sect:commonly-used-html-tags}에서 다룰 예정이다.

\subsection*{Basic Structure of HTML}
HTML 문서의 구조에는 가장 기본적인 틀이 정해져 있다. 먼저 \cd{doctype}을 이용하여 문서가 HTML 문서임을 명시히고, 가장 상위 태그로 \cd{html} 태그, 그 아래에 \cd{head}와 \cd{body} 태그가 위치해야 한다. 코드로 나타낸다면 \coderef{code:html-basic-struct}\와 같다.\footnote{HTML에서는 주석을 \cd{<!--}, \cd{-->}로 표현한다.}

\begin{codeenv}{code:html-basic-struct}{Basic Structure of HTML}
\begin{verbatim}
<!doctype html>
<html>
    <head>
        <!-- Head Element Content -->
    </head>

    <body>
        <!-- Body Element Content -->
    </body>
</html>
\end{verbatim}
\end{codeenv}

\cd{head} 태그는 문서에 대한 전반적인 정보를 담고 있는 태그로, 이 태그에는 \cd{title}, \cd{meta}, \cd{link} 등 다양한 태그들이 사용된다. 이번 절에서는 \cd{title}과 \cd{meta} 두 태그만 살펴보자. 먼저, \cd{title} 태그 내부의 텍스트는 웹 페이지의 제목으로, 웹 브라우저를 통해 웹 페이지에 접속했을 때 브라우저의 상단에 표시된다. \cd{meta} 태그는 HTML 문서의 인코딩 방법, viewport, 키워드 등 중요한 정보들을 지정할 수 있는 태그이다.

\cd{body} 태그는 웹 페이지에서 사용자에게 보여질 부분이 포함되는 태그이다. 개발자가 웹 페이지를 통해 사용자에게 보여주고자 하는 텍스트, 이미지 등의 내용은 \cd{body} 태그 안에 작성되어야 하며, \sectref{sect:commonly-used-html-tags}에서 본격적으로 웹 페이지의 내용을 작성하는 방법을 다룬다.
