\section{HTML Exercises}\label{sect:html-exercises}

\subsection*{\excstref{exc:simple-survey-page}: Simple Survey Page}

KWEB 동아리 설문 조사에 쓰일 HTML 문서를 \figref{fig:kweb-survey-page}\와 같이 작성하여라. KWEB 로고 이미지와 홈페이지 주소는 아래와 같으며, HTML의 표준 구조를 준수하고, 각종 태그를 사용해서 구현한다. 제시되지 않았더라도 각 요소에 적당한 속성과 속성값을 적절히 부여하고, \cd{div}, \cd{span} 태그와 class, id 등을 적절히 활용하여 HTML 문서를 구조화하고, 가독성을 높여라.

\begin{itemize}
    \item KWEB 로고: https://i.imgur.com/hSSFVSw.png
    \item KWEB 페이스북 페이지: https://kwebofficial.com/
\end{itemize}

\figures{fig:kweb-survey-page}{KWEB survey page example}
    {\fig{kweb-survey-page.png}{.6}}

\subsection*{\excstref{exc:structurizing-html-code}: Structurizing HTML Code}

\sectref{sect:basic-structure-of-html}에서 학습한 HTML의 기본 구조, \sectref{sect:commonly-used-html-tags}에서 학습한 \cd{div}, \cd{span} 태그와 \sectref{sect:class-and-id-attributes}에서 학습한 class, id를 이용하여, 앞의 \coderef{code:input-tags}\를 HTML 표준에 맞게 수정하고, 자유롭게 구조화해보자. (정해진 정답은 없다)
