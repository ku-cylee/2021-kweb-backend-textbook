\section{CSS Exercises}\label{sect:css-exercises}

\subsection*{\excstref{exc:text-styling}: Text Styling}
\coderef{code:text-styling-html-code}\가 \figref{fig:text-styling-example}\과 같이 나타나도록 text-styling.css를 작성하여라.

\figures{fig:text-styling-example}{\excref{exc:text-styling} example}
    {\fig{text-styling-example.png}{.7}}

\subsection*{\excstref{exc:ars-buttons}: ARS Buttons}

\coderef{code:ars-buttons-html-code}\가 \figref{fig:ars-buttons-example}\과 같이 나타나도록 ars-buttons.css를 작성하여라.

\figures{fig:ars-buttons-example}{\excref{exc:ars-buttons} example}
    {\fig{ars-buttons-example.png}{.8}}

\subsection*{\excstref{exc:scroll-button}: Scroll Button}

\coderef{code:scroll-button-html-code}\를 참고하여, \figref{fig:scroll-button-example}\과 같이 우측 하단에 고정되어 있으면서, 클릭하였을 때 웹 페이지의 제일 상단과 제일 하단으로 이동하는 버튼을 \cd{\#scroll-button} 내부를 수정하고, scroll-button.css를 작성하여 구현하여라. \cd{a} 태그의 \cd{href} 속성값을 HTML 요소의 id로 설정하면 해당 위치로 이동할 수 있으며, 위쪽 삼각형(▲)과 아래쪽 삼각형(▼)의 개체 번호는 각각 9650, 9660이다.

\figures{fig:scroll-button-example}{\excref{exc:scroll-button} example}
    {\fig{scroll-button-example.png}{.7}}

\subsection*{\excstref{exc:responsive-color-page}: Responsive Color Page}
\coderef{code:responsive-color-page-html-code}\를 참고하여 responsive-color-page.css를 작성하여라. \cd{body} 태그의 배경색이 600px 미만일 때는 \cd{skyblue}, 600px 이상 1200px 미만일 때는 \cd{blue}, 1200px 이상일 때는 \cd{darkblue}이어야 하며, \cd{h1} 태그의 색은 900px 미만일 때는 \cd{white}, 900px 이상일 때는 \cd{yellow}이어야 한다.

\subsection*{\excstref{exc:responsive-layout}: Responsive Layout}
\coderef{code:responsive-layout-html-code}\를 참고하여 웹 페이지의 너비가 800px 이상이면 \figref{fig:responsive-layout-wide}, 800px 미만이면 \figref{fig:responsive-layout-narrow}\와 같이 렌더링 되도록 responsive-layout.css 파일을 작성하여라.

\figures{fig:responsive-layout-view}{Webpage view of \excref{exc:responsive-layout}}{
    \subfig{fig:responsive-layout-wide}{Wider than 800px}
        {responsive-layout-wide.png}{.6378}
    \subfig{fig:responsive-layout-narrow}{Narrower than 800px}
        {responsive-layout-narrow.png}{.2122}
}
