\section{JavaScript with Front-end}\label{sect:javascript-with-front-end}

이번 장에서는 \chapref{ch:js-basics}에서 학습한 JS의 기본 문법을 이용하여 HTML 문서의 동적 기능의 구현을 학습한다.

\subsection*{Application of JS on HTML}

Front-end에서 JS는 HTML 문서를 제어하고 수정하기 위해 존재하는 언어이므로 CSS와 마찬가지로 HTML 문서에 연동되어야 한다. JS 코드를 HTML 문서에 대해 실행하는 방법은 두 가지가 있는데, 첫 번째 방법은 HTML 문서에서 \cd{script} 태그를 만들어 내부에 JS 코드를 작성하는 것이다. \coderef{code:js-internal}\을 웹 브라우저에서 열면 개발자 도구의 Console 탭에서 \cd{Hello World!} 문구가 출력된 것을 확인할 수 있다.

\begin{codeenv}{code:js-internal}{Internal application of JS}\begin{verbatim}
<script>
    console.log('Hello World!');
</script>

<h1>HTML and JS</h1>
\end{verbatim}
\end{codeenv}

두 번째 방법은 별도의 JS 파일을 만들어, JS 파일에는 JS 스크립트만 작성하고, HTML 파일에는 웹페이지의 구조만 작성하는 방법이다. HTML 파일에는 \coderef{code:js-external-html}\과 같이 \cd{script} 태그의 \cd{src} 속성에 JS 파일의 주소를 작성한다.

\begin{codeenv}{code:js-external-html}{External application of JS - HTML}\begin{verbatim}
<script type="text/javascript" src="./script.js"></script>

<h1>HTML and JS</h1>
\end{verbatim}
\end{codeenv}

script.js 파일을 \coderef{code:js-external-js}\와 같이 작성한다.

\begin{codeenv}{code:js-external-js}{External application of JS - JS}\begin{verbatim}
console.log('Hello World!');
\end{verbatim}
\end{codeenv}

이렇게 HTML과 JS 파일을 작성하고 HTML 파일을 열면 \coderef{code:js-internal}\과 동일한 결과를 확인할 수 있다. 다만, 웹 브라우저는 HTML 문서의 앞부분부터 로딩하므로 JS 파일이 연동된 \cd{script} 태그 뒤에 작성된 HTML 요소를 JS 코드가 인식하지 못할 수 있다는 점을 주의하여야 한다. 이를 방지하기 위해 \cd{script} 태그를 \cd{head} 태그가 아닌 \cd{body} 태그 내부의 중간이나 끝에 작성하기도 하는데, 이 방식은 HTML의 가독성을 떨어뜨린다.

\cd{script} 태그에 \cd{async}나 \cd{defer} 속성을 활성화하면 이러한 문제를 해결할 수 있다. \cd{async} 속성을 속성값 없이 명시하면 해당 JS 코드는 비동기적으로 실행되고, \cd{defer} 속성을 속성값 없이 명시하면 해당 JS 코드는 문서 로딩이 완전히 끝난 뒤에 실행된다.
