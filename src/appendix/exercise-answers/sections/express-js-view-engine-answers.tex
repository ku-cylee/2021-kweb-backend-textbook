\section{Express.js and View Engine Exercise Answers}\label{sect:express-js-view-engine-answers}

\subsection*{\excref{exc:express-query-body}}

\begin{code}{code:express-query-body-answer}{\excref{exc:express-query-body} Answer}
\begin{minted}{js}
const express = require('express');

const app = express();
const port = 3000;

app.use(express.urlencoded({ extended: true }));

const objectToString = obj =>
    Object.keys(obj).map(k => `${k}: ${obj[k]}`).join('\n');

app.get('/', (req, res) => res.send(objectToString(req.query)));

app.post('/', (req, res) => res.send(objectToString(req.body)));

app.put('/', (req, res) => res.send(objectToString(req.body)));

app.delete('/', (req, res) => res.send(objectToString(req.body)));

app.listen(port, () => console.log(`Server listening on port ${port}!`));
\end{minted}
\end{code}

\subsection*{\excref{exc:express-params}}

\begin{code}{code:express-params-answer}{\excref{exc:express-params} Answer}
\begin{minted}{js}
const express = require('express');

const app = express();
const port = 3000;

app.get('/board/page/:page', (req, res) => {
    const { page } = req.params;
    res.send(`This is page #${page}`);
})

app.listen(port, () => console.log(`Server listening on port ${port}!`));
\end{minted}
\end{code}
\clearpage

\subsection*{\excref{exc:express-factorial}}

\begin{code}{code:express-factorial-answer}{\excref{exc:express-factorial} Answer}
\begin{minted}{js}
const express = require('express');

const app = express();
const port = 3000;

app.get('/factorial', (req, res) => {
    const { number } = req.query;
    return res.redirect(`/factorial/${number}`);
});

app.get('/factorial/:number', (req, res) => {
    const { number } = req.params;
    const parsedNumber = parseInt(number, 10);
    let result = 1;
    for (let i = 0; i < parsedNumber; i++) result *= (i + 1);
    return res.send(`${parsedNumber}! = ${result}`);
});

app.listen(port, () => console.log(`Server listening on port ${port}!`));
\end{minted}
\end{code}

\subsection*{\excref{exc:form-login}}

\begin{code}{code:form-login-answer-index-js}{\excref{exc:form-login} Answer (index.js)}
\begin{minted}{js}
const express = require('express');

const app = express();
const port = 3000;

app.set('views', `${__dirname}/views`);
app.set('view engine', 'pug');

app.use(express.urlencoded({ extended: true }));

app.get('/', (req, res) => res.render('login.pug'));

app.post('/login', (req, res) => {
    const { username, password } = req.body;
    return res.send(`Username: ${username} / Password: ${password}`);
});

app.listen(port, () => console.log(`Server listening on port ${port}!`));
\end{minted}
\end{code}

\begin{code}{code:form-login-answer-login-pug}{\excref{exc:form-login} Answer (views/login.pug)}
\begin{minted}[escapeinside=||]{pug}
doctype html
html
    head
        title Login Page
    body
    |\PYG{n+nt}{form}(\PYG{n+na}{method=}\PYG{l+s}{\PYGZsq{}post\PYGZsq{}}\PYG{n+nt}{, action}=\PYG{l+s}{\PYGZsq{}/login\PYGZsq{}}\PYG{n+nt}{)}|
            div
                label Username:
                |\PYG{n+nt}{input}\PYG{n+nf}{\PYGZsh{}username\PYGZhy{}input}(\PYG{n+na}{name=}\PYG{l+s}{\PYGZsq{}username\PYGZsq{}}\PYG{n+nt}{, type}=\PYG{l+s}{\PYGZsq{}text\PYGZsq{}}\PYG{n+nt}{)}|
            div
                label Password:
                |\PYG{n+nt}{input}\PYG{n+nf}{\PYGZsh{}password\PYGZhy{}input}(\PYG{n+na}{name=}\PYG{l+s}{\PYGZsq{}password\PYGZsq{}}\PYG{n+nt}{, type}=\PYG{l+s}{\PYGZsq{}password\PYGZsq{}}\PYG{n+nt}{)}|
            div
                div Introduce yourself
                |\PYG{n+nt}{textarea}\PYG{n+nf}{\PYGZsh{}introduction\PYGZhy{}input}(\PYG{n+na}{name=}\PYG{l+s}{\PYGZsq{}introduction\PYGZsq{}}\PYG{n+nt}{)}|
            |\PYG{n+nt}{button}(\PYG{n+na}{type=}\PYG{l+s}{\PYGZsq{}submit\PYGZsq{}}\PYG{n+nt}{) }\PYG{n+nt}{Submit}|
\end{minted}
\end{code}

% doctype html
% html
%     head
%         title Login Page
%     body
%         form(method='post', action='/login')
%             div
%                 label Username:
%                 input#username-input(name='username', type='text')
%             div
%                 label Password:
%                 input#password-input(name='password', type='password')
%             div
%                 div Introduce yourself
%                 textarea#introduction-input(name='introduction')
%             button(type='submit') Submit
