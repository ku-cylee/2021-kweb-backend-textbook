\section{HTML Exercise Answers}\label{sect:html-exercise-answers}

\subsection*{\excref{exc:simple-survey-page}}

\begin{codeenv}{code:simple-survey-page-answer}{\excref{exc:simple-survey-page} Answer}\begin{verbatim}
<!doctype html>
<html>
<head>
    <title>KWEB Survey Page</title>
</head>
<body>
    <div id="survey-header">
        <h2>KWEB 설문조사</h2>
        <div id="kweb-logo">
            <img src="http://info.korea.ac.kr/_res/info/img/community/img_kweb.gif"
                 alt="KWEB Logo" width=250>
        </div>
        <a href="https://www.facebook.com/kwebfamily/">KWEB 페이스북 페이지</a>
    </div>
    <div id="survey-body">
        <div class="survey-question">
            <h3>1. 이름을 작성해주세요.</h3>
            <input type="text" name="name">
        </div>
        <div class="survey-question">
            <h3>2. 학번을 작성해주세요.</h3>
            <input type="text" name="studentNumber">
        </div>
        <div class="survey-question">
            <h3>3. 현재 회원 등급을 선택해주세요.</h3>
            <input type="radio" name="level" value="intern">준회원
            <input type="radio" name="level" value="qualified">정회원
            <input type="radio" name="level" value="break">휴회원
        </div>
        <div class="survey-question">
            <h3>4. 본인이 관심있는 분야를 모두 선택해주세요.</h3>
            <input type="checkbox" name="field" value="frontend">프론트엔드 프로그래밍
            <input type="checkbox" name="field" value="backend">백엔드 프로그래밍
            <input type="checkbox" name="field" value="crawling">웹 크롤링
            <input type="checkbox" name="field" value="security">웹 보안
            <input type="checkbox" name="field" value="etc">기타
            <input type="text" name="fieldText">
        </div>
        <div class="survey-question">
            <h3>
                5. 2020년 봄학기
                KWEB 스터디에 대한 전반적인 평가와 피드백을 작성해주세요.
            </h3>
            <textarea name="feedbacks" cols="100" rows="20"></textarea>
        </div>
    </div>
    <div id="survey-footer">
        <button type="submit">제출</button>
    </div>
</body>
</html>
\end{verbatim}
\end{codeenv}
