\section{Selectors} \label{sect:selectors}

\sectref{sect:basic-structure-of-css}에서 CSS의 구조에 관해 다루면서 선택자(selector)를 소개하였습니다. CSS에서는 선택자에 따라 원하는 HTML 요소에 원하는 스타일을 지정할 수 있습니다. 이번 절에서는 선택자를 작성하는 방법에 대해 학습해봅시다. 

\subsection*{Universal Selector}

전체 선택자(universal selector)는 HTML 문서의 모든 요소를 선택합니다. 전체 선택자는 \verb|*|로 표현됩니다. 

\begin{codeenv}{code:universal-selector}{Universal Selector}\begin{verbatim}


* { width: 80% }
\end{verbatim}
\end{codeenv}

\subsection*{Tag, Class, Id Selector}
태그 이름, 클래스 이름, 아이디를 기준으로 요소를 선택하는 선택자입니다. 태그 이름은 태그 이름 그대로, 클래스 이름은 \verb|.class-name|의 형태로, 아이디는 \verb|#id|의 형태로 작성합니다. 또한, 태그 이름, 클래스 이름, 아이디 등으로 표현된 선택자 element로 선택되는 요소 중 클래스 이름이 \verb|class-name|인 요소를 선택하고자 하는 경우 \verb|element.class-name|로 선택자를 작성할 수 있습니다.

\begin{codeenv}{code:tag-class-id-selector}{Examples of Tag, Class, Id Selector}\begin{verbatim}


ul { list-style: none }
.title { font-size: 20px }
#article-list { padding: 0 }
.title.recent-article { font-weight: bold }
\end{verbatim}
\end{codeenv}

\subsection*{Child Selector and Descendants Selector}
자식 선택자(child selector)와 자손 선택자(descendants selector)는 두 개 이상의 선택자를 이용하여 요소를 선택하는 선택자입니다. 자식 선택자는 \verb|parent > child|의 형태로 쓰며, \verb|parent| 선택자로 선택된 각 요소의 바로 밑에 있는 자식 요소 중 \verb|child| 선택자를 만족하는 요소를 선택합니다. 반면, 자손 선택자는 \verb|parent child|의 형태로 쓰며, \verb|parent| 선택자로 선택된 각 요소의 밑에 있는 모든 자식 요소 중 \verb|child| 선택자를 만족하는 요소를 

\begin{codeenv}{code:child-desc-selector}{Understanding Child Selector and Descendants Selector}\begin{verbatim}


<div class="class1">
    <span id="span1"></span>
</div>
<div class="class2">
    <span id="span2"></span>
</div>
<div class="class1">
    <div>
        <span id="span3"></span>
    </div>
</div>
\end{verbatim}
\end{codeenv}

\coderef{code:child-desc-selector}을 참고하여 정확하게 이해해봅시다. 먼저 자식 선택자인 \verb|.class1 > span|은 \verb|#span1|만 선택합니다. 그러나 자손 선택자인 \verb|.class1 span|은 \verb|#span1과 #span3|를 선택합니다. 

\subsection*{Pseudo-class Selector}
가상 클래스 선택자(pseudo-class selector)는 특정한 상태에 놓여있는 요소들을 선택하는 선택자입니다. 예를 들어 \verb|button| 태그로 구현한 버튼에 일반적인 상태, 호버(hover)된 상태, 비활성화된 상태 등에 대해 각기 다른 스타일을 적용할 수 있습니다. 이러한 가상 클래스 선택자는 \verb|:pseudo-class|의 형태로 쓰고, \verb|element:pseudo-class|로 표현된 선택자는 \verb|element| 선택자로 선택된 요소 중 \verb|pseudo-class|에 해당하는 요소들을 선택합니다. \verb|a| 태그는 \verb|:link|, \verb|:visited|, \verb|:hover|, \verb|:active| 등의 가상 클래스를 가질 수 있습니다. 

\begin{codeenv}{code:pseudo-class-selector-1}{Example of Pseudo-class Selector (1)}\begin{verbatim}


<style>
    button { color: black }
    button:hover { color: red }
    button:disabled { background-color: yellow }
</style>

<button>Enabled Button</button>
<button disabled>Disabled Button</button>
\end{verbatim}
\end{codeenv}

가상 클래스에는 HTML 요소의 상태뿐만 아니라 구조와 관련된 가상 클래스도 있습니다.

\begin{itemize}
    \item \verb|:first-child| – 선택된 요소 중 가장 첫 번째 요소를 선택합니다. 
    \item \verb|:last-child| – 선택된 요소 중 가장 마지막 요소를 선택합니다. 
    \item \verb|:nth-child(e(n))| – $n$으로 표현된 식 $e(n)$에 대해, 선택된 요소 중 $e(0)$, $e(1)$, $e(2)$, $\cdots$번째 요소를 모두 선택합니다. \verb|e(n)|은 $an+b$의 꼴만 가능하며, \verb|e(n)| 대신 \verb|odd|나 \verb|even|을 사용할 수 있습니다.
    \item \verb|:nth-last-child(e(n))| – \verb|nth-child|와 유사하게 동작하나, 뒤에서부터 선택합니다. 
\end{itemize}

이 외에도, \verb|:not(selector)|은 \verb|selector|에 의해 선택되지 않은 요소들만 선택합니다.

\begin{codeenv}{code:pseudo-class-selector-2}{Example of Pseudo-class Selector (2)}\begin{verbatim}


<style>
    li:first-child { color: red }
    li:last-child { color: blue }
    li:nth-child(4n+3) { color: yellow }
    li:nth-last-child(4n+3) { color: green }
    li:not(#current) { font-style: italic }
</style>

<ul>
    <li>01</li>
    <li>02</li>
    <li>03</li>
    <li>04</li>
    <li>05</li>
    <li id="current">06</li>
    <li>07</li>
    <li>08</li>
    <li>09</li>
    <li>10</li>
</ul>
\end{verbatim}
\end{codeenv}

가상 클래스의 종류는 매우 많으므로, 자주 쓰이는 몇 가지를 제외하고는 필요할 때마다 찾아서 사용하면 됩니다. 더 많은 가상 클래스는 아래의 MDN reference에서 확인할 수 있습니다. 

\begin{itemize}
    \item https://developer.mozilla.org/en-US/docs/Web/CSS/Pseudo-classes
\end{itemize}

지금까지 다양한 선택자에 대하여 알아보았습니다. 이 외에도 속성 선택자, 동위 선택자, 가상 요소 선택자 등 다양한 선택자가 있습니다. 자료에서 언급되지 않은 선택자들은 제시된 링크들에서 확인할 수 있습니다. 

\begin{itemize}
    \item https://www.w3schools.com/cssref/css\_selectors.asp
    \item https://developer.mozilla.org/en-US/docs/Web/CSS/CSS\_Selectors
    \item https://css-tricks.com/almanac/selectors
\end{itemize}
