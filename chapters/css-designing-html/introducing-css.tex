\section{Introducing CSS} \label{sect:introducing-css}

CSS는 Cascading Style Sheet의 약자로, HTML로 작성된 문서가 실제로 웹 브라우저에 어떻게 표현될지 명시해주는 컴퓨터 언어입니다. CSS를 이용하여 HTML 문서의 지정된 요소에 의도하고자 하는 디자인을 적용할 수 있습니다. CSS는 정적 디자인을 하는 데 주로 사용되며, 각 요소의 상태나 웹 페이지가 표시되는 화면의 크기에 따라 동적으로 디자인할 수도 있습니다. 

\subsection*{CSS 등장 배경}
과거 CSS가 존재하지 않았을 때는 HTML 문서에 웹 페이지의 구조뿐만 아니라 디자인 요소까지 작성하였습니다. \verb|li| 태그는 정보를 저장하는 태그에 지나지 않았고, 텍스트에 스타일을 저장하기 위해 \verb|font|, \verb|b| 등의 태그를 사용하여 스타일을 저장하곤 했습니다. 그러나 스타일과 레이아웃에 대한 정보를 훨씬 많이 저장하게 되면서 HTML 문서는 원래의 목적인 구조를 저장하는 문서에서 디자인까지 서술하는 문서가 되었습니다. 그로 인해 HTML은 인간이 읽기에도 그 구조를 파악하기 힘들고, 웹 브라우저가 사용자에게 웹 페이지를 띄워주기 위해 분석하는 작업조차 힘든, 비효율적인 언어가 되었습니다. 

이렇게 HTML이 가지는 비효율적인 면을 개선하기 위해 1996년 CSS가 발표되었고, HTML과 CSS를 분리하면서 HTML에는 가급적 문서에 대한 구조만 서술하고, CSS에는 각 요소에 대한 스타일이나 레이아웃만을 서술하도록 권고되었습니다. CSS의 도입으로 HTML은 본연의 목적을 되찾아 문서의 구조를 표현하는 효율적인 언어가 되었으며, 더 나아가 웹 브라우저가 여러 웹 페이지에서 공통으로 사용되는 CSS 문서를 서버로부터 이중, 삼중으로 다운로드할 필요가 없어져 웹 페이지 로딩에도 성능 향상을 가져왔습니다.

\begin{codeenv}{code:early-html}{Example of Early HTML}\begin{verbatim}


<body>
    <li><font color="red">HTML before CSS existence.</font></li>
    <b>This is a bold text. </b>
    <i>This text is italicized.</i>
</body>
\end{verbatim}
\end{codeenv}
