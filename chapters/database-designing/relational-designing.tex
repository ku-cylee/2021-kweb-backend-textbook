\section{Relational Designing}\label{sect:relational-designing}

\subsection*{Relation Between Tables}

\sectref{sect:basics-of-designing}에서 테이블을 설계하는 가장 기본적인 방법에 대해 스터디하였습니다. 정말 간단한 형태의 데이터를 DB에 저장하고자 할 때는 앞에서 스터디한 내용만으로도 충분히 데이터를 저장할 수 있습니다. 그러나 실제 세상은 그렇게 간단하지 않고, 여러 테이블 간에 관계(relation)가 존재합니다.

예를 들어, 전공과목에 관한 정보를 담은 테이블을 가정해봅시다. 이 테이블에서 전공과목 각각의 데이터에는 어느 학부에 속한 과목인지에 관한 정보가 포함되어 있어야 합니다. 만약 각 학부에 관한 추가적인 정보가 없다면 전공과목 테이블에 학부 이름 column을 추가하면 되지만, 각 학부에 관한 다른 정보가 저장된 테이블이 따로 존재한다면 앞에서 설계한 방식대로 데이터를 온전히 나타낼 수 없습니다.

위와 같이 여러 테이블 간에 관계(relation)가 존재하는 경우에 대한 DB 설계는 매우 중요하며, 실제 웹 애플리케이션에서도 서로 관계있는 테이블을 굉장히 많이 사용합니다. 이번 장에서는 서로 관계가 있는 테이블이 있는 DB를 다루는 데에 강점을 가진 관계형 데이터베이스 모델을 이용하여 설계하는 스터디를 진행할 것입니다.

\subsection*{Primary Key and Foreign Key}

테이블 간 관계가 존재하는 데이터베이스를 디자인해봅시다. 앞에서 예시로 제시된 학부, 전공과목과 관련된 테이블을 각각 \cd{departments}, \cd{courses}라고 합시다. \cd{departments} 테이블을 다음과 같이 나타낼 수 있습니다.

\begin{tblenv}
    {tab:table-example-departments}
    {Example of \cd{departments} Table}
    {>{\colc}m{0.15\tw}|>{\colc}m{0.15\tw}|>{\colc}m{0.15\tw}}
    \thickhline
    \rowcolor{tblheadcolor}
    \cd{kor\_name} & \cd{eng\_name} & \cd{code}\tabularnewline
    \hline
    \rowcolor{tblheadcolor}
    \cd{VC(8)} & \cd{VC(32)} & \cd{TINYINT}\tabularnewline
    \hline
    공과대학 & Engineering & 17\tabularnewline
    \hline
    정보대학 & Informatics & 32\tabularnewline
    \hline
    이과대학 & Science & 16\tabularnewline
    \thickhline
\end{tblenv}

이제 전공과목과 관련된 테이블 \cd{courses}를 설계해봅시다. 앞의 \sectref{sect:basics-of-designing}에서 \tblref{tab:table-example-categories}의 테이블을 설계한 것과 같이 courses 테이블에 department column을 추가하여 각 전공과목이 어떤 학부에 소속된 과목인지 나타낼 수 있습니다. 이렇게 설계한 테이블은 \tblref{tab:table-example-courses}\와 같이 나타낼 수 있습니다.

\begin{tblenv}
    {tab:table-example-courses}
    {Example of \cd{courses} Table}
    {>{\colc}m{0.18\tw}|>{\colc}m{0.12\tw}|>{\colc}m{0.15\tw}|>{\colc}m{0.12\tw}|>{\colc}m{0.12\tw}}
    \thickhline
    \rowcolor{tblheadcolor}
    \cd{name} & \cd{code} & \cd{department} & \cd{credit} & \cd{is\_necessary}\tabularnewline
    \hline
    \rowcolor{tblheadcolor}
    \cd{VC(16)} & \cd{VC(8)} & \cd{VC(8)} & \cd{TINYINT} & \cd{TINYINT}\tabularnewline
    \hline
    컴퓨터프로그래밍II & COSE102 & 정보대학 & 3 & 1\tabularnewline
    \hline
    자료구조 & COSE213 & 정보대학 & 3 & 1\tabularnewline
    \hline
    기계학습 & COSE362 & 정보대학 & 3 & 0\tabularnewline
    \hline
    전자회로 & GEEE301 & 공과대학 & 3 & 1\tabularnewline
    \hline
    무기화학II & CHEM308 & 이과대학 & 3 & 1\tabularnewline
    \thickhline
\end{tblenv}

\tblref{tab:table-example-courses}의 설계는 \cd{courses.department} column이 \cd{department.kor\_name}을 참조한 것입니다. 이와 같은 설계는 전공과목 테이블에서 전공과목이 소속된 학부의 이름뿐만 아니라 다른 정보도 조회할 수 있다는 장점이 있습니다. 예를 들어, 자료구조 과목이 소속된 학부의 학과 코드를 알고 싶다면, 먼저 자료구조 row의 \cd{department} 값을 확인하고(``정보대학''), \cd{department} 테이블에서 \cd{kor\_name}의 값이 ``정보대학''인 row를 찾아서 \cd{code} column의 값(32)을 찾으면 됩니다.

이와 같은 설계는 언뜻 문제가 없어 보이지만, 실제로는 몇 가지 중요한 문제점을 가지고 있습니다. 먼저, \cd{courses} 테이블에서 \cd{department} 테이블의 \cd{kor\_name}을 참조하는데, \cd{kor\_name} column의 값이 같은 row가 두 개 이상 있을 수 있습니다. \tblref{tab:table-example-departments}\와 \tblref{tab:table-example-courses}\는 고려대학교의 학부와 전공과목만 나타낸 테이블이지만, 이를 확장하여 다른 학교의 학부와 전공과목을 저장한다고 가정해봅시다. \cd{department} 테이블에 ``연세대학교 공과대학''이 저장된다면, \cd{kor\_name}의 값이 ``공과대학''인 row는 두 개입니다. 이 경우 \cd{courses} 테이블에서 전자회로 과목이 속한 학부의 과목 코드를 확인하고자 할 때 고려대학교 공과대학을 참조할지, 연세대학교 공과대학을 참조할지 알 수 없습니다.

두 번째 문제는, 참조되는 값이 바뀔 수 있다는 점입니다. 예를 들어, \cd{departments} 테이블에서 ``정보대학''의 이름이 ``정보통신대학''으로 바뀌었다고 가정하면, 기존에 \cd{courses} 테이블에서 ``정보대학''을 참조하던 ``컴퓨터프로그래밍II'', ``자료구조'', ``기계학습'' 과목들의 \cd{department} 값이 모두 ``정보통신대학''으로 바뀌어야 합니다.

위와 같은 문제점을 해결하기 위해 도입된 개념이 기본 키(primary key)입니다. 기본 키는 다음 두 조건을 반드시 만족해야 합니다.

\begin{itemize}
    \item 테이블 내에서 기본 키의 값이 동일한 서로 다른 row가 존재하지 않는다. 즉, 모든 기본 키는 해당 테이블 내에서 유일(unique)하다.
    \item 기본 키의 각 값은 수정되지 않아야 하며, \cd{NULL} 값이 할당되어서도 안 된다.
\end{itemize}

기본 키는 한 테이블 내에서 하나만 설정할 수 있고, 한 column의 값으로 설정될 수도, 여러 column의 값의 조합으로 설정될 수도 있습니다. 관계형 데이터베이스에서는 참조되는 테이블이 아니더라도 모든 테이블에 기본 키를 설정할 것을 권장합니다.

일반적으로 기본 키는 1부터 시작하여 1씩 증가하는 정수로 설정합니다. 첫 번째 row의 기본 키 값은 1, 두 번째 row의 기본 키 값은 2 등으로 증가하도록 설정하는 것이 일반적입니다.

\begin{tblenv}
    {tab:table-example-departments-primary}
    {Example of \cd{departments} Table with Primary Key}
    {>{\colc}m{0.08\tw}|>{\colc}m{0.15\tw}|>{\colc}m{0.15\tw}|>{\colc}m{0.15\tw}}
    \thickhline
    \rowcolor{tblheadcolor}
    \cd{id} & \cd{kor\_name} & \cd{eng\_name} & \cd{code}\tabularnewline
    \hline
    \rowcolor{tblheadcolor}
    \cd{INT} & \cd{VC(8)} & \cd{VC(32)} & \cd{TINYINT}\tabularnewline
    \hline
    1 & 공과대학 & Engineering & 17\tabularnewline
    \hline
    2 & 정보대학 & Informatics & 32\tabularnewline
    \hline
    3 & 이과대학 & Science & 16\tabularnewline
    \thickhline
\end{tblenv}

\tblref{tab:table-example-departments-primary}\는 \tblref{tab:table-example-departments}에 \cd{id} column을 추가하고, 해당 column에 기본 키 설정을 한 테이블입니다.

\begin{codeenv}{code:create-table-primary-key}{Create Table with Primary Key}\begin{verbatim}
CREATE TABLE departments (
    id INT NOT NULL AUTO_INCREMENT,
    kor_name VARCHAR(8) NOT NULL,
    eng_name VARCHAR(32) NOT NULL,
    code TINYINT NOT NULL,
    PRIMARY KEY (id)
) ENGINE=InnoDB DEFAULT CHARSET=utf8;
\end{verbatim}
\end{codeenv}

\tblref{code:create-table-primary-key}의 테이블을 \coderef{code:create-table-primary-key}\와 같이 DB에 생성할 수 있습니다.

이제 \cd{courses} 테이블을 수정하여, 각 row가 \cd{departments} 테이블의 \cd{id}를 참조하도록 설정해주어야 합니다. 이때 사용되는 개념이 외래 키(foreign key)입니다. 외래 키는 다른 테이블의 기본 키를 가리키는 키이고, 그러므로 외래 키의 자료형은 참조하는 키의 자료형과 동일해야 합니다. 이 예제의 경우 \cd{courses.department} column이 \cd{departments.id}를 참조하는 외래 키입니다.

\begin{tblenv}
    {tab:table-example-courses-primary-foreign}
    {Example of \cd{courses} Table with Primary Key and Foreign Key}
    {>{\colc}m{0.08\tw}|>{\colc}m{0.18\tw}|>{\colc}m{0.12\tw}|>{\colc}m{0.15\tw}|>{\colc}m{0.12\tw}|>{\colc}m{0.12\tw}}
    \thickhline
    \rowcolor{tblheadcolor}
    \cd{id} & \cd{name} & \cd{code} & \cd{department} & \cd{credit} & \cd{is\_necessary}\tabularnewline
    \hline
    \rowcolor{tblheadcolor}
    \cd{INT} & \cd{VC(16)} & \cd{VC(8)} & \cd{INT} & \cd{TINYINT} & \cd{TINYINT}\tabularnewline
    \hline
    1 & 컴퓨터프로그래밍II & COSE102 & 2 & 3 & 1\tabularnewline
    \hline
    2 & 자료구조 & COSE213 & 2 & 3 & 1\tabularnewline
    \hline
    3 & 기계학습 & COSE362 & 2 & 3 & 0\tabularnewline
    \hline
    4 & 전자회로 & GEEE301 & 1 & 3 & 1\tabularnewline
    \hline
    5 & 무기화학II & CHEM308 & 3 & 3 & 1\tabularnewline
    \thickhline
\end{tblenv}

\tblref{tab:table-example-courses-primary-foreign}\은 \tblref{tab:table-example-courses}에 \cd{id} column을 추가하고, \cd{department} column이 \cd{departments.id}를 참조하도록 외래 키 설정을 한 테이블입니다.

만약 다른 테이블로부터 참조를 받는 row가 삭제되면 어떻게 될까요? 예를 들어 \cd{departments} 테이블에서 3번 ``이과대학'' row가 삭제된다면, \cd{courses} 테이블에서 이를 참조하는 5번 ``무기화학II'' 과목은 참조할 데이터가 사라지고, 이로 인해 문제가 생길 수 있습니다.

이렇게 참조하는 column이 삭제되었을 때의 action을 \cd{ON DELETE} 구문을 이용하여 설정할 수 있습니다. 가장 많이 쓰이는 cascade는 참조하는 column이 삭제되면 같이 삭제되는 동작이고, 이 외에도 set null, no action, set default, restrict 등의 동작이 존재합니다.

\begin{codeenv}{code:create-table-foreign-key}{Create Table with Foreign Key}\begin{verbatim}
CREATE TABLE courses (
    id INT NOT NULL AUTO_INCREMENT,
    name VARCHAR(16) NOT NULL,
    code VARCHAR(8) NOT NULL,
    department INT NOT NULL,
    credit TINYINT NOT NULL,
    is_necessary TINYINT NOT NULL DEFAULT 1,
    PRIMARY KEY (id),
    FOREIGN KEY (department)
    REFERENCES departments(id) ON DELETE CASCADE
) ENGINE=InnoDB DEFAULT CHARSET=utf8;
\end{verbatim}
\end{codeenv}

\coderef{code:create-table-foreign-key}\는 \tblref{tab:table-example-courses-primary-foreign}의 테이블을 DB에 생성한 예제입니다. 여담으로, \cd{DEFAULT} 키워드를 이용하여 column의 기본 값을 설정할 수 있습니다.
