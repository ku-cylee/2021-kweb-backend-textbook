\section{Relational Designing}\label{sect:relational-designing}

\subsection*{Relation Between Tables}

\sectref{sect:basics-of-designing}에서는 \cd{courses} 테이블을 설계하면서 테이블을 설계하는 가장 기본적인 방법을 배웠다. 정말 간단한 형태의 데이터를 DB에 저장할ㄸ때는 기본적인 방법으로도 충분히 데이터를 저장할 수 있으나, 실제 세상은 여러 테이블 간의 관계(relation)가 존재하여 훨씬 복잡하다.

\cd{courses} 테이블(\tblref{tab:courses-table-subdivisions})의 \cd{department} column에는 학과 강의가 속해있는 학과 이름이 저장되어 있는데, \cd{courses} 테이블과 같이 학과 이름만 저장할 때는 이러한 방법으로도 충분하지만 각 학과에 대한 정보를 저장하고자 할때는 이러한 방식으로 온전히 데이터를 저장할 수 없고, 별도의 테이블에 저장하여야 한다.

위와 같이 테이블 간 관계(relation)가 존재하는 경우에 대한 DB 설계는 매우 중요하며, 실제 서비스에서도 서로 관계가 있는 테이블이 매우 많이 사용된다. 이번 장에서는 서로 관계가 있는 테이블을 다루는 데에 강점을 가진 관계형 데이터베이스 모델을 이용하여 앞의 \cd{courses} 테이블과 관계가 있는 \cd{departments} 테이블을 설계해본다.

\subsection*{Primary Key and Foreign Key}

먼저 학과와 관련된 \cd{departments} 테이블을 \sectref{sect:basics-of-designing}\과 같은 방법으로 설계하면 \tblref{tab:departments-table-basic}\과 같이 나타낼 수 있다.

\begin{tblenv}
    {tab:departments-table-basic}
    {\cd{departments} Table}
    {?>{\colc}m{0.12\tw}|>{\colc}m{0.3\tw}|>{\colc}m{0.12\tw}?}
    \thickhline
    \rowcolor{tblheadcolor}
    \cd{kor\_name} & \cd{eng\_name} & \cd{college}\tabularnewline
    \hline
    \rowcolor{tblheadcolor}
    \cd{VC(16)} & \cd{VC(50)} & \cd{VC(16)}\tabularnewline
    \hline
    컴퓨터학과 & Computer Science and Engineering & 정보대학\tabularnewline
    \hline
    화학과 & Chemistry & 이과대학\tabularnewline
    \hline
    물리학과 & Physics & 이과대학\tabularnewline
    \hline
    언어학과 & Linguistics & 문과대학\tabularnewline
    \thickhline
\end{tblenv}

이제 \cd{courses} 테이블과 \cd{departments} 테이블을 비교해보자. 예를 들어 ``운영체제'' 강의가 속한 학과의 영문 이름을 조회하는 경우 \cd{courses} 테이블에서 \cd{name} 값이 ``운영체제''인 row의 \cd{department} 값을 찾아 ``컴퓨터학과''임을 알아낸 뒤, \cd{departments} 테이블에서 \cd{kor\_name}의 값이 ``컴퓨터학과''인 row의 \cd{eng\_name} 값을 확인하면 된다. 이때 \cd{courses.department} column이 \cd{department.kor\_name} column을 참조(reference)한다고 한다. 이렇게 설계하면 강의 이름을 통해 그 강의가 속한 단과대 이름 등을 다양하게 조회할 수 있다는 장점이 있다.

다만, 이러한 설계는 몇 가지 중요한 문제점을 가지고 있다. 먼저 \cd{courses} 테이블에서 \cd{department} 테이블의 \cd{kor\_name} column을 참조하는데, \cd{kor\_name} column 값이 같은 row가 존재할 수 있다. 예를 들어, \tblref{tab:departments-table-basic}에 다른 학교에 소속된 화학과 학과를 추가한다고 가정하면 ``무기화학II'' 과목이 속한 학과의 단과대 이름을 조회하려고 할때 우리 학교의 화학과를 참조하는지, 다른 학교의 화학과를 참고하는지 알 수 없다.

또한 참조되는 값이 바뀔 수 있다는 문제도 있다. 예를 들어 \cd{departments} 테이블에서 ``화학과''의 이름이 ``화학공학과''로 바뀐다면 \cd{courses} 테이블에서 ``화학과''를 참조하던 ``유기화학실험''과 ``무기화학II'' row의 \cd{department} 값이 모두 ``화학공학과''로 바뀌어야 한다.

위와 같은 문제점을 해결하기 위해 기본 키(primary key)를 사용하는데, 기본 키는 다음 두 조건을 반드시 만족해야 한다.

\begin{itemize}
    \item 테이블 내에서 기본 키의 값이 동일한 서로 다른 row가 존재하지 않는다. 즉, 모든 기본 키는 해당 테이블 내에서 유일(unique)하다.
    \item 기본 키의 각 값은 수정되지 않아야 하며, NULL 값이어서도 안 된다.
\end{itemize}

기본 키는 한 테이블 내에서 하나만 설정할 수 있고, 한 column의 값으로 설정될 수도 있고, 여러 column 값의 조합으로 설정될 수도 있다. 일반적으로 기본 키는 1부터 시작하여 1씩 증가하는 정수로 설정하며, 첫 번째 row의 기본 키 값은 1, 두 번째 row의 기본 키 값은 2 등으로 증가하도록 설정하는 것이 일반적이다. 관계형 데이터베이스에서는 참조되는 테이블이 아니더라도 기본 키를 설정할 것을 권장한다. 

\begin{tblenv}
    {tab:departments-table-pk}
    {\cd{departments} Table with Primary Key}
    {?>{\colc}m{0.05\tw}|>{\colc}m{0.12\tw}|>{\colc}m{0.3\tw}|>{\colc}m{0.12\tw}?}
    \thickhline
    \rowcolor{tblheadcolor}
    \cd{id} & \cd{kor\_name} & \cd{eng\_name} & \cd{college}\tabularnewline
    \hline
    \rowcolor{tblheadcolor}
    \cd{INT} & \cd{VC(16)} & \cd{VC(50)} & \cd{VC(16)}\tabularnewline
    \hline
    1 & 컴퓨터학과 & Computer Science and Engineering & 정보대학\tabularnewline
    \hline
    2 & 화학과 & Chemistry & 이과대학\tabularnewline
    \hline
    3 & 물리학과 & Physics & 이과대학\tabularnewline
    \hline
    4 & 언어학과 & Linguistics & 문과대학\tabularnewline
    \thickhline
\end{tblenv}

\tblref{tab:departments-table-pk}\는 \tblref{tab:departments-table-basic}에 기본 키에 해당하는 \cd{id} column을 추가한 테이블이다.

\begin{codeenv}{code:create-table-primary-key}{Create \cd{departments} Table with Primary Key}\begin{verbatim}
CREATE TABLE `departments` (
    `id` INT NOT NULL AUTO_INCREMENT,
    `kor_name` VARCHAR(8) NOT NULL,
    `eng_name` VARCHAR(32) NOT NULL,
    `college` VARCHAR(16) NOT NULL,
    PRIMARY KEY (`id`)
) ENGINE=InnoDB DEFAULT CHARSET=utf8;
\end{verbatim}
\end{codeenv}

\coderef{code:create-table-primary-key}\는 \tblref{tab:departments-table-pk}의 테이블을 SQL문을 이용하여 DB에 생성한 코드이다. \cd{AUTO\_INCREMENT} 키워드를 이용하여 데이터를 추가할 때 \cd{id} 값을 명시하지 않아도 자동으로 직전 row의 \cd{id} 값보다 1이 큰 값을 할당하도록 하고, \cd{PRIMARY KEY} 키워드를 이용하여 \cd{id} column이 기본 키임을 설정할 수 있다.

이제 \cd{courses} 테이블을 수정하여 각 row가 \cd{departments} 테이블의 \cd{id}를 참조할 수 있도록 해주자. 이렇게 다른 테이블의 기본 키를 가리키는 키를 외래 키(foreign key)라고 하고, 외래 키의 자료형은 참조하는 키의 자료형과 동일해야 한다. 이 예제에서는 \cd{courses.department} column이 \cd{departments.id} column을 참조하는 외래 키이다.

\begin{tblenv}
    {tab:courses-table-pk-fk}
    {\cd{courses} Table with Primary Key and Foreign Key}
    {?>{\colc}m{0.05\tw}|>{\colc}m{0.16\tw}|>{\colc}m{0.11\tw}|>{\colc}m{0.1\tw}|>{\colc}m{0.1\tw}|>{\colc}m{0.12\tw}|>{\colc}m{0.07\tw}|>{\colc}m{0.07\tw}?}
    \thickhline
    \rowcolor{tblheadcolor}
    \cd{id} & \cd{name} & \cd{department} & \cd{code} & \cd{is\_major} & \cd{is\_required} & \cd{credit} & \cd{period}\tabularnewline
    \hline
    \rowcolor{tblheadcolor}
    \cd{INT} & \cd{VC(20)} & \cd{VC(16)} & \cd{VC(8)} & \cd{TINYINT(1)} & \cd{TINYINT(1)} & \cd{INT} & \cd{INT}\tabularnewline
    \hline
    1 & 모두를위한파이썬 & 1 & COSE156 & 0 & 0 & 3 & 4\tabularnewline
    \hline
    2 & 이산수학 & 1 & COSE211 & 1 & 0 & 3 & 3\tabularnewline
    \hline
    3 & 운영체제 & 1 & COSE341 & 1 & 1 & 3 & 3\tabularnewline
    \hline
    4 & 기계학습 & 1 & COSE362 & 1 & 0 & 3 & 3\tabularnewline
    \hline
    5 & 캡스톤디자인 & 1 & COSE489 & 1 & 0 & 3 & 6\tabularnewline
    \hline
    6 & 유기화학실험 & 2 & CHEM214 & 1 & 1 & 2 & 4\tabularnewline
    \hline
    7 & 고체물리학 & 3 & PHYS482 & 1 & 0 & 3 & 3\tabularnewline
    \hline
    8 & 무기화학II & 2 & CHEM308 & 1 & 1 & 3 & 3\tabularnewline
    \thickhline
\end{tblenv}

\tblref{tab:courses-table-pk-fk}\는 \tblref{tab:courses-table-subdivisions}에 \cd{id} column을 추가하고 \cd{department} column이 \cd{departments.id} column을 참조하도록 변경한 테이블이다. 이제 ``운영체제'' 강의가 속한 학과의 단과대 이름을 조회하려면 ``운영체제'' 강의의 \cd{department} 값이 ``1''임을 찾고, \cd{departments} 테이블에서 \cd{id} 값이 ``1''인 row의 \cd{college} 값을 찾으면 된다.

다른 테이블로부터 참조를 받는 row가 삭제되는 경우 수행할 작업을 on delete action이라고 한다. 예를 들어 \cd{departments} 테이블에서 2번 ``화학과'' row가 삭제되면 \cd{courses} 테이블에서 이를 참조하는 8번 ``무기화학II'' 강의는 참조할 데이터가 사라진다 이렇게 row가 삭제되었을 때 이를 참조하는 row를 단계적으로 모두 삭제하는 cascade action을 수행하도록 설정할 수 있고, 이 외에도 set null, no action, set default, restrict 등이 있다.

\begin{codeenv}{code:create-table-foreign-key}{Create \cd{courses} Table with Foreign Key}\begin{verbatim}
CREATE TABLE `courses` (
    `id` INT NOT NULL AUTO_INCREMENT,
    `name` VARCHAR(20) NOT NULL,
    `department` INT NOT NULL,
    `code` VARCHAR(8) NOT NULL,
    `is_major` TINYINT(1) NOT NULL,
    `is_required` TINYINT(1) NOT NULL,
    `credit` INT NOT NULL,
    `period` INT NOT NULL,
    PRIMARY KEY (`id`),
    FOREIGN KEY (`department`)
    REFERENCES `departments`(`id`) ON DELETE CASCADE
) ENGINE=InnoDB DEFAULT CHARSET=utf8;
\end{verbatim}
\end{codeenv}

\coderef{code:create-table-foreign-key}\는 \tblref{tab:courses-table-pk-fk}의 테이블을 SQL문을 이용하여 DB에 생성한 코드이다.

\begin{tblenv}
    {tab:colleges-table}
    {\cd{colleges} Table}
    {?>{\colc}m{0.05\tw}|>{\colc}m{0.12\tw}|>{\colc}m{0.25\tw}|>{\colc}m{0.08\tw}?}
    \thickhline
    \rowcolor{tblheadcolor}
    \cd{id} & \cd{kor\_name} & \cd{eng\_name} & \cd{code}\tabularnewline
    \hline
    \rowcolor{tblheadcolor}
    \cd{INT} & \cd{VC(16)} & \cd{VC(50)} & \cd{INT}\tabularnewline
    \hline
    1 & 문과대학 & College of Liberal Arts & 13\tabularnewline
    \hline
    2 & 이과대학 & College of Science & 16\tabularnewline
    \hline
    3 & 정보대학 & College of Informatics & 32\tabularnewline
    \thickhline
\end{tblenv}

지금까지 관계가 있는 테이블을 기본 키와 외래 키를 이용하여 설계하는 방법에 대해 알아보았다. 이러한 구조는 nested 될 수 있으며, 단과대의 정보를 담은 \cd{colleges} 테이블을 만들어 \cd{departments} 테이블과의 관계를 생성할 수도 있다. \tblref{tab:colleges-table}\은 이러한 \cd{colleges} 테이블의 구조를 나타낸 것이다.

\begin{tblenv}
    {tab:departments-table-fk}
    {\cd{departments} Table with Foreign Key}
    {?>{\colc}m{0.05\tw}|>{\colc}m{0.12\tw}|>{\colc}m{0.3\tw}|>{\colc}m{0.1\tw}?}
    \thickhline
    \rowcolor{tblheadcolor}
    \cd{id} & \cd{kor\_name} & \cd{eng\_name} & \cd{college}\tabularnewline
    \hline
    \rowcolor{tblheadcolor}
    \cd{INT} & \cd{VC(16)} & \cd{VC(50)} & \cd{INT}\tabularnewline
    \hline
    1 & 컴퓨터학과 & Computer Science and Engineering & 3\tabularnewline
    \hline
    2 & 화학과 & Chemistry & 2\tabularnewline
    \hline
    3 & 물리학과 & Physics & 3\tabularnewline
    \hline
    4 & 언어학과 & Linguistics & 1\tabularnewline
    \thickhline
\end{tblenv}

\cd{colleges} 테이블의 데이터를 참조할 수 있도록 \cd{departments} 테이블을 \tblref{tab:departments-table-fk}\와 같이 수정할 수 있다.
