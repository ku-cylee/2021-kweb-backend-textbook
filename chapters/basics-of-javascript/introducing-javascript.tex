\section{Introducing Javascript} \label{sect:introducing-javascript}

자바스크립트(Javascript; JS)는 웹 분야에서 광범위하게 사용되는 인터프리터형 프로그래밍 언어이다. Front-end에서는 JS를 이용하여 웹 페이지의 동적인 기능을 구현하며, 프론트엔드 프레임워크인 Angular.js와 React.js 등에서도 쓰인다. 또한, front-end 뿐만 아니라 back-end 애플리케이션 역시 Node.js 엔진을 이용하여 JS로 구현할 수 있다. 이러한 JS의 특징으로 인해 2021년 기준 Stackoverflow에서 조사한 컴퓨터 언어의 사용 빈도 순위에서 7년째 1위를 차지할 정도로 여러 분야에서 광범위하게 응용되고 있다.

\subsection*{컴파일형 언어와 인터프리터형 언어}

컴퓨터 언어는 그 언어로 작성된 프로그램의 실행 방식에 따라 크게 컴파일러형 언어와 인터프리터형 언어로 나뉜다. 컴파일러형 언어로 작성된 프로그램은 실행하기 위해 소스 코드를 기계어로 변환하는 과정, 즉 컴파일 과정이 필요하며, C와 C++, Java, Rust, Go 등이 대표적인 컴파일러형 언어이다.

반면 인터프리터형 언어는 컴파일 과정 없이 인터프리터를 통해 소스 코드를 바로 실행할 수 있는 언어로, Python과 JS, Lisp 계열 언어들이 대표적인 인터프리터형 언어이다. 인터프리터형 언어들은 컴파일 과정이 불필요하므로 인터프리터에 하나의 표현식(expression)을 입력한 후 바로 그 코드의 실행 결과를 확인할 수 있으며, REPL(Read-Eval-Print-Loop) Shell에서 하나의 표현식을 실행한 후 결과를 바로 확인할 수 있다.\footnote{파일 형태로 작성하는 것도 당연히 가능하다. JS 파일 작성과 실행은 \chapref{ch:javascript-dynamic-frontend}에서 다룬다.} 그러나 인터프리터형 언어는 표현식을 분석하고 컴파일하는 과정을 해당 표현식이 실행될 때마다 반복해야 하므로 컴파일형 언어에 비해 성능은 낮은 편이다.

\begin{codeenv}{code:js-simple-example}{Simple Example of JS}\begin{verbatim}
> console.log('Hello World!');
undefined
Hello World!
> alert('Hello World!');
undefined
> 3 + 5;
8
\end{verbatim}
\end{codeenv}

웹 브라우저 개발자 도구의 Console 탭은 JS의 REPL Shell 역할을 하므로, Console 탭에서 \coderef{code:js-simple-example}\을 실행하여 결과를 관찰해보자.

첫 번째 표현식을 실행하면 콘솔에 \texttt{undefined}와 \texttt{Hello World!}라는 문자열이 출력되는데, 먼저 표현식의 반환값인 \texttt{undefined}가 console에 출력되고, 이 표현식이 \texttt{Hello World!}라는 문자열을 출력하라는 뜻을 가지므로 해당 문자열 역시 console에 출력된다. 두 번째 표현식을 실행하면 브라우저에 \texttt{Hello World!} 문자열을 담은 경고창이 나타나며, 역시 해당 표현식의 반환값인 \texttt{undefined}가 console에 출력된다. 또한, 마지막 표현식을 실행하면 3과 5의 덧셈 결과값이 출력된다. 앞으로 본 교재에서는 표현식의 반환값이 \texttt{undefined}인 경우에는 결과 표시를 생략한다.
