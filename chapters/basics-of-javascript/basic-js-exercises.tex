\section{Basics of JS Exercises} \label{sect:basic-js-exercises}

\subsection*{Exercise 1}
인자로 받은 값이 1 이상 9 이하의 정수인지 판별하여 결과를 반환하는 함수 \texttt{isValidNumber}를 구현하여라. \texttt{isValidNumber} 함수는 \coderef{code:basic-js-exercise-1}\과 같이 동작하여야 한다.

\begin{codeenv}{code:basic-js-exercise-1}{Exercise 1 Example}\begin{verbatim}
> isValidNumber(9);
true
> isValidNumber('4');
true
> isValidNumber('abc');
false
> isValidNumber(-5);
false
> isValidNumber(3.5);
false
> isValidNumber(3 / 0);
false
\end{verbatim}
\end{codeenv}

\subsection*{Exercise 2}
인자로 받은 정수의 모든 양의 약수(約數, divisor) 배열을 작은 순서대로 반환하는 함수 \texttt{getDivisors}를 구현하여라. 정수 $x$의 약수는 $\sqrt{x}$까지만 탐색하여도 모두 구할 수 있음을 이용하고, 배열의 \texttt{sort} 메서드를 이용하여라. \texttt{getDivisors} 함수는 \coderef{code:basic-js-exercise-2}\와 같이 동작하여야 하며, 인자로 받은 값이 유효한 값인지 확인할 필요는 없다.

\begin{codeenv}{code:basic-js-exercise-2}{Exercise 2 Example}\begin{verbatim}
> getDivisors(5);
[ 1, 5 ]
> getDivisors(24);
[ 1, 2, 3, 4, 6, 8, 12, 24 ]
> getDivisors(196);
[ 1, 2, 4, 7, 14, 28, 49, 98, 196 ]
\end{verbatim}
\end{codeenv}
\newpage

\subsection*{Exercise 3}
\coderef{code:basic-js-exercise-3}에 주어진 \texttt{ellipse} 객체에 타원의 넓이, 둘레의 길이, 이심률을 구하여 반환하는 함수 \texttt{getArea}, \texttt{getPerimeter}, \texttt{getEccentricity}를 구현하여라. 타원의 \texttt{width}를 $w$, \texttt{height}를 $h$라고 하였을 때($w \geq h$), 각 값을 구하는 식은 다음과 같다.

$$ \begin{aligned}
    (\mathrm{Area}) &= \pi w h \\
    (\mathrm{Perimeter}) &\approx 2 \pi \sqrt{\frac{w^2 + h^2} 2} \\ 
    (\mathrm{Eccentricity}) &= \sqrt{1 - \left( \frac h w \right)^2}
\end{aligned} $$

\begin{codeenv}{code:basic-js-exercise-3}{Exercise 3}\begin{verbatim}
> const ellipse = {
      width: 10,
      height: 5,
  };
> ellipse.getArea();
157.07963267948966
> ellipse.getPerimeter();
49.6729413289805
> ellipse.getEccentricity();
0.8660254037844386
\end{verbatim}
\end{codeenv}
