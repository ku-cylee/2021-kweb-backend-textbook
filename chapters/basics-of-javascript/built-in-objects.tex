\section{Built-in Objects} \label{sect:built-in-objects}

\sectref{sect:statements-and-functions}에서 객체에 대해 다루면서, 메서드, 클래스, 생성자, 인스턴스 등의 개념과 그 관계에 대해 다루었습니다. \sectref{sect:data-types}에서 다루었던 JS의 자료형이나 배열 등은 사실 모두 객체입니다. 즉, number형, string형의 변수나 상수들은 모두 인스턴스이며 속성, 메서드 등을 가지며, 이렇게 JS가 자체적으로 보유하고 있는 객체를 내장 객체(built-in objects)라고 합니다. 이번 절에서는 내장 함수나 내장 객체의 유용한 속성이나 메서드 등에 대해서 다룰 것입니다. 

\subsection*{Global Functions}

객체의 메서드 형태가 아닌, 함수 형태로 사용할 수 있는 함수들입니다.

\begin{itemize}
    \item \texttt{eval}: string형으로 주어진 계산식을 계산하여 반환합니다. 
    \item \texttt{isFinite}: 주어진 값이 \texttt{Infinity}인지 판별하여 결과를 반환합니다. 
    \item \texttt{isNaN}: 주어진 값이 \texttt{NaN}인지 판별하여 결과를 반환합니다. 
    \item \texttt{parseInt}: 주어진 값을 정수로 변환할 수 있다면 변환하여 반환합니다. 
\end{itemize}

\begin{codeenv}{code:global-functions}{Global Functions}\begin{verbatim}


> eval('3 + 5');
8
> isFinite(3/0);
false
> isNaN(0/0);
true
> parseInt('3.5');
3
\end{verbatim}
\end{codeenv}

\subsection*{String Objects}

string형 객체의 속성이나 메서드입니다.

\begin{itemize}
    \item \texttt{length}: 문자열의 길이 속성입니다. 
    \item \texttt{concat}: 두 문자열을 이어 붙여 새 문자열을 반환합니다. 
    \item \texttt{split}: 문자열을 정해진 문자열을 기준으로 분리하여, 분리된 문자열들의 배열을 반환합니다. 
    \item \texttt{substring}: 문자열을 정해진 index에서 또 다른 정해진 index까지 잘라 반환합니다.
    \item \texttt{toUpperCase}: 문자열의 모든 문자를 대문자로 바꾸어 반환합니다. 
    \item \texttt{includes}: 주어진 문자열을 원래 문자열이 포함하는지 판별하여 결과를 반환합니다. 
    \item \texttt{trim}: 주어진 문자열의 양 끝 whitespace 문자를 잘라냅니다. 
\end{itemize}

\begin{codeenv}{code:string-methods}{String Object Methods}\begin{verbatim}


> let str = '  Hello World!\t';
> str.length;
15
> str.concat('JS is Fun');
'  Hello World!\tJS is Fun'
> str.split('o');
[ '  Hell', ' W', 'rld!\t' ]
> str.substring(6);
'o World!\t'
> str.toUpperCase();
'  HELLO WORLD!\t'
> str.includes('ll');
true
> str.trim();
'Hello World!'
\end{verbatim}
\end{codeenv}

\subsection*{Array Objects}

배열도 일종의 객체로, 배열에 대한 속성이나 메서드입니다. 

\begin{itemize}
    \item \texttt{length}: 배열의 길이 속성입니다. 
    \item \texttt{push}: 배열의 끝에 배열의 원소를 새로 넣습니다. 
    \item \texttt{pop}: 배열의 끝에 있는 원소를 제거하고, 제거한 원소를 반환합니다. 
    \item \texttt{concat}: 두 배열을 이어 붙여 새 배열을 반환합니다. 
    \item \texttt{sort}: 배열의 원소들을 정렬하여 반환합니다. 
    \item \texttt{reverse}: 배열의 원소들의 순서를 반대로 바꾸어 반환합니다. 
    \item \texttt{map}: 배열의 각 원소들에 주어진 함수를 적용하여 반환된 값들의 배열을 반환합니다. 
    \item \texttt{join}: 배열의 각 원소들을 주어진 문자열을 이용하여 이은 문자열을 반환합니다. 
\end{itemize}

\begin{codeenv}{code:array-methods}{Array Methods}\begin{verbatim}


> let arr = ['apple', 'orange', 'banana'];
> arr.concat(['cherry', 'grapes']);
[ 'apple', 'orange', 'banana', 'cherry', 'grapes' ]
> arr.sort();
[ 'apple', 'banana', 'orange' ]
> arr.push('cherry', 'grapes');
5
> arr.pop();
'grapes'
> arr.reverse();
[ 'cherry', 'orange', 'banana', 'apple' ]
> arr.join(' + ');
'cherry + orange + banana + apple'
\end{verbatim}
\end{codeenv}

\subsection*{Date Objects}

\texttt{Date} 객체는 날짜와 시간을 다루는 객체입니다. \texttt{Date} 객체를 생성하는 방법은 네 가지가 있습니다.

\begin{itemize}
    \item \texttt{new Date();}
    \item \texttt{new Date(milliseconds);}
    \item \texttt{new Date(dateString);}
    \item \texttt{new Date(year, month, day, hours, minutes, seconds, milliseconds);}
\end{itemize}

이렇게 선언된 \texttt{Date} 객체 간에는 서로의 시간 차이를 계산하는 뺄셈 계산이 가능하며, 수많은 메서드들이 있습니다. 이 중 \texttt{getTime} 메서드는 UTC 기준의 날짜/시각과 1970년 1월 1일 사이의 차이를 밀리초(ms)로 변환하여 반환하는 메서드로, 날짜/시각을 이러한 방식으로 저장하면 날짜/시각 간의 선후 관계를 계산하거나 DB 등에 날짜/시각을 저장할 때 매우 유용합니다. 

\begin{codeenv}{code:date-methods}{Date Object Methods}\begin{verbatim}


> let date1 = new Date('2020/05/25');
> date1;
2020-05-24T15:00:00.000Z
> let date2 = Date.now();
> date2
1590459716425
> date2 - date1;
127316425
> date1.getTime();
1590332400000
> date1.getDay();
1
\end{verbatim}
\end{codeenv}

\subsection*{Math Objects}

\texttt{Math} 객체는 수학적인 연산을 다루는 객체입니다. \texttt{Math} 객체는 모든 메서드가 정적 메서드인 것이 특징입니다. 

\begin{codeenv}{code:math-methods}{Math Object Methods}\begin{verbatim}


> Math.abs(-3);
3
> Math.max(3, 5, 7);
7
> Math.min(3, 5, 7);
3
> Math.round(2.62);
3
> Math.random();
0.2024096270009006
> Math.PI;
3.141592653589793
\end{verbatim}
\end{codeenv}
