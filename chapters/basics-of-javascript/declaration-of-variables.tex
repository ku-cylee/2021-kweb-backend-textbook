\section{Declaration of Variables} \label{sect:declaration-of-variables}

\subsection*{ECMAScript}

ECMAScript는 스크립트 언어에 관한 규약이고, 줄여서 ES라고 한다. JS는 ES의 표준을 따르며 ES는 JS를 이루는 코어 스크립트 언어이다. ES1이 1997년 출시된 이후 2015년 ES6가 출시되며 언어의 표준에 큰 변화가 발생하였고, 이후 약간의 업데이트를 통해 2019년 ES9가 출시되어 현재\footnote{2021년}에 이르고 있다. 본 교재에서는 ES9을 기준으로 JS를 다룬다.

\subsection*{식별자의 선언}

JS는 다른 언어들과 마찬가지로 변수나 상수 등의 식별자(identifier)를 정의하고 사용할 수 있고, 모든 식별자에는 자료형이 있다. 다만 C나 Java 등과는 달리 식별자를 선언할 때 자료형을 명시해주지 않고, 인터프리터가 식별자에 할당되는 값을 분석하여 자료형을 스스로 지정한다.

변수, 상수, 함수 등은 camel case(예: \texttt{parsedSourceInput}), 클래스는 pascal case(예: \texttt{RequestHandler}), 상수는 대문자와 underscore(\texttt{\_})로 작명(예: \texttt{LENGTH\_LIMIT})한다. 또한, 숫자로 시작하지 않고 가급적 알파벳으로 시작하며, 식별자가 나타내는 바를 명확하게 알 수 있도록 작명한다. 

식별자는 \texttt{var}, \texttt{let}, \texttt{const} 등의 키워드를 이용하여 선언한다. 먼저, \texttt{var}는 지금까지 널리 사용되어 온 키워드로, \texttt{var}로 선언한 변수는 재선언할 수 있고, 값을 재할당할 수도 있다.

\begin{codeenv}{code:var-keyword}{Declaring Variable using \texttt{var}}\begin{verbatim}
> var num1 = 3;
> num1;
3
> num1 + 5;
8
> num1 = 5;
5
> var num1 = 20;
\end{verbatim}
\end{codeenv}

반면 \texttt{let}으로 선언한 변수는 값을 재할당할 수는 있으나, 재선언은 불가능하다.\footnote{Chrome의 개발자 도구를 이용하였을 때 재선언이 가능한 버그가 보고되었다. (2021년 기준)}

\begin{codeenv}{code:let-keyword}{Declaring Variable using \texttt{let}}\begin{verbatim}
> let num2 = 3;
> num2 + 5;
8
> num2 = 5;
5
> let num2 = 20;
SyntaxError: redeclaration of let num2
\end{verbatim}
\end{codeenv}

\texttt{const}는 constant, 즉 상수(常數)의 약자로, \texttt{const}로 선언한 상수는 재선언과 재할당이 모두 불가능하다.

\begin{codeenv}{code:const-keyword}{Declaring Constant using \texttt{const}}\begin{verbatim}
> const constNum = 3;
undefined
> constNum + 5;
8
> constNum = 5;
TypeError: invalid assignment to const 'constNum'
> const constNum = 20;
Thrown:
SyntaxError: redeclaration of const constNum
\end{verbatim}
\end{codeenv}

\texttt{var}는 JS 초창기부터 사용되어 왔던 식별자 선언 키워드인 반면, \texttt{let}과 \texttt{const}는 ES6 출시와 함께 추가된 식별자 선언 키워드이다. \texttt{var}과 \texttt{let}/\texttt{const}는 재선언 가능성에서도 차이점이 있지만, 식별자의 접근 가능한 범위(scope)에도 차이가 있다.

\begin{codeenv}{code:var-let-scope}{Scope of var and let}\begin{verbatim}
> { var num1 = 1; }
> num1;
1
> { let num2 = 2; }
> num2;
Uncaught ReferenceError: num2 is not defined
\end{verbatim}
\end{codeenv}

\coderef{code:var-let-scope}\를 실행하면, \texttt{var}로 선언된 \texttt{num1}의 값은 정상적으로 출력되나, \texttt{let}으로 선언된 \texttt{num2}는 출력되지 않고 정의되지 않았다는 에러가 발생한다. 이렇듯 \texttt{var}로 선언된 식별자는 선언된 블록 바깥에서 접근될 수 있고 함수 밖에서는 접근될 수 없는 function-scope이며, \texttt{let}과 \texttt{const}로 선언된 식별자는 선언된 블록 바깥에서 접근될 수 없는 block-scope이다.

\begin{tblenv}
    {tab:var-let-const-table}
    {Differences between \texttt{var}, \texttt{let} and \texttt{const}}
    {
        >{\raggedright}m{0.08\textwidth}
        >{\raggedright}m{0.25\textwidth}
        >{\raggedright}m{0.25\textwidth}
        >{\raggedright}m{0.20\textwidth}
    }
    \thickhline
    키워드 & 재할당 (Reassignment) & 재선언 (Redeclaration) & 스코프 (Scope) \tabularnewline
    \hline
    \texttt{var} & 가능 & 가능 & function-scope \tabularnewline
    \texttt{let} & 가능 & 불가능 & block-scope \tabularnewline
    \texttt{const} & 불가능 & 불가능 & block-scope \tabularnewline
    \thickhline
\end{tblenv}

ES6 이전에는 \texttt{var}를 이용하여 변수를 선언하였으나, ES6 이후에는 \texttt{let}과 \texttt{const}를 이용하여 변수와 상수를 선언하는 것이 권장되며, \texttt{var}의 사용은 지양되는 추세이다. 그 이유는 \texttt{let}과 \texttt{const}가 재선언이 불가능하고, block-scope이기 때문이다. \texttt{var}는 function-scope이기 때문에 코드의 네임스페이스를 오염\footnote{https://stackoverflow.com/questions/22903542/what-is-namespace-pollution}시키며, run time에서 의도치 않은 논리적 오류가 발생할 수 있다. 반면 \texttt{let}과 \texttt{const}로 선언된 식별자는 block-scope이므로 네임스페이스 오염을 최소화할 수 있고, \texttt{const}로 선언된 상수는 재할당이 불가능하므로 run time에서 상수값을 변경되어 발생하는 논리적 오류를 방지할 수 있다.
