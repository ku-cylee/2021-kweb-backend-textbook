\section{Declaration of Variables} \label{sect:declaration-of-variables}

\subsection*{ECMAScript}

ECMAScript란 자바스크립트를 이루는 코어 스크립트 언어로, 줄여서 ES라고 부릅니다. ES는 스크립트 언어에 관한 규약으로, JS는 ES 표준을 따르는 언어입니다. ES5가 2009년에 출시된 후 ES6가 2015년에 출시되어 현재까지 사용되고 있고, 현재는 ES8이 개발되고 있습니다. 준회원 스터디에서는 ES6를 따르는 JS에 관해 다룰 것입니다.

\subsection*{변수의 선언}

JS는 다른 언어들과 마찬가지로 변수나 상수를 정의하고 사용할 수 있습니다. 다만 C, Java와 달리 JS에서는 변수나 상수를 선언할 때 자료형을 명시해주지 않습니다. 그러나 자료형을 명시해주지 않는다고 해서 자료형이 없는 것은 아니고, 변수나 상수에 할당되는 값을 분석하여 자료형을 스스로 지정합니다. 변수 및 상수는 \texttt{var}, \texttt{let}, \texttt{const} 등의 구문을 이용하여 선언합니다. 

먼저, 가장 널리 사용되는 \texttt{var}로 선언한 변수는 재선언할 수 있고, 값을 재할당할 수도 있습니다.

\begin{codeenv}{code:var-keyword}{Declaring Variable using \texttt{var}}\begin{verbatim}


> var num1 = 3;
undefined
> num1;
3
> num1 + 5;
8
> num1 = 5;
5
> var num1 = 20;
undefined
\end{verbatim}
\end{codeenv}

반면 \texttt{let}으로 선언한 변수는 값을 재할당할 수는 있으나, 변수를 재선언하는 것은 불가능합니다.\footnote{현재 Chrome의 개발자 도구를 이용하였을 때 재선언이 가능한 버그가 보고됨}

\begin{codeenv}{code:let-keyword}{Declaring Variable using \texttt{let}}\begin{verbatim}


> let num2 = 3;
undefined
> num2 + 5;
8
> num2 = 5;
5
> let num2 = 20;
SyntaxError: redeclaration of let num2
\end{verbatim}
\end{codeenv}

\texttt{const}는 constant, 즉 상수(常數)의 약자로, \texttt{const}로 선언한 상수는 재선언과 재할당이 모두 불가능합니다.

\begin{codeenv}{code:const-keyword}{Declaring Constant}\begin{verbatim}


> const constNum = 3;
undefined
> constNum + 5;
8
> constNum = 5;
TypeError: invalid assignment to const `constNum'
> const constNum = 20;
Thrown:
SyntaxError: redeclaration of const constNum
\end{verbatim}
\end{codeenv}

과거에는 \texttt{var}을 이용하여 변수를 선언하였으나, 최근에는 \texttt{let}으로 선언하는 추세이며, 권장되는 방법입니다. 그 이유는 \texttt{var}과 \texttt{let}으로 변수를 선언했을 때 각 변수에 접근 가능한 범위(scope)가 다르기 때문입니다. \texttt{var}로 선언된 변수는 선언된 함수 내 어디서든 접근할 수 있지만, \texttt{let}과 \texttt{const}로 선언된 변수 및 상수의 경우 같은 블록 내에서만 접근이 가능합니다. 

\begin{codeenv}{code:var-let-scope}{Scope of var and let}\begin{verbatim}


let num = 10;

if (num % 2 == 0) {
    var isOdd = 0;
} else {
    var isOdd = 1;
}

console.log(isOdd);

if (num % 2 === 0) {
    let isEven = 1;
} else {
    let isEven = 0;
}

console.log(isEven);
\end{verbatim}
\end{codeenv}

\coderef{code:var-let-scope}를 개발자 도구를 이용하여 실행하면, \texttt{var}로 선언된 \texttt{isOdd}의 값은 정상적으로 출력되지만, \texttt{let}으로 선언된 \texttt{isEven}은 출력되지 않고, 정의되지 않았다는 에러가 나옵니다. 따라서 \texttt{var}로 선언된 변수는 선언된 블록 바깥에서도 접근할 수 있는 function-scope이며, \texttt{let}으로 선언된 변수는 선언된 블록 바깥에서 접근할 수 없는 block-scope입니다.

\begin{tblenv}
    {tab:var-let-const-table}
    {Differences between \texttt{var}, \texttt{let} and \texttt{const}}
    {
        >{\raggedright}m{0.08\textwidth}
        >{\raggedright}m{0.27\textwidth}
        >{\raggedright}m{0.27\textwidth}
        >{\raggedright}m{0.22\textwidth}
    }
    \thickhline
    키워드 & 재할당 (Reassignment) & 재선언 (Redeclaration) & 스코프 (Scope) \tabularnewline
    \hline
    \texttt{var} & 가능 & 가능 & function-scope \tabularnewline
    \texttt{let} & 가능 & 불가능 & block-scope \tabularnewline
    \texttt{const} & 불가능 & 불가능 & block-scope \tabularnewline
    \thickhline
\end{tblenv}

JS에서 \texttt{let}을 사용해야 하는 이유는 바로 \texttt{let}으로 선언된 변수가 block-scope이기 때문입니다. function-scope인 \texttt{var}의 특성상 \texttt{var}로 선언된 변수는 코드의 네임스페이스를 오염\footnote{https://stackoverflow.com/questions/22903542/what-is-namespace-pollution}시키며, 프로그램을 실행했을 때 의도치 않은 논리적 오류를 유발할 수 있습니다. 반면 \texttt{let}과 \texttt{const}로 선언된 변수 및 상수는 block-scope로, 변수 선언에 따른 네임스페이스 오염을 최소화할 수 있으며, \texttt{const}로 선언된 상수의 경우 선언 이후의 변경이 불가능하므로 프로그램 동작하면서 의도치 않게 상수값을 변형하여 발생하는 논리적 오류를 방지할 수 있습니다. 
