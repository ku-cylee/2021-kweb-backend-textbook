\section{Data Types} \label{sect:data-types}

JS에는 다음과 같은 7개의 자료형이 존재합니다. 

\begin{itemize}
    \item Primitive Types: number, string, boolean, undefined, null, symbol
    \item Compound Type: object
\end{itemize}

변수나 상수의 자료형은 \texttt{typeof} 함수로 확인할 수 있습니다. 이번 절에서는 symbol형을 제외한 6가지 자료형에 대해 다루고, 배열에 관해 다룹니다. 

\subsection*{Number}

number형은 수(數)를 저장하는 자료형입니다. number형에는 정수, 유리수 등을 구별 없이 저장할 수 있으며, 상호 연산이 가능합니다.

\begin{codeenv}{code:number-type}{Number Type}\begin{verbatim}


> let num1 = 2;
> let num2 = 7.5;
> let num3 = 2 / 5;
> typeof(num1);
'number'
> typeof(num2);
'number'
> num1 * num2;
15
> num3;
0.4
\end{verbatim}
\end{codeenv}

number 자료형에는 일반적인 수 이외에도 \texttt{Infinity}와 \texttt{NaN}(not a number)도 있습니다. \texttt{Infinity}와 \texttt{NaN}은 0으로 나누는 연산이 작용하면 발생합니다. 

\begin{codeenv}{code:infinity-nan}{Infinity and NaN}\begin{verbatim}


> 3/0;
Infinity
> typeof(3/0);
'number'
> 3/(-0);
-Infinity
> 0/0;
NaN
> typeof(0/0);
'number'
\end{verbatim}
\end{codeenv}

\subsection*{String}

string형은 문자열을 저장하는 자료형입니다. string형 문자열 간에는 상호 계산이 가능합니다.

\begin{codeenv}{code:string-type}{String Type}\begin{verbatim}


> let hello = 'Hello';
> let world = 'World';
> let helloWorld = hello + ' ' + world;
> helloWorld;
'Hello World'
> helloWorld.length;
11
> helloWorld[3];
'l'
> typeof(helloWorld);
'string'
\end{verbatim}
\end{codeenv}

string형의 변수에 값을 할당할 때 큰따옴표(\texttt{"})나 작은따옴표(\texttt{'})를 사용하여 문자열을 감쌉니다. 다만, 문자열을 서로 다른 문자로 감싸는 것은 불가능합니다. 또한, 문자열 내에 감싸는 문자가 포함되어 있다면 역슬래시(\verb|\|)를 이용하여 escape 시켜야 합니다. 

\begin{codeenv}{code:string-type-assign}{Assigning Value on String Type Variable}\begin{verbatim}


> "String";
'String'
> 'String';
'String'
> "String';
Uncaught SyntaxError: Invalid or unexpected token
> "McDonald's";
'McDonald\'s'
> console.log("McDonald's");
McDonald's
\end{verbatim}
\end{codeenv}

\subsection*{Boolean}

boolean형은 논리적인 요소를 나타내며, \texttt{true}와 \texttt{false} 두 가지의 값이 가능합니다. 

\begin{codeenv}{code:boolean-type}{Boolean Type}\begin{verbatim}


> 1 == 1;
true
> 1 > 3;
false
> true == false;
false
> true != false;
true
> let isNumberEven = (5 % 2 == 0);
> isNumberEven;
false
\end{verbatim}
\end{codeenv}

\subsection*{Object}

object형, 즉 객체형이란 속성(property)과 그에 대한 값(value)을 가질 수 있는 자료형입니다. 예를 들어, 어떠한 학생에 대한 정보를 다룰 때, 그 학생의 학번, 이름, 학과, 재학 여부 등을 별도의 변수에 저장하지 않고, \texttt{student}라는 객체형 변수에 저장하여, 일괄적으로 다룰 수 있습니다. 

객체에서는 속성을 key, 그에 대응하는 속성값을 value라고 합니다. Key는 문자열이어야 하고, 같은 계층에 있는 다른 key와 겹치는 key가 있을 수 없습니다. value에는 숫자, 문자열뿐만 아니라 객체 등 모든 종류의 데이터형이 가능합니다. 즉, 객체 내부에 또 다른 객체가 포함될 수 있어, 복잡한 데이터 구조를 형성하는 것이 가능합니다. 

\begin{codeenv}{code:object-type}{Object Type}\begin{verbatim}


> let person = {
    age: 21,
    name: 'Frank',
    height: 170,
    isMale: true,
};
> person;
{ age: 21, name: 'Frank', height: 170, isMale: true }
> person.age;
21
> person['age'];
21
> !person.isMale;
false
> person.weight = 60;
60
> person;
{ age: 21, name: 'Frank', height: 171, isMale: true, weight: 60 }
\end{verbatim}
\end{codeenv}

\subsection*{Undefined and Null}

undefined 자료형은 값이 할당되지 않은 변수의 값이며, null 자료형은 값을 모르는, \texttt{null} 값을 갖습니다. 

\begin{codeenv}{code:undefined-type}{Undefined Type}\begin{verbatim}


> let variable;
> variable;
undefined
> typeof(variable);
'undefined'
\end{verbatim}
\end{codeenv}

null 자료형은 \texttt{typeof} 함수를 이용하면 잘못된 자료형을 반환받습니다. 이는 JS의 매우 오래된 버그이기 때문에 고치는 것이 불가능합니다. 따라서 변수가 \texttt{null}임을 확인하기 위해서는 값을 직접 비교해야 합니다. 

\begin{codeenv}{code:null-type}{Null Type}\begin{verbatim}


> let nullvar = null;
> typeof(null);
'object'
> nullvar == null;
true
\end{verbatim}
\end{codeenv}

\subsection*{Array}

배열은 index를 가지는, 복수의 자료를 저장할 수 있는 자료구조입니다. 배열의 각 원소의 자료형에는 제한이 없으며, 배열의 크기가 정해져 있지 않기 때문에 메서드(method)를 이용하여 원소를 자유롭게 넣고 뺄 수 있습니다. 

\begin{codeenv}{code:js-array}{Array}\begin{verbatim}


> let arr = [0, 1, 1.5, true, 'hungry'];
> arr.length;
5
> arr[3];
true
> arr[4];
'hungry'
> arr.push('new element');
6
> arr;
[ 0, 1, 1.5, true, 'hungry', 'new element' ]
\end{verbatim}
\end{codeenv}
