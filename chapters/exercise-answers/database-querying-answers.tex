\section{Database Querying Exercise Answers}\label{sect:database-querying-answers}

\subsection*{\excref{exc:simple-crud}}

\begin{codeenv}{code:simple-crud-answer}{\excref{exc:simple-crud} Answer}\begin{verbatim}
SELECT `id`, `c3`, `c5` FROM `crud` WHERE `c1` = 11 AND `c2` = 2;
SELECT * FROM `crud` WHERE `c1` > 18 OR `c2` < 2;
INSERT INTO `crud` (`c1`, `c2`, `c3`, `c5`) VALUES (7, 4, 'col101', 0);
INSERT INTO `crud` VALUES (103, 3, 3, 'col103', DEFAULT, 1);
SELECT * FROM `crud` WHERE `id` > 100;
UPDATE `crud` SET `c3` = 'col0', `c5` = 0 WHERE `c1` > 4 AND `c1` < 9 AND `c2` = 1;
SELECT * FROM `crud` WHERE `c1` > 4 AND `c1` < 9 AND `c2` = 1;
DELETE FROM `crud` WHERE `c5` = 0;
SELECT * FROM `crud` WHERE `c5` = 0;
\end{verbatim}
\end{codeenv}

\subsection*{\excref{exc:train-system-sql}}

\begin{codeenv}{code:train-system-sql-answer}{\excref{exc:train-system-sql} Answer}\begin{verbatim}
-- (1)
SELECT `types`.`name` AS `type`, `src`.`name` AS `srcStn`,
        `dst`.`name` AS `dstStn`, `trains`.`departure`, `trains`.`arrival`
FROM   `trains`
       INNER JOIN `types` ON `types`.`id` = `trains`.`type`
       INNER JOIN `stations` AS `src` ON `src`.`id` = `trains`.`source`
       INNER JOIN `stations` AS `dst` ON `dst`.`id` = `trains`.`destination`
ORDER  BY `departure`;

-- (2)
SELECT `trains`.`id`, `types`.`name` AS `type`,
       `src`.`name` AS `srcStn`, `dst`.`name` AS `dstStn`,
       Count(*) AS `occupied`, `types`.`maxseats` AS `maximum`
FROM   `tickets`
       INNER JOIN `trains` ON `trains`.`id` = `tickets`.`train`
       INNER JOIN `types` ON `types`.`id` = `trains`.`type`
       INNER JOIN `stations` AS `src` ON `src`.`id` = `trains`.`source`
       INNER JOIN `stations` AS `dst` ON `dst`.`id` = `trains`.`destination`
GROUP  BY `tickets`.`train`
ORDER  BY `trains`.`id`;

-- (3)
SELECT `users`.`id`, `users`.`name`, Sum(`trains`.`distance`) / 10 AS `totalDistance`
FROM   `tickets`
       INNER JOIN `trains` ON `trains`.`id` = `tickets`.`train`
       INNER JOIN `users` ON `users`.`id` = `tickets`.`user`
GROUP  BY `users`.`id`
ORDER  BY `totaldistance` DESC
LIMIT  20;
\end{verbatim}
\end{codeenv}

\subsection*{\excref{exc:train-system-server}}

\begin{codeenv}{code:train-system-server-answer}{\excref{exc:train-system-server} Answer}\begin{verbatim}
const express = require('express');
const { runQuery } = require('./database');

const app = express();
const port = 3000;

app.get('/passengers', async (req, res, next) => {
    try {
        const { tid } = req.query;
        const sql = 'SELECT users.id, users.name, tickets.seatNumber FROM tickets ' +
            'INNER JOIN trains ON trains.id = tickets.train AND trains.id = ? ' +
            'INNER JOIN users ON users.id = tickets.user ORDER BY seatNumber'
        const passengers = await runQuery(sql, [tid]);
        return res.send(passengers.map(passenger => {
            const { id, name, seatNumber } = passenger;
            return `${name} (#${id}): Seat #(${seatNumber})`;
        }).join('\n'));
    } catch (err) {
        console.error(err);
        return res.sendStatus(500);
    }
});

app.get('/train/status', async (req, res, next) => {
    try {
        const { tid } = req.query;
        const sql = 'SELECT Count(*) AS occupied, maxSeats AS maximum FROM tickets ' +
            'INNER JOIN trains ON trains.id = tickets.train AND trains.id = ? ' +
            'INNER JOIN types ON types.id = trains.type';
        const { occupied, maximum } = (await runQuery(sql, [tid]))[0];
        const isSeatsLeft = occupied < maximum
        return res.send(`Train ${tid} is ${isSeatsLeft ? 'not sold out' : 'sold out'}`);
    } catch (err) {
        console.error(err);
        return res.sendStatus(500);
    }
});

app.listen(port, () => console.log(`Server listening on port ${port}!`));
\end{verbatim}
\end{codeenv}

database.js 모듈은 \coderef{code:runquery-function}\과 \coderef{code:prepared-statement}\를 참고하여라.
