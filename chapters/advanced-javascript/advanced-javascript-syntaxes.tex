\section{Advanced Javascript Syntaxes}\label{sect:advanced-javascript-syntaxes}

\subsection*{Template Literal}

프로그램에서 문자열을 생성할 때, 상황에 따라 문자열 내 특정 부분의 값을 바꾸고 싶은 경우가 많다. 이 경우, 변수를 이용하여 상황에 따라 변하는 부분의 값을 바꾸어 문자열을 생성할 수 있다.

예컨대, \cd{x}, \cd{y} 두 인자를 받아 \cd{x}의 \cd{y} 제곱의 값을 출력하는 함수는 \coderef{code:string-format-without-template-literal}과 같이 구현될 수 있다.

\begin{codeenv}{code:string-format-without-template-literal}{String Formatting without Template Literal}\begin{verbatim}
const printPower = (x, y) => {
    const msg = 'Value of ' + x + ' to the power ' + y + ': ' + Math.pow(x, y);
    console.log(msg);
};

printPower(3, 4);
\end{verbatim}
\end{codeenv}

\coderef{code:string-format-without-template-literal}은 문제없이 제 기능을 수행하는 함수이지만, 두 가지 문제점을 갖는다. 먼저 msg 문자열에서 변하지 않는 부분과 변하는 부분이 분리되어 있어, 실제 문자열의 형태를 알아보기 어렵다. 또한, 문자열을 감싸는 문자, 덧셈 기호 등으로 인해 코드가 불필요하게 길어져, 가독성을 저해시킨다. 이러한 문제점을 해결하기 위해 JS에서는 template literal이라는 문법을 제공하여, 문자열 포맷팅을 더 편리하게 할 수 있도록 한다.

먼저 template literal을 이용하여 문자열을 생성하기 위해서는 일반적인 문자열과는 달리 문자열을 backtick(백틱; \cd{`}) 문자로 감싸야 한다. 또한, 문자열 내에서 \cd{\$\{}과 \cd{\}}로 감싸진 부분은 문자열이 아닌, JS 표현식으로 인식된다. 문자열이 생성될 때는 표현식을 나타내는 부분이, 감싸진 부분의 표현식의 반환값으로 대체된다.
\coderef{code:string-format-with-template-literal}\은 \coderef{code:string-format-without-template-literal}의 함수를 template literal을 이용하여 표현한 코드이다. 코드가 기존에 비해 매우 직관적이고 깔끔해진 것을 확인할 수 있다.

\begin{codeenv}{code:string-format-with-template-literal}{String Formatting with Template Literal}\begin{verbatim}
const printPower = (x, y) => {
    const msg = `Value of ${x} to the power ${y}: ${Math.pow(x, y)}`;
    console.log(msg);
};

printPower(3, 4);
\end{verbatim}
\end{codeenv}

여담으로, \cd{\$\textbraceleft\textbraceright} 부분을 표현식이 아닌 일반 문자열로 사용하고 싶은 경우, 역슬래시(\cd\textbackslash)를 이용하여 escape 해주면 된다.

\subsection*{Destructuring Assignment}

Destructuring Assignment(비구조화 할당)는 배열의 각 요소나 객체의 각 값을 서로 다른 변수에 저장하는 작업을 편리하게 할 수 있게 하는 문법이다.

\begin{codeenv}{code:array-destructuring}{Array Destructuring}\begin{verbatim}
const arr = [1, 2, 3, 4];

const [a1, a2, a3, a4] = arr;
const [b1, , b3] = arr;
const [, , , c4, c5, c6 = 10] = arr;
\end{verbatim}
\end{codeenv}

\coderef{code:array-destructuring}\은 배열의 비구조화를 나타낸 예제이다. 구문에 대한 간단한 설명은 다음과 같다.

\begin{itemize}
    \item 대입 연산자(\cd{=})의 좌변에 배열의 각 요소의 값이 할당될 변수를 배열 형태로 나타내고, 우변에는 분해하고자 하는 배열을 나타낸다.
    \item 좌변에서 \cd{n}번째에 배치된 요소에는 우변에서 \cd{n}번째에 배치된 요소의 값이 할당된다.
    \item 우변의 일부 값은 좌변에서 생략함으로써 무시할 수 있다.
    \item 좌변의 변수 중 우변에 대응되는 값이 없는 경우 \cd{undefined}가 할당된다.
    \item 좌변에서 특정 변수의 기본값을 설정해주면, 비구조화의 결과로 해당 변수의 값이 \cd{undefined}일 때, 그 변수에는 설정한 기본값이 할당된다.
\end{itemize}

\begin{codeenv}{code:object-destructuring}{Object Destructuring}\begin{verbatim}
const obj = { x: 1, y: 2, z: 3 };
const { x, z, u, v = 10 } = obj;
const { y: y1 } = obj;
\end{verbatim}
\end{codeenv}

\coderef{code:object-destructuring}\은 객체의 비구조화을 나타낸 예제이다. 배열의 비구조화와 전체적으로 유사하며, 차이점은 다음과 같다.

\begin{itemize}
    \item 배열의 경우 위치(index)를 기준으로 match하였다면, 객체는 속성(property) 이름을 기준으로 match한다.
    \item 객체의 원래 속성 이름과는 다른 이름의 변수에 값을 할당할 수 있다.
\end{itemize}

객체의 비구조화는 객체의 원래 속성명을 변수로써 사용할 때에는 권장되는 문법이다. 예컨대, \coderef{code:object-destructuring}에서 \cd{obj} 객체의 \cd{x} 속성을 사용할 때, \cd{const x = obj.x} 같은 표현보다는 \cd{const \{ x \} = obj} 같은 표현이 권장된다.

\subsection*{Truthy and Falsy}

JS의 모든 값은 암묵적으로(implicitly) 참이나 거짓의 값으로 변환될 수 있다. 암묵적으로 참(\cd{true})로 변환되는 값들을 truthy, 거짓(\cd{false})으로 변환되는 값들을 falsy라고 한다. 이렇게 암묵적인 참/거짓 변환은 조건문 등에서 조건의 참/거짓을 확인할 때 유용하게 사용된다.

다음 7가지 값은 falsy이며, 나머지 값들은 모두 truthy로 취급된다.

\begin{itemize}
    \item \cd{false} / \cd{0} / \cd{-0} / \cd{null} / \cd{undefined} / \cd{""} / \cd{NaN}
\end{itemize}

Truthy, falsy 표현식은 \coderef{code:truty-and-falsy}\와 같이 사용될 수 있다.

\begin{codeenv}{code:truty-and-falsy}{Truthy and Falsy Expressions}\begin{verbatim}
if (arr.length > 0) { ... }
// can be converted to
if (arr.length) { ... }

if (foo === undefined) { ... }
// can be converted to
if (!foo) { ... }
\end{verbatim}
\end{codeenv}

JS에서 논리적 OR을 계산하는 방식과, truthy/falsy 표현을 이용하여 short-circuit evaluation(단축 평가)을 실시할 수 있다. 논리적 OR을 계산하는 이항 연산자 \cd{||}는 연산자 앞에 오는 피연산자가 truthy하면 앞에 오는 항을 반환하고, falsy하면 뒤에 오는 피연산자를 반환한다.

\begin{codeenv}{code:short-circuit-eval}{Short Circuit Evaluation}\begin{verbatim}
const port = config.port || 3000;
\end{verbatim}
\end{codeenv}

Short-circuit evaluation은 \coderef{code:short-circuit-eval}\과 같이 변수의 기본값을 설정할 때 유용하게 사용된다. \cd{config.port}의 값이 \cd{truthy}하면 \cd{port}의 값은 \cd{config.port}의 값이 되고, falsy하면 \cd{3000}이라는 기본값을 갖는다.

\subsection*{Error Handling}

프로그램에서 에러가 발생했을 때 이를 처리하는 작업은 매우 중요하다. 코드에서 에러가 발생하면 그 이후의 코드는 더 이상 실행되지 않기 때문에, 에러가 제대로 처리되지 않은 경우 프로그램 자체가 종료되어 버릴 수 있다. 간단한 프로그램이 아닌, 실제 production level의 서비스에서 예기치 않은 에러가 발생하여 프로그램이 종료되어 버린다면 심각한 문제가 발생할 수 있다. 이런 종류의 문제가 발생하는 것을 방지하기 위해 JS에서는 에러 처리를 위한 문법을 제공한다.

\begin{codeenv}{code:function-without-error-handling}{Function Without Error Handling}\begin{verbatim}
const getStatusCode = res => res.status.code;

const code1 = getStatusCode({ status: { code: 400 } });
const code2 = getStatusCode({});
\end{verbatim}
\end{codeenv}

\coderef{code:function-without-error-handling}\은 에러 처리를 하지 않은 코드이다. \cd{getStatusCode} 함수에 정상적인 형태의 인자가 들어가면 의도한 값을 반환하지만, 빈 객체가 인자로 들어가면 에러가 발생한다. 이 함수에 정상적인 형태의 인자가 들어간다는 보장이 없다면, 이 함수를 포함한 프로그램은 에러가 발생했을 때 프로그램이 종료되어 버린다. 따라서 에러 처리를 반드시 해주어야 한다.

try...catch 문이 에러 처리를 위해 사용된다. try...catch 문은 \cd{try}, \cd{catch}, \cd{finally}로 이루어져 있고, \cd{try} block에서 에러가 발생하면 \cd{catch} block의 코드가 실행되고, 에러 여부와 무관하게 \cd{finally} block의 코드가 실행된다. try...catch 문에서 \cd{try} block은 반드시 필요하고, \cd{catch} block이나 \cd{finally} block 중 적어도 하나는 반드시 있어야 한다. 또한, \cd{catch} 문은 에러 객체를 인자로 받는다. 에러 객체에는 발생한 에러와 관련된 속성과 값이 포함되어 있고, 개발자의 필요에 따라 포함시킬 수 있다.

\coderef{code:function-without-error-handling}\은 try...catch 문을 이용하여 에러 처리를 한 코드이다.

\begin{codeenv}{code:function-with-error-handling}{Function With Error Handling}\begin{verbatim}
const getStatusCode = res => {
    try {
        return res.status.code;
    } catch (e) {
        return 0;
    } finally {
        console.log('getStatusCode called');
    }
};

const code1 = getStatusCode({ status: { code: 400 } });
const code2 = getStatusCode({});
\end{verbatim}
\end{codeenv}

JS에서 기본적으로 제공하는 에러 이외에도, 개발자의 필요에 따라 에러를 발생시킬 수 있다. 새로운 에러 클래스를 이용하여 에러 객체를 생성하고, \cd{throw} 키워드를 이용하여 생성한 에러 객체를 던지면, 에러가 발생한다.

에러 처리를 이용하여 코드의 흐름을 조절할 수 있다. A 함수에서 B 함수를 호출했을 때, B 함수에서 예외 상황이 발생하였다고 가정하자. 개발자는 B 함수에서 발생한 예외 상황를 즉시 처리하지 않고, A 함수에서 처리하기를 희망할 수 있다. 에러 처리가 되어있지 않은 함수는 에러 발생 즉시 종료되므로, B 함수에서 예외 상황과 관련된 에러를 발생시키고, A 함수에서 에러를 처리할 수 있다.

\begin{codeenv}{code:flow-control-with-error-handling}{Controlling Flow with Error Handling}\begin{verbatim}
const validateData = data => {
    if (!data) throw new Error(500);
    if (!data.length) throw new Error(404);
    return data;
};

const createMessage = data => {
    try {
        const checkedData = validateData(data);
        return `Success: ${data}`;
    } catch (e) {
        return `Failed: ${e.message}`;
    }
};

console.log(createMessage());
console.log(createMessage([]));
console.log(createMessage([1, 2, 3]));
\end{verbatim}
\end{codeenv}

\coderef{code:flow-control-with-error-handling}\은 이렇게 try...catch 문을 이용하여 코드 흐름을 조절한 예제이다. \cd{validateData}에서 데이터를 검증할 때, 예외적인 데이터 형태에 따라 서로 다른 에러를 발생시키면서 예외 상황의 처리를 즉시 수행하지 않는다. \cd{createMessage}에서는 \cd{validateData}에서 에러가 발생하였다면 \cd{catch} 문으로 코드 흐름이 넘어가고, 발생하지 않았다면 \cd{try} 문의 코드가 실행된다. 이처럼 try...catch 문을 이용하면 예외 상황에 대한 처리를 편리하고 깔끔하게 할 수 있고, 코드가 복잡해지고 함수 호출의 단계가 깊어질수록 try...catch 문을 이용한 코드 흐름 조절이 편리해진다.
