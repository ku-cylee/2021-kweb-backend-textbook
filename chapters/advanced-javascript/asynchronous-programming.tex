\section{Asynchronous Programming}\label{sect:asynchronous-programming}

\subsection*{동기식 프로그래밍}

동기식 프로그래밍 모델이란 프로그램의 코드가 순차적(sequentially)으로 실행되는 프로그래밍 모델이다. 즉, 코드가 나열된 순서와 실행되는 순서가 일치한다. 지금까지 우리가 봐왔던 코드는 \coderef{code:synchronous-program}\과 같이 동기식 프로그래밍 모델을 사용한 코드이다.

\begin{codeenv}{code:synchronous-program}{Synchronous Program}\begin{verbatim}
console.log('foo');

let a = 1;
while (a < 5000) a = a * 2;
console.log('a = ' + a);

console.log('bar');
\end{verbatim}
\end{codeenv}

코드가 하는 일이 단순 계산과 같이 프로세서(CPU)만을 필요로 하는 일이라면, 동기식 프로그래밍 모델만으로도 충분하다. 그러나 대부분 프로그램은 키보드나 디스크 등 여러 보조 장치와 데이터를 주고받아야 한다. 이러한 보조 장치와 데이터를 주고받는 I/O 작업은 프로세서에서의 처리 속도보다 느리므로 동기식으로 처리하면 데이터를 기다리는 동안 프로세서는 유휴(idle) 상태에 놓이게 된다.

\begin{codeenv}{code:big-file-read-sync}{Reading Big File Synchronously}\begin{verbatim}
const fs = require('fs');

console.log('foo');

const data = fs.readFileSync('./bigfile.bin');
console.log(`Data length: ${data.length} bytes`);

console.log('bar');
\end{verbatim}
\end{codeenv}

\coderef{code:big-file-read-sync}\는 bigfile.bin이라는, 크기가 큰 파일을 처음부터 끝까지 읽어 그 길이를 표시하는 코드이다. 코드에서 사용한 \cd{fs.readFileSync} 메서드는 인자로 주어진 파일을 동기식으로 읽어서 그 내용을 반환합니다. 그러므로 \coderef{code:big-file-read-sync}에서는 크기가 큰 bigfile.bin 파일을 메모리로 읽어 들이는 동안 프로그램의 실행은 멈추게 되고, 그로 인해 파일의 입출력과는 관련이 없는, bar 문자열을 출력하는 메서드는 입출력 작업이 끝날 때까지 기다려야 합니다.

\subsection*{Callback을 이용한 비동기 프로그래밍}

비동기 프로그래밍 모델이란 프로그램의 실행이 순차적이지 않은 프로그래밍 모델이다. 즉, 코드가 나열된 순서와 실행되는 순서가 다를 수 있다.

\begin{codeenv}{code:big-file-read-async}{Reading Big File Asynchronously}\begin{verbatim}
const fs = require('fs');

console.log('foo');

fs.readFile('./bigfile.bin', (err, data) => {
    if (!err) console.log(`Data length: ${data.length} bytes`);
    else console.error(err);
});

console.log('bar');
\end{verbatim}
\end{codeenv}

\coderef{code:big-file-read-async}\는 \coderef{code:big-file-read-sync}\를 비동기식으로 작성한 코드로, bigfile.bin 파일을 다 불러올 때까지 대기하지 않고 다음 코드를 먼저 실행한 후, 파일을 다 불러들인 후에 관련된 데이터에 의존적인 코드를 실행한다. 즉, 비동기 코드를 호출하면 코드의 흐름이 갈라진다.

\figures{fig:sync-async-code-flow}{Code Flow of Synchronous Code and Asynchronous Code}{
    \subfig{fig:sync-code-flow}{Synchronous Code Flow}
        {images/advanced-javascript/sync-code-flow.png}{.734}
    \vspace{10pt}
    \subfig{fig:async-code-flow}{Asynchronous Code Flow}
        {images/advanced-javascript/async-code-flow.png}{.8}
}

\figref{fig:sync-async-code-flow}\는 코드를 동기식으로 구현했을 때와 비동기식으로 구현했을 때 코드의 흐름을 개략적으로 나타낸 도식이다. 동기식 코드에서 작업 3은 작업 2가 종료된 이후 수행되는 반면, 비동기식 코드에서는 작업 3이 작업 2와 동시에 수행된다.

\coderef{code:big-file-read-async}에서 볼드체로 표시된 부분이 콜백(callback) 함수이다. 콜백 함수는 메서드의 실행 결과를 인자로 받는 함수로, \cd{fs.readFile} 메서드는 파일 읽기를 실패한 경우 \cd{err} 인자에 에러 객체를 넘겨주고, 성공한 경우 읽은 데이터를 \cd{data} 인자에 넘겨줍니다. 이렇듯 콜백 함수를 사용하여 비동기적으로 수행된 코드의 결과를 사용할 수 있습니다. 

그러나 콜백을 이용한 비동기 프로그래밍은 여러 가지 문제점이 있다. 가장 대표적인 문제점은 콜백 지옥(callback hell)이다. 콜백 함수 내에서 또 비동기 함수를 사용하여 콜백을 호출하는 등의 코드가 반복되면, 들여쓰기(indent)가 계속되어 코드의 깊이가 점점 깊어진다. \coderef{code:callback-hell}\은 callback hell이 나타날 수 있는 하나의 예시이다. 적당한 들여쓰기는 코드의 흐름을 파악하기 쉽게 하지만, 지나친 들여쓰기는 코드의 가독성을 현저히 저해시킨다. 

\begin{codeenv}{code:callback-hell}{Example of Callback Hell}\begin{verbatim}
const fs = require('fs');

fs.readFile('db.info', (err, data) => {
    const tableLocation = identifyTable(data, 'A');
    fs.readFile(tableLocation, (err, data) => {
        const recordLocation = identifyRecord(data, 'B');
        fs.readFile(recordLocation, (err, data) => {
            const record = parseRecord(data);
            // do something based on record's data
            console.log(processedData);
        });
    });
});
\end{verbatim}
\end{codeenv}

따라서, callback 함수의 문제점을 해결하여 비동기 처리를 할 필요가 있다. 

\subsection*{async/await}

ES6부터는 비동기 처리를 훨씬 간편하게 할 수 있는 async/await 방식이 도입되었다. async/await 방식을 이용하면 기존의 동기식 코드에 가까운 형태로 비동기식 코드를 작성할 수 있다.

\begin{codeenv}{code:big-file-read-async-await}{Reading Big File Using async/await}\begin{verbatim}
const fs = require('fs');
const util = require('util');

const readFile = util.promisify(fs.readFile);

const printFileSize = async filename => {
    try {
        const data = await readFile(filename);
        console.log(`Data length: ${data.length}`);
    } catch (err) {
        console.error(err);
    }
};

console.log('foo');
printFileSize('./bigfile.bin');
console.log('bar');
\end{verbatim}
\end{codeenv}

\coderef{code:big-file-read-async-await}\은 \coderef{code:big-file-read-async}\를 async/await 방식으로 작성한 코드이다. \coderef{code:big-file-read-async}에 비해 코드가 훨씬 간결해지고, 가독성이 향상된 것을 볼 수 있다. async/await 방식은 몇 가지 특징이 있다. 

\begin{itemize}
    \item 비동기 함수는 함수의 앞에 \cd{async} 키워드를 붙여 비동기 함수임을 표시한다.
    \item 비동기 함수를 호출할 때 \cd{await} 키워드를 사용하면 함수가 끝날 때까지 기다리며, 사용하지 않으면 기다리지 않고 다음 코드를 실행한다.
    \item await 키워드는 비동기 함수 내에서만 사용 가능하다. 
    \item 사용하고자 하는 비동기 함수를 \cd{util.promisify} 메서드를 이용하여 promisify한 후 사용해야 한다.
\end{itemize}

\begin{codeenv}{code:callback-hell-resolved}{Callback Hell Resolved with async/await}\begin{verbatim}
const fs = require('fs');
const readFile = util.promisify(fs.readFile);

(async () => {
    try {
        const dbInfo = await readFile('db.info');
        const tableInfo = await readFile(identityTable(dbInfo, 'A'));
        const rawRecord = await readFile(identifyRecord(tableInfo));
        const record = parseRecord(data);
        // do something based on record's data
        console.log(processedData);
    } catch (err) {
        console.error(err);
    }
})();
\end{verbatim}
\end{codeenv}

async/await 방식을 사용하면, callback hell 역시 해결할 수 있습니다. \coderef{code:callback-hell-resolved}\는 \coderef{code:callback-hell}\을 async/await 방식으로 작성한 코드로, callback hell이 해결되어 가독성이 높아진 것을 확인할 수 있습니다. 

이렇듯 코드를 간결하게 바꾸어주는 async/await 방식도 단점이 존재한다. async/await 방식은 자체적인 에러 처리가 불가능하므로 에러 처리는 \coderef{code:big-file-read-async-await}\과 같이 JS에 내장된 try...catch문을 사용하여야 한다. try...catch문을 사용한 에러 처리가 가독성 측면에선 깔끔하지만, JS의 try...catch문은 속도가 느리다는 단점을 가지고 있어 async/await 방식을 이용하였을 때 성능 저하가 나타날 수 있다.
