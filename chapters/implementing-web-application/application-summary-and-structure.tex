\section{Application Summary and Structure}\label{sect:application-summary-and-structure}

지금까지 back-end에 해당하는 웹 서버에서 필요한 기초 개념들과 사용되는 각종 기술 스택들을 공부하였다. 이번 장에서는 이러한 개념들과 기술 스택을 이용하여 회원가입, 로그인, 게시글 열람, 작성, 수정, 삭제 등의 기능을 수행하는 간단한 게시판 웹 애플리케이션을 구현해본다.

\subsection*{Application Summary and Structure}

다음 GitHub 저장소에서 스켈레톤 코드를 clone한다. 이 스켈레톤 코드에 주요 기능을 구현하여 애플리케이션을 완성한다.

\begin{itemize}
    \item https://github.com/KWEBofficial/node-simple-board-skeleton
\end{itemize}

\begin{codeenv}{code:project-package-json}{package.json}\begin{verbatim}
...
    "scripts": {
        "start": "nodemon ./src/index.js"
    },
    ...
    "dependencies": {
        "dotenv": "^10.0.0",
        "express": "^4.17.1",
        "express-session": "^1.17.2",
        "morgan": "^1.10.0",
        "mysql2": "^2.3.0",
        "pug": "^3.0.2"
    },
    "devDependencies": {
        "nodemon": "^2.0.13"
    }
}
\end{verbatim}
\end{codeenv}

\coderef{code:project-package-json}\은 서버 프로젝트의 package.json 파일이다. \cd{dependencies}에 명시된 모듈들은 코드 내에서 사용되는 모듈이고, \cd{devDependencies} 항목에 명시된 모듈은 소스 코드 내에서 직접 사용되지는 않으나 개발 과정에서 필요한 모듈이다. \cd{npm install} 명령어를 통해 명시된 모듈을 일괄적으로 설치할 수 있으며, \cd{--only=prod} 옵션을 주면 \cd{dependencies} 모듈들만 설치한다.

dotenv은 설정값을 가져오는 모듈, morgan은 로그를 출력하는 미들웨어를 제공하는 모듈이다. nodemon 모듈은 Node.js 애플리케이션 개발 과정에서 소스 코드가 수정되면 자동으로 애플리케이션을 재시작하는 모듈로, \cd{node} 명령어로 실행했을 때에 비해 번거로움이 줄어든다.
\newpage

\begin{shellenv}{shell:npm-run-script}{npm run script}\begin{verbatim}
$ npm run start
\end{verbatim}
\end{shellenv}

\cd{scripts} 항목에는 명령어와 그 이름을 미리 작성하여 등록할 수 있으며, \cd{npm run <script-name>} 형태로 해당 명령어를 실행할 수 있다. \coderef{code:project-package-json}에는 nodemon으로 서버를 실행하는 명령어가 start 스크립트로 등록되어 있고, 이는 \shellref{shell:npm-run-script}\와 같이 실행할 수 있다.

\begin{codeenv}{code:dotenv}{Writing .env(dotenv) file}\begin{verbatim}
MODE=<run-mode>

PORT=<web-app-port>

SESSION_SECRET=<session-secret>

DB_HOST=<db-hostname>
DB_PORT=<db-port>
DB_USER=<db-user>
DB_PASS=<db-password>
DB_NAME=<db-name>
\end{verbatim}
\end{codeenv}

./.env 파일을 생성하고 \coderef{code:dotenv}\와 같이 작성하여 서버의 설정 파일인 dotenv 파일을 작성한다. \cd{<run-mode>}는 실행하는 모드로, 개발 과정에서는 \cd{dev}로 작성한다. \cd{<web-app-port>}는 서버를 실행할 포트 번호, \cd{<session-secret>}은 세션 ID를 생성(\coderef{code:express-session-config})하기 위한 key이다. \cd{<db-host>}, \cd{<db-port>}, \cd{<db-user>}, \cd{<db-password>}, \cd{<db-name>}에는 각각 사용할 DB의 hostname, 포트, 사용자 이름, 사용자 비밀번호, 이름을 작성한다. 이때 hostname과 포트의 기본값은 localhost와 3306이다.

.env 파일은 .gitignore 파일에 의해 추적되지 않으며, 이렇게 실행 환경 등에 따라 달라지는 설정 등은 .env와 같은 별도의 파일에 작성하여야 한다.

\begin{shellenv}{shell:project-structure-root}{Project Structure (/)}\begin{verbatim}
$ tree -L 1 ./
./
├── package-lock.json
├── package.json
├── public                    : front-end static files (CSS, JS, etc)
├── schema                    : database schema
├── src                       : server source codes
└── views                     : pug templates
\end{verbatim}
\end{shellenv}

\shellref{shell:project-structure-root}\는 프로젝트 최상위 디렉토리를 나타낸 것이다. public/ 디렉토리에는 프론트엔드에서 필요한 CSS와 JS 파일, schema/ 디렉토리에는 DB 테이블을 생성하는 SQL 파일, src/ 디렉토리에는 서버의 소스코드, views/ 디렉토리에는 pug로 작성된 템플릿 파일이 있다.

\begin{shellenv}{shell:project-structure-views}{Project Structure (views/)}\begin{verbatim}
$ tree ./views/
./views/
├── articles                  : pages associated with articles
│   ├── details.pug           : article content page
│   ├── editor.pug            : article editor page
│   └── index.pug             : article list page
├── auth                      : pages associated with authorization
│   ├── sign-in.pug           : sign in page
│   └── sign-up.pug           : sign up page
├── error.pug                 : error page
├── index.pug                 : homepage
└── layout.pug                : layout
\end{verbatim}
\end{shellenv}

views/ 디렉토리 내에는 pug 템플릿 파일이 있으며, templates/layout.pug를 제외한 모든 템플릿 파일에는 첫 두줄에 템플릿 파일이 필요로 하는 인자가 주석으로 명시되어 있다.

\begin{shellenv}{shell:project-structure-src}{Project Structure (src/)}\begin{verbatim}
$ tree ./src
./src
├── app.js                    : server configuration
├── env.js                    : dotenv parser
├── index.js                  : server initialization
└── lib                       : library modules
    ├── authentication.js     : password generation and verification
    ├── database.js           : database query processor
    └── error-handler.js      : error handler
\end{verbatim}
\end{shellenv}

\shellref{shell:project-structure-src}\는 src/ 디렉토리의 구조이다. 먼저 src/lib/database.js에는 \cd{runQuery} 함수(\coderef{code:runquery-prepared-statement})가 구현되어 있고, src/lib/authentication.js에는 비밀번호를 생성하고 인증하는 함수(\coderef{code:generating-password}, \coderef{code:verifying-password})가 구현되어 있다.

src/lib/error-handler.js에는 HTTP 요청 처리 과정에서 발생한 모든 에러를 일괄적으로 처리하는 \cd{errorHandler} 함수가 구현되어 있다. 이 함수는 400 Bad Request, 401 Unauthorized, 404 Not Found에 해당하는 에러들을 각각 적절히 처리하고, 그 외의 에러들은 500 Internal Server Error 상태를 응답으로 보내어 처리한다. 4개의 인자를 갖는 \cd{errorHandler} 미들웨어를 가장 마지막에 등록하면 다른 미들웨어에서 에러가 발생하였을 때 바로 \cd{errorHandler}가 실행된다. 이러한 에러 처리 방식을 이용하면 다른 미들웨어에서 발생한 에러를 즉시 처리하지 않아도 되고, 200 OK가 아닌 다른 상태에 해당하는 응답을 보낼 때 고의로 원하는 상태에 해당하는 에러를 발생시킬 수도 있어 코드의 가독성을 향상시킨다.

src/env.js에서는 .env 파일을 분석하여 각 값을 \cd{process.env} 객체에 저장한다. 예를 들어 .env 파일의 \cd{PORT} 값은 \cd{process.env.PORT}에 할당된다. src/app.js에서는 서버 객체를 생성하고, view engine을 pug, pug 템플릿 파일의 디렉토리를 views/, 프론트엔드 정적 파일의 디렉토리를 public/으로 설정한 뒤 서버에서 사용할 각종 미들웨어를 등록한다. src/index.js에서는 src/env.js와 src/app.js 모듈을 import하여 .env 파일을 분석하고 서버 객체를 생성한 뒤 서버를 실행한다.

\subsection*{Design Pattern and Objectives}

\sectref{sect:implementing-daos}에서는 src/DAO/ 디렉토리에 DAO를 구현한다. DAO는 Data Access Object의 약자로, DB를 통해 데이터를 조회하거나 조작하는 작업을 전담하는 객체이다. \sectref{sect:implementing-user-routes}과 \sectref{sect:implementing-article-routes}에서는 각 라우트에 대한 controller 함수들을 구현한다.

이렇게 model(DAO), view(pug 템플릿), controller의 세 부분으로 나누어 소프트웨어를 설계하는 방식을 MVC 패턴(\figref{fig:mvc-pattern})이라고 한다. Model에 해당하는 DAO에서는 오로지 특정 조건을 만족하는 데이터를 DB에서 가져오거나 데이터를 삽입하는 역할을 하고, view에 해당하는 pug 템플릿은 오로지 사용자에게 보여지는 영역, 즉 front-end 역할만 한다. 그리고 controller는 model을 적절히 활용하여 HTTP 요청을 처리하고, 그 결과를 view를 통해 클라이언트에 전달한다.

\figures{fig:mvc-pattern}{MVC Pattern}{
    \fig{images/implementing-web-application/mvc-pattern.png}{.7}
}

이러한 MVC 패턴과 같은 디자인 패턴은 프로젝트에서 각 코드의 역할을 명확히 구분하고 분류하여 생산성의 향상을 가져오고, 추상화 수준을 높여 코드의 재사용성을 높이고 유지 및 보수를 용이하게 한다. 따라서 프로젝트의 성격에 따라 적절한 디자인 패턴을 정하고 그 패턴을 준수하는 것은 중요하다.
