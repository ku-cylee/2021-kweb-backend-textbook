\section{Setting Development Environment} \label{sect:setting-development-environment}

\subsection*{Visual Studio Code}
웹 개발을 보다 간편하게 할 수 있도록 개발 환경을 구축해보자. 준회원 스터디에서는 널리 이용되는 소스 코드 편집기인 Visual Studio Code를 사용한다. 물론, Atom이나 Vim 등 본인이 더 선호하는 편집기가 있다면 사용하여도 무관하나, 사용 과정에서 문제가 발생했을 때 도움을 제공하기는 힘들다는 점을 미리 말해둔다.

VS Code 홈페이지인 https://code.visualstudio.com/\#alt-downloads에 접속하여 본인의 운영체제에 맞게 설치한다. 

\subsection*{Chrome / Firefox}
Chrome과 Firefox는 널리 사용되는 대표적인 인터넷 브라우저이다. Internet Explorer 역시 널리 알려진 인터넷 브라우저이나, 지속적으로 업데이트되는 웹 표준을 제때 따라오지 못해 개발 시에 다소 어려움을 유발하고, Chrome이나 Firefox가 더 편리한 디버깅 도구를 제공하므로, Chrome이나 Firefox를 사용하는 것을 권장한다. 본 스터디에서는 Chrome을 주로 사용한다. 
