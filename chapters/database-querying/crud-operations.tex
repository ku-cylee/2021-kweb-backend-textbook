\section{CRUD Operations}\label{sect:crud-operations}

\sectref{sect:intro-to-database}에서 DB가 반드시 갖추어야 하는 CRUD 기능에 대해 알아보았다. 이번 장에서는 SQL문을 이용하여 CRUD 기능에 해당하는 기능을 수행해본다.

\subsection*{INSERT}

CRUD 기능 중 Create(생성)에 해당하는 기능은 \cd{INSERT} 키워드를 통해 수행된다.

\begin{sqlenv}{sql:insert}{SQL \cd{INSERT} syntax}\begin{verbatim}
INSERT INTO <table-name> (<c1>, <c2>, ..., <cn>) VALUES (<v1>, <v2>, ..., <vn>)
\end{verbatim}
\end{sqlenv}

\sqlref{sql:insert}\는 \cd{INSERT} 키워드를 사용한 SQL문의 기본적인 형태로, \cd{table-name} 테이블에 \cd{c1} column의 값이 \cd{v1}, \cd{c2} column의 값이 \cd{v2}, \ldots, \cd{cn} column의 값이 \cd{vn}인 row를 생성하는 SQL문이다.

\begin{shellenv}{shell:sql-insert-example}{\cd{INSERT} Example}\begin{verbatim}
> INSERT INTO `departments` (`name`, `code`) VALUES ('Liberal Arts', 13);
\end{verbatim}
\end{shellenv}

\shellref{shell:sql-insert-example}\은 \cd{INSERT} 키워드를 이용하여 \cd{departments} 테이블에 \cd{name}은 ``Liberal Arts'', \cd{code}는 13인 row를 삽입하는 예제이다. 이때 \cd{id}는 \cd{AUTO\_INCREMENT} 옵션이 있어 값을 명시하지 않아도 그 값이 알아서 추가된다. 이렇게 기본값이 존재하는 column에는 값을 삽입하지 않으면 기본값이 생성된다.

\begin{shellenv}{shell:insert-without-column-names}{\cd{INSERT} Without Column Names Example}\begin{verbatim}
> INSERT INTO `departments` VALUES (4, 'Liberal Arts', 13);
\end{verbatim}
\end{shellenv}

모든 column의 값을 명시하여 row를 생성하는 경우 \shellref{shell:insert-without-column-names}\와 같이 column 이름을 작성하지 않고 column의 순서대로 값을 나열하여 생성할 수 있다.

\begin{shellenv}{shell:insert-with-default-values}{\cd{INSERT} With Default Values Example}\begin{verbatim}
> INSERT INTO `departments` VALUES (DEFAULT, 'Liberal Arts', 13);
\end{verbatim}
\end{shellenv}

기본값이 있는 column에 column 이름과 값을 생략할 때는 \shellref{shell:insert-with-default-values}\와 같이 \cd{DEFAULT} 키워드를 이용하여 생성할 수 있다.
\newpage

\subsection*{SELECT}

CRUD 기능 중 Read(조회)에 해당하는 기능은 \cd{SELECT} 키워드를 통해 수행된다.

\begin{sqlenv}{sql:select}{SQL \cd{SELECT} Syntax}\begin{verbatim}
SELECT <c1>, <c2>, ..., <cn> FROM <table-name> WHERE <condition>
\end{verbatim}
\end{sqlenv}

\sqlref{sql:select}\는 \cd{SELECT} 키워드를 사용한 SQL문의 기본적인 형태로, \cd{table-name} 테이블에서 \cd{condition}을 만족하는 row들의 \cd{c1}, \cd{c2}, ..., \cd{cn} column의 값을 조회하는 SQL문이다.

\begin{shellenv}{shell:sql-select-example}{\cd{SELECT} Example}\begin{verbatim}
> SELECT `stdnt_num`, `name` FROM `students` WHERE `grade` = 'A' AND `credits` > 52;
\end{verbatim}
\end{shellenv}

\shellref{shell:sql-select-example}\은 \cd{SELECT} 키워드를 이용하여 \cd{students} 테이블에서 \cd{grade} 값이 ``A''이고, \cd{credits}의 값이 52보다 큰 row 중 \cd{stdtn\_num}과 \cd{name} column의 값을 조회하는 예제이다. 이렇게 조건(condition) 부분은 column 이름과 값의 비교 형태로 작성될 수 있고, 그러한 조건들의 조합으로 작성될 수 있다.

\begin{shellenv}{shell:sql-select-all-columns}{\cd{SELECT} All Columns Example}\begin{verbatim}
> SELECT * FROM `students` WHERE `grade` = 'A' AND `credits` > 52;
\end{verbatim}
\end{shellenv}

주어진 조건을 만족하는 row의 모든 column 값을 조회할 때는 \shellref{shell:sql-select-all-columns}\와 같이 column 이름을 모두 작성하지 않고 \cd{*} 문자로 대신할 수 있다.

\begin{shellenv}{shell:sql-select-without-condition}{\cd{SELECT} Without Condition Example}\begin{verbatim}
> SELECT * FROM `students`;
\end{verbatim}
\end{shellenv}

테이블 내 모든 데이터를 조회할 때는 \shellref{shell:sql-select-without-condition}\과 같이 \cd{WHERE} clause를 생략할 수 있다.

\subsection*{UPDATE}

CRUD 기능 중 Update(수정)에 해당하는 기능은 \cd{UPDATE} 키워드를 통해 수행된다.

\begin{sqlenv}{sql:update}{SQL \cd{UPDATE} sytnax}\begin{verbatim}
UPDATE <table-name> SET <c1>=<v1>, <c2>=<v2>, ..., <cn>=<vn> WHERE <condition>
\end{verbatim}
\end{sqlenv}

\sqlref{sql:update}\는 \cd{UPDATE} 키워드를 사용한 SQL문의 기본적인 형태로, \cd{table-name} 테이블에서 \cd{condition}을 만족하는 row들의 column 중 \cd{c1} column의 값을 \cd{v1}, \cd{c2} column의 값을 \cd{v2}, ..., \cd{cn} column의 값을 \cd{vn}으로 수정하는 SQL문입니다.

\begin{shellenv}{shell:sql-update-example}{\cd{UPDATE} Example}\begin{verbatim}
> UPDATE `students` SET `credits` = 73, `is_attending` = 0 WHERE `dptmt` = 32;
\end{verbatim}
\end{shellenv}

\shellref{shell:sql-update-example}\은 \cd{UPDATE} 키워드를 이용하여 \cd{students} 테이블에서 \cd{dptmt}의 값이 32인 모든 row의 \cd{credits} 값을 73, \cd{is\_attending} 값을 0으로 수정하는 예제이다.

앞의 \cd{SELECT} 키워드와 마찬가지로 \cd{WHERE} clause를 생략하여 테이블 내의 모든 데이터를 일괄적으로 수정할 수 다.

\subsection*{DELETE}

CRUD 기능 중 Delete(삭제)에 해당하는 기능은 \cd{DELETE} 키워드를 통해 수행된다.

\begin{sqlenv}{sql:delete}{SQL \cd{DELETE} Syntax}\begin{verbatim}
DELETE FROM <table-name> WHERE <condition>
\end{verbatim}
\end{sqlenv}

\sqlref{sql:delete}\는 \cd{DELETE} 키워드를 사용한 SQL문의 기본적인 형태로, \cd{table-name} 테이블에서 \cd{condition}을 만족하는 모든 row를 삭제하는 SQL문이다.

\begin{shellenv}{shell:sql-delete-example}{\cd{DELETE} Example}\begin{verbatim}
> DELETE FROM `students` WHERE `dptmt` = 13;
\end{verbatim}
\end{shellenv}

\shellref{shell:sql-delete-example}\은 \cd{DELETE} 키워드를 이용하여 \cd{students} 테이블에서 \cd{dptmt}의 값이 13인 row를 모두 삭제하는 SQL문이다. \cd{DELETE} 키워드를 사용할 때 \cd{WHERE} clause를 생략하면 테이블 내의 모든 데이터가 삭제되므로 유의하여야 한다.

\subsection*{Summary}

이번 장에서는 CRUD 기능을 수행하는 SQL문을 작성하는 방법에 대해 학습하였다.

\begin{sqlenv}{sql:crud-operations}{SQL CRUD Operations Syntaxes}\begin{verbatim}
INSERT INTO <table-name> (<c1>, <c2>, ..., <cn>) VALUES (<v1>, <v2>, ..., <vn>)
SELECT <c1>, <c2>, ..., <cn> FROM <table-name> WHERE <condition>
UPDATE <table-name> SET <c1>=<v1>, <c2>=<v2>, ..., <cn>=<vn> WHERE <condition>
DELETE FROM <table-name> WHERE <condition>
\end{verbatim}
\end{sqlenv}

\subsection*{Aggregate Functions}

앞의 5.3장 실습에서 사용된 테이블에서, 특정 강의를 수강하는 학생들의 수를 구하는 방법을 생각해봅시다. \cd{SELECT}를 이용하여 학생들의 데이터를 모두 가져온 뒤, row의 수를 세면 학생들의 수를 구할 수 있습니다. 그러나 SQL문의 aggregate 함수를 이용하면, 이러한 계산을 직접 수행할 필요가 없습니다.

Aggregate 함수란, 여러 row로 이루어진 데이터 그룹을 하나의 값으로 표현할 수 있는 함수로, 평균을 구하는 함수 \cd{Avg}, 총합을 구하는 함수 \cd{Sum}, 데이터의 개수를 구하는 함수 \cd{Count}, 최솟값을 구하는 함수 \cd{Min}, 최댓값을 구하는 함수 \cd{Max} 등이 있습니다. 예를 들어, \shellref{shell:aggregate-function}\은 credit이 4인 과목의 개수를 구하는 SQL문입니다.

\begin{shellenv}{shell:aggregate-function}{Example of Using Aggregate Function}\begin{verbatim}
> SELECT Count(id) FROM courses WHERE credit = 4;
\end{verbatim}
\end{shellenv}

\subsection*{Advanced Select Options}

% https://dev.mysql.com/doc/refman/8.0/en/select.html
% GROUP BY, HAVING / ORDER BY, ASC, DESC / LIMIT, OFFSET
