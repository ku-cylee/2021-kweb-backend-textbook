\section{CRUD Operations}\label{sect:crud-operations}

\sectref{sect:intro-to-database}에서 데이터를 다루는 프로그램이 기본적으로 갖추고 있어야 하는 데이터 처리 기능인 CRUD 기능에 대해 학습하였습니다. 이번 장에서는 SQL문을 이용하여 CRUD 기능에 해당하는 querying을 해볼 것입니다.

\subsection*{\cd{INSERT}}

CRUD 기능 중 Create(생성)에 해당하는 기능은 \cd{INSERT} 키워드를 이용하여 수행합니다.

\begin{sqlenv}{sql:insert}{SQL \cd{INSERT} syntax}\begin{verbatim}
INSERT INTO <table-name> (<c1>, <c2>, ..., <cn>) VALUES (<v1>, <v2>, ..., <vn>)
\end{verbatim}
\end{sqlenv}

\sqlref{sql:insert}\는 INSERT 키워드의 기본적인 형태로, \cd{table-name} 테이블에 \cd{c1} column의 값이 \cd{v1}, \cd{c2} column의 값이 \cd{v2}, ..., \cd{cn} column의 값이 \cd{vn}인 row를 삽입하는 SQL문입니다.

\begin{shellenv}{shell:sql-insert-example}{\cd{INSERT} Example}\begin{verbatim}
> INSERT INTO departments (name, code) VALUES ('Liberal Arts', 13);
\end{verbatim}
\end{shellenv}

\shellref{shell:sql-insert-example}\은 \cd{INSERT} 키워드를 이용하여 \cd{departments} 테이블에 \cd{name}은 Liberal Arts, \cd{code}는 13인 row를 삽입하는 예제입니다. 이때 \cd{id}는 \cd{AUTO\_INCREMENT} 옵션이 걸려있어 값을 제시해주지 않아도 DBMS가 자동으로 \cd{id} 값을 추가합니다. 이렇게 기본값이 존재하는 column에 값을 삽입하지 않으면 기본값이 삽입됩니다.

\begin{shellenv}{shell:insert-without-column-names}{\cd{INSERT} Without Column Names Example}\begin{verbatim}
> INSERT INTO departments VALUES (4, 'Liberal Arts', 13);
\end{verbatim}
\end{shellenv}

테이블의 모든 column에 값을 삽입하는 경우에는 \shellref{shell:insert-without-column-names}\와 같이 column name을 작성하지 않고, column의 순서대로 값을 나열하여 SQL문을 작성할 수 있습니다.

\begin{shellenv}{shell:insert-with-default-values}{\cd{INSERT} With Default Values Example}\begin{verbatim}
> INSERT INTO departments VALUES (DEFAULT, 'Liberal Arts', 13);
\end{verbatim}
\end{shellenv}

테이블의 column 이름을 생략하면서, 기본값이 있는 column에 기본값을 삽입하고자 할 때는 \shellref{shell:insert-with-default-values}\와 같이 \cd{DEFAULT} 키워드를 삽입하여 SQL문을 작성할 수 있습니다.

\subsection*{\cd{SELECT}}

CRUD 기능 중 Read(조회)에 해당하는 기능은 \cd{SELECT} 키워드를 이용하여 수행합니다.

\begin{sqlenv}{sql:select}{SQL \cd{SELECT} Syntax}\begin{verbatim}
SELECT <c1>, <c2>, ..., <cn> FROM <table-name> WHERE <condition>
\end{verbatim}
\end{sqlenv}

\sqlref{sql:select}\는 \cd{SELECT} 키워드의 기본적인 형태로, \cd{table-name} 테이블에서 \cd{condition}을 만족하는 row들의 \cd{c1}, \cd{c2}, ..., \cd{cn} column의 값을 조회하는 SQL문입니다.

\begin{shellenv}{shell:sql-select-example}{\cd{SELECT} Example}\begin{verbatim}
> SELECT stdnt_num, name FROM students WHERE grade = 'A' AND total_credit > 52;
\end{verbatim}
\end{shellenv}

\shellref{shell:sql-select-example}\은 \cd{SELECT} 키워드를 이용하여 \cd{students} 테이블에서 \cd{grade} 값이 A이고, \cd{total\_credit}의 값이 52보다 큰 row 중 \cd{stdtn\_num}과 \cd{name}의 값을 조회하는 예제입니다. 이렇게 조건(condition) 부분은 column 이름과 값의 비교 형태로 작성될 수 있고, 그러한 조건들의 조합으로 작성될 수 있습니다.

\begin{shellenv}{shell:sql-select-all-columns}{\cd{SELECT} All Columns Example}\begin{verbatim}
> SELECT * FROM students WHERE grade = 'A' AND total_credit > 52;
\end{verbatim}
\end{shellenv}

주어진 조건을 만족하는 row의 모든 column 값을 조회하고 싶은 경우 \shellref{shell:sql-select-all-columns}\와 같이 column 이름을 모두 작성하지 않고, \cd{*} 문자로 대신할 수 있습니다.

\begin{shellenv}{shell:sql-select-without-condition}{\cd{SELECT} Without Condition Example}\begin{verbatim}
> SELECT * FROM students;
\end{verbatim}
\end{shellenv}

테이블 내 모든 데이터를 조회하고자 하는 경우 \shellref{shell:sql-select-without-condition}\과 같이 \cd{WHERE} clause를 생략하여 SQL문을 작성할 수 있습니다.

\subsection*{\cd{UPDATE}}

CRUD 기능 중 Update(수정)에 해당하는 기능은 \cd{UPDATE} 키워드를 이용하여 수행합니다.

\begin{sqlenv}{sql:update}{SQL \cd{UPDATE} sytnax}\begin{verbatim}
UPDATE <table-name> SET <c1>=<v1>, <c2>=<v2>, ..., <cn>=<vn> WHERE <condition>
\end{verbatim}
\end{sqlenv}

\sqlref{sql:update}\는 \cd{UPDATE} 키워드의 기본적인 형태로, \cd{table-name} 테이블에서 \cd{condition}을 만족하는 row들의 column 중 \cd{c1} column의 값을 \cd{v1}, \cd{c2} column의 값을 \cd{v2}, ..., \cd{cn} column의 값을 \cd{vn}으로 수정하는 SQL문입니다.

\begin{shellenv}{shell:sql-update-example}{\cd{UPDATE} Example}\begin{verbatim}
> UPDATE students SET telephone = '010-1234-5678', total_credit = 73 WHERE dptmt = 32;
\end{verbatim}
\end{shellenv}

\shellref{shell:sql-update-example}\은 \cd{UPDATE} 키워드를 이용하여 \cd{students} 테이블에서 \cd{dptmt}의 값이 32인 모든 row에 대해 \cd{telephone}의 값을 010-1234-5678, \cd{total\_credit}의 값을 73으로 수정하는 예제입니다.

테이블 내의 모든 데이터에 대해 수정하고자 하는 경우 앞의 \cd{SELECT} 키워드와 마찬가지로 \cd{WHERE} clause를 생략하여 SQL문을 작성할 수 있습니다.

\subsection*{\cd{DELETE}}

\begin{sqlenv}{sql:delete}{SQL \cd{DELETE} Syntax}\begin{verbatim}
DELETE FROM <table-name> WHERE <condition>
\end{verbatim}
\end{sqlenv}

\sqlref{sql:delete}\는 \cd{DELETE} 키워드의 기본적인 형태로, \cd{table-name} 테이블에서 \cd{condition}을 만족하는 모든 row를 삭제하는 SQL문입니다.

\begin{shellenv}{shell:sql-delete-example}{\cd{DELETE} Example}\begin{verbatim}
> DELETE FROM students WHERE dptmt=13;
\end{verbatim}
\end{shellenv}

\shellref{shell:sql-delete-example}\은 \cd{DELETE} 키워드를 이용하여 \cd{students} 테이블에서 \cd{dptmt}의 값이 13인 모든 row를 삭제하는 SQL문입니다. \cd{DELETE} 키워드를 사용할 때 \cd{WHERE} clause를 생략하면 테이블 내의 모든 데이터를 삭제하므로, 유의하여 사용하여야 합니다.

\subsection*{Summary}

데이터를 다루는 가장 기본적인 기능인 CRUD 기능을 수행하는 SQL문을 작성하는 방법에 대해 학습하였습니다.

\begin{sqlenv}{sql:crud-operations}{SQL CRUD Operations Syntaxes}\begin{verbatim}
INSERT INTO <table-name> (<c1>, <c2>, ..., <cn>) VALUES (<v1>, <v2>, ..., <vn>)
SELECT <c1>, <c2>, ..., <cn> FROM <table-name> WHERE <condition>
UPDATE <table-name> SET <c1>=<v1>, <c2>=<v2>, ..., <cn>=<vn> WHERE <condition>
DELETE FROM <table-name> WHERE <condition>
\end{verbatim}
\end{sqlenv}
