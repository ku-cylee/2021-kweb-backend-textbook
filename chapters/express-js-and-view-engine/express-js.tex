\section{Express.js}\label{sect:express-js}

\subsection*{Framework and Express.js}

\chapref{ch:introduction-to-back-end}에서 웹 서버가 수행해야 하는 기능과 웹 서버 구현에 필요한 기초적인 개념을 학습하였고, 유효한 기능을 수행하는 간단한 서버를 http 모듈을 이용하여 구현해보았다.

그러나 http 모듈은 HTTP 통신을 통해 데이터를 주고받는 메서드를 제공할 뿐, 그 외에 웹 서버가 요구하는 수많은 기능을 제공하지는 않는다. 이러한 기능에는 라우팅(routing), query 파싱, 정적 파일 로딩, 에러 처리 등이 있다. 물론 이러한 기능들을 직접 구현하는 것이 불가능하지는 않지만, 통상적인 웹 서버에서 공통적으로 요구하는 기능이므로 매번 구현하는 것은 비효율적이다.

이러한 종류의 소프트웨어를 개발할 때는 효율적인 개발을 위해 프레임워크를 사용하곤 한다. 프레임워크(framework)는 개발자가 구현하고자 하는 기능의 구현에만 집중할 수 있도록 공통적이며 기본적으로 요구되는 기능들을 미리 갖춰둔 기본 뼈대이다. 웹 서버는 프레임워크를 사용하여 구현하는 것이 일반적이기 때문에 수많은 웹 프레임워크가 존재하며, Python의 Django와 Flask, Java와 Kotlin의 Spring, PHP의 Laravel 등은 대표적인 웹 프레임워크이다.

Node.js에서 동작하는 JS의 웹 프레임워크에는 대표적으로 Express.js, Koa.js, hapi.js 등이 있다. Express.js는 가장 보편적으로 사용되는 프레임워크이며, Koa.js는 Express.js의 기존 개발팀이 Express.js의 소유권이 이전되면서 장기적인 유지 및 보수가 제대로 이루어지지 않을 것을 우려해 만든 프레임워크로, 최근에는 Express.js로 구현된 프로젝트들이 Koa.js로 migration되는 추세이다. 그러나 Express.js의 점유율은 여전히 높으며 그동안 널리 사용되어왔기 때문에 포럼 등 온라인에서 얻을 수 있는 자료는 Express.js가 훨씬 많으므로, 본 교재에서는 Express.js 프레임워크를 사용하여 웹 서버를 구현한다.

\subsection*{Simple Express.js Server}

Express.js 프레임워크를 이용하여 간단한 웹 서버를 구현해보자.

\begin{shellenv}{shell:create-express-server}{Create Express Server}\begin{verbatim}
$ npm init -y
$ npm install express
\end{verbatim}
\end{shellenv}

먼저 \shellref{shell:create-express-server}\와 같이 Node.js 프로젝트를 생성하고, express 모듈을 설치한 뒤, 적절한 gitignore 파일을 생성한다. \cd{npm init} 명령은 Node.js 프로젝트에 관한 정보를 담은 package.json 파일이 생성되며, \cd{-y} 옵션을 붙이면 기본 설정을 그대로 사용한다. 프로젝트에서 사용하는 모듈을 npm을 이용하여 설치하면 모듈들의 이름과 버전이 package.json 파일의 \cd{dependencies} 속성에 기록되어 다른 개발 환경에서 프로젝트를 실행할 때 모듈 파일들을 직접 복사하지 않고도 모듈을 설치할 수 있고, 그렇기 때문에 모듈들이 저장된 node\_modules 디렉토리는 gitignore 파일에 의해 추적되지 않는다.

이제 index.js를 생성하고, \coderef{code:simple-express-server}의 코드를 작성한 후 실행한다.

\begin{codeenv}{code:simple-express-server}{Simple Express Server}\begin{verbatim}
const express = require('express');
const app = express();
const port = 3000;

app.get('/', (req, res) => res.send('Hello World!'));

app.listen(port, () => console.log(`Server listening on port ${port}!`));
\end{verbatim}
\end{codeenv}

\coderef{code:simple-express-server}\는 \coderef{code:nodejs-simple-web-server}(\pageref{code:nodejs-simple-web-server}쪽)\를 Express.js로 다시 작성한 코드로, 거의 같은 기능을 수행하면서 코드는 매우 간결해진 것을 확인할 수 있다.

코드를 살펴보면, 먼저 express 모듈을 이용하여 웹 서버의 설계에 대한 모든 정보를 담고있는 \cd{app} 객체를 생성한다. 그리고 \cd{app} 객체의 \cd{get} 메서드를 통해 서버에 GET / 메서드로 요청이 들어오면 ``Hello World!'' 문자열을 응답으로 보낸다. 반면 \coderef{code:nodejs-simple-web-server}의 서버는 모든 요청 라우트에 대한 응답이 구현되어 있지만, \coderef{code:simple-express-server}의 서버는 GET / 라우트에 대한 응답만 구현되어 있고 다른 라우트로 요청을 보내면 404 Not Found 에러를 응답한다는 차이점이 있다.

\subsection*{Routing}

웹 애플리케이션에서 요청 경로와 메서드, 즉 라우트는 사용자가 서버에게 요청하는 작업, 즉 서버가 수행해야 하는 작업을 나타내는 매우 중요한 요소이다. 따라서 서버 개발자는 요청 라우트에 따라 수행할 작업을 적절히 나누어야 하고, 이러한 작업을 라우팅(routing)이라고 한다.

예를 들어 게시물을 작성하고 열람할 수 있는 웹 애플리케이션의 라우트는 다음과 같이 설계할 수 있다.

\begin{itemize}
    \item GET /auth/login: 로그인을 할 수 있는 페이지 응답
    \item POST /auth/login: 아이디와 비밀번호를 받아서 사용자의 인증 요청을 처리하고, 그 결과를 응답
    \item GET /board/page/1: 게시판의 첫 번째 페이지 응답
    \item GET /post/3: 서버에 저장된 게시물 중 3번 게시물을 찾아서 응답
    \item POST /post: 제목과 내용 등을 받아서 서버에 새로운 게시물을 생성하고 저장
\end{itemize}

Express.js는 이러한 라우팅을 할 수 있는 라우팅 메서드를 제공한다. \cd{app} 객체의 \cd{get}, \cd{post} 메서드 등은 GET, POST 메서드 요청을 처리하고, \cd{put}, \cd{delete} 등의 메서드 역시 유사한 기능을 수행한다. 라우팅 메서드는 첫 번째 인자로 매칭하고자 하는 경로를 정규표현식 형태로 받고, 이후 라우트에 매칭되었을 때 실행할 함수, 즉 controller 함수를 순서대로 인자로 받는다. 예를 들어, \coderef{code:simple-express-server}에서 \cd{app.get('/', ...)} 코드는 GET 메서드, / 경로로 들어온 요청을 처리하는 코드이다.

\subsection*{Request and Response Objects}

\sectref{sect:hypertext-transfer-protocol}에서 다루었듯 HTTP 요청과 응답에는 메서드, 경로, 상태 코드뿐만 아니라 body, header, cookie 등의 정보를 담고 있다. 이러한 정보에 따라 수행되는 작업이나 작업에 사용되는 데이터의 값이 달라질 수 있으므로 controller 함수에서는 이러한 정보에 접근할 수 있어야 한다.

\coderef{code:simple-express-server}에서 컨트롤러 함수는 두 인자 \cd{req}와 \cd{res}를 받는다. 두 인자는 각각 HTTP 요청의 각 정보를 담고 있는 요청 객체와 HTTP 응답의 각 정보를 담고 있는 응답 객체이다. 요청 객체에는 HTTP 요청을 분석한 데이터가 \cd{query}, \cd{body}, \cd{header}, \cd{cookie} 등의 속성에 할당되어 있어 개발자가 controller 함수의 구현 과정에서 필요한 값에 접근할 수 있다. 응답 객체에는 응답으로 보낼 상태 코드를 정할 수 있는 \cd{status} 메서드, 응답 내용을 정하고 보낼 수 있는 \cd{send} 메서드, 다른 경로로 리다이렉트 시킬 수 있는 \cd{redirect} 메서드 등이 있다.

\subsection*{Middleware}

Express.js로 생성된 애플리케이션은 일련의 미들웨어를 순차적으로 실행하는 애플리케이션이라고 요약할 수 있다. Express.js에서 미들웨어(middleware)는 요청 객체와 응답 객체, 그리고 요청 - 응답 처리 과정 중 그 다음 미들웨어 함수에 대한 접근 권한을 갖는 함수이다.

\begin{codeenv}{code:middleware-example}{Middleware Example}\begin{verbatim}
const express = require('express');
const app = express();
const port = 3000;

app.use((req, res, next) => {
    const randomNumber = Math.floor(Math.random() * 10);
    console.log(`Random Number: ${randomNumber}`);
    if (randomNumber % 3) return next();
    else return res.sendStatus(403);
});

app.use((req, res, next) => {
    const { method, path } = req;
    console.log(`${method} ${path}`);
    return next();
});

app.get('/', (req, res) => res.send('GET /'));

app.post('/', (req, res) => res.send('POST /'));

app.listen(port, () => console.log(`Server listening on port ${port}!`));
\end{verbatim}
\end{codeenv}

\coderef{code:middleware-example}\은 두 미들웨어를 \cd{use} 메서드를 이용하여 웹 서버 애플리케이션에 bind하는 코드이다. 첫 번째 미들웨어는 0 이상 10 미만 랜덤 정수 \cd{randomNumber}를 뽑아 출력하고, \cd{randomNumber}가 3의 배수이면 403 Forbidden 에러를 HTTP 응답으로 보내고, 3의 배수가 아니면 다음 함수(\cd{next})를 호출한다. 두 번째 미들웨어는 요청 객체(\cd{req})에서 요청의 메서드와 경로를 가져와 라우트를 출력하고, 다음 함수를(\cd{next})를 호출한다. 그 이후 라우팅을 통해 각 요청에 맞는 응답을 보낸다.

이 예제를 실행해보면 다음과 같은 몇 가지 사실을 관찰할 수 있다. 먼저 미들웨어는 \cd{use} 메서드를 통해 bind된 순서대로 실행되며, 후술하겠지만 라우팅 메서드 역시 미들웨어로 취급되고 미들웨어 형태로 bind할 수 있다. 두 번째, 미들웨어 역시 요청 객체(\cd{req})와 응답 객체(\cd{res})에 접근할 수 있다. 세 번째, 미들웨어에서 \cd{next} 함수를 호출하면 그 다음 미들웨어가 실행되며, 그렇지 않고 응답 객체(\cd{res})를 이용하여 응답을 보내게 되면 그 이후의 미들웨어는 더 이상 실행되지 않는다.

이러한 미들웨어의 특징을 도식으로 나타내면 \figref{fig:middlewares-flow}\와 같다.

\figures{fig:middlewares-flow}{Flow of Code using Middlewares}{
    \fig{images/express-js-and-view-engine/middlewares-flow.png}{.9}
}

이러한 미들웨어의 특징과 일련의 미들웨어의 순차적 실행이라는 Express.js의 특징으로 인해 Express.js 애플리케이션에서 미들웨어는 매우 중요한 역할을 한다. 요청 객체에 대한 전처리, logging, 에러 처리, 라우팅 등은 모두 미들웨어의 형태로 작성할 수 있으며, Express.js 애플리케이션을 위해 개발된 모듈은 미들웨어 형태로 되어있는 것이 대부분이다. 특히, 미들웨어를 이용한 에러 처리는 \chapref{ch:web-application-implementation}에서 다룰 예정이다.

라우팅은 express 모듈의 \cd{Router} 객체를 이용하여 미들웨어 형태로 작성될 수 있다. 먼저 router.js를 생성하고, \coderef{code:middleware-routing-router}\와 같이 작성한다.

\begin{codeenv}{code:middleware-routing-router}{Routing with Middleware - router.js}\begin{verbatim}
const { Router } = require('express');

const router = Router();

router.get('/', (req, res) => res.send('GET /'));

router.post('/', (req, res) => res.send('POST /'));

module.exports = router;
\end{verbatim}
\end{codeenv}
\newpage

이후 index.js를 \coderef{code:middleware-routing-index}\와 같이 작성한다.

\begin{codeenv}{code:middleware-routing-index}{Routing with Middleware - index.js}\begin{verbatim}
const express = require('express');
const router = require('./router');
const port = 3000;

const app = express();

app.use('/', router);

app.listen(port, () => console.log(`Server listening on port ${port}!`));
\end{verbatim}
\end{codeenv}

이 프로그램은 라우팅 기능을 미들웨어 형태로 router.js에 구현하고, 이를 index.js에서 모듈 형태로 import하여 \cd{app} 객체에 추가하였다. 이렇게 \cd{Router} 객체와 모듈을 이용하면 라우팅과 관련된 코드를 분리하여 프로젝트를 구조화할 수 있다.

\coderef{code:middleware-routing-index}의 \cd{use} 메서드에는 두 인자가 있는데, 두 번째 인자인 미들웨어는 첫 번째 인자인 HTTP 경로로 시작하는 요청 경로에 대해서만 실행되며, 첫 번째 인자가 생략되면 모든 경로에 대해 실행된다. 즉, \cd{'/'} 인자는 /로 시작하는 경로만 \cd{router}에서 처리하라는 뜻으로, 이 값을 \cd{'/api'}로 바꾸면 GET /api, POST /api 등의 라우트로 접속해야 적절한 응답을 받을 수 있다. 여기에 router.js에서 \cd{get} 메서드의 인자를 \cd{'/'}에서 \cd{'/post'}로 바꾸면 GET /api/post로 요청을 보내야 ``GET /'' 응답을 받을 수 있다.
