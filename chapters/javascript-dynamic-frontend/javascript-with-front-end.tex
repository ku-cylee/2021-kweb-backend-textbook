\section{Javscript with Front-end} \label{sect:javascript-with-front-end}

\chapref{ch:basics-of-javascript}에서 JS의 기본 문법에 대하여 학습하였습니다. 이제 JS를 이용하여 HTML 문서를 동적으로 작동하게끔 만들 것입니다. 

\subsection*{HTML과 JS의 연동}

Front-end에서 JS는 HTML 문서를 제어하고 수정하기 위해 존재하는 언어입니다. 그렇기 때문에 CSS와 마찬가지로 HTML 문서에 연동되어야 합니다. JS 코드를 HTML 문서에서 실행하는 방법은 두 가지가 있습니다. 첫 번째 방법은 HTML 문서에서 \texttt{script} 태그를 열고, 그 내부에 JS 코드를 작성하는 것입니다. \coderef{code:js-internal}을 웹 브라우저에서 열고, 개발자 도구의 Console 탭을 열어보면 위의 문구가 출력된 것을 확인할 수 있습니다. 

\begin{codeenv}{code:js-internal}{Internal Application of JS}\begin{verbatim}


<script>
    console.log('Hello World!');
</script>

<h1>
    HTML and JS
</h1>
\end{verbatim}
\end{codeenv}

두 번째 방법은 별도의 JS 파일을 만들어, JS 파일에는 JS 스크립트만 작성하고, HTML 파일에는 웹페이지의 구조만 서술하는 방법입니다. \coderef{code:js-external-html}\과 같이 \texttt{script} 태그의 \texttt{src} 속성에 JS 파일의 주소를 명시해줍니다. 

\begin{codeenv}{code:js-external-html}{External Application of JS - HTML}\begin{verbatim}


<script type="text/javascript" src="./script.js"></script>

<h1>
    HTML and JS
</h1>
\end{verbatim}
\end{codeenv}

그리고는 script.js 파일을 열어, JS 코드를 작성해줍니다. 

\begin{codeenv}{code:js-external-js}{External Application of JS - JS}\begin{verbatim}


console.log('Hello World!');
\end{verbatim}
\end{codeenv}

\coderef{code:js-external-js}\와 같이 JS 파일을 작성하고 앞의 HTML 파일을 열어도, 개발자 도구의 Console 란에 문구가 출력된 것을 볼 수 있습니다. 다만 여기에서 주의할 점이 있습니다. 웹 브라우저는 HTML 문서의 앞부분부터 로딩하기 때문에 JS 파일을 불러오는 코드를 다른 HTML 요소들보다 앞에 작성하면, 그 뒤에 위치한 요소들을 인식하지 못하는 현상이 생깁니다. 

이를 방지하기 위해 JS를 불러오는 코드를 \texttt{head} 태그가 아닌 \texttt{body} 태그의 중간이나 끝에 작성하는 방식을 사용하게 되었고, 이는 HTML 파일의 가독성을 떨어뜨리는 요인이 되었습니다. 이러한 문제를 해결하기 위해 \texttt{async}와 \texttt{defer} 키워드를 이용하게 되었습니다. \texttt{script} 태그의 속성으로 속성값 없이 \texttt{async} 속성을 명시해주면, 해당 JS 코드는 비동기적으로 실행됩니다. 속성값 없이 \texttt{defer} 속성을 명시해주면, 해당 JS 코드는 문서 로딩이 완전히 끝난 뒤에야 실행됩니다. 
