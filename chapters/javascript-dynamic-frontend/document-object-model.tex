\section{Document Object Model (DOM)} \label{sect:document-object-model}

\textbf{문서 객체 모델(Document Object Model; DOM)}은 HTML 문서의 프로그래밍 인터페이스이자 구조를 나타내는 모델이며, DOM을 이용하여 HTML 문서로부터 원하는 HTML 요소들을 선택하고, 선택된 요소들의 정보에 접근하고 수정하는 등 HTML 문서를 제어할 수 있다. 예를 들어 DOM을 이용하여 HTML 요소의 내부 텍스트(inner text), 속성과 속성값 등을 읽거나 수정할 수 있다.

\subsection*{HTML 문서의 구조}

HTML은 원소들의 계층 구조로 되어 있어 객체로 나타내기 적합하므로 DOM은 HTML 문서를 객체 형태로 표현한다.

\begin{codeenv}{code:html-tree-struct-html}{HTML Tree Structure - HTML}\begin{verbatim}
<!doctype html>
<html>
<head>
    <title>Document Object Model</title>
    <meta name="viewport" content="width=device-width, initial-scale=1" >
</head>
<body>
    <div class="container" id="title">
        <h1>Document Object Model</h1>
    </div>
    <div class="container" id="content">
        <p style="color: red">This article is about Document Object Model.</p>
        <p>Document Object Model can be abbreviated to DOM.</p>
    </div>
</body>
</html>    
\end{verbatim}
\end{codeenv}

간단한 형태의 HTML 문서인 \coderef{code:html-tree-struct-html}\을 객체의 형태로 나타낼 수 있겠는가?

\begin{codeenv}{code:html-tree-struct-object}{HTML Tree Structure - Object}\begin{verbatim}
const htmlDocument = [
  {
    tag: 'html',
    children: [
      {
        tag: 'head',
        children: [
          {
            tag: 'title',
            text: 'Document Object Model'
          }, {
            tag: 'meta',
            attributes: {
              name: 'viewport',
              content: 'width=device-width, initial-scale=1',
            },
          },
        ],
      }, {
        tag: 'body',
        children: [
          {
            tag: 'div',
            attributes: { class: ['container'], id: 'title' },
            children: [{ tag: 'h1', text: 'Document Object Model' }],
          }, {
            tag: 'div',
            attributes: { class: ['container'], id: 'content' },
            children: [
              {
                tag: 'p',
                attributes: { style: 'color: red' },
                text: 'This article is about Document Object Model.',
              }, {
                tag: 'p',
                text: 'Document Object Model can be abbreviated to DOM.'
              },
            ],
          },
        ],
      },
    ],
  },
];
\end{verbatim}
\end{codeenv}

\coderef{code:html-tree-struct-object}\는 \coderef{code:html-tree-struct-html}\을 간단한 객체 형태로 나타낸 예시이다. (절대 DOM의 구조는 아니다.)

\figures{fig:html-tree-struct}{Tree Structure of HTML Elements}
    {\fig{images/javascript-dynamic-frontend/html-tree-struct.png}{.8}}

\coderef{code:html-tree-struct-html}\은 \figref{fig:html-tree-struct}\와 같은 구조로 표현될 수도 있다. 즉 HTML 문서는 tree로 표현될 수 있고, 모든 HTML 요소는 tree의 node로 표현될 수 있으며, \texttt{document} 객체에 접근하여 tree의 root node에 접근할 수 있다. DOM은 root node를 비롯한 모든 node, 즉 HTML 요소에 대해 어떤 요소의 정보에 접근하는 메서드, 어떤 요소를 기준으로 다른 요소를 선택할 수 있는 메서드 등을 제공한다.

이번 절에서는 DOM에서 제공하는 메서드를 이용하여 HTML 요소들을 제어하는 방법에 대해 학습한다.

\subsection*{HTML 요소 조회}

DOM에서 제공하는 메서드를 이용하여 HTML 요소를 조회(query)할 수 있다. \coderef{code:dom-example-html}\을 웹 브라우저에서 열고, 개발자 도구의 REPL Shell을 이용하여 DOM에서 제공하는 메서드를 사용해보며 HTML 문서가 어떻게 수정되는지 확인해보자.

\begin{codeenv}{code:dom-example-html}{DOM Example HTML}\begin{verbatim}
<div id="container">
    <h1>Articles List</h1>
    <div class="article-title">
        <a href="/article/5">Ch 5. JS: Dynamic Frontend</a>
    </div>
    <div class="article-title">
        <a class="anchor current" href="/article/4">Ch 4. Basics of Javascript</a>
    </div>
    <div class="article-title">
        <a href="/article/3">Ch 3. CSS: Designing HTML</a>
    </div>
    <div class="article-title">
        <a href="/article/2">Ch 2. HTML: The Basic Structure</a>
    </div>
    <div class="article-title">
        <a href="/article/1">Ch 1. Introduction to Front-end</a>
    </div>
</div>
\end{verbatim}
\end{codeenv}

먼저, 어떤 요소의 자손 요소 중 id, class, 태그, CSS 선택자 등을 기준으로 요소를 조회할 수 있다.

\begin{itemize}
    \item \texttt{getElementById(id) => HTMLElement}: 요소의 자손 요소 중 주어진 문자열을 id로 갖는 요소를 반환하는 메서드
    \item \texttt{getElementsByTagName(tagName) => HTMLCollection}: 요소의 자손 요소 중 주어진 문자열을 태그로 갖는 요소들의 유사 배열을 반환하는 메서드
    \item \texttt{getElementsByClassName(className) => HTMLCollection}: 요소의 자손 요소 중 주어진 문자열을 class로 갖는 요소들의 유사 배열을 반환하는 메서드
    \item \texttt{querySelector(selector) => HTMLElement}: 요소의 자손 요소 중 주어진 선택자로 선택되는 첫 번째 요소를 반환하는 메서드
    \item \texttt{querySelectorAll(selector) => NodeList}: 요소의 자손 요소 중 주어진 선택자로 선택되는 요소들의 유사 배열을 반환하는 메서드
\end{itemize}

유사 배열이란 for문 등으로 iteration을 수행할 수 있으나 배열은 아닌 \texttt{HTMLCollection}, \texttt{NodeList} 클래스의 인스턴스를 표현하기 위한 용어로, 일반적으로 사용되는 용어는 아니다. 두 클래스 모두 요소들의 개수를 확인할 수 있는 \texttt{length} 속성이 있으나, \texttt{HTMLCollection} 클래스와 달리 \texttt{NodeList} 클래스는 \texttt{forEach} 메서드를 갖는다.

\coderef{code:dom-query-descendents}\를 실행해보며 결과를 확인해보자. 각 표현식의 결과값은 생략한다.
\newpage

\begin{codeenv}{code:dom-query-descendents}{DOM Query Descendents Examples}\begin{verbatim}
> document.getElementById('container');
> document.getElementsByClassName('article-title');
> const aTagsList1 = document.getElementsByTagName('a');
> const aTagsList2 = document.querySelectorAll('div.article-title > a');
> aTagsList1;
> aTagsList2;
> for (const aTag of aTagsList1) console.log(aTag);
> aTagsList2.forEach(elmt => console.log(elmt));
> const title = document.getElementsByClassName('article-title')[2];
> title.getElementsByTagName('a');
\end{verbatim}
\end{codeenv}

또한, 어떤 요소를 기준으로 상대적인 위치에 있는 요소를 조회할 수 있다.

\begin{itemize}
    \item \texttt{parentElement}: 요소의 부모 요소를 값으로 갖는 속성
    \item \texttt{children}: 요소의 자식 요소들의 유사 배열을 값으로 갖는 속성
    \item \texttt{firstElementChild}: 요소의 첫 번째 자식 요소를 값으로 갖는 속성
    \item \texttt{lastElementChild}: 요소의 마지막 자식 요소를 값으로 갖는 속성
    \item \texttt{previousElementSibling}: 요소의 직전 형제 요소를 값으로 갖는 속성
    \item \texttt{nextElementSibling}: 요소의 직후 형제 요소를 값으로 갖는 속성
\end{itemize}

\coderef{code:dom-query-adjacents}\를 실행해보며 결과를 확인해보자. 각 표현식의 결과값은 생략한다.

\begin{codeenv}{code:dom-query-adjacents}{DOM Query Adjacents Examples}\begin{verbatim}
> const container = document.getElementById('container');
> const title = container.children[1];
> title.parentElement;
> title.children;
> title.firstElementChild;
> container.lastElementChild;
> title.previousElementSibling;
> title.nextElementSibling;
\end{verbatim}
\end{codeenv}

\subsection*{HTML 요소 및 속성 접근 및 수정}

요소 객체에 접근하여 속성을 읽거나 쓰고 수정할 수 있다.

\begin{itemize}
    \item \texttt{innerText}: 요소 내부의 텍스트에 접근할 수 있는 속성; 이 속성의 값을 변경하면 요소 내부 텍스트도 변경된다.
    \item \texttt{setAttribute(name, value)}: 요소의 \texttt{name} 속성의 값을 \texttt{value}로 설정(변경)하는 메서드
    \item \texttt{getAttribute(name) => string}: 요소의 \texttt{name} 속성에 대한 속성값을 반환하는 메서드
    \item \texttt{hasAttribute(name) => boolean}: 요소에 \texttt{name} 속성이 있는지 판별하는 메서드
    \item \texttt{removeAttribute(name)}: 요소에서 \texttt{name} 속성을 제거하는 메서드
    \item \texttt{style}: 요소의 CSS 스타일에 접근할 수 있는 객체; 각 CSS 속성은 \texttt{kebab-case}가 아니라 \texttt{camelCase}로 작성한다. (\coderef{code:html-elmt-attrs} 참고)
    \newpage
    \item \texttt{classList}: 요소의 클래스의 유사 배열에 접근할 수 있는 객체
    \begin{itemize}
      \item \texttt{length}: 클래스의 개수를 값으로 갖는 속성
      \item \texttt{value}: 요소의 \texttt{class} 속성값을 값으로 갖는 속성
      \item \texttt{contains(class) => boolean}: 클래스의 존재 유무를 판별하여 반환하는 메서드
      \item \texttt{add(class1, class2, ..., classN)}: 주어진 클래스들을 추가하는 메서드
      \item \texttt{remove(class1, class2, ..., classN)}: 주어진 클래스들을 제거하는 메서드
      \item \texttt{toggle(class) => boolean}: 주어진 클래스가 클래스 유사배열에 존재하면 제거하고, 존재하지 않으면 삽입하며, 주어진 클래스가 추가되었으면 \texttt{true}, 제거되었으면 \texttt{false}를 반환하는 메서드
    \end{itemize}
\end{itemize}

\begin{codeenv}{code:html-elmt-attrs}{Accessing Attributes of HTML Elements}\begin{verbatim}
> const title = document.getElementsByClassName('article-title')[1];
> const link = title.getElementsByTagName('a')[0];
> link.innerText = 'JS Grammar';
> link.getAttribute('href');
> link.setAttribute('href', '/article/40');
> link.removeAttribute('href');
> title.hasAttribute('href');
> title.style.color = 'red';
> title.style.borderTop = '1px solid black';
> link.classList;
> link.add('class1', 'class2');
> link.remove('anchor');
> link.toggle('class2');
> link.toggle('class2');
\end{verbatim}
\end{codeenv}

또한, 요소에 자식 요소를 추가하거나 제거할 수 있다.

\begin{itemize}
    \item \texttt{createElement(tagName) => element}: 주어진 태그 이름을 태그로 한 새로운 HTML 요소를 생성하여 반환하는 메서드; \texttt{document} 객체만 사용할 수 있는 메서드
    \item \texttt{appendChild(element) => element}: 주어진 요소를 요소의 자식 요소로 삽입하고, 삽입된 요소를 반환하는 메서드
    \item \texttt{removeChild(element) => element}: 주어진 요소를 요소의 자식 요소에서 제거하고, 제거된 요소를 반환하는 메서드
\end{itemize}

\begin{codeenv}{code:child-elmts-modify}{Modifying Child Elements}\begin{verbatim}
> const container = document.getElementById('container');
> const newButton = document.createElement('button');
> newButton.type = 'submit';
> newButton.innerText = 'Submit';
> container.appendChild(newButton);
> const title = document.getElementsByClassName('article-title')[1];
> container.removeChild(title);
\end{verbatim}
\end{codeenv}
