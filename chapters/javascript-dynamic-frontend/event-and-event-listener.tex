\section{Event and Event Listener}\label{sect:event-and-event-listener}

\subsection*{What is Event and Event Listener?}

프로그램에서 이벤트(event)란 개념은 어떠한 사건을 의미합니다. Front-end에서 발생할 수 있는 이벤트에는 요소를 클릭하는 행위, 키보드를 이용하여 키를 입력하는 행위, 드래그하는 행위 등이 있습니다. 이러한 사건이 일어나면 ``이벤트가 발생했다''라고 표현합니다.

이벤트를 이용하면 마침내 사용자에 의해 동적으로 작동하는 웹 페이지를 만들 수 있습니다. 버튼을 눌러 원하는 함수를 작동시키고, 키보드를 이용하여 게임 오브젝트를 컨트롤 할 수도 있습니다. 이러한 기능을 구현하기 위해서는 특정한 이벤트가 발생하였을 때 실행되어야 하는 작업을 명시해주어야 합니다. 어떠한 요소에서 특정한 이벤트가 발생하였을 때 실행될 함수가 이벤트 리스너(event listener)입니다. 이번 절에서는 이벤트와 이벤트 리스너를 다루는 방법에 대해 다룹니다.

\subsection*{Events}

이벤트의 종류는 매우 많습니다. 따라서 필요에 따라 찾아 쓰는 경우가 많습니다. 자주 쓰이는 몇 가지 이벤트에 대해 소개하겠습니다. 먼저 마우스 동작과 관련된 이벤트입니다.

\begin{itemize}
    \item \texttt{click}: 요소를 좌클릭하였을 때 이벤트가 발생합니다.
    \item \texttt{mousedown}: 요소 위에 커서를 대고 마우스를 눌렀을 때 발생합니다.
    \item \texttt{mouseenter}: 마우스 커서가 요소 안으로 들어올 때 발생합니다.
    \item \texttt{mouseleave}: 마우스 커서가 요소 밖으로 나갈 때 발생합니다.
\end{itemize}

다음은 키보드를 이용한 키 입력과 관련된 이벤트입니다.

\begin{itemize}
    \item \texttt{keydown}: 키보드의 키를 눌렀을 때 발생합니다.
    \item \texttt{keyup}: 키보드의 키를 떼었을 때 발생합니다.
    \item \texttt{keypress}: 키보드의 키를 눌렀을 때 발생합니다. 다만, Ctrl, Alt와 같은 기능키에 대해서는 작동하지 않습니다.
\end{itemize}

소개된 이벤트 외에도 웹 페이지에서 발생할 수 있는 다른 이벤트들을 아래의 링크에서 확인해볼 수 있습니다.

\begin{itemize}
    \item https://www.w3schools.com/jsref/dom\_obj\_event.asp
\end{itemize}

\subsection*{Event Listener}

특정 요소에서 이벤트가 발생했을 때 이벤트 리스너를 실행시키기 위해서는 해당 요소와 이벤트 리스너를 등록해야 합니다. 그 방법은 크게 세 가지가 있습니다.

먼저, HTML 요소 내부에 직접 명시해주는 방법입니다. HTML 요소의 속성에 이벤트의 이름 앞에 \texttt{on}을 붙여 작성하고, 속성값에 실행할 JS 코드, 즉 이벤트 리스너를 작성합니다. 예를 들어, \texttt{click} 이벤트에 대응하는 속성은 \texttt{onclick}, \texttt{keypress} 이벤트에 대응하는 속성은 \texttt{onkeypress}입니다. Inline 방식으로 이벤트 리스너를 연동하면 해당 요소에서 특정 이벤트가 발생하였을 때 어떤 작업이 발생하는지 직관적으로 파악할 수 있습니다. 그러나 \sectref{sect:basic-structure-of-css}에서 언급한 바와 같이 HTML 문서의 목적에 위배되며, 유지 및 보수의 관점에서 매우 비효율적입니다.

\begin{codeenv}{code:event-listener-inline}{Event Listener - inline}\begin{verbatim}


<button id="btn" onclick="console.log('Clicked!');">Click Button</button>
\end{verbatim}
\end{codeenv}

\coderef{code:event-listener-inline}에서 HTML 요소의 \texttt{onclick} 속성에 이벤트 리스너를 속성값으로 부여했으므로, 요소 객체의 속성 \texttt{onclick}을 이용하여 그 값을 바꿀 수 있습니다. 즉, 이벤트 리스너를 등록하고자 하는 요소 객체의 속성 중 이벤트와 관련된 속성에 이벤트 리스너를 함수의 형태로 할당하는 방법도 있습니다.

\begin{codeenv}{code:event-listener-property-mod}{Event Listener – Modifying Property}\begin{verbatim}


document.getElementById('btn').onclick = () => {
    console.log('Clicked!');
};
\end{verbatim}
\end{codeenv}

마지막 방법은 \texttt{addEventListener} 메서드를 활용하는 것입니다. \texttt{addEventListener} 메서드는 이벤트의 이름과 이벤트 리스너를 인자로 받아, 이벤트 리스너를 추가합니다.

\begin{codeenv}{code:event-listener-addeventlistener}{Event Listener – Using \texttt{addEventListener}}\begin{verbatim}


document.getElementById('btn').addEventListener('click', () => {
    console.log('Clicked!');
});
\end{verbatim}
\end{codeenv}

객체의 속성을 직접 수정하는 방법은 IE 8 이전 버전의 브라우저에서도 호환되고, 간결하다는 것이 장점입니다. 그러나 IE 8 이전 버전의 브라우저는 지나치게 오래된 브라우저이므로 지원할 필요성이 너무 적고, 무엇보다 한 요소의 한 이벤트에 여러 이벤트 리스너를 등록할 수 없습니다. 이러한 이유로 \texttt{addEventListener} 메서드를 사용하여 이벤트 리스너를 등록하는 방법이 제일 권장되는 방법입니다.

\subsection*{Event Object}

지금까지 이벤트의 종류와 이벤트 리스너를 등록하는 방법에 대해 다뤘습니다. 그런데 이벤트 리스너를 구현하다 보면 발생한 이벤트에 대한 정보를 얻을 수 없다는 큰 문제에 부딪히게 됩니다. 예를 들어 \texttt{keypress} 이벤트가 발생했을 때 사용자가 어느 키를 눌렀는지, \texttt{mousedown} 이벤트가 발생했을 때 사용자가 좌클릭했는지 우클릭했는지, 어느 위치에 커서를 대고 눌렀는지 등에 대한 정보를 얻지 못한다는 문제에 부딪힙니다.

이러한 문제를 해결하기 위해 이벤트가 발생하면, 발생한 이벤트에 대한 여러 정보를 담은 이벤트 객체를 생성하여 이벤트 리스너에 전달합니다. 이벤트 리스너는 이벤트 객체를 함수의 인자로 받아서 사용할 수 있습니다. 물론, JS의 함수의 특성상 이벤트 리스너에서 이벤트 객체가 필요하지 않다면 앞에서 본 예제처럼 인자를 아예 받지 않을 수도 있습니다.

\begin{codeenv}{code:event-object-from-listener}{Event Listener with Event Object}\begin{verbatim}


document.addEventListener('mousedown', (event) => {
    console.log(event);
});
\end{verbatim}
\end{codeenv}

이벤트 객체의 속성 중 자주 쓰이는 속성을 몇 가지 소개합니다. 마우스 동작과 관련된 이벤트 객체에는 다음과 같은 속성들이 있습니다.

\begin{itemize}
    \item \texttt{button}: 마우스의 버튼 번호입니다. \texttt{0}, \texttt{1}, \texttt{2}, \texttt{3}, \texttt{4}의 값이 가능하며 각각 왼쪽 버튼, 휠 버튼, 오른쪽 버튼, 뒤로 가기 버튼, 앞으로 가기 버튼입니다. 
    \item \texttt{clientX}, \texttt{clientY}: 마우스 이벤트가 발생한 위치를 웹 페이지를 기준으로 나타낸 값입니다. 
    \item \texttt{offsetX}, \texttt{offsetY}: 마우스 이벤트가 발생한 위치를 대상 요소의 위치를 기준으로 나타낸 값입니다. 
    \item \texttt{screenX}, \texttt{screenY}: 마우스 이벤트가 발생한 위치를 웹 페이지가 존재하는 화면을 기준으로 나타낸 값입니다. 
\end{itemize}

키보드를 이용한 키 입력과 관련된 이벤트 객체의 속성에는 \texttt{keyCode}가 있습니다. Key code는 키보드의 각 키를 정수로 나타낸 값으로, 다음 링크에서 그 값을 쉽게 확인할 수 있습니다. 

\begin{itemize}
    \item https://keycode.info/
\end{itemize}
