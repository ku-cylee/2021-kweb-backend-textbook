\section{Event and Event Listener Exercises}\label{sect:event-and-event-listener-exercises}

\subsection*{Problem 1}

쇼핑몰 사이트에서 물품을 하나씩 카트에 담으면, 지불할 금액의 총합이 그 물품의 액수만큼 증가하는 웹 페이지를 제작하는 문제입니다. 제시된 단계를 차례대로 수행하여 프로그램을 완성해보세요.

\begin{codeenv}{code:event-exercise-1}{Event Exercise 1}\begin{verbatim}


<!doctype html>
<head>
    <style>
        body > div { margin: 20px 0; }
        .item { 
            width: 500px; padding: 10px; margin: 20px 0; 
            background-color: #d3d3d3; border: 2px solid gray;
        }
        .item > div { display: inline-block; vertical-align: top; }
        .item-image > img { width: 100px }
        .item-info { margin-left: 10px }
        .item-info > div { margin-bottom: 10px; }
        .item-name { font-size: 20px; font-weight: bold }
    </style>
    <script src="./script.js" type="text/javascript" defer></script>
</head>
<body>
    <div id="header">
        <h1>Marketplace</h1>
        <div>Total Cost: <span id="cost">0</span> KRW</div>
    </div>

    <div id="items">
        <div class="item" id="apple">
            <div class="item-image">
                <img src="https://i.imgur.com/PMmhG47.jpg" >
            </div>
            <div class="item-info">
                <div class="item-name">Apple</div>
                <div class="item-price">Unit Price: 700 KRW</div>
                <button class="add-to-cart">Add to Cart</button>
            </div>
        </div>
        <div class="item" id="orange">
            <div class="item-image">
                <img src="https://i.imgur.com/IOEmEO1.jpg" >
            </div>
            <div class="item-info">
                <div class="item-name">Orange</div>
                <div class="item-price">Unit Price: 800 KRW</div>
                <button class="add-to-cart">Add to Cart</button>
            </div>
        </div>
        <div class="item" id="lemon">
            <div class="item-image">
                <img src="https://i.imgur.com/ZYLR99S.jpg" >
            </div>
            <div class="item-info">
                <div class="item-name">Lemon</div>
                <div class="item-price">Unit Price: 900 KRW</div>
                <button class="add-to-cart">Add to Cart</button>
            </div>
        </div>
    </div>
</body>
\end{verbatim}
\end{codeenv}

\begin{itemize}
    \item Step 1: \texttt{.item} 요소들은 그 물품의 이름을 \texttt{id} 값으로 가지고 있습니다. 각 \texttt{id}가 \texttt{key}, 해당 물품의 단가를 \texttt{value}로 하는 객체를 만드세요. 그리고, 필요한 변수/상수들을 선언하고 할당합니다.
    \item Step 2: 이 문제에서는 여러 요소에 이벤트 리스너를 등록해야 하므로 등록할 요소들을 선택합니다.
    \item Step 3: for-of문을 사용하여 각 요소에 이벤트 리스너를 등록합니다.
    \item Step 4: 이벤트 리스너 내부를 구현합니다. 힌트: 이벤트 리스너가 등록된 요소가 속한 \texttt{.item} 요소의 \texttt{id} 값을 가져와야 합니다. \sectref{sect:objects}에서 자기 자신을 지칭하는 키워드가 무엇이라고 소개했는지 기억나시나요?
\end{itemize}

\subsection*{Problem 2}

사각형 게임 오브젝트를 방향키를 이용하여 움직여서, 원형의 게임 오브젝트를 잡으면 이기는 간단한 게임을 만드는 문제입니다. 제시된 단계를 차례대로 수행하여 프로그램을 완성해보세요.

\begin{codeenv}{code:event-exercise-2}{Event Exercise 2}\begin{verbatim}


<!doctype html>
<head>
    <style>
        #game-panel {
            width: 800px; height: 500px; position: relative; border: 4px solid black
        }
        #game-panel > div {
            width: 20px; height: 20px; position: absolute
        }
        .circle { border: 2px solid red; border-radius: 20px }
        .square { border: 2px solid darkblue; background-color: blue }
    </style>
    <script src="./script.js" type="text/javascript" defer></script>
</head>
<body>
    <h1>Catch the circle</h1>
    <div id="game-panel"></div>
    <div id="success-message"></div>
</body>
\end{verbatim}
\end{codeenv}

\begin{itemize}
    \item Step 1: \texttt{\#game-panel}의 크기는 가로 800px, 세로 500px입니다. \texttt{\#game-panel}을 가로와 세로 모두 100px 단위로 나누어 단위 사각형을 40개 만들 수 있습니다. 게임 오브젝트들은 항상 단위 사각형의 중앙에 위치합니다. 게임이 시작할 때 사각형 게임 오브젝트는 가장 왼쪽 하단에 있는 단위 사각형, 원형 게임 오브젝트는 가장 오른쪽 상단에 있는 단위 사각형에 위치합니다. 이 정보를 이용하여, 각 게임 오브젝트의 좌표를 \texttt{x}, \texttt{y}를 속성으로 갖는 좌표 객체로 표현하세요.
    \begin{itemize}
        \item Hint 1: \texttt{x} 좌표의 기준은 \texttt{\#game-panel}의 왼쪽 면, \texttt{y} 좌표의 \texttt{\#game-panel}의 위쪽 면입니다.
        \item Hint 2: 요소의 위치는 그 요소의 왼쪽 상단이 기준이 됩니다. 그렇지만 문제에서는 요소의 중심이 단위 사각형의 중심과 일치할 것을 요구하고 있습니다.
    \end{itemize}
    \item Step 2: 위, 아래, 왼쪽, 오른쪽 방향키의 key code는 각각 \texttt{38}, \texttt{40}, \texttt{37}, \texttt{39}입니다. 이 상수들을 저장할 수 있는 객체를 만드세요. 이 외에도 필요한 변수나 상수들을 선언하고, 할당하세요.
    \item Step 3: \texttt{setGameObjectPosition}: 요소 객체와 좌표 객체를 인자로 받아, 요소 객체의 위치를 좌표 객체에 따라 이동시키는 함수를 작성하세요.
    \item Step 4: \texttt{moveGameObjectByKey}: 좌표 객체와 key code를 인자로 받아, key code에 따라서 이동된 새 좌표 객체를 반환하는 함수를 작성하세요.
    \item Step 5: \texttt{isGameObjectOutOfRange}: 좌표 객체를 인자로 받아, 해당 좌표가 \texttt{\#game-panel}을 벗어나는지를 판별하여 그 값을 반환하는 함수를 작성하세요.
    \item Step 6: HTML 문서에 사각형 게임 오브젝트와 원형 게임 오브젝트를 각각 추가하세요. 사각형 게임 오브젝트의 클래스는 square, 원형 게임 오브젝트의 클래스는 circle입니다.
    \item Step 7: HTML 문서에 이벤트 리스너를 적절한 이벤트에 연동시킵니다. 이벤트 리스너에서는 먼저 사각형 오브젝트를 key code에 따라 이동시킨 새로운 좌표를 알아냅니다. 새로운 좌표가 \texttt{\#game-panel}을 벗어나면 경고창을 띄우고, 그렇지 않다면 새로운 좌표로 오브젝트를 이동시킵니다. 만일 사각형 오브젝트가 원형 오브젝트에 도달하는 데 성공했다면, 원형 오브젝트를 \texttt{\#game-panel}에서 삭제하고, \texttt{\#success-message} 요소에 성공했다는 메시지를 전달합니다.
\end{itemize}

Exercise에 대한 해답은 \sectref{sect:js-exercise-answers}에서 제공됩니다.
