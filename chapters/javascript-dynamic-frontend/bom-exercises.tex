\section{BOM Exercises} \label{sect:bom-exercises}

\subsection*{Problem 1}

주어진 레벨업 시스템을 버튼을 눌러 작동시켜서, 레벨 0에서 레벨 10까지 레벨업을 계속 시도하는 문제입니다.

\begin{itemize}
    \item 레벨업을 시도하면 확률적으로 성공할 수도, 실패할 수도 있습니다. 성공하면 레벨이 1만큼 오릅니다.
    \item 레벨업이 성공할 확률은 현재의 레벨에 따라 다릅니다.
    \item 웹 페이지에서는 현재 레벨과 시도 횟수를 표시합니다.
    \item 현재 레벨이 파란색 게이지 바의 길이에 반영되어야 합니다. (레벨 $n$인 경우 전체 게이지 바의 $\frac n {10}$을 차지)
    \item 프로그램은 50ms에 한 번씩 레벨업을 시도합니다.
\end{itemize}

\begin{codeenv}{code:bom-exercise}{BOM Exercise}\begin{verbatim}


<!doctype html>
<head>
    <style>
        #gauge-background {
            width: 300px; height: 10px;
            background-color: red; position: relative; margin: 20px 0
        }
        #gauge-bar {
            background-color: blue; width: 0; height: 10px; position: absolute
        }
    </style>
    <script src="./script.js" type="text/javascript" defer></script>
</head>
<body>
    <h1>Level-Up System</h1>
    <div id="gauge-background">
        <div id="gauge-bar"></div>
    </div>
    <div>Current Level: <span id="current-level">0</span></div>
    <div>Attempts: <span id="attempts">0</span></div>
    <button id="start-btn" onclick="work()">Start Level Up</button>
</body>
\end{verbatim}
\end{codeenv}

\begin{itemize}
    \item 1단계: DOM을 이용하여 필요한 요소들을 상수를 선언하여 저장합니다. 이 외에도 필요한 변수들을 선언합니다. 
    \item 2단계: 레벨업 성공 확률을 \% 단위, number형의 배열로 저장합니다. 레벨 0에서 레벨 9까지 다음 레벨로의 레벨업이 성공할 확률은 각각 100\%, 60\%, 36\%, 22\%, 13\%, 8\%, 5\%, 3\%, 2\%, 1\%입니다. 
    \item 3단계: \% 단위의 확률을 인자로 받아, 제시된 확률에 맞게 무작위로 성공 여부를 판단하여, 성공하면 \texttt{true}, 실패하면 \texttt{false}를 반환하는 함수를 구현합니다. 
    \begin{itemize}
        \item 예를 들어, \texttt{32}를 인자로 받으면 32\%의 확률로 \texttt{true}, 68\%의 확률로 \texttt{false}를 반환해야 합니다. 
        \item Hint: 0 이상 20 미만의 정수 중 하나를 무작위로 뽑았을 때 13 미만의 정수는 65\% 확률로 뽑힙니다. 
    \end{itemize}
    \item 4단계: 버튼을 눌렀을 때 작동할 기능은 \texttt{work} 함수 내부에 작성합니다. 
\end{itemize}

Exercise에 대한 해답은 \sectref{sect:js-exercise-answers}에서 제공됩니다.
