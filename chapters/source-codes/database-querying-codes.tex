\section{Database Querying Source Codes}\label{sect:database-querying-codes}

\begin{codeenv}{code:join-example-sql}{SQL Join Example Data}\begin{verbatim}
CREATE TABLE `tb1` (`id` INT, `data1` VARCHAR(4), PRIMARY KEY (`id`)) ENGINE=InnoDB;
CREATE TABLE `tb2` (`id` INT, `data2` VARCHAR(4), PRIMARY KEY (`id`)) ENGINE=InnoDB;

INSERT INTO `tb1` VALUES (1,'1-1'),(6,'1-6'),(7,'1-7'),(8,'1-8'),(10,'1-10'),
(11,'1-11'),(12,'1-12'),(13,'1-13'),(14,'1-14'),(15,'1-15');
INSERT INTO `tb2` VALUES (2,'2-2'),(3,'2-3'),(4,'2-4'),(5,'2-5'),(7,'2-7'),(8,'2-8'),
(9,'2-9'),(11,'2-11'),(13,'2-13'),(14,'2-14');
\end{verbatim}
\end{codeenv}

\begin{codeenv}{code:scores-table-sql}{\cd{scores} Table SQL}\begin{verbatim}
CREATE TABLE `scores` (
    `id` INT NOT NULL AUTO_INCREMENT,
    `student` VARCHAR(32) NOT NULL,
    `course` VARCHAR(32) NOT NULL,
    `score` INT NOT NULL,
    PRIMARY KEY (`id`)
) ENGINE=InnoDB DEFAULT CHARSET=utf8;

INSERT INTO `scores` VALUES (1,'Barack','Discrete Mathematics',87),
    (2,'Joe','Discrete Mathematics',92),(3,'Barack','Machine Learning',61),
    (4,'Donald','Operating Systems',98),(5,'Joe','Machine Learning',78),
    (6,'Donald','Discrete Mathematics',58),(7,'Donald','Machine Learning',82),
    (8,'Joe','Operating Systems',66);
\end{verbatim}
\end{codeenv}
