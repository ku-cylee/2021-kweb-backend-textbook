\section{Class and Id Attributes} \label{sect:class-and-id-attributes}

\sectref{sect:commonly-used-html-tags}에서 HTML 문서를 작성하는 기본적인 방법에 대하여 학습하였다. \chapref{ch:css-designing-html}부터는 CSS와 JS를 배우면서 HTML에 적용하는 과정을 다루는데, CSS에서는 각 요소에 원하는 디자인을 적용할 수 있고, JS에서는 각 요소를 추가 및 삭제하거나, 그 속성을 수정하는 등의 작업을 할 수 있다. 이러한 CSS와 JS를 HTML 문서에 적용할 때, 특정 요소 혹은 특정 분류의 모든 요소에 CSS나 JS를 적용하게 된다.

\sectref{sect:commonly-used-html-tags}에서 HTML 문서를 구조화하기 위해 \cd{div}나 \cd{span} 태그의 쓰임새에 대해 학습하였다. 그러나 태그만으로는 기능이나 역할 등 개발자가 원하는 분류 기준에 따라 HTML 요소들을 분류하는 것은 어려우며, HTML 문서가 매우 길어진다면 단순히 태그 이름만으로 요소들을 구분하는 것은 불가능하다.

이렇게 특정 기준에 따라 요소들을 분류하거나, 특정 HTML 요소를 지정할 때 필요한 속성이 \textbf{class}와 \textbf{id}이다. Class와 id는 모든 HTML 요소에 적용할 수 있으며, CSS, JS를 HTML과 연동할 때 매우 중요한 역할을 한다.

\subsection*{Class Attribute}
먼저, class 속성은 HTML 요소들을 \textbf{특정한 기준에 따라 분류(classify)}할 때 사용되는 속성이며, class 속성의 값을 class name이라고 한다. 특정한 기준으로 분류하였을 때 하나의 묶음으로 묶이는 요소들에는 각각 같은 이름의 class를 사용한다. 하나의 HTML 요소는 여러 class를 가질 수 있고, 각 class name은 공백을 이용하여 구분한다. \coderef{code:class-attr-ex}\은 class 속성을 활용한 예제이다.

\begin{codeenv}{code:class-attr-ex}{Class Attributes Example}\begin{verbatim}
<div class="page-thumbnail new">
    <img src="/resources/week2_handout.jpg">
    <span class="page-title">
        <a href="/study/201R/3">Week 2 Handout</a>
    </span>
</div>
<div class="page-thumbnail">
    <img src="/resources/week1_asgmt.jpg">
    <span class="page-title">
        <a href="/study/201R/2">Week 1 Assignment</a>
    </span>
</div>
<div class="page-thumbnail">
    <img src="/resources/week1_handout.jpg">
    <span class="page-title">
        <a href="/study/201R/1">Week 1 Assignment</a>
    </span>
</div>
\end{verbatim}
\end{codeenv}


\subsection*{Id Attribute}
Id 속성은 \textbf{특정한 HTML 요소 하나를 식별(identification)}하기 위해 사용되는 속성이다. 하나의 요소는 여러 id를 가질 수 없고, 특정 id의 값을 갖는 HTML 요소가 여러 개가 될 수 없다.\footnote{이 규칙을 위반하더라도 HTML 문서는 정상적으로 렌더링된다.} 다만, 각 HTML 요소는 class와 id를 동시에 가질 수 있다. \coderef{code:id-attr-ex}\은 id 속성을 활용한 예제이다.

\begin{codeenv}{code:id-attr-ex}{Id Attributes Example}\begin{verbatim}
<div id="article-form">
    <input id="article-title" name="title">
    <textarea id="article-content" name="content"></textarea>
    <button>Submit</button>
</div>
\end{verbatim}
\end{codeenv}

\subsection*{Naming Convention}
Class name이나 id를 작성할 때 반드시 준수해야 하는 작명 규칙(naming convention)은 없다. 그러나 협업이나 유지보수 등 생산성의 향상을 위해 널리 통용되고 권장되는 규칙을 소개한다.\footnote{아래 소개되는 작명 규칙보다 더 자세한 규칙은 다음 링크를 참조하길 바란다: https://bogmong.tistory.com/14}

\begin{itemize}
    \item 대문자의 사용은 지양하고, 소문자로만 구성한다. 숫자로 시작하지 않는다.
    \item 이름은 class나 id의 의미에 잘 부합하여 어떠한 기준으로 지어진 이름인지 알기 쉽게 작명한다.
    \item 여러 단어의 조합은 하이픈(\cd{-})으로 연결하여 작명한다. (예: \cd{multiple-words})
\end{itemize}
