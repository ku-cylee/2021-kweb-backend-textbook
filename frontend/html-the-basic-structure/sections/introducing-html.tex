\section{Introducing HTML} \label{sect:introducing-html}

HTML은 HyperText Markup Language의 약자로, 웹 브라우저에 웹 페이지의 구조를 체계적으로 표현하는 컴퓨터 언어이다. HTML은 웹 페이지의 가장 기본적인 뼈대를 이루며, 어떤 front-end framework를 사용하던 HTML을 필수적으로 이해하고 있어야 한다.

\subsection*{HTML is not a Programming Language?}

소프트웨어 개발과 관련된 밈(meme)을 접하다 보면 가장 자주 듣는 밈 중 하나가 바로 ``HTML is not a programming language'', 즉 HTML은 프로그래밍 언어가 아니라는 밈이다. HTML은 프로그래밍 언어가 아닌, 구조를 서술하기 위한 언어로, 쉽게 접할 수 있는 프로그래밍 언어인 C, Java, Python 등과 달리 조건문, 반복문, 변수 선언 등의 기능이 전혀 없다. 구조를 서술하기 위해 존재하는 언어이니만큼, \textbf{미리 정해진 규칙}에 따라 HTML을 작성하여 \textbf{구조를 표현}한다는 개념으로 접근하면 HTML이라는 언어를 쉽게 이해할 수 있을 것이다.

\figures{fig:html-not-pl}{HTML is not a programming language}
    {\fig{html-is-not-a-programming-language.png}{.5}}
