\section{Style Properties} \label{sect:style-properties}

\sectref{sect:basic-structure-of-css}에서 설명한 바와 같이 CSS의 속성은 매우 많다. 모든 속성과 속성값의 역할을 모두 알고 있는 것이 나쁘지는 않으나, 주로 사용되는 속성들을 잘 습득하고 그 외에 필요한 속성은 필요할 때 찾아서 사용하는 것이 바람직하다. 이번 절에서 스타일과 관련된 속성 중 자주 사용되는 속성들을 알아보자.

\subsection*{텍스트와 관련된 속성}

\begin{itemize}
    \item \cd{text-align} – 텍스트를 수평적으로 어떻게 정렬할지에 관한 속성이다. 속성값으로는 \cd{center}, \cd{left}, \cd{right}, \cd{justify}가 있으며, 각각 가운데, 왼쪽, 오른쪽, 양쪽 정렬을 뜻한다.
    \item \cd{text-decoration} – 텍스트를 선(line)을 이용하여 꾸민다. \cd{text-decoration-line}, \cd{text-decoration-color}, \cd{text-decoration-style}의 3가지 세부 속성이 있고, 각 속성의 값을 \cd{text-decoration} 속성의 값으로 차례대로 나열하여 표현하여도 된다.	
    \item \cd{line-height} – 줄 간격에 관한 속성. 100\%는 \cd{1}, 150\%는 \cd{1.5} 등 단위 없이 표현한다.
    \item \cd{letter-spacing} – 글자 간 간격에 관한 속성. 단위는 \%, em, px 등을 사용한다.
    \item \cd{vertical-align} – 텍스트를 세로 방향으로 어떻게 정렬할지에 관한 속성. 속성값으로는 \cd{baseline}, \cd{top}, \cd{bottom}, \cd{text-top}, \cd{text-bottom}, \cd{middle} 등이 있다.
\end{itemize}

\subsection*{폰트와 관련된 속성}

\begin{itemize}
    \item \cd{font-size} – 글자의 크기에 관한 속성. 단위는 em, px 등을 사용하여 표현한다.
    \item \cd{font-weight} – 글자의 두께에 관한 속성입니다. \cd{100}, \cd{200}, \ldots, \cd{900}의 값이 가능하다. 이 외에도 \cd{normal}, \cd{bolder} 등의 값이 가능하며, 각각 \cd{400}, \cd{700}에 해당하는 값이다.
    \item \cd{font-family} – 글자의 서체를 지정하는 속성. 웹 브라우저에서 지원하는 서체뿐만 아니라 로컬 컴퓨터나 외부 서버에 있는 서체도 가능하다.
    \item \cd{color} – 글자의 색상에 관한 속성. \cd{red}, \cd{blue} 등의 색상 이름이나, \cd{rgb(26, 219, 158)} 또는 \cd{rgba(26, 219, 158, .5)}와 같이 RGB(A) 형태로 나타낸 값, \cd{hsl(161, 79\%, 48\%)}나 \cd{hsla(161, 79\%, 48\%, .5)}와 같이 HSL(A) 형태로 나타낸 값, \cd{\#1ADB9E}와 같이 HEX 형태로 나타낸 값 모두 가능하다.
\end{itemize}

\subsection*{배경과 관련된 속성}

\begin{itemize}
    \item \cd{background-color} – 배경의 색상에 관한 속성으로, 앞의 \cd{color}과 동일한 형태의 값을 가질 수 있다.
    \item \cd{background-image} – 배경에 이미지를 삽입할 수 있는 속성
    \item \cd{background-repeat} – 배경 이미지가 반복되는 형태를 지정할 수 있다. 속성값으로는 \cd{repeat}, \cd{repeat-x}, \cd{repeat-y}, \cd{no-repeat} 등이 있다.
    \item \cd{background-size} – 배경 이미지의 크기에 관한 속성
\end{itemize}
