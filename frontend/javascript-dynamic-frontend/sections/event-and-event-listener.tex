\section{Event and Event Listener}\label{sect:event-and-event-listener}

프로그램에서 \textbf{이벤트(event)}란 개념은 어떠한 사건을 의미하며, front-end에서 발생할 수 있는 이벤트에는 요소를 클릭하는 행위, 키보드를 이용하여 키를 입력하는 행위, 드래그하는 행위 등이 있다. 이러한 사건이 일어나면 ``이벤트가 발생했다''라고 표현한다.

이벤트를 이용하면 버튼을 눌러 원하는 함수를 작동시키고, 키보드를 이용하여 게임 오브젝트를 컨트롤 하는 등 사용자에 의해 동적으로 작동하는 웹 페이지를 구현할 수 있다. 이러한 기능은 특정 이벤트가 발생하였을 때 실행되어야 하는 작업을 명시하여 구현하며, 이렇게 특정 요소에서 특정 이벤트가 발생하였을 때 실행될 함수를 \textbf{이벤트 리스너(event listener)}라고 한다.

이번 절에서는 이벤트와 이벤트 리스너를 이용하여 웹 페이지를 구현하는 방법을 다룬다.

\subsection*{Events}

웹 페이지에서 발생할 수 있는 이벤트의 종류는 다양하며, 본 교재에서는 자주 쓰이는 몇 가지만 소개한다. 먼저 마우스 동작과 관련된 이벤트이다.

\begin{itemize}
    \item \cd{click}: 요소가 좌클릭되었을 때 발생하는 이벤트
    \item \cd{mousedown}: 요소 위에 커서를 대고 마우스를 눌렀을 때 발생하는 이벤트
    \item \cd{mouseenter}: 마우스 커서가 요소 안으로 들어올 때 발생하는 이벤트
    \item \cd{mouseleave}: 마우스 커서가 요소 밖으로 나갈 때 발생하는 이벤트
\end{itemize}

다음은 키보드를 이용한 키 입력과 관련된 이벤트이다.

\begin{itemize}
    \item \cd{keydown}: 키보드의 키를 눌렀을 때 발생하는 이벤트
    \item \cd{keyup}: 키보드의 키를 떼었을 때 발생하는 이벤트
    \item \cd{keypress}: 키보드의 키를 눌렀을 때 발생하는 이벤트; Ctrl, Alt 등의 기능키에는 작동하지 않음
\end{itemize}

소개된 이벤트 외의 이벤트의 종류는 매우 많으므로 필요에 따라 찾아서 사용하는 것이 좋다. 웹 페이지에서 발생할 수 있는 이벤트는 아래의 링크에서 확인해볼 수 있다.

\begin{itemize}
    \item https://www.w3schools.com/jsref/dom\_obj\_event.asp
\end{itemize}

\subsection*{Event Listener}

특정 요소에서 이벤트가 발생했을 때 이벤트 리스너를 실행시키기 위해서는 해당 요소와 이벤트 리스너를 등록해야 하며, 그 방법은 크게 세 가지가 있다.

먼저, HTML 요소의 속성과 속성값으로 직접 명시하는 방법이다. 먼저 속성의 이름은 이벤트의 이름 앞에 접두사 \cd{on}을 붙인다. 예를 들어 \cd{click} 이벤트에 대응하는 속성은 \cd{onclick}, \cd{keypress} 이벤트에 대응하는 속성은 \cd{onkeypress}이다. 이후 속성값에 이벤트가 발생했을 때 수행될 JS 코드를 작성한다. 이러한 방식으로 이벤트 리스너를 연동하면 특정 요소에서 특정 이벤트가 발생하였을 때 어떤 작업이 수행되는지 직관적으로 파악할 수 있으나, \sectref{sect:basic-structure-of-css}에서 다룬 CSS의 inline style의 단점과 마찬가지로 HTML 문서의 목적에 위배되며, 유지 및 보수의 관점에서 매우 비효율적이다.

\begin{codeenv}{code:event-listener-inline}{Event Listener - Inline Method}\begin{verbatim}
<button id="btn" onclick="console.log('Clicked!');">Click Button</button>
\end{verbatim}
\end{codeenv}

두 번째 방법은 HTML 요소 객체에 이벤트 리스너를 메서드로 할당하는 방법이다. 메서드의 이름은 앞의 방식과 마찬가지로 접두사 \cd{on}을 붙여 작성하고, 메서드의 값에 이벤트 리스너를 할당한다. 이 방식은 JS 파일에 작성할 수 있어 HTML의 목적에 위배되지 않고, IE 8 이전 버전의 브라우저에서도 호환되며, 간결하다는 장점을 갖는다. 그러나 IE 8\footnote{2009년 3월에 발표되었다} 이전의 브라우저는 지나치게 오래되어 지원할 필요성이 너무 적고, 한 요소의 한 이벤트에 하나의 이벤트 리스너만 등록할 수 있다는 단점이 있다.

\begin{codeenv}{code:event-listener-method-override}{Event Listener – Overriding Method}\begin{verbatim}
document.getElementById('btn').onclick = () => {
    console.log('Clicked!');
};
\end{verbatim}
\end{codeenv}

마지막 방법은 요소의 \cd{addEventListener} 메서드를 사용하는 방법으로, \cd{addEventListener} 메서드는 이벤트의 이름과 이벤트 리스너를 인자로 받아, 이벤트 리스너를 추가하는 메서드이다. 이렇게 \cd{addEventListener} 메서드를 이용한 방법이 가장 권장되는 방법이다.

\begin{codeenv}{code:event-listener-addeventlistener}{Event Listener – Using \cd{addEventListener}}\begin{verbatim}
document.getElementById('btn').addEventListener('click', () => {
    console.log('Clicked!');
});
\end{verbatim}
\end{codeenv}

\subsection*{Event Object}

지금까지 이벤트의 종류와 이벤트 리스너를 등록하는 방법에 대해 다루었다. 그런데 지금까지 학습한 방법으로는 이벤트 리스너에서 발생한 이벤트에 대한 정보를 얻을 수 없다는 문제가 발생한다. 예를 들어 \cd{keypress} 이벤트가 발생했을 때 사용자가 어느 키를 눌렀는지, \cd{mousedown} 이벤트가 발생했을 때 사용자가 좌클릭했는지 우클릭했는지, 어느 위치에 커서를 대고 눌렀는지 등에 대한 정보를 얻지 못한다.

이러한 문제는 이벤트 리스너가 이벤트 객체를 받음으로써 해결된다. 발생한 이벤트에 대한 여러 정보를 담은 이벤트 객체가 생성되어 이벤트 리스너에 전달되며, 이 이벤트 객체를 인자로 받아서 사용할 수 있다. JS의 함수의 특성상 이벤트 리스너에서 이벤트 객체가 필요하지 않다면 앞의 \coderef{code:event-listener-addeventlistener}\와 같이 인자를 아예 받지 않을 수도 있다.


\begin{codeenv}{code:event-object}{Event Listener with Event Object}\begin{verbatim}
document.addEventListener('mousedown', event => {
    console.log(event);
});
\end{verbatim}
\end{codeenv}

이벤트 객체의 속성 중 자주 쓰이는 속성을 몇 가지 소개한다.

\begin{itemize}
    \item \cd{target}: 이벤트가 발생한 요소 객체를 값으로 갖는 속성
    \item \cd{button}: 마우스 이벤트를 발생시킨 마우스의 버튼 번호를 값으로 갖는 속성; \cd{0}, \cd{1}, \cd{2}, \cd{3}, \cd{4}는 각각 왼쪽 버튼, 휠 버튼, 오른쪽 버튼, 뒤로 가기 버튼, 앞으로 가기 버튼을 뜻한다.
    \item \cd{clientX}, \cd{clientY}: 뷰포트를 기준으로 마우스 이벤트가 발생한 위치를 값으로 갖는 속성
    \item \cd{offsetX}, \cd{offsetY}: 이벤트가 발생한 요소를 기준으로 마우스 이벤트가 발생한 위치를 값으로 갖는 속성
    \item \cd{screenX}, \cd{screenY}: 웹 페이지가 존재하는 화면을 기준으로 마우스 이벤트가 발생한 위치를 값으로 갖는 속성
    \item \cd{code}: 키 입력과 관련된 이벤트를 발생시킨 키를 문자열\footnote{https://keycode.info/에서 확인할 수 있다.}로 나타낸 속성
\end{itemize}

\coderef{code:event-example}\은 이벤트 객체를 이용하여 \cd{\#panel} 요소를 좌클릭 또는 우클릭하였을 때 \cd{\#panel} 요소의 위쪽 테두리와 왼쪽 테두리에 내린 수선을 표시되며, 좌클릭하였을 때는 빨간색, 우클릭하였을 때는 파란색 수선이 표시되는 예제이다. (편의상 CSS와 JS를 모두 HTML 문서에 작성하였다.)

\begin{codeenv}{code:event-example}{Event Object Example}\begin{verbatim}
<div id="panel">
    <div id="offset"></div>
</div>

<style>
    #panel {
        width: 800px;
        height: 400px;
        border: 3px solid black;
    }
</style>

<script>
    document.getElementById('panel').addEventListener('mousedown', event => {
        if (event.button !== 0 && event.button !== 2) return;
        const fooElmt = document.getElementById('offset');
        const borderColor = event.button === 0 ? 'red' : 'blue'
        fooElmt.style.borderRight = '1px solid ' + borderColor;
        fooElmt.style.borderBottom = '1px solid ' + borderColor;
        fooElmt.style.width = event.offsetX + 'px';
        fooElmt.style.height = event.offsetY + 'px';
    });
</script>
\end{verbatim}
\end{codeenv}
