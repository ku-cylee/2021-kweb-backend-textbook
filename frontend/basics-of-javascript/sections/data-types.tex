\section{Data Types} \label{sect:data-types}

JS에는 다음과 같은 7개의 자료형이 존재하며, 이번 절에서는 symbol형을 제외한 6가지 자료형과 배열을 다룬다.

\begin{itemize}
    \item Primitive Types: number, string, boolean, undefined, null, symbol
    \item Compound Type: object
\end{itemize}

\subsection*{Number}

number형은 수(數)를 저장하는 자료형으로, 정수, 유리수 등을 구별 없이 저장할 수 있고, 상호 연산이 가능하다.

\begin{codeenv}{code:number-type}{Number Type}\begin{verbatim}
> const num1 = 2;
> const num2 = 7.5;
> const num3 = 2 / 5;
> typeof num1;
'number'
> typeof num2;
'number'
> num1 * num2;
15
> num3;
0.4
\end{verbatim}
\end{codeenv}

number 자료형에는 일반적인 실수 이외에 \cd{Infinity}와 \cd{NaN}(not a number)이 있다. 두 값은 0으로 나누는 연산이 수행되면 발생한다.

\begin{codeenv}{code:infinity-nan}{Infinity and NaN}\begin{verbatim}
> 3 / 0;
Infinity
> typeof(3 / 0);
'number'
> 3 / (-0);
-Infinity
> 0 / 0;
NaN
> typeof(0 / 0);
'number'
\end{verbatim}
\end{codeenv}

ES7부터 지수 연산자를 지원하며, \coderef{code:exponentation-operator}\와 같이 사용한다.

\begin{codeenv}{code:exponentation-operator}{Exponentation Operator}\begin{verbatim}
> 3 ** 5
243
> 4 ** (-0.5)
0.5
\end{verbatim}
\end{codeenv}

\subsection*{String}

string형은 문자열을 저장하는 자료형으로, string형 문자열 간에도 상호 계산이 가능하다.

\begin{codeenv}{code:string-type}{String Type}\begin{verbatim}
> const hello = 'Hello';
> const world = 'World';
> const helloWorld = hello + ' ' + world;
> helloWorld;
'Hello World'
> helloWorld.length;
11
> helloWorld[4];
'o'
> typeof helloWorld;
'string'
\end{verbatim}
\end{codeenv}

문자열을 표현할 때 큰따옴표(\cd{"})나 작은따옴표(\cd{'})를 사용하여 문자열을 감싸며, 서로 다른 문자로 감싸는 것은 불가능하다. 또한, 문자열 내에 감싸는 문자 등이 포함되어 있는 경우 역슬래시(\verb|\|)를 이용하여 escape 시켜야 한다.

\begin{codeenv}{code:string-type-assign}{Assigning Value on String Type Variable}\begin{verbatim}
> "String";
'String'
> 'String';
'String'
> "String';
Uncaught SyntaxError: Invalid or unexpected token
> "McDonald's";
'McDonald\'s'
> console.log("McDonald's");
McDonald's
\end{verbatim}
\end{codeenv}

\subsection*{Boolean}

boolean형은 논리적인 요소를 나타내는 자료형으로, \cd{true}와 \cd{false} 두 가지의 값이 가능하다.

\begin{codeenv}{code:boolean-type}{Boolean Type}\begin{verbatim}
> 1 == 1;
true
> 1 > 3;
false
> true == false;
false
> true != false;
true
> const isNumberEven = (5 % 2 == 0);
> isNumberEven;
false
\end{verbatim}
\end{codeenv}

\subsection*{Object}

object형, 즉 객체형이란 key와 value를 가질 수 있는 자료형이다. 예를 들어, 어떠한 학생에 대한 정보를 다룰 때, 학생의 학번, 이름, 학과, 재학 여부 등을 서로 다른 변수에 저장하지 않고, \cd{student}라는 객체형 식별자에 저장하여 일괄적으로 다룰 수 있다. Key는 같은 계층에서 유일한 문자열이고, value에는 숫자, 문자열 등 자료형에 상관없이 모든 종류의 값이 할당될 수 있다. 즉, 객체 내부에 또 다른 객체가 포함될 수 있어, 계층이 있는 데이터(hierarchial data)를 형성하는 것이 가능하다.

\begin{codeenv}{code:object-type}{Object Type}\begin{verbatim}
> const person = {
    age: 21,
    name: 'Frank',
    height: 170,
    isMale: true,
};
> person;
{ age: 21, name: 'Frank', height: 170, isMale: true }
> person.age;
21
> person['age'];
21
> !person.isMale;
false
> person.weight = 60;
60
> person;
{ age: 21, name: 'Frank', height: 171, isMale: true, weight: 60 }
\end{verbatim}
\end{codeenv}

\subsection*{Undefined and Null}

undefined형과 null형은 각각 값이 할당되지 않은 식별자와 값을 모르는 식별자가 갖는 자료형이며, 두 자료형의 값은 각각 항상 \cd{undefined}와 \cd{null}이다.

\begin{codeenv}{code:undefined-type}{Undefined Type}\begin{verbatim}
> let undefVar;
> undefVar;
undefined
> typeof undefVar;
'undefined'
\end{verbatim}
\end{codeenv}

JS에는 null 자료형 식별자를 \cd{typeof} 함수를 이용하여 확인할 때 잘못된 자료형을 반환받는 버그를 가지고 있다. 이는 매우 오래된 버그이므로 고치는 것이 불가능하다고 하며, 변수가 \cd{null}임을 확인하기 위해서는 값을 직접 비교하여야 한다.


\begin{codeenv}{code:null-type}{Null Type}\begin{verbatim}
> const nullVar = null;
> typeof null;
'object'
> nullVar == null;
true
\end{verbatim}
\end{codeenv}

\subsection*{Array}

배열은 index를 가지는, 복수의 자료를 저장할 수 있는 자료구조이다. JS에서는 각 원소의 자료형에 제한이 없어 서로 다른 자료형의 원소를 하나의 배열에 저장할 수 있으며, 배열의 크기가 정해져 있지 않아 각종 메서드(method)를 이용하여 원소를 자유롭게 넣고 뺄 수 있다.

\begin{codeenv}{code:js-array}{Array}\begin{verbatim}
> const arr = [0, 1, 1.5, true, 'hungry'];
> arr.length;
5
> arr[3];
true
> arr[4];
'hungry'
> arr.push('new element');
6
> arr;
[ 0, 1, 1.5, true, 'hungry', 'new element' ]
\end{verbatim}
\end{codeenv}
