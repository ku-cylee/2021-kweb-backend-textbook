\section{Built-in Objects} \label{sect:built-in-objects}

이번 절에서는 \sectref{sect:data-types}에서 다룬 객체에 대해 더 자세히 알아본다. JS에서 number형이나 string형의 식별자, 배열, 함수 등은 모두 객체\footnote{object형은 아니다.}이고, 그러므로 객체라는 개념을 정확히 이해하기 위해 객체 지향형 프로그래밍(Object-Oriented Programming)을 공부해야 하지만, 본 교재에서 다루는 내용과는 다소 거리가 있으므로 이번 절에서는 클래스와 메서드 등의 기초 개념만 다룬다. 또한, JS에서 기본적으로 제공되는 객체(built-in objects)의 종류와 속성과 메서드 등의 사용 방법에 대해 학습한다.

\subsection*{Property와 Method}

\coderef{code:object-type}(\pageref{code:object-type}쪽)에서 \cd{person}이라는 객체에 \cd{age}, \cd{name}, \cd{height}, \cd{isMale}의 네 속성과 그 값을 부여하였다. 이때 \cd{person} 객체의 속성의 값을 이용하여 수행하는 작업을 생각해보자. 예를 들어 새해가 되어 나이를 증가시켜주는 작업은 \cd{age}의 값을 \cd{1}만큼 증가시키는 작업이다. 이 작업은 매우 간단하지만, 복잡하고 긴 과정을 거쳐야 하는 작업일 수도 있으므로 \cd{increaseAge}라는 함수를 만들어 사용하는 상황을 가정하자.

JS 파일에 이러한 함수를 만들어두고 사용하면 어떤 부작용이 있겠는가? 객체가 많은 프로그램의 경우는 함수의 수가 지나치게 많아져 가독성이 저하되고, 개발자가 헷갈리기 쉬워지고 코드 분석도 어려워지는 등 효율성이 저하될 수 있다. 또한, 다른 객체에 대해 유사한 작업을 수행하는 함수가 있다면 함수 이름이 겹쳐 namespace 오염이 일어날 수 있고, 함수 이름을 비효율적으로 짓게 된다. 따라서 객체의 속성값을 이용하는 함수는 객체 내부에 정의하여 namespace 오염과 부작용을 방지하고, 이를 \textbf{메서드(method)}라고 한다.

\begin{codeenv}{code:object-method}{Method of Object}\begin{verbatim}
> const person = {
      age: 21,
      name: 'Frank',
      height: 170,
      isMale: true,
      increaseAge: function () { this.age++ },
  };
> person.increaseAge();
> person.age;
22
\end{verbatim}
\end{codeenv}

메서드는 \coderef{code:object-method}\와 같이 key에 메서드 이름을 작성하고 value에 메서드가 수행할 작업을 함수의 형태로 할당한다. 앞의 \cd{increaseAge} 메서드는 자신이 속한 \cd{person} 객체의 \cd{age} 값을 \cd{1}만큼 증가시켜야 하는데, 함수 내에 \cd{age}가 선언되어 있지 않아 직접 접근할 수 없다. 이때 \cd{this} 키워드를 사용하면 자기 자신이 속한 객체에 접근할 수 있다.\footnote{\cd{this}는 자기 자신의 객체 외에도 다른 종류의 객체를 가리킬 수도 있다.} 즉 \cd{increaseAge} 메서드 내부의 \cd{this}는 \cd{person}을 가리키고, \cd{increaseAge}는 자신의 속성 \cd{age}의 값을 \cd{1}만큼 증가시킨다.

\begin{codeenv}{code:object-simple-method}{Simple Method Syntax}\begin{verbatim}
> const person = {
      age: 21,
      increaseAge() { this.age++ },
  };
> person.increaseAge();
> person.age;
22
\end{verbatim}
\end{codeenv}

ES6의 축약 메서드 표현법에 따라 \coderef{code:object-simple-method}\와 같이 축약하여 표현할 수 있다.

메서드를 표현할 때 arrow function을 이용하면 오류가 발생할 여지가 크다는 점을 주의해야 한다. 화살표 함수는 \cd{this}가 가리키는 객체를 정하는 방식이 일반적인 함수와 다르므로\footnote{https://poiemaweb.com/es6-arrow-function} 메서드를 표현할 때는 arrow function보다는 전통적인 함수 표현 방식을 사용하는 것이 좋다.

\subsection*{Class and Instance}

앞에서는 Frank라는 이름을 가진 사람에 대한 정보를 다루기 위해 \cd{person}이라는 객체를 생성하고 속성과 메서드를 정의하였다. 그런데 Frank라는 사람 한 명뿐만 아니라 같은 속성을 갖는 수많은 사람에 대한 정보를 다룰 때 \coderef{code:object-method}\와 같이 객체를 일일이 생성하는 방식은 매우 비효율적이며 재사용성이 크게 떨어진다.

이렇게 동일한 형태의 객체들을 생성하기 위해 \textbf{클래스(class)}를 이용하여 객체들의 구조를 표현하며, 클래스는 설계도와 같은 역할을 한다. 클래스는 각 객체가 갖는 속성, 상수, 메서드 등에 대한 정보의 집합이다. 그리고 이러한 설계도, 즉 클래스에 따라 생성된 각 객체를 \textbf{인스턴스(instance)}라고 한다. 예를 들어 \cd{Person}이라는 클래스가 있다면 이 클래스에 따라 생성된 객체 \cd{frank}, \cd{david}, \cd{jennie} 등은 인스턴스이다.

인스턴스를 생성하려면 클래스라는 설계도를 이용하여 객체를 생성하는 매개체가 필요하다. 이 매개체를 \textbf{생성자(constructor)}라고 부르며, 생성자는 인스턴스를 생성할 때 필요한 정보를 받아, 설계도에 따라 객체의 속성이나 메서드를 정한다. 예를 들어, 앞의 \cd{Person} 클래스의 생성자는 나이, 이름, 키, 성별 등의 정보를 받아 \cd{age}, \cd{name}, \cd{height}, \cd{isMale}의 속성값에 저장할 것이다. 또한, 생성자는 여러 유의미한 정보가 응집되어 있는 정보를 나누어 서로 다른 속성에 저장하기도 한다. 예를 들어, 학생의 학번을 받아서 입학 연도, 소속 학과, 개인 일련번호 등의 정보를 저장하는 \cd{Student}라는 클래스를 가정하면 이 클래스의 생성자는 학번을 분석하여 \cd{admissionYear}, \cd{department}, \cd{serialNumber} 등의 속성에 분석한 값들을 저장할 수 있다.

본 교재에서는 생성자를 직접 작성하는 것은 다루지 않고, 생성자를 이용하여 인스턴스를 생성할 때는 \cd{new} 키워드를 사용한다. 예를 들어 \cd{Person} 클래스는 \cd{age}, \cd{name}, \cd{height}, \cd{isMale} 값을 인자로 받아 인스턴스를 생성한다면 \cd{jennie}라는 인스턴스는 \cd{jennie = new Person(24, 'Jennie', 163, false);}와 같이 생성할 수 있다. 또한, 어떤 속성이나 메서드는 생성자에 의해 할당되는 속성이나 메서드가 필요하지 않을 수 있는데, 이러한 메서드를 정적(static) 속성 및 메서드라고 하며, 생성자를 사용하지 않고 클래스에서 직접 속성이나 메서드를 호출할 수 있다.

여담으로, 객체는 \cd{const} 키워드로 선언되어도 그 속성이나 메서드는 재할당될 수 있다.

\subsection*{Global Methods(Functions)}

다음 메서드들은 객체를 명시하지 않고 함수 형태로 사용할 수 있는 메서드들이다. 이 함수들은 사실 전역 객체(global object)의 메서드이므로 함수처럼 사용할 수 있는 것이다.

\begin{itemize}
    \item \cd{isFinite(testValue) => boolean}: 주어진 값이 유한한지 판별하는 메서드; \cd{testValue}를 인자로 받아 \cd{Infinity}, \cd{NaN}, \cd{undefined}이면 \cd{false}, 아니면 \cd{true}를 반환한다.
    \item \cd{isNaN(testValue) => boolean}: 주어진 값이 \cd{NaN}인지 판별하는 메서드; \cd{testValue}를 인자로 받아 \cd{NaN}이면 \cd{false}, 아니면 \cd{true}를 반환한다.
    \item \cd{parseInt(string[, radix]) => number}: 주어진 문자열을 숫자로 변환하여 반환하는 메서드; \cd{string}과 \cd{radix}를 인자로 받아, \cd{string}을 \cd{radix}진법의 정수로 변환하여 반환한다. 주어진 값을 정수로 변환할 수 없다면 \cd{NaN}을 반환하며, \cd{radix}가 주어지지 않을 때는 \cd{string}이 \cd{0x}나 \cd{0X}로 시작하면 16진수로, 그 외에는 10진수로 변환한다.
\end{itemize}

\begin{codeenv}{code:global-methods}{Global Methods}\begin{verbatim}
> isFinite(3/0);
false
> isNaN(0/0);
true
> parseInt('3.5');
3
> parseInt('hello');
NaN
\end{verbatim}
\end{codeenv}

\subsection*{String Objects}

string형 식별자, 즉 \cd{String} 객체의 자주 쓰이는 속성이나 메서드이며, MDN 문서\footnote{https://developer.mozilla.org/en-US/docs/Web/JavaScript/Reference/Global\_Objects/String}에서 더 자세하게 확인할 수 있다.

\begin{itemize}
    \item \cd{length}: 문자열의 길이를 값으로 갖는 속성
    \item \cd{includes(searchString[, position]) => boolean}: 문자열에 주어진 문자열이 포함되는지 판별하는 메서드; 원래 문자열에 \cd{searchString} 문자열이 \cd{position}번째 문자 이후에 포함되면 \cd{true}, 포함되지 않으면 \cd{false}를 반환한다. \cd{position}의 기본값은 \cd{0}이다.
    \item \cd{replace(substr, newSubstr) => string}: 문자열의 일부분을 다른 문자열로 대치(代置)하여 반환하는 메서드; 원래 문자열에서 처음으로 나오는 \cd{substr} 문자열 부분을 \cd{newSubstr}로 대치한다. \cd{substr}은 정규식 형태로 주어질 수도 있다.
    \item \cd{replaceAll(substr, newSubstr) => string}: 문자열의 일부분을 모두 다른 문자열로 대치하여 반환하는 메서드; \cd{replace} 메서드와 유사하게 동작하나 \cd{substr} 문자열 부분을 모두 \cd{newSubstr}로 대치한다.
    \item \cd{split([separator, limit]) => Array<string>}: 문자열을 주어진 문자열을 기준으로 분리하여 문자열의 배열로 반환하는 메서드; 원래 문자열에서 \cd{separator}를 기준으로 분리하여 최대 \cd{limit}개의 문자열을 반환한다. \cd{separator}가 주어지지 않으면 분리되지 않으며, \cd{limit}이 주어지지 않으면 반환하는 문자열 배열의 길이에 제한이 없다.
    \item \cd{substring(indexStart[, indexEnd]) => string}: 문자열의 일부분을 추출하는 메서드; \cd{indexStart}번째 문자부터 \cd{indexEnd}번째 문자 직전까지의 문자열을 추출하여 반환하며, \cd{indexEnd}의 기본값은 \cd{length} 속성값이다.
    \item \cd{toLowerCase() => string}: 문자열 중 알파벳을 모두 소문자로 바꾸어 반환하는 메서드
    \item \cd{toUpperCase() => string}: 문자열 중 알파벳을 모두 대문자로 바꾸어 반환하는 메서드
    \item \cd{trim() => string}: 문자열의 앞뒤 whitespace를 모두 제거하여 반환하는 메서드
\end{itemize}

\begin{codeenv}{code:string-methods}{String Object Methods}\begin{verbatim}
> const str = '  Hello World!\t';
> str.length;
15
> str.includes('ll');
true
> str.replace('World', 'Korea');
'  Hello Korea!\t'
> str.split('o');
[ '  Hell', ' W', 'rld!\t' ]
> str.substring(6);
'o World!\t'
> str.toUpperCase();
'  HELLO WORLD!\t'
> str.trim();
'Hello World!'
\end{verbatim}
\end{codeenv}

\subsection*{Array Objects}

배열에서 자주 쓰이는 속성이나 메서드이며, MDN 문서\footnote{https://developer.mozilla.org/en-US/docs/Web/JavaScript/Reference/Global\_Objects/Array}에서 더 자세하게 확인할 수 있다. 많은 메서드가 high-order 함수임을 확인할 수 있다.

\begin{itemize}
    \item \cd{length}: 배열의 길이를 값으로 갖는 속성
    \item \cd{concat([arr1, arr2, ..., arrN]) => Array}: 원래 배열의 뒤에 여러 배열을 이어붙여 반환하는 메서드
    \item \cd{filter(fn(element[, index, array]) => boolean) => Array}: 배열의 원소 중 주어진 test를 통과한 원소들의 배열을 반환하는 메서드. 배열의 원소 중 \cd{fn}의 반환값이 \cd{true}인 원소들의 배열을 반환한다. \cd{fn}은 원소의 값, index, 원본 배열을 인자로 받아 boolean 값을 반환한다.
    \item \cd{find(fn(element[, index, array]) => boolean) => Any}: 배열의 원소 중 주어진 함수를 만족하는 첫 번째 원소를 반환하는 메서드
    \item \cd{findIndex(fn(element[, index, array]) => boolean) => number}: 배열의 원소 중 주어진 함수를 만족하는 첫 번째 원소의 index를 반환하는 메서드
    \item \cd{forEach(fn(element[, index, array]))}: 배열의 각 원소에 대해 주어진 함수를 실행하는 메서드
    \item \cd{includes(searchElement[, fromIndex]) => boolean}: 배열의 원소에 주어진 원소가 있는지 판별하는 메서드; 배열의 \cd{fromIndex}번째 원소 이후에 \cd{searchElement}가 있으면 \cd{true}를 반환한다. \cd{fromIndex}의 기본값은 \cd{0}이다.
    \item \cd{join([separator]) => string}: 배열의 모든 원소를 주어진 문자열로 이어붙인 문자열을 반환하는 메서드
    \item \cd{map(fn(element[, index, array]) => Any) => Array}: 배열의 각 원소에 대한 주어진 함수의 반환값의 배열을 반환하는 메서드
    \item \cd{pop() => Any}: 배열의 마지막 원소를 제거하고, 마지막 원소의 값을 반환하는 메서드
    \item \cd{push([val0, val1, ..., valN]) => number}: 배열의 끝에 주어진 값을 배열의 원소로 차례대로 삽입하고, 값이 모두 삽입된 배열의 길이를 반환하는 메서드
    \item \cd{reverse() => Array}: 배열의 원소들의 순서를 반대로 뒤집어 반환하는 메서드; 원본 배열도 뒤집힌다는 점을 주의하여야 한다.
    \item \cd{slice([start, end]) => Array}: 배열의 일부분을 추출하는 메서드; \cd{start}번째 원소부터 \cd{end}번째 원소 직전까지의 원소를 반환하며, \cd{start}의 기본값은 \cd{0}, \cd{end}의 기본값은 \cd{length} 속성이다.
    \item \cd{sort([compareFn(firstEl, secondEl) => number])}: 배열의 원소들을 주어진 함수에 따라 정렬하여 반환하는 메서드; \cd{compareFn}은 두 비교 대상 원소를 인자로 받으며, 반환값이 음수인 경우 \cd{firstEl}이, 양수인 경우 \cd{secondEl}이 앞에 정렬된다. \cd{compareFn}이 주어지지 않은 경우 알파벳 순서대로 정렬된다. 원본 배열도 정렬된다는 점을 주의하여야 한다.
\end{itemize}

\begin{codeenv}{code:array-methods}{Array Methods}\begin{verbatim}
> const arr = ['apple', 'orange', 'banana'];
> arr.length;
3
> arr.concat(['cherry', 'grapes']);
[ 'apple', 'orange', 'banana', 'cherry', 'grapes' ]
> arr.filter(fruit => fruit.length === 6);
[ 'orange', 'banana' ]
> arr.findIndex(fruit => fruit[0] === 'b');
2
> arr.includes('lemon');
false
> arr.join(' + ');
'apple + orange + banana'
> arr.map((elmt, index) => '#' + index + ': ' + elmt);
[ '#0: apple', '#1: orange', '#2: banana' ]
> arr.pop();
'banana'
> arr.push('cherry', 'grapes');
4
> arr.reverse();
[ 'grapes', 'cherry', 'orange', 'apple' ]
> arr.sort();
[ 'apple', 'cherry', 'grapes', 'orange' ]
> arr.sort((first, second) => {
      const firstLength = first.split('e')[0].length;
      const secondLength = second.split('e')[0].length;
      return firstLength - secondLength;
  });
[ 'cherry', 'apple', 'grapes', 'orange' ]
\end{verbatim}
\end{codeenv}


\subsection*{Date Objects}

\cd{Date} 클래스는 날짜와 시간을 다루는 클래스이며, \cd{Date} 클래스의 인스턴스를 생성하기 위해서는 생성자를 이용하여야 하며, \cd{Date} 클래스의 생성자는 네 가지가 있다.

\begin{itemize}
    \item \cd{new Date()}: 현재 날짜/시각을 나타내는 \cd{Date} 객체를 생성한다.
    \item \cd{new Date(msec)}: Epoch 시각\footnote{1970년 1월 1일 자정 UTC}으로부터 \cd{msec}밀리초가 지난 후의 날짜/시각을 나타내는 \cd{Date} 객체를 생성한다.
    \item \cd{new Date(dateStr)}: \cd{dateStr}을 분석하여 문자열이 표현하는 날짜/시각을 나타내는 \cd{Date} 객체를 생성한다.
    \item \cd{new Date(year, mon[, day, hrs, min, sec, msec])}: \cd{year}년 \cd{mon}월 \cd{day}일 \cd{hrs}시 \cd{min}분 \cd{sec}초 \cd{msec} 밀리초에 해당하는 날짜/시각을 나타내는 \cd{Date} 객체를 생성한다. \cd{mon}의 값은 \cd{0}이 1월, \cd{11}이 12월이다.
\end{itemize}

\cd{Date} 객체의 메서드 몇 가지를 나열하면 다음과 같으며, MDN 문서\footnote{https://developer.mozilla.org/en-US/docs/Web/JavaScript/Reference/Global\_Objects/Date}에서 더 자세하게 확인할 수 있다.

\begin{itemize}
    \item \cd{now() => number}: Epoch 시각으로부터 현재까지 흐른 시간을 밀리초 단위로 반환하는 정적 메서드
    \item \cd{getFullYear() => number}: \cd{Date} 객체의 날짜/시각의 연도를 반환하는 메서드
    \item \cd{getMinutes() => number}: \cd{Date} 객체의 날짜/시각의 분(分)을 반환하는 메서드
    \item \cd{getUTCDate() => number}: \cd{Date} 객체의 날짜/시각의 UTC 날짜를 반환하는 메서드
    \item \cd{getUTCHours() => number}: \cd{Date} 객체의 날짜/시각의 UTC 시각을 반환하는 메서드
    \item \cd{toLocaleString() => string}: \cd{Date} 객체의 날짜/시각을 현재 지역에서 나타내는 형태의 문자열로 변환하여 반환하는 메서드
\end{itemize}

\begin{codeenv}{code:date-methods}{Date Object Methods}\begin{verbatim}
> const date = new Date(2021, 6, 19);
undefined
> date;
2021-07-18T15:00:00.000Z
> date.getFullYear();
2021
> date.getHours();
0
> date.getUTCHours();
15
> date.toLocaleString();
'2021. 7. 19. 오전 12:00:00'
> const current = new Date();
undefined
> current.getMonth();
4                                   // result may vary
> current.getMinutes();
52                                  // result may vary
> current.getMilliseconds();
584                                 // result may vary
\end{verbatim}
\end{codeenv}


\subsection*{Math Objects}

\cd{Math} 객체는 수학 연산과 관련된 객체이며, 모든 속성과 메서드가 정적이라는 특징을 갖는다. \cd{Math} 객체의 속성과 메서드 몇 가지를 소개하며, MDN 문서\footnote{https://developer.mozilla.org/en-US/docs/Web/JavaScript/Reference/Global\_Objects/Math}에서 더 자세하게 확인할 수 있다.

\begin{itemize}
    \item \cd{E}: 자연상수 $e (\approx 2.718)$를 값으로 갖는 속성
    \item \cd{LN2}: $\log 2 (\approx 0.693)$를 값으로 갖는 속성
    \item \cd{PI}: 원주율 $\pi (\approx 3.142)$를 값으로 갖는 속성
    \item \cd{SQRT2}: $\sqrt 2 (\approx 1.414)$를 값으로 갖는 속성
    \item \cd{abs(x) => number}: 주어진 값의 절댓값($|x|$)을 반환하는 메서드
    \item \cd{ceil(x) => number}: 주어진 값보다 큰 가장 작은 정수(올림, $\lceil x\rceil$)를 반환하는 메서드
    \item \cd{cos(x) => number}: 주어진 값의 코사인 값($\cos x$)을 반환하는 메서드 (\cd{x}의 단위는 radian)
    \item \cd{exp(x) => number}: 주어진 값의 exponential 값($e^x$)을 반환하는 메서드
    \item \cd{floor(x) => number}: 주어진 값보다 작은 가장 큰 정수(버림, $\lfloor x\rfloor$)를 반환하는 메서드
    \item \cd{log(x) => number}: 주어진 값의 자연로그 값($\log x$)을 반환하는 메서드
    \item \cd{max([val1, val2, ..., valN]) => number}: 주어진 값들 중 가장 큰 값을 반환하는 메서드
    \item \cd{min([val1, val2, ..., valN]) => number}: 주어진 값들 중 가장 작은 값을 반환하는 메서드
    \item \cd{pow(x, y) => number}: $x^y$의 값을 반환하는 메서드\footnote{ES7 이후에는 \sectref{sect:data-types}에서 다룬 지수 연산자를 사용하는 것이 더 간결하다.}
    \item \cd{random() => number}: 0 이상 1 미만의 랜덤값을 반환하는 메서드
    \item \cd{round(x) => number}: 주어진 값에 가장 가까운 정수(반올림, $\lfloor x + 0.5\rfloor$)를 반환하는 메서드
    \item \cd{sqrt(x) => number}: 주어진 값의 square root값($\sqrt{x}$)을 반환하는 메서드
\end{itemize}

\begin{codeenv}{code:math-methods}{Math Object Methods}\begin{verbatim}
> Math.SQRT2;
1.4142135623730951
> Math.max(3, 4, 5);
5
> 0.5 * Math.PI * Math.pow(3, 2);
14.137166941154069
Math.tan(Math.PI / 3) * Math.sqrt(3);
2.9999999999999987
> Math.ceil((Math.random() * 10) + 10);
17                                  // result may vary
\end{verbatim}
\end{codeenv}
